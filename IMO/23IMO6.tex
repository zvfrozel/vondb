author: Ankan Bhattacharya, Luke Robitaille (USA)
desc: $480$-degree geometry
hardness: 45
source: IMO 2023/6
url: https://aops.com/community/p28104331
tags: [2023-11, pop, good, yod]

---

Let $ABC$ be an equilateral triangle.
Let $A_1$, $B_1$, $C_1$ be interior points of $ABC$
such that $BA_1=A_1C$, $CB_1=B_1A$, $AC_1=C_1B$, and
\[ \angle BA_1C + \angle CB_1A + \angle AC_1B = 480\dg. \]
Let $A_2 = \ol{BC_1} \cap \ol{CB_1}$, $B_2 = \ol{CA_1} \cap \ol{AC_1}$,
$C_2 = \ol{AB_1} \cap \ol{BA_1}$.
Prove that if triangle $A_1B_1C_1$ is scalene,
then the circumcircles of triangles $AA_1A_2$, $BB_1B_2$, and $CC_1C_2$
all pass through two common points.

---

This is the second official solution from the marking scheme,
also communicated to me by Michael Ren.
Define $O$ as the center of $ABC$ and set the angles
\begin{align*}
  \alpha &\coloneqq \angle A_1CB = \angle CBA_1 \\
  \beta &\coloneqq \angle ACB_1 = \angle B_1AC \\
  \gamma &\coloneqq \angle C_1AB = \angle C_1BA
\end{align*}
so that
\[ \alpha + \beta + \gamma = 30\dg. \]
In particular, $\max(\alpha,\beta,\gamma) < 30\dg$,
so it follows that $A_1$ lies inside $\triangle OBC$, and similarly for the others.
This means for example that $C_1$ lies between $B$ and $A_2$, and so on.
Therefore the polygon $A_2C_1B_2A_1C_2B_1$ is convex.

\begin{center}
\begin{asy}
size(14cm);
pair A = dir(90);
pair B = dir(210);
pair C = dir(330);
real s = abs(B-C)/2;
pair A_1 = midpoint(B--C) + dir(A) * s*Tan(11);
pair B_1 = midpoint(C--A) + dir(B) * s*Tan(13);
pair C_1 = midpoint(A--B) + dir(C) * s*Tan(6);
pair A_1 = A_1;
pair B_1 = B_1;
pair C_1 = C_1;
pair O = origin;
filldraw(A--B--C--cycle, opacity(0.1)+lightcyan, blue);
draw(A--A_1, blue);
draw(B--B_1, blue);
draw(C--C_1, blue);
pair A_2 = extension(B, C_1, B_1, C);
pair B_2 = extension(C, A_1, C_1, A);
pair C_2 = extension(A, B_1, B, A_1);
draw(B--A_2--C, lightred);
draw(C--B_2--A, lightred);
draw(A--C_2--B, lightred);
markangle("$\alpha$", n=1, radius=32.0, C, B, A_1, deepgreen);
markangle("$\alpha$", n=1, radius=32.0, A_1, C, B, deepgreen);
markangle("$\beta$",  n=2, radius=56.0, A, C, B_1, deepgreen);
markangle("$\beta$",  n=2, radius=56.0, B_1, A, C, deepgreen);
markangle("$\gamma$", n=3, radius=64.0, B, A, C_1, deepgreen);
markangle("$\gamma$", n=3, radius=64.0, C_1, B, A, deepgreen);

draw(circumcircle(B_1, C_1, B_2), grey);
draw(circumcircle(C_1, A_1, C_2), grey);
draw(circumcircle(A_1, B_1, A_2), grey);

pair Z = A_1;
pair X = -Z+2*foot(circumcenter(A_1, A_2, B_1), A, O);
pair Y = -Z+2*foot(circumcenter(A_1, A_2, C_1), A, O);

dot("$A$", A, dir(A));
dot("$B$", B, dir(B));
dot("$C$", C, dir(C));
dot("$A_1$", A_1, dir(A_1));
dot("$B_1$", B_1, dir(B_1));
dot("$C_1$", C_1, dir(C_1));
dot("$O$", O, 1.4*dir(240));
dot("$A_2$", A_2, dir(A_2));
dot("$B_2$", B_2, dir(B_2));
dot("$C_2$", C_2, dir(320));
dot("$X$", X, 1.7*dir(295));
dot("$Y$", Y, dir(55));

/* -----------------------------------------------------------------+
|                 TSQX: by CJ Quines and Evan Chen                  |
| https://github.com/vEnhance/dotfiles/blob/main/py-scripts/tsqx.py |
+-------------------------------------------------------------------+
!size(14cm);
A = dir 90
B = dir 210
C = dir 330
!real s = abs(B-C)/2;
!pair A_1 = midpoint(B--C) + dir(A) * s*Tan(11);
!pair B_1 = midpoint(C--A) + dir(B) * s*Tan(13);
!pair C_1 = midpoint(A--B) + dir(C) * s*Tan(6);
A_1 = A_1
B_1 = B_1
C_1 = C_1
O 1.4R240 = origin
A--B--C--cycle / 0.1 lightcyan / blue
A--A_1 / blue
B--B_1 / blue
C--C_1 / blue
A_2 = extension B C_1 B_1 C
B_2 = extension C A_1 C_1 A
C_2 320 = extension A B_1 B A_1
B--A_2--C / lightred
C--B_2--A / lightred
A--C_2--B / lightred
!markangle("$\alpha$", n=1, radius=32.0, C, B, A_1, deepgreen);
!markangle("$\alpha$", n=1, radius=32.0, A_1, C, B, deepgreen);
!markangle("$\beta$",  n=2, radius=56.0, A, C, B_1, deepgreen);
!markangle("$\beta$",  n=2, radius=56.0, B_1, A, C, deepgreen);
!markangle("$\gamma$", n=3, radius=64.0, B, A, C_1, deepgreen);
!markangle("$\gamma$", n=3, radius=64.0, C_1, B, A, deepgreen);
circumcircle B_1 C_1 B_2 / grey
circumcircle C_1 A_1 C_2 / grey
circumcircle A_1 B_1 A_2 / grey
Z := A_1
X 1.7R295 = -Z+2*foot (circumcenter A_1 A_2 B_1) A O
Y 55 = -Z+2*foot (circumcenter A_1 A_2 C_1) A O
*/

\end{asy}
\end{center}

We start by providing the ``interpretation'' for the $480\dg$ angle in the statement:
\begin{claim*}
  Point $A_1$ is the circumcenter of $\triangle A_2BC$, and similarly for the others.
\end{claim*}
\begin{proof}
  We have $\angle BA_1C = 180\dg - 2\alpha$, and
  \begin{align*}
    \angle BA_2C &= 180\dg - \angle CBC_1 - \angle B_1CB \\
    &= 180\dg - \left( 60\dg - \gamma \right) - \left( 60\dg - \beta \right) \\
    &= 60\dg + \beta + \gamma = 90\dg - \alpha = \half \angle BA_1 C.
  \end{align*}
  Since $A_1$ lies inside $\triangle BA_2C$, it follows $A_1$ is exactly the circumcenter.
\end{proof}

\begin{claim*}
  Quadrilateral $B_2C_1B_1C$ can be inscribed in a circle, say $\gamma_a$.
  Circles $\gamma_b$ and $\gamma_c$ can be defined similarly.
  Finally, these three circles are pairwise distinct.
\end{claim*}
\begin{proof}
  Using directed angles now, we have
  \[ \dang B_2B_1C_2 = 180\dg-\dang AB_1B_2
    = 180\dg - 2 \dang ACB = 180\dg-2(60\dg-\alpha) = 60\dg+2\alpha. \]
  By the same token, $\dang B_2C_1C_2 = 60\dg+2\alpha$.
  This establishes the existence of $\gamma_a$.

  The proof for $\gamma_b$ and $\gamma_c$ is the same.
  Finally, to show the three circles are distinct, it would be enough to
  verify that the convex hexagon $A_2C_1B_2A_1C_2B_1$ is not cyclic.

  Assume for contradiction it was cyclic. Then
  \[ 360\dg = \angle C_2A_1B_1 + \angle B_2C_1A_2 + \angle A_2B_1C_2
    = \angle BA_1C + \angle CB_1A + \angle AC_1B = 480\dg \]
  which is absurd.
  This contradiction eliminates the degenerate case, so the three circles are distinct.
\end{proof}

For the remainder of the solution, let $\opname{Pow}(P, \omega)$
denote the power of a point $P$ with respect to a circle $\omega$.

Let line $AA_1$ meet $\gamma_b$ and $\gamma_c$ again at $X$ and $Y$,
and set $k_a \coloneqq \frac{AX}{AY}$.
Consider the locus of all points $P$ such that
\[ \mathcal C_a \coloneqq \Big\{ \text{points } P \text{ in the plane satisfying }
    \opname{Pow}(P, \gamma_b) = k_a \opname{Pow}(P, \gamma_c) \Big\}. \]
We recall the \emph{coaxiality lemma}\footnote{We quickly outline a proof of this lemma:
  in the Cartesian coordinate system, the expression $\opname{Pow}((x,y), \omega)$
  is an expression of the form $x^2 + y^2 + {\bullet} x + {\bullet} y + {\bullet}$
  for some constants ${\bullet}$ whose value does not matter.
  Substituting this into the equation
  $\frac{k_a \opname{Pow}(P, \gamma_c) - \opname{Pow}(P, \gamma_b)}{k_a-1} = 0$
  gives the equation of a circle provided $k_a \neq 1$,
  and when $k_a = 1$, one instead recovers the radical axis.},
which states that (given $\gamma_b$ and $\gamma_c$ are not concentric)
the locus $\mathcal C_a$ must be either a circle (if $k_a \neq 1$) or a line (if $k_a=1$).

On the other hand, $A_1$, $A_2$, and $A$ all obviously lie on $\mathcal C_a$.
(For $A_1$ and $A_2$, the powers are both zero, and for the point $A$,
we have $\opname{Pow}(P, \gamma_b) = AX \cdot AA_1$
and $\opname{Pow}(P, \gamma_c) = AY \cdot AA_1$.)
So $\mathcal C_a$ must be exactly the circumcircle of $\triangle AA_1A_2$
from the problem statement.

We turn to evaluating $k_a$ more carefully.
First, note that
\[ \angle A_1XB_1 = \angle A_1B_2B_1 = \angle CB_2B_1
  = 90\dg - \angle B_2AC = 90\dg - (60\dg-\gamma) = 30\dg + \gamma. \]
Now using the law of sines, we derive
\begin{align*}
  \frac{AX}{AB_1} &= \frac{\sin \angle AB_1X}{\sin \angle AXB_1}
  = \frac{\sin(\angle A_1XB_1 - \angle XAB_1)}{\sin \angle A_1XB_1} \\
  &= \frac{\sin\left( (30\dg+\gamma)-(30\dg-\beta) \right)}{\sin (30\dg+\gamma)}
  = \frac{\sin(\beta+\gamma)}{\sin (30\dg+\gamma)}.
\end{align*}
Similarly, $AY = AC_1 \cdot \frac{\sin(\beta+\gamma)}{\sin(30\dg+\beta)}$, so
\[ k_a = \frac{AX}{AY}
  = \frac{AB_1}{AC_1} \cdot \frac{\sin(30\dg+\beta)}{\sin(30\dg+\gamma)}. \]
Now define analogous constants $k_b$ and $k_c$
and circles $\mathcal C_b$ and $\mathcal C_c$.
Owing to the symmetry of the expressions, we have the key relation
\[ k_a k_b k_c = 1. \]

In summary, the three circles in the problem statement may be described as
\begin{align*}
  \mathcal C_a &= (AA_1A_2)
  = \left\{ \text{points $P$ such that } \opname{Pow}(P, \gamma_b) = k_a \opname{Pow}(P, \gamma_c) \right\} \\
  \mathcal C_b &= (BB_1B_2)
  = \left\{ \text{points $P$ such that } \opname{Pow}(P, \gamma_c) = k_b \opname{Pow}(P, \gamma_a) \right\} \\
  \mathcal C_c &= (CC_1C_2)
  = \left\{ \text{points $P$ such that } \opname{Pow}(P, \gamma_a) = k_c \opname{Pow}(P, \gamma_b) \right\}.
\end{align*}
Since $k_a$, $k_b$, $k_c$ have product $1$,
it follows that any point on at least two of the circles
must lie on the third circle as well.
The convexity of hexagon $A_2C_1B_2A_1C_2B_1$ mentioned earlier ensures these
any two of these circles do intersect at two different points, completing the solution.
