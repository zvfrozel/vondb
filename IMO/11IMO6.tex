desc: Nice Japan problem, complex after guessing concurrency
source: IMO 2011/6
tags: [2016-08, rich, nice, complex, criticalclaim, brave, yod]
author: Japan
hardness: 45
url: https://aops.com/community/p2365045

---

Let $ABC$ be an acute triangle with circumcircle $\Gamma$.
Let $\ell$ be a tangent line to $\Gamma$, and let $\ell_a$, $\ell_b$, $\ell_c$ be the lines obtained
by reflecting $\ell$ in the lines $BC$, $CA$, and $AB$, respectively.
Show that the circumcircle of the triangle determined by the lines $\ell_a$, $\ell_b$, and $\ell_c$
is tangent to the circle $\Gamma$.

---

This is a hard problem with many beautiful solutions.
The following solution is not very beautiful but not too hard to find during an olympiad,
as the only major insight it requires is the construction of $A_2$, $B_2$, and $C_2$.

\begin{center}
  \begin{asy}
    size(11cm);
    pair A = dir(110);
    dot("$A$", A, dir(A));
    pair B = dir(195);
    dot("$B$", B, dir(160));
    pair C = dir(325);
    dot("$C$", C, 1.4*dir(30));
    pair P = dir(270);
    dot("$P$", P, dir(P));
    draw(unitcircle);
    draw(A--B--C--cycle);

    pair U = P+(2,0);
    pair V = 2*P-U;

    pair X_1 = reflect(B,C)*P;
    pair Y_1 = reflect(C,A)*P;
    pair Z_1 = reflect(A,B)*P;
    pair X_2 = extension(B, C, U, V);
    dot(X_2);
    pair Y_2 = extension(C, A, U, V);
    dot(Y_2);
    pair Z_2 = extension(A, B, U, V);
    dot(Z_2);
    draw(B--Z_2, dotted);
    draw(C--Y_2, dotted);
    draw(C--X_2, dotted);
    draw(X_2--Z_2);

    pair A_1 = extension(Y_1, Y_2, Z_1, Z_2);
    dot("$A_1$", A_1, dir(A_1));
    pair B_1 = extension(Z_1, Z_2, X_1, X_2);
    dot("$B_1$", B_1, dir(B_1));
    pair C_1 = extension(X_1, X_2, Y_1, Y_2);
    dot("$C_1$", C_1, dir(50));

    draw(A_1--B_1--C_1--cycle, black+1);
    draw(C_1--X_2, dotted);
    pair O_1 = circumcenter(A_1, B_1, C_1);
    draw(arc(O_1, abs(O_1-A_1), -80, 140));

    pair A_2 = A*A/P;
    dot("$A_2$", A_2, dir(-20));
    pair B_2 = B*B/P;
    dot("$B_2$", B_2, dir(130));
    pair C_2 = C*C/P;
    dot("$C_2$", C_2, dir(C_2));
    draw(A_2--B_2--C_2--cycle, black+1);

    pair T = extension(A_1, A_2, B_1, B_2);
    dot("$T$", T, dir(T));
    draw(T--A_1, dashed);
    draw(T--B_1, dashed);
    draw(T--C_1, dashed);

    /*
    A = dir 110
    B = dir 195
    C = dir 325
    P = dir 270

    unitcircle
    A--B--C--cycle blue
    U := P+(2,0)
    V := 2*P-U
    X = reflect(B,C)*P
    Y = reflect(C,A)*P
    Z = reflect(A,B)*P
    Y--Z heavygreen

    X1 := extension B C U V
    Y1 := extension C A U V
    Z1 := extension A B U V
    Line Y1 Z1

    A_1 = extension Y Y1 Z Z1
    B_1 = extension Z Z1 X X1
    C_1 = extension X X1 Y Y1
    A_1--B_1--C_1--cycle red
    circumcircle A_1 B_1 C_1

    circumcircle B Z X dotted
    circumcircle C X Y dotted
    circumcircle A Y Z dotted

    M = extension A_1 A*A/P B_1 B*B/P
    */
  \end{asy}
\end{center}

We apply complex numbers with $\omega$ the unit circle and $p=1$.  Let $A_1 = \ell_B \cap \ell_C$, and let $a_2 = a^2$ (in other words, $A_2$ is the reflection of $P$ across the diameter of $\omega$ through $A$).  Define the points $B_1$, $C_1$, $B_2$, $C_2$ similarly.

We claim that $\ol{A_1A_2}$, $\ol{B_1B_2}$, $\ol{C_1C_2}$ concur at a point on $\Gamma$.

We begin by finding $A_1$. If we reflect the points $1+i$ and $1-i$ over $\ol{AB}$, then we get two points $Z_1$, $Z_2$ with
\begin{align*}
  z_1 &= a+b-ab(1-i) = a+b-ab+abi \\
  z_2 &= a+b-ab(1+i) = a+b-ab-abi. \\
  \intertext{Therefore,}
  z_1 - z_2 &= 2abi  \\
  \ol{z_1}z_2 - \ol{z_2}{z_1}
    &= -2i \left( a+b+\frac1a+\frac1b-2 \right).
\end{align*}

Now $\ell_C$ is the line $\ol{Z_1Z_2}$,
so with the analogous equation $\ell_B$ we obtain:
\begin{align*}
  a_1 &= \frac{ -2i\left( a+b+\frac1a+\frac1b-2 \right)\left( 2ac i \right) +
    2i\left( a+c+\frac1a+\frac1c-2 \right)(2abi) }
    { \left( -\frac{2}{ab}i \right)
    \left( 2ac i \right) - \left( -\frac{2}{ac}i \right) \left( 2abi \right)} \\
  &= \frac{\left[ c-b \right]a^2 + \left[ \frac cb - \frac bc - 2c + 2b \right]a + (c-b)  }{\frac cb - \frac bc} \\
  &= a + \frac{(c-b)\left[ a^2-2a+1 \right]}{(c-b)(c+b)/bc} \\
  &= a + \frac{bc}{b+c} (a-1)^2.
\end{align*}
Then the second intersection of $\ol{A_1A_2}$ with $\omega$ is given by
\begin{align*}
  \frac{a_1-a_2}{1-a_2\ol{a_1}}
  &= \frac{a+\frac{bc}{b+c}(a-1)^2-a^2}{1-a-a^2 \cdot \frac{(1-1/a)^2}{b+c}} \\
  &= \frac{a + \frac{bc}{b+c}(1-a)}{1 - \frac{1}{b+c}(1-a)} \\
  &= \frac{ab+bc+ca - abc}{a+b+c-1}.
\end{align*}
Thus, the claim is proved.

Finally, it suffices to show $\ol{A_1B_1} \parallel \ol{A_2B_2}$.
One can also do this with complex numbers;
it amounts to showing $a^2-b^2$, $a-b$, $i$
(corresponding to $\ol{A_2 B_2}$, $\ol{A_1 B_1}$, $\ol{PP}$)
have their arguments an arithmetic progression, equivalently
\[ \frac{(a-b)^2}{i(a^2-b^2)} \in \RR
  \iff
  \frac{(a-b)^2}{i(a^2-b^2)}
  = \frac{\left( \frac 1a-\frac1b \right)^2}
  {\frac1i\left(\frac{1}{a^2}-\frac{1}{b^2}\right)}
\]
which is obvious.
\begin{remark*}
One can use directed angle chasing for this last part too.
Let $\ol{BC}$ meet $\ell$ at $K$ and $\ol{B_2C_2}$ meet $\ell$ at $L$.
Evidently
\begin{align*}
  -\dang B_2LP &= \dang LPB_2 + \dang PB_2L \\
  &= 2 \dang KPB + \dang PB_2C_2 \\
  &= 2 \dang KPB + 2\dang PBC \\
  &= -2\dang PKB \\
  &= \dang PKB_1 \\
\end{align*}
as required.
\end{remark*}

---



In this solution we will use a more general notion of a directed angle
between two lines.
Suppose lines $\ell_1$ and $\ell_2$ intersect at $X$.
Then we define
\[ \dang \left( \ell_1, \ell_2 \right) \]
to be equal to the directed angle $\dang A_1XA_2$,
where $A_1$ and $A_2$ are points on $\ell_1$ and $\ell_2$,
respectively, distinct from $X$.
(It is clear that the directed angle $\dang A_1XA_2$ does not depend
on the choice of $A_1$ and $A_2$).

\begin{figure}[ht]
  \centering
  \begin{asy}
    size(11.5cm);
    pair A = dir(110);
    pair B = dir(195);
    pair C = dir(325);
    pair P = dir(270);

    draw(unitcircle);
    draw(A--B--C--cycle, blue);
    pair U = P+(2,0);
    pair V = 2*P-U;
    pair X = reflect(B,C)*P;
    pair Y = reflect(C,A)*P;
    pair Z = reflect(A,B)*P;
    draw(Y--Z, heavygreen);

    pair X1 = extension(B, C, U, V);
    pair Y1 = extension(C, A, U, V);
    pair Z1 = extension(A, B, U, V);
    draw(Line(Y1, Z1));

    pair A_1 = extension(Y, Y1, Z, Z1);
    pair B_1 = extension(Z, Z1, X, X1);
    pair C_1 = extension(X, X1, Y, Y1);
    draw(A_1--B_1--C_1--cycle, red);
    draw(circumcircle(A_1, B_1, C_1));

    draw(circumcircle(B, Z, X), dotted);
    draw(circumcircle(C, X, Y), dotted);
    draw(circumcircle(A, Y, Z), dotted);

    pair M = extension(A_1, A*A/P, B_1, B*B/P);

    dot("$A$", A, dir(A));
    dot("$B$", B, dir(B));
    dot("$C$", C, dir(C));
    dot("$P$", P, dir(P));
    dot("$X$", X, dir(X));
    dot("$Y$", Y, dir(Y));
    dot("$Z$", Z, dir(Z));
    dot("$A_1$", A_1, dir(A_1));
    dot("$B_1$", B_1, dir(B_1));
    dot("$C_1$", C_1, dir(C_1));
    dot("$M$", M, dir(M));

    pair O_1 = circumcenter(A_1, B_1, C_1);
    clip(CR(O_1, 4));

    /* Source generated by TSQ */
  \end{asy}
\end{figure}

Let $P$ be the contact point of $\ell$ and suppose the
triangle formed has vertices $A_1$, $B_1$, $C_1$ and circumcircle $\Gamma_1$.
Let the reflection of $P$ across $\ol{BC}$ be $X$, and define $Y$, $Z$ similarly.
Then $X$, $Y$, $Z$ is the image of the Simson line from $P$ to $ABC$
with homothety $2$, so these three points are collinear
(and pass through the orthocenter of $ABC$).

We claim that $A$, $Y$, $Z$, $A_1$ are concyclic,
and we denote their circumcircle by $\Gamma_A$.
Through a symmetry around $\ol{AC}$ and then $\ol{BC}$ we get
\[ \dang \left( YA, YA_1 \right)
  = -\dang \left( AP, \ell \right)
  = \dang \left( ZA, ZA_1 \right). \]
Thus, $\dang AYA_1 = \dang AZA_1$, and $A$, $A_1$, $Y$, $Z$ are concyclic.
We can thus define $\Gamma_B$ and $\Gamma_C$ similarly.

Observe that lines $A_1B_1$, $B_1C_1$, $C_1A_1$, $YZ$
determine a complete quadrilateral with Miquel point $M$ lying
on each of the circles $\Gamma_1$, $\Gamma_A$, $\Gamma_B$, $\Gamma_C$.
We will show first that $M$ also lies on $\Gamma$.
Upon noting that $AY = AP = AZ$, we can compute
\[
  \dang (MB, MC)
  = \dang (MB, MX) + \dang (MX, MC)
  = \dang (ZB, ZX) + \dang (YX, YC)
  = -\dang (BX, CX)
  = \dang (BP, CP). \]
Hence, $B$, $P$, $C$, $M$ are cyclic.

Consider a tangent line $\ell^\ast$ tangent to $\Gamma$ at $M$.
Then
\[
  \dang \left( \ell^\ast, MC_1 \right)
  = \dang \left( \ell^\ast, MC \right) + \dang \left( MC, MC_1 \right)
  = \dang \left( BM, BC \right) + \dang \left( MC, MC_1 \right)
  . \]
We know that
\[ \dang \left( MC, MC_1 \right)
  = \dang \left( XC, XC_1 \right)
  = -\dang \left( CP, \ell \right). \]
Thus,
\[ \dang \left( \ell^\ast, MC_1 \right)
  =  \dang (BM, BC) + \dang (\ell, PC)
  = \dang (BM, BX) + \dang (BX, BC) + \dang (\ell, PC)
  = \dang (BM, BX).
  \]
Finally,
\[ \dang \left( \ell^\ast, MC_1 \right) = \dang (BM, BX) = \dang (B_1M, B_1X) = \dang (MB_1, MC_1). \]
Thus $\ell^\ast$ is also tangent to $\Omega_1$ as required.
