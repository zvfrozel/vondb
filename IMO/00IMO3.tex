desc: Fleas and bowling balls
source: IMO 2000/3
tags: [2019-04, equalitycase, find, bestpossible, intuitive, good, algorithm, understand, invariant, process, zayin]
hardness: 35
url: https://aops.com/community/p354112
author: Sergei Shikh and Igor Voronovich (BLR)

---

Let $n \ge 2$ be a positive integer
and $\lambda$ a positive real number.
Initially there are $n$ fleas on a horizontal line,
not all at the same point.
We define a move as choosing two fleas at some points $A$ and $B$,
with $A$ to the left of $B$,
and letting the flea from $A$ jump over the flea from $B$ to the point $C$
so that $\frac{BC}{AB} = \lambda$.

Determine all values of $ \lambda$ such that,
for any point $M$ on the line
and for any initial position of the $n$ fleas,
there exists a sequence of moves that will take
them all to the position right of $M$.

---

The answer is $\lambda \ge \frac{1}{n-1}$.

We change the problem by replacing the fleas
with \textbf{bowling balls} $B_1$, $B_2$, \dots, $B_n$ in that order.
Bowling balls aren't exactly great at jumping,
so each move can now be described as follows:
\begin{itemize}
\ii Select two indices $i < j$.
Then ball $B_i$ moves to $B_{i+1}$'s location,
$B_{i+1}$ moves to $B_{i+2}$'s location, and so on;
until $B_{j-1}$ moves to $B_j$'s location,
\ii Finally, $B_j$ moves some distance forward;
the distance is at most $\lambda \cdot |B_j B_i|$
and $B_j$ may not pass $B_{j+1}$.
\end{itemize}

\begin{claim*}
  If $\lambda < \frac{1}{n-1}$
  the bowling balls have bounded movement.
\end{claim*}
\begin{proof}
  Let $a_i \ge 0$ denote the initial distance
  between $B_i$ and $B_{i+1}$,
  and let $\Delta_i$ denote the distance travelled by ball $i$.
  Of course we have
  $\Delta_1 \le a_1 + \Delta_2$,
  $\Delta_2 \le a_2 + \Delta_3$,
  \dots,
  $\Delta_{n-1} \le a_{n-1} + \Delta_n$
  by the relative ordering of the bowling balls.
  Finally, distance covered by $B_n$ is always
  $\lambda$ times distance travelled by other bowling balls, so
  \begin{align*}
    \Delta_n &\le \lambda \sum_{i=1}^{n-1} \Delta_i
    \le \lambda \sum_{i=1}^{n-1}
    \left( \left( a_i + a_{i+1} + \dots + a_{n-1} \right)
      + \Delta_n \right) \\
    &= (n-1)\lambda \cdot \Delta_n + \sum_{i=1}^{n-1} i a_i
  \end{align*}
  and since $(n-1)\lambda > 1$, this gives an upper bound.
\end{proof}

\begin{remark*}
  Equivalently, you can phrase the proof without
  bowling balls as follows:
  if $x_1 < \dots < x_n$ are the positions of the fleas,
  the quantity
  \[ L = x_n - \lambda(x_1 + \dots + x_{n-1}) \]
  is a monovariant which never increases;
  i.e.\ $L$ is bounded above.
  Since $L > (1-(n-1)\lambda) x_n$, it follows
  $\lambda < \frac{1}{n-1}$ is enough to stop the fleas.
\end{remark*}

\begin{claim*}
When $\lambda \ge \frac{1}{n-1}$,
it suffices to always jump the leftmost flea
over the rightmost flea.
\end{claim*}
\begin{proof}
If we let $x_i$ denote the distance travelled by $B_1$
in the $i$th step,
then $x_i = a_i$ for $1 \le i \le n-1$
and $x_i = \lambda(x_{i-1} + x_{i-2} + \dots + x_{i-(n-1)})$.

In particular, if $\lambda \ge \frac{1}{n-1}$
then each $x_i$ is at least the average of the previous $n-1$ terms.
So if the $a_i$ are not all zero,
then $\{x_{n}, \dots, x_{2n-2}\}$ are all positive
and thereafter $x_i \ge \min \left\{ x_n, \dots, x_{2n-2} \right\} > 0$
for every $i \ge 2n-1$.
So the partial sums of $x_i$ are unbounded, as desired.
\end{proof}

\begin{remark*}
  Other inductive constructions are possible.
  Here is the idea of one of them,
  although the details are more complicated.

  We claim in general that given $n-1$ fleas at $0$
  and one flea at $1$,
  we can get all the fleas arbitrarily close to
  $\frac{1}{1-(n-1)\lambda}$
  (or as far as we want if $\lambda > \frac{1}{n-1}$.).
  The proof is induction by $n \ge 2$;
  for $n=2$ we get a geometric series.
  For $n \ge 3$, we leave one flea at zero
  and move the remainder close to $\frac{1}{1-(n-2)\lambda}$,
  then jump the last flea to
  $\frac{1+\lambda}{1-(n-2)\lambda}$.

  Now we're in the same situation,
  except we shifted $\frac{1}{1-(n-2)\lambda}$ right
  and have then scaled everything by
  $r = \frac{\lambda}{1-(n-2)\lambda}$.
  If we repeat this process again and check the geometric series,
  we see the fleas converge to
  \[ \frac{1}{1-(n-2)\lambda}
    \left( 1 + r + r^2 + r^3 + \dots \right)
    = \frac{1}{1-(n-2)\lambda} \cdot \frac{1}{1-r}
    = \frac{1}{1-(n-1)\lambda}. \]
\end{remark*}
