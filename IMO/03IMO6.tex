desc:  $n^p-p$ not divisible by $q$
source: IMO 2003/6
tags:  [fermat, divis, construct, primes, mods, order, instructive, good, zayin]
hardness: 35
url: https://aops.com/community/p266
author: Johan Yebbou (FRA)

---

Let $p$ be a prime number.
Prove that there exists a prime number $q$ such that for every integer $n$,
the number $n^p-p$ is not divisible by $q$.

---

By orders, we must have $q=pk+1$ for this to be possible
(since if $q \not \equiv 1 \pmod p$, then $n^p$ can be any residue modulo $q$).
Since $p \equiv n^p \pmod q \implies p^k \equiv 1 \pmod q$,
it suffices to prevent the latter situation from happening.

So we need a prime $q \equiv 1 \pmod p$ such that $p^k \not\equiv 1 \pmod q$.
To do this, we first recall the following lemma.
\begin{lemma*}
  Let $\Phi_p(X) = 1 + X + X^2 + \dots + X^{p-1}$.
  For any integer $a$, if $q$ is a prime divisor of $\Phi_p(a)$ other than $p$,
  then $a \pmod q$ has order $p$. (In particular, $q \equiv 1 \pmod p$.)
\end{lemma*}
\begin{proof}
  We have $a^p-1 \equiv 0 \pmod q$, so either the order is $1$ or $p$.
  If it is $1$, then $a \equiv 1 \pmod q$, so $q \mid \Phi_p(1) = p$, hence $q = p$.
\end{proof}

% Wishfully we hope the order of $p$ is $p$ and $k \nmid p$.
Now the idea is to extract a prime factor $q$
from the cyclotomic polynomial
\[ \Phi_p(p) = \frac{p^p-1}{p-1} \equiv 1+p \pmod{p^2} \]
such that $q \not\equiv 1 \pmod{p^2}$;
hence $k \not\equiv 0 \pmod p$,
and as $p \pmod q$ has order $p$ we have $p^k \not\equiv 1 \pmod q$.

---

In this case it's possible to narrow down the search
space right at the beginning.
\begin{walk}
  \ii Show that if $q \not\equiv 1 \pmod p$ then this fails.
  So we will restrict our attention
  to $q = pk + 1$.
  \ii Prove that it's sufficient to have
  $p^k \not\equiv 1 \pmod q$, for the $k$ in (a).
\end{walk}
Okay, so that means for our fixed prime $p$,
we want to find a $q = pk+1$ such that $q \nmid p^k-1$.
(Funny aside: it would be sufficient that $p$
is a primitive root modulo $q$, but this is open.)

Dirichlet's theorem at least assures us there
are infinitely many $q \equiv 1 \pmod p$,
but where can we find them?
(Aside: something is fishy here;
Dirichlet is not an easy theorem to prove,
so it is very surprising that a contest
is asking us to prove some related result.
In fact, even the statement ``there exists \emph{any} prime
which is $1 \pmod p$'' is not trivial.)
\begin{walk}[resume]
  \ii Suppose $X \not\equiv 1 \pmod p$.
  Prove (if you have not seen it already)
  that any prime factor $q$ of
  \[ \Phi(X) = \frac{X^p-1}{X-1} \]
  is always $1 \pmod p$, and in fact $X \pmod q$ has order $p$.
  (This is called the $p$th cyclotomic polynomial.)

  \ii Prove that if $q \not\equiv 1 \pmod{p^2}$,
  then $p \nmid k$.

  \ii Putting together (c) and (d),
  pick a suitable value of $X$
  and use it to find a way to pick the desired $q$.
\end{walk}
