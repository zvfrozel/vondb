author: Marek Pechal (CZE)
desc: Areas of RPK and RQL are equal
source: IMO 2007/4
tags: [2019-04, trig, length, anglechase, trivial]
hardness: 5
url: https://aops.com/community/p894655

---

In triangle $ABC$ the bisector of $\angle BCA$
meets the circumcircle again at $R$,
the perpendicular bisector of $\ol{BC}$ at $P$,
and the perpendicular bisector of $\ol{AC}$ at $Q$.
The midpoint of $\ol{BC}$ is $K$ and the midpoint of $\ol{AC}$ is $L$.
Prove that the triangles $RPK$ and $RQL$ have the same area.

---

We first begin by proving the following claim.
\begin{claim*}
  We have $CQ = PR$ (equivalently, $CP = QR$).
\end{claim*}
\begin{proof}
  Let $O = \ol{LQ} \cap \ol{KP}$ be the circumcenter.
  Then
  \[ \dang OPQ = \dang KPC = 90\dg - \dang PCK
  = 90\dg - \dang LCQ = \dang \dang CQL = \dang PQO. \]
  Thus $OP = OQ$.
  Since $OC = OR$ as well, we get the conclusion.
\end{proof}

Denote by $X$ and $Y$ the feet from $R$ to $\ol{CA}$
and $\ol{CB}$, so $\triangle CXR \cong \triangle CYR$.
Then, let $t = \frac{CQ}{CR} = 1 - \frac{CP}{CR}$.

\begin{center}
\begin{asy}
pair C = dir(90);
pair R = -C;
pair X = dir(205);
pair Y = -conj(X);
pair Q = 0.41*R+0.59*C;
pair P = R+C-Q;
pair L = foot(Q, C, X);
pair K = foot(P, C, Y);
pair A = 2*L-C;
pair B = 2*K-C;
pair O = extension(L, Q, P, K);

filldraw(L--Q--X--cycle, opacity(0.1)+orange, dotted+orange);
filldraw(P--K--Y--cycle, opacity(0.1)+orange, dotted+orange);
draw(C--A--B--cycle, lightblue);
draw(A--X, lightblue);
draw(circumcircle(C, A, B), lightblue+dashed);
draw(C--R, deepgreen);
draw(X--R--Y, deepgreen);
draw(L--O, blue);
draw(P--K, blue);

dot("$C$", C, dir(C));
dot("$R$", R, dir(R));
dot("$X$", X, dir(180));
dot("$Y$", Y, dir(0));
dot("$Q$", Q, dir(45));
dot("$P$", P, dir(215));
dot("$L$", L, dir(L));
dot("$K$", K, dir(K));
dot("$A$", A, dir(A));
dot("$B$", B, dir(B));
dot("$O$", O, dir(315));

/* TSQ Source:

C = dir 90
R = -C
X = dir 205 R180
Y = -conj(X) R0
Q = 0.41*R+0.59*C R45
P = R+C-Q R215
L = foot Q C X
K = foot P C Y
A = 2*L-C
B = 2*K-C
O = extension L Q P K R315

L--Q--X--cycle 0.1 orange / dotted orange
P--K--Y--cycle 0.1 orange / dotted orange
C--A--B--cycle lightblue
A--X lightblue
circumcircle C A B lightblue dashed
C--R deepgreen
X--R--Y deepgreen
L--O blue
P--K blue

*/
\end{asy}
\end{center}

Then it follows that
\[ [RQL] = [XQL] = t(1-t) \cdot [XRC]
= t(1-t) \cdot [YCR] = [YKP] = [RKP] \]
as needed.

\begin{remark*}
  Trigonometric approaches are very possible
  (and easier to find) as well:
  both areas work out to be $\frac 18 ab \tan \half C$.
\end{remark*}
