author: Dominik Burek (POL) and Tomasz Ciesla (POL)
desc: $AD+DT+TX+XA=CDF+DY+YZ+ZC$
source: IMO 2021/4
url: https://aops.com/community/p22698001
tags: [2023-01, length, spiralsim, troll, aleph]
hardness: 15

---

Let $\Gamma$ be a circle with center $I$, and $ABCD$ a convex quadrilateral
such that each of the segments $AB$, $BC$, $CD$ and $DA$ is tangent to $\Gamma$.
Let $\Omega$ be the circumcircle of the triangle $AIC$.
The extension of $BA$ beyond $A$ meets $\Omega$ at $X$,
and the extension of $BC$ beyond $C$ meets $\Omega$ at $Z$.
The extensions of $AD$ and $CD$ beyond $D$ meet $\Omega$ at $Y$ and $T$, respectively.
Prove that
\[ AD + DT + TX + XA = CD + DY + YZ + ZC. \]

---

Let $PQRS$ be the contact points of $\Gamma$ an $\ol{AB}$, $\ol{BC}$,
$\ol{CD}$, $\ol{DA}$.

\begin{center}
\begin{asy}
  /*
    Converted from GeoGebra by User:Azjps using Evan's magic cleaner
    https://github.com/vEnhance/dotfiles/blob/main/py-scripts/export-ggb-clean-asy.py
*/
pair I = (0.,0.);
pair P = (-0.40442,0.91457);
pair Q = (-0.35670,-0.93421);
pair R = (0.99998,0.00468);
pair S = (0.70710,0.70710);
pair A = (0.22244,1.19177);
pair B = (-2.62593,-0.06777);
pair C = (1.00681,-1.45484);
pair D = (0.99806,0.41615);
pair X = (2.38106,2.14630);
pair Z = (1.47042,-1.63185);
pair Y = (2.86072,-1.44650);
pair T = (0.99083,1.96044);
pair E = (0.99283,1.53243);
pair F = (4.01929,-2.60507);

size(12.41979cm);
pen qqwuqq = rgb(0.,0.39215,0.);
pen fuqqzz = rgb(0.95686,0.,0.6);
pen zzttqq = rgb(0.6,0.2,0.);
pen cqcqcq = rgb(0.75294,0.75294,0.75294);
draw(P--Q--R--S--cycle, linewidth(0.6) + zzttqq);

draw(circle(I, 1.), linewidth(0.6) + qqwuqq);
draw(circle((1.92609,0.25714), 1.94318), linewidth(0.6) + fuqqzz);
draw(P--Q, linewidth(0.6) + zzttqq);
draw(Q--R, linewidth(0.6) + zzttqq);
draw(R--S, linewidth(0.6) + zzttqq);
draw(S--P, linewidth(0.6) + zzttqq);
draw(circle((0.50340,-0.72742), 0.88462), linewidth(0.6) + fuqqzz);
draw(circumcircle(I,P,S), dotted + fuqqzz);
draw(A--F, linewidth(0.6) + qqwuqq);
draw(B--X, linewidth(0.6) + qqwuqq);
draw(B--F, linewidth(0.6) + qqwuqq);
draw(C--T, linewidth(0.6) + qqwuqq);

dot("$I$", I, dir(160));
dot("$P$", P, dir((-17.392, 6.881)));
dot("$Q$", Q, dir((-17.176, -15.513)));
dot("$R$", R, dir((2.333, 5.318)));
dot("$S$", S, dir((2.248, 5.460)));
dot("$A$", A, dir((-11.357, 7.426)));
dot("$B$", B, dir((-5.839, 9.792)));
dot("$C$", C, dir((2.205, -22.196)));
dot("$D$", D, dir(45));
dot("$X$", X, dir(80));
dot("$Z$", Z, dir((-3.145, -21.676)));
dot("$Y$", Y, dir(90));
dot("$T$", T, dir((-10.053, 11.473)));
dot("$E$", E, dir((10.253, -19.990)));
dot("$F$", F, dir((2.446, 4.708)));
\end{asy}
\end{center}

\begin{claim*}
  We have $\triangle IQZ \cong \triangle IRT$.
  Similarly, $\triangle IPX \cong \triangle ISY$.
\end{claim*}
\begin{proof}
  By considering $(CQIR)$ and $(CITZ)$,
  there is a spiral similarity similarity
  mapping $\triangle IQZ$ to $\triangle IRT$.
  Since $IQ = IR$, it is in fact a congruence.
\end{proof}

This congruence essentially solves the problem.
First, it implies:
\begin{claim*}
  $TX = YZ$.
\end{claim*}
\begin{proof}
  Because we saw $IX = IY$ and $IT = IZ$.
\end{proof}
Then, we can compute
\begin{align*}
  AD + DT + XA
  &= AD + (RT - RD) + (XP-AP) \\
  &= (AD-RD-AP) + RT + XP = RT + XP
\end{align*}
and
\begin{align*}
  CD + DY + ZC &= CD + (SY-SD) + (ZQ-QC) \\
  &= (CD-SD-QC) + SY + ZQ = SY + ZQ
\end{align*}
but $ZQ = RT$ and $XP = SY$, as needed.
