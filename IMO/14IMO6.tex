desc: $\sqrt{n}$ lines can be colored with no completely blue
source:  IMO 2014/6
tags:  [instructive, local, extreme, greedy, good, global, dumb, gimel]
author: Gerhard Woeginger (AUT)
hardness: 35
url: https://aops.com/community/p3543151

---

A set of lines in the plane is in \emph{general position}
if no two are parallel and no three pass through the same point.
A set of lines in general position cuts the plane into regions,
some of which have finite area; we call these its \emph{finite regions}.
Prove that for all sufficiently large $n$,
in any set of $n$ lines in general position
it is possible to colour at least $\sqrt{n}$ lines blue
in such a way that none of its finite regions
has a completely blue boundary.

---

Suppose we have colored $k$ of the lines blue, and that
it is not possible to color any additional lines.
That means any of the $n-k$ non-blue lines
is the side of some finite region with
an otherwise entirely blue perimeter.
For each such line $\ell$, select one such region,
and take the next counterclockwise vertex;
this is the intersection of two blue lines $v$.
We'll say $\ell$ is the \emph{eyelid} of $v$.

\begin{center}
 \begin{asy}
  size(2cm);
  pair A = dir(  0); dot(A);
  pair B = dir( 72); dot(B);
  pair C = dir(144); dot(C);
  pair D = dir(216); dot(D);
  pair E = dir(288); dot(E);
  draw(D--E--A--B--C, blue);
  draw(C--D, red);
  label("$\ell$", 0.5*C+0.5*D, dir(180));
  label("$v$", E, dir(E));
 \end{asy}
\end{center}

You can prove without too much difficulty that every intersection of two blue lines
has at most two eyelids.
Since there are $\binom k2$ such intersections, we see that
\[ n-k \le 2 \binom k2 = k^2 - k\]
so $n \le k^2$, as required.

\begin{remark*}
In fact, $k = \sqrt n$ is ``sharp for greedy algorithms'',
as illustrated below for $k=3$:
\begin{center}
\begin{asy}
size(2.5cm);
real R = 7;
real r = 3;
real e = 2;
pair A = R * dir(90);
pair B = R * dir(210);
pair C = R * dir(330);
draw( (A-r*dir(B-A)) -- (B-r*dir(A-B)), blue );
draw( (B-r*dir(C-B)) -- (C-r*dir(B-C)), blue );
draw( (C-r*dir(A-C)) -- (A-r*dir(C-A)), blue );
void yanpi(pair P) {
dot(P, blue);
draw( (P+e*dir(P)*dir(70)) -- (P+e*dir(P)*dir(-70)), red);
draw( (P+e*dir(P)*dir(110)) -- (P+e*dir(P)*dir(-110)), red);
}
yanpi(A);
yanpi(B);
yanpi(C);
\end{asy}
\end{center}
\end{remark*}

---

The problem is going to go by the greedy algorithm:
color lines blue until it is not possible to color any more blue (RUST).
Call a line red if it is not blue.
Let $k$ be the number of blue lines.
\begin{walk}
  \ii Show that each of the $n-k$ red lines
  borders an ``almost-blue'' polygon.
  \ii For each red line $\ell$, select one such almost-blue polygon,
  and \emph{attach} $\ell$ to the next counterclockwise vertex
  (which the intersection of two blue lines).
  Show that each blue vertex has at most two red lines attached to it.
  \ii Using (b), write down an inequality in $n$, $k$
  relating the number of red lines to the number of
  intersections of two blue lines.
  \ii Solve the relation in (b) to conclude $k \ge \sqrt n$.
\end{walk}
The fact that this was an IMO6 mostly goes to show that (at least at the time)
IMO contestants were not familiar with the idea of a greedy algorithm.
