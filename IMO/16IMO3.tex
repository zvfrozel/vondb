desc:  Geometry of numbers
source:  IMO 2016/3
tags:  [nice, criticalclaim, meta, vp, inversion, induct, stronger, yod]
author: Aleksandr Gaifullin (RUS)
hardness: 40
url: https://aops.com/community/p6637660

---

Let $P=A_1A_2\dots A_k$ be a convex polygon in the plane.
The vertices $A_1$, $A_2$, \dots, $A_k$ have integral coordinates
and lie on a circle. Let $S$ be the area of $P$.
An odd positive integer $n$ is given such that
the squares of the side lengths of $P$ are integers divisible by $n$.
Prove that $2S$ is an integer divisible by $n$.

---

Solution by Jeck Lim:
We will prove the result just for $n = p^e$
where $p$ is an odd prime and $e \ge 1$.
The case $k=3$ is resolved by Heron's formula directly:
we have $S = \frac14\sqrt{2(a^2b^2 + b^2c^2 + c^2a^2) - a^4-b^4-c^4}$,
so if $p^e \mid \gcd(a^2,b^2,c^2)$ then $p^{2e} \mid S^2$.

Now we show we can pick a diagonal and induct down on $k$ by using inversion.

Let the polygon be $A_1 A_2 \dots A_{k+1}$
and suppose for contradiction that all sides are divisible by $p^e$
but no diagonals are.
Let $O = A_{k+1}$ for notational convenience.
By applying inversion around $O$ with radius $1$,
we get the ``generalized Ptolemy theorem''
\[
  \frac{A_1A_2}{OA_1 \cdot OA_2}
  + \frac{A_2A_3}{OA_2 \cdot OA_3}
  + \dots
  + \frac{A_{k-1} A_k}{OA_{k-1} \cdot OA_k}
  = \frac{A_1 A_k}{OA_1 \cdot OA_k}
\]
or, making use of square roots,
\[
  \sqrt{\frac{A_1A_2^2}{OA_1^2 \cdot OA_2^2}}
  + \sqrt{\frac{A_2A_3^2}{OA_2^2 \cdot OA_3^2}}
  + \dots
  + \sqrt{\frac{A_{k-1} A_k^2}{OA_{k-1}^2 \cdot OA_k^2}}
  = \sqrt{\frac{A_1 A_k^2}{OA_1^2 \cdot OA_k^2}}
\]
Suppose $\nu_p$ of all diagonals is strictly less than $e$.
Then the relation becomes
\[ \sqrt{q_1} + \dots + \sqrt{q_{k-1}} = \sqrt q \]
where $q_i$ are positive rational numbers.
Since there are no nontrivial relations between square roots
(see \href{https://qchu.wordpress.com/2009/07/02/square-roots-have-no-unexpected-linear-relationships/}{this link})
there is a positive rational number $b$
such that $r_i = \sqrt{q_i/b}$ and $r = \sqrt{q/b}$
are all rational numbers.
Then
\[ \sum r_i = r. \]
However, the condition implies that $\nu_p(q_i^2) > \nu_p(q^2)$ for all $i$
(check this for $i=1$, $i=k-1$ and $2 \le i \le k-2$),
and hence $\nu_p(r_i) > \nu_p(r)$.
This is absurd.

\begin{remark*}
  I think you basically have to use some Ptolemy-like geometric property,
  and also all correct solutions I know of for $n = p^e$
  depend on finding a diagonal and inducting down.
  (Actually, the case $k=4$ is pretty motivating;
  Ptolemy implies one can cut in two.)
\end{remark*}

---

Warm-ups:
\begin{walk}
  \ii Show that it's sufficient to prove
  the result for odd prime powers.
  \ii Prove the result for $k = 3$ by using Heron's formula.
  \ii Optionally: prove the result for $k = 4$ by using Brahmagupta's formula.
  \ii Optionally: prove the result for $k = 4$ by using Ptolemy's theorem
  to show that one of the diagonals has square length divisible by $p^e$.
  Thus we can dissect the polygon and apply (b).
\end{walk}
Unfortunately, the formulas for $k \ge 5$ are not tractable
to my knowledge, so the approach in (c)
does not work for higher values.
However, the approach in (d) \emph{does} extend. Why?
Well, Ptolemy's theorem is what happens when you invert
a cyclic quadrilateral $ABCD$ around $D$ with radius $1$,
then assert $A^\ast B^\ast + B^\ast C^\ast = A^\ast C^\ast$.
\begin{walk}[resume]
  \ii Prove Ptolemy's theorem in the way described.
  \ii Now let $P = A_1 \dots A_{k+1}$ and fix $O = A_{k+1}$.
  Perform an inversion at $O$ with radius $1$ and deduce that
  \[
    \frac{A_1A_2}{OA_1 \cdot OA_2}
    + \frac{A_2A_3}{OA_2 \cdot OA_3}
    + \dots
    + \frac{A_{k-1} A_k}{OA_{k-1} \cdot OA_k}
    = \frac{A_1 A_k}{OA_1 \cdot OA_k}.
  \]
\end{walk}
The numbers here are square roots of integers,
say \[ \sqrt{q_1} + \dots + \sqrt{q_{k-1}} = \sqrt q. \]
You can take for granted there exists a rational number $b$
such that $\sqrt{q_i/b} = r_i$ is rational for every $i$
(and $\sqrt{q/b} = r$ is rational too).
This follows from the fact that ``there are no nontrivial dependencies
between $\sqrt{-}$'', see
\url{https://qchu.wordpress.com/2009/07/02/square-roots-have-no-unexpected-linear-relationships/}
for example.
Then $\sum r_i = r$ and
we are free to do $v_p$ analysis in the usual way.

Now assume for contradiction that no diagonals are divisible by $p^e$
but all sides are.
\begin{walk}[resume]
  \ii Prove that $\nu_p(r_i) > \nu_p(r)$ for all $i$.
  (Be a little careful here, the edge cases $i=1,k-1$
  need to be treated separately.)
  \ii Conclude.
\end{walk}

---

Right, so let me write up what I have for the case where $n = p$ a prime.
I don't know how to extend it to prime powers,
and I suspect that it's not easy.

First, the circumcenter $O$ has rational coordinates,
so by scaling and translation assume its the origin.
Let $A_i = (x_i, y_i)$ and $R$ the circumradius.
From the fact that $n$ divides $(x_i-y_i)^2 + (x_{i+1}-y_{i+1})^2$ we derive
\[ R^2 \equiv x_i^2 + y_i^2
  \equiv x_i x_{i+1} + y_i y_{i+1} \pmod p \qquad \forall i. \]

For convenience, call a pair $(x,y)$ nonzero if $x \neq 0$ or $y \neq 0$.

Now, we consider two cases.
First $R^2 \not\equiv 0 \pmod p$.
Suppose $(a,b)$ and $(x,y)$ are adjacent vertices.
We claim they're equal mod $p$.
Check that \[ a^2R^2 \equiv (R^2-by)^2 + a^2y^2 \implies R^2(y-b)^2 \equiv 0; \]
hence $y \equiv b$ and similarly $x \equiv a$.
Thus all vertices coincide modulo $p$ in this case and we're done.

If $R^2 \equiv 0 \pmod p$,
we can assume $p \equiv 1 \pmod 4$ since otherwise all vertices are zero.
So assert $p \equiv 1 \pmod 4$ and fix $c = \sqrt{-1} \pmod p$.
Then every nonzero vertex has $x = \pm cy$,
so call it positive or negative for signs $+$ and $-$ respectively.
It's easy to see adjacent nonzero vertices are either both positive or both negative.
Thus the original polygon can be triangulated into triangles
such that no triangle has both positive and negative vertices;
in each case the area of the triangle vanishes modulo $p$.
