desc: Radical axis with some angle chase
source: IMO 2000/1
tags: [2018-02, good, anglechase, pop, adhoc, instructive, aleph]
hardness: 10
url: https://aops.com/community/p354110
author: Sergei Berlov (RUS)

---

Two circles $G_1$ and $G_2$ intersect at two points $M$ and $N$.
Let $AB$ be the line tangent to these circles at $A$ and $B$,
respectively, so that $M$ lies closer to $AB$ than $N$.
Let $CD$ be the line parallel to $AB$
and passing through the point $M$,
with $C$ on $G_1$ and $D$ on $G_2$.
Lines $AC$ and $BD$ meet at $E$; lines $AN$ and $CD$ meet at $P$;
lines $BN$ and $CD$ meet at $Q$.
Show that $EP = EQ$.

---

First, we have $\dang EAB = \dang ACM = \dang BAM$
and similarly $\dang EBA = \dang BDM = \dang ABM$.
Consequently, $\ol{AB}$ bisects $\angle EAM$ and $\angle EBM$,
and hence $\triangle EAB \cong \triangle MAB$.

\begin{center}
\begin{asy}
pair M = dir(120);
pair U = dir(170);
pair V = dir(10);
pair N = 0.3*V+0.7*U;

pair O_1 = circumcenter(M, U, N);
pair O_2 = circumcenter(M, V, N);

real r1 = abs(U-O_1);
real r2 = abs(V-O_2);
pair Se = (r1*O_2-r2*O_1)/(r1-r2);
path w1 = CP(O_1,N);
path w2 = CP(O_2,N);
filldraw(w1, opacity(0.1)+lightcyan, blue);
filldraw(w2, opacity(0.1)+lightcyan, blue);
pair A = IP(w1, CP(midpoint(Se--O_1), Se));
pair B = IP(w2, CP(midpoint(Se--O_2), Se));

draw(A--B, red);
pair C = -M+2*foot(O_1, M, M+B-A);
pair D = -M+2*foot(O_2, M, M+B-A);
draw(C--D, red);

pair E = extension(A, C, B, D);
pair P = extension(A, N, C, D);
pair Q = extension(B, N, C, D);
draw(C--E--D, blue);
draw(A--N--B, blue);
draw(P--E--Q, heavygreen);

pair T = extension(M, N, A, B);
draw(T--N, heavycyan);
draw(A--M--B, purple);
draw(E--M, red);

dot("$M$", M, dir(350));
dot("$N$", N, dir(260));
dot("$A$", A, dir(A));
dot("$B$", B, dir(B));
dot("$C$", C, dir(C));
dot("$D$", D, dir(D));
dot("$E$", E, dir(E));
dot("$P$", P, dir(260));
dot("$Q$", Q, dir(330));
dot("$T$", T, dir(160));

/* TSQ Source:

M = dir 120 R350
U := dir 170
V := dir 10
N = 0.3*V+0.7*U R260

O_1 := circumcenter M U N R-90
O_2 := circumcenter M V N R-90

!real r1 = abs(U-O_1);
!real r2 = abs(V-O_2);
Se := (r1*O_2-r2*O_1)/(r1-r2)
!path w1 = CP(O_1,N);
!path w2 = CP(O_2,N);
w1 0.1 lightcyan / blue
w2 0.1 lightcyan / blue
A = IP w1 CP midpoint Se--O_1 Se
B = IP w2 CP midpoint Se--O_2 Se

A--B red
C = -M+2*foot O_1 M M+B-A
D = -M+2*foot O_2 M M+B-A
C--D red

E = extension A C B D
P = extension A N C D R260
Q = extension B N C D R330
C--E--D blue
A--N--B blue
P--E--Q heavygreen

T = extension M N A B R160
T--N heavycyan
A--M--B purple
E--M red

*/
\end{asy}
\end{center}

Now it is well-known that $\ol{MN}$ bisects $\ol{AB}$
and since $\ol{AB} \parallel \ol{PQ}$
we deduce that $M$ is the midpoint of $\ol{PQ}$.
As $\ol{AB}$ is the perpendicular bisector of $\ol{EM}$,
it follows that $EP = EQ$ as well.
