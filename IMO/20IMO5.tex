desc: AM/GM of cards
hardness: 20
source: IMO 2020/5
tags: [2020-09, primes, nice, divis, size, aleph]
author: Oleg Ko\v{s}ik (EST)
url: https://aops.com/community/p17821528

---

A deck of $n > 1$ cards is given.
A positive integer is written on each card.
The deck has the property that the arithmetic mean of the
numbers on each pair of cards is also the
geometric mean of the numbers on some collection of one or more cards.
For which $n$ does it follow that the numbers on the cards are all equal?

---

The assertion is true for all $n$.

\bigskip

\textbf{Setup (boilerplate).}
Suppose that $a_1$, \dots, $a_n$ satisfy the required properties
but are not all equal.
Let $d = \gcd(a_1, \dots, a_n) > 1$
then replace $a_1$, \dots, $a_n$ by
$\frac{a_1}{d}$, \dots, $\frac{a_n}{d}$.
Hence without loss of generality we may assume
\[ \gcd(a_1, a_2, \dots, a_n) = 1. \]
WLOG we also assume \[ a_1 \ge a_2 \ge \dots \ge a_n. \]

\bigskip

\textbf{Main proof.}
As $a_1 \ge 2$, let $p$ be a prime divisor of $a_1$.
Let $k$ be smallest index such that $p \nmid a_k$ (which must exist).
In particular, note that $a_1 \neq a_k$.

Consider the mean $x = \frac{a_1+a_k}{2}$; by assumption,
it equals some geometric mean, hence
\[ \sqrt[m]{a_{i_1} \dots a_{i_m}} = \frac{a_1 + a_k}{2} > a_k. \]
Since the arithmetic mean is an integer not divisible by $p$,
all the indices $i_1$, $i_2$, \dots, $i_m$
must be at least $k$.
But then the GM is at most $a_k$, contradiction.

\begin{remark*}
  A similar approach could be attempted by using
  the smallest numbers rather than the largest ones,
  but one must then handle the edge case $a_n = 1$
  separately since no prime divides $1$.
\end{remark*}

\begin{remark*}
  Since $\frac{27+9}{2} = 18 = \sqrt[3]{27 \cdot 27 \cdot 8}$,
  it is not true that in general the AM of two largest different cards
  is not the GM of other numbers in the sequence
  (say the cards are $27, 27, 9, 8, \dots$).
\end{remark*}
