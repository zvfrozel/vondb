desc: Chaotic evil pebbles
hardness: 40
source: IMO 2020/3
tags: [2020-09, additivecombo, graph, stronger, troll, good, zayin]
author: Milan Haiman (HUN), Carl Schildkraut (USA)
url: https://aops.com/community/p17821656

---

There are $4n$ pebbles of weights $1, 2, 3, \dots, 4n$.
Each pebble is coloured in one of $n$ colours
and there are four pebbles of each colour.
Show that we can arrange the pebbles into two piles
the total weights of both piles are the same,
and each pile contains two pebbles of each colour.

---

The first key idea is the deep fact that
\[ 1+4n = 2+(4n-1) = 3+(4n-2) = \dots. \]
So, place all four pebbles of the same colour in a box (hence $n$ boxes).
For each  $k=1,2,\dots,2n$
we tape a piece of string between pebble $k$ and $4n+1-k$.
To solve the problem, it suffices to paint each string
either blue or green such that each box has two blue strings
and two green strings
(where a string between two pebbles in the same box counts double).
\begin{center}
\begin{asy}
size(12cm);
// picture box;

pair A = 0.5*dir(135);
pair C = 0.5*dir(225);
pair D = 0.5*dir(315);
pair B = 0.5*dir(45);

path boxoutline = (dir(135)+dir(200)*0.4)--dir(135)--dir(225)
    --dir(315)--dir(45)--(dir(45)+dir(-20)*0.4);

transform[] t = new transform[5];
for (int i=0; i<5; ++i) {
  t[i] = shift(3.4*dir(90+72*i));
  draw(t[i]*boxoutline, brown+1.5);
  label("Box " + array("ABCDE")[i], t[i]*dir(-90), brown);
}

void rope(pair X, pair Y, string s1, string s2, pen p) {
  draw(X--Y, p);
  dot(s1, X, dir(-90), p + fontsize(9pt));
  dot(s2, Y, dir(-90), p + fontsize(9pt));
}
rope(t[0]*A, t[0]*B, "1", "20", blue);
rope(t[0]*C, t[1]*A, "4", "17", deepgreen);
rope(t[0]*D, t[3]*D, "3", "18", deepgreen);
rope(t[1]*B, t[4]*A, "8", "13", blue);
rope(t[1]*D, t[4]*C, "7", "14", deepgreen);
rope(t[4]*B, t[2]*A, "15", "6", deepgreen);
rope(t[1]*C, t[2]*B, "11", "10", blue);
rope(t[4]*D, t[3]*B, "9", "12", blue);
rope(t[2]*C, t[3]*A, "19", "2", blue);
rope(t[2]*D, t[3]*C, "5", "16", deepgreen);

/*
for (int i=0; i<=3; ++i) {
  dot(box, 0.5*dir(45+90*i));
}
*/
\end{asy}
\end{center}

We can therefore rephrase the problem as follows,
if we view boxes as vertices and strings as edges:
\begin{claim*}
  Given a $4$-regular multigraph on $n$ vertices
  (where self-loops are allowed and have degree $2$),
  one can color the edges blue and green
  such that each vertex has two blue and two green edges.
\end{claim*}
\begin{proof}
  Each connected component of the graph can be decomposed
  into an Eulerian circuit, since $4$ is even.
  A connected component with $k$ vertices has $2k$
  edges in its Eulerian circuit,
  so we may color the edges in this circuit alternating green and blue.
  This may be checked to work.
\end{proof}
