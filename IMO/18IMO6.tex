desc: Angle conditions implies isogonal conjugate
author: Tomasz Ciesla (POL)
source: IMO 2018/6
tags: [2018-07, inversion, config, weird, yod]
hardness: 45
url: https://aops.com/community/p10632360

---

A convex quadrilateral $ABCD$ satisfies $AB \cdot CD = BC \cdot DA$.
Point $X$ lies inside $ABCD$ so that
\[ \angle XAB=\angle XCD \quad \text{ and } \quad \angle XBC=\angle XDA. \]
Prove that $\angle BXA + \angle DXC=180\dg$.

---

We present two solutions by inversion.
The first is the official one.
The second is a solution via inversion,
completed by USA5 Michael Ren.

\paragraph{Official solution by inversion.}
In what follows a convex quadrilateral is called
\emph{quasi-harmonic} if $AB \cdot CD = BC \cdot DA$.

\begin{claim*}
  A quasi-harmonic quadrilateral is determined
  up to similarity by its angles.
\end{claim*}
(This could be expected by degrees of freedom;
a quadrilateral has four degrees of freedom up to similarity;
the pseudo-harmonic condition is one
while the angles provide three conditions.)
\begin{proof}
  Do some inequalities.
\end{proof}

Performing an inversion at $X$, one obtains a
second quasi-harmonic quadrilateral
$A^\ast B^\ast C^\ast D^\ast$ which has the same angles
as the original one, $\angle D^\ast = \angle A$,
$\angle A^\ast = \angle B$, and so on.
Thus by the claim we obtain similarity
\[ D^\ast A^\ast B^\ast C^\ast \sim ABCD. \]
If one then maps $D^\ast A^\ast B^\ast C^\ast$,
onto $ABCD$, the image of $X^\ast$
becomes a point isogonally conjugate to $X$.
In other words, $X$ has an isogonal conjugate in $ABCD$.

It is well-known that this is equivalent to
$\angle BXA + \angle DXC = 180\dg$,
for example by inscribing an ellipse with foci $X$ and $X^\ast$.

\paragraph{Second solution: ``rhombus inversion'', by Michael Ren.}
Since
\[ \frac{AB}{AD} = \frac{CB}{CD} \]
and
\[ \frac{BA}{BC} = \frac{DA}{DC} \]
it follows that $B$ and $D$ lie on an Apollonian circle $\omega_{AC}$
through $A$ and $C$,
while $A$ and $C$ lie on an Apollonian circle $\omega_{BD}$
through $B$ and $D$.
We let these two circles intersect at a point $P$ inside $ABCD$.

The main idea is then to
perform an inversion about $P$ with radius $1$.
We obtain:
%\paragraph{Rhombus construction.}
\begin{lemma*}
  The image of $ABCD$ is a rhombus.
\end{lemma*}
\begin{proof}
  By the inversion distance formula, we have
  \[ \frac{1}{A'B'} = \frac{PA}{AB} \cdot PB = \frac{PC}{BC} \cdot PB = \frac{1}{B'C'} \]
  and so $A'B' = B'C'$.
  In a similar way, we derive $B'C' = C'D' = D'A'$,
  so the image is a rhombus as claimed.
\end{proof}

Let us now translate the angle conditions.
We were given that $\dang XAB = \dang XCD$, but
\begin{align*}
  \dang XAB &= \dang XAP + \dang PAB = \dang PX'A' + \dang A'B'P \\
  \dang XCD &= \dang XCP + \dang PCD = \dang PX'C' + \dang C'D'P
  \intertext{so subtracting these gives}
  \dang A'X'C' &= \dang A'B'P + \dang PD'C' = \dang (A'B', B'P) + \dang (PD', C'D') \\
  &= \dang (A'B', B'P) + \dang (PD', A'B') = \dang D' P B'. \tag{1}
\end{align*}
since $\ol{A'B'} \parallel \ol{C'D'}$.
Similarly, we obtain
\[ \dang B'X'D' = \dang A'PC' \tag{2}. \]
We now translate the desired condition.
Since
\begin{align*}
  \dang AXB &= \dang AXP + \dang PXB = \dang PA'X' + \dang X'B'P \\
  \dang CXD &= \dang CXP + \dang PXD = \dang PC'X' + \dang X'DP'
\end{align*}
we compute
\begin{align*}
  \dang AXB + \dang CXD &= (\dang PA'X' + \dang X'B'P) + (\dang PC'X' + \dang X'D'P) \\
  &= -\left[ \left( \dang A'X'P + \dang X'PA' \right)
    + \left( \dang PX'B' + \dang B'PX' \right) \right] \\
  &\quad- \left[ \left( \dang C'X'P + \dang X'PC' \right)
    + \left( \dang PX'D' + \dang D'PX'  \right) \right] \\
  &= \left[ \dang PX'A' + \dang BX'P + \dang PX'C' + \dang D'X'P  \right] \\
  &\quad+ \left[ \dang A'PX' + \dang X'PB' + \dang C'PX' + \dang X'PD' \right] \\
  &= \dang A'PB' + \dang C'PD' + \dang B'X'C + \dang D'X'A
\end{align*}
and we wish to show this is equal to zero, i.e.\
the desired becomes
\[ \dang A'PB' + \dang C'PD' + \dang B'X'C + \dang D'X'A = 0. \tag{3} \]
In other words, the problem is to show (1) and (2) implies (3).

Henceforth drop apostrophes.
Here is the inverted diagram (with apostrophes dropped).
\begin{center}
\begin{asy}
size(12cm);
pair A = (0,3);
pair B = (-8,0);
pair C = (0,-3);
pair D = (8,0);

pair X = (-5.6,7.8);
pair Y = reflect(circumcenter(A,C,X), circumcenter(D,B,X))*X;
pair Q = IP(circumcircle(A, Y, D), circumcircle(B, Y, C));
pair P = conj(Q);

filldraw(A--B--C--D--cycle, opacity(0.1)+lightred, red);
draw(A--C, red);
draw(B--D, red);

draw(B--X--D, lightgreen);
draw(A--P--C, lightgreen);
draw(C--X--A, lightblue);
draw(B--P--D, lightblue);

draw(circumcircle(A, X, C), heavycyan);
draw(circumcircle(B, X, D), heavycyan);
draw(circumcircle(B, Q, C), lightcyan);
draw(circumcircle(A, Q, D), lightcyan);
draw(Q--P, red);

dot("$A$", A, dir(A));
dot("$B$", B, dir(B));
dot("$C$", C, dir(C));
dot("$D$", D, dir(D));
dot("$X$", X, dir(X));
dot("$Y$", Y, dir(Y));
dot("$Q$", Q, dir(Q));
dot("$P$", P, dir(P));

/* TSQ Source:

A = (0,3)
B = (-8,0)
C = (0,-3)
D = (8,0)

X = (-5.6,7.8)
Y = OP circumcircle A X C circumcircle B X D
Q = IP circumcircle A Y D circumcircle B Y C
P = conj(Q)

A--B--C--D--cycle 0.1 lightred / red
A--C red
B--D red

B--X--D lightgreen
A--P--C lightgreen
C--X--A lightblue
B--P--D lightblue

circumcircle A X C heavycyan
circumcircle B X D heavycyan
circumcircle B Q C lightcyan
circumcircle A Q D lightcyan
Q--P red

*/
\end{asy}
\end{center}

Let $Q$ denote the reflection of $P$
and let $Y$ denote the second intersection of $(BQC)$ and $(AQD)$.
Then
\begin{align*}
  -\dang AXC &= -\dang DPB = \dang BQD = \dang BQY + \dang YQD = \dang BCY + \dang YAD \\
  &= \dang(BC,CY) + \dang(YA,AD) = \dang YCA = -\dang AYC.
\end{align*}
% again using $\ol{AB} \parallel \ol{CD}$.
Hence $XACY$ is concyclic; similarly $XBDY$ is concyclic.

\begin{claim*}
  $X \neq Y$.
\end{claim*}
\begin{proof}
  To see this: Work pre-inversion assuming $AB < AC$.
  Then $Q$ was the center of $\omega_{BD}$.
  If $T$ was the second intersection of $BA$ with $(QBC)$,
  then $QB = QD = QT = \sqrt{QA \cdot QC}$, by shooting lemma.
  Since $\angle BAD < 180\dg$,
  it follows $(QBCY)$ encloses $ABCD$ (pre-inversion).
  (This part is where the hypothesis that
  $ABCD$ is convex with $X$ inside is used.)
\end{proof}

Finally, we do an angle chase to finish:
\begin{align*}
  \dang DXA &= \dang DXY + \dang YXA = \dang DBY + \dang YCA \\
  &= \dang (DB, YB) + \dang (CY, CA) = \dang CYB + 90\dg \\
  &= \dang CQB + 90\dg = -\dang APB + 90\dg \tag{4}.
\end{align*}
Similarly,
\[ \dang BXC = \dang DPC + 90\dg. \tag{5} \]
%% \dang AXB = \dang DYC + 90\dg, \quad
%% \dang CXD = \dang BYC + 90\dg.
Summing (4) and (5) gives (3).
% $\dang BXC + \dang DXA = (\dang BPA + 90\dg) + (\dang DPC + 90\dg) = -(\dang APB + \dang CPD)$.


\begin{remark*}
  A difficult part of the problem in many solutions
  is that the conclusion is false in the directed sense,
  if the point $X$ is allowed to lie outside the quadrilateral.
  We are saved in the first solution because the equivalence
  of the isogonal conjugation requires $X$ inside the quadrilateral.
  On the other hand, in the second solution,
  the issue appears in the presence of the second point $Y$.
\end{remark*}

---

\paragraph{Third solution by moving points (Anant Mudgal, un-edited).}
Let $P=\ol{AD} \cap \ol{BC}$ and $Q=\ol{AB} \cap \ol{CD}$. Let $x=\tfrac{1}{2}(\angle A-\angle C)$ and $y=\tfrac{1}{2}(\angle B-\angle D)$. WLOG $x,y \ge 0$. Now construct point $X$ inside $ABCD$ such that $\angle BXA=90^{\circ}-y$ and $\angle BXC=90^{\circ}-x$. Then $\angle AXC=180^{\circ}-\tfrac{1}{2}(\angle A+\angle B-\angle C-\angle D)=\angle C+\angle D=180^{\circ}-\angle APC$ hence $X \in \odot(APC)$.


\begin{claim*}
  $QBXD$ is cyclic.
\end{claim*}
\begin{proof}
Invert at $B$ with radius $\sqrt{BA \cdot BC}$ followed by reflection in internal bisector of $\angle ABC$. We omit the $'$s for readability. Thus, in $\triangle ABC$ with $M$ midpoint of side $\ol{AC}$ we have \begin{itemize}
    \item point $D$ satisfies $\ol{DM} \perp \ol{AC}$;
    \item point $Q \neq B$ lies on line $\ol{AB}$ and $\odot(BDC)$;
    \item point $X$ satisfies $\angle BAX=90^{\circ}+\theta$ and $\angle BCX=90^{\circ}-\angle B+\alpha$; where $\theta=\angle (\ol{BD}, \ol{DM})$ and $\alpha=\angle ADM$.
\end{itemize}

We need to prove $X$ lies on line $\ol{DQ}$. Define $T$ as the point with $\ol{TC} \perp \ol{BC}$ and $TA=TC$. Let $m$ be the line with $T \in m$ and $m \perp \ol{BA}$. Now define $X_1 \in m$ with $\angle BAX_1=90^{\circ}+\theta$ and $X_2 \in m$ with $\angle BCX_2=90^{\circ}-\angle B+\alpha$. We show that  $X_1 \equiv X_2$.

\vspace{0.25cm}


Move point $D$ on the perpendicular bisector $\ell$ of $\ol{AC}$. Then pencils $\ol{AX_1}$ and $\ol{BD}$ move with equal angular velocity (not constant though). Hence $D \mapsto X_1$ is a projective map. Likewise since pencils $\ol{AD}$ and $\ol{CX_2}$ move with equal angular velocity (not constant though), $D \mapsto X_2$ is also a projective map. Thus there is a projective mapping $\pi : X_1 \mapsto X_2$. In order to show $\pi$ is the identity; we only need to show it has three fixed points.

\begin{itemize}
    \item For $D=\ell_{\infty}$ note that $\alpha=-180^{\circ}$ and $\theta=0^{\circ}$ so $\ol{AX_1} \perp \ol{AB}$ and $\ol{CX_2} \perp \ol{AB}$ hence $X_1=X_2=m_{\infty}$.
    \item For $D=m \cap \ol{BA}$ note that $\alpha=90^{\circ}-\angle A$ so $\ol{CX_2} \equiv \ol{AC}$ and $\theta=90^{\circ}+\angle A$ so $\ol{AX_1} \equiv \ol{AC}$.
    \item For $D=O$ (circumcenter of $\triangle ABC$) note that $\alpha=\angle B$ so $X_2=T$. Since $TA=TC$ we obtain $\angle TAB=90^{\circ}+\angle A-\angle C$ but $\theta=\angle A-\angle C$ so $X_1=T$ as well.
\end{itemize}

Now $X=\ol{AX_1} \cap \ol{CX_2}$ hence $X \overset{\text{def}}{:=} X_1 \equiv X_2$. So $\angle CDT=\alpha$ and $\angle CXT=\angle B-(\angle B-\alpha)$ hence $X \in \odot(TCB)$. Finally, observe that $\angle XDC=\angle (\ol{CT}, m)=\angle B$ hence $\angle CDX+\angle CDQ=\angle CBA+\angle CBQ=180^{\circ}$ so the claim is proved. Apply directed angles to conclude the same for other configurations.
\end{proof}


Then \[ X \in \odot(QBD) \implies \angle BXA-\angle CXD=\angle BXD-\angle AXC=\angle C+\angle D-\angle C-\angle B=\angle D-\angle B=-2x \]
and $\angle BXA=(90^{\circ}-x)$ so $\angle CXD=(90^{\circ}+x)$;
hence $\angle BXA+\angle DXC=180^{\circ}$.
Now $PAXC$ and $QBXD$ are cyclic so $\angle{XAD} = \angle{XCB}$ and $\angle{XDC} = \angle{XBA}$; so we're done!
