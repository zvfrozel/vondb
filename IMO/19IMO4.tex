desc: $k! = (2^n-1)(2^n-2)\dots (2^n-2^{n-1})$
source: IMO 2019/4
tags: [2019-07, find, size, vp, cases, neatness, aleph]
author: Gabriel Chicas Reyes (SLV)
hardness: 10
url: https://aops.com/community/p12752761

---

Solve over positive integers the equation
\[ k! = \prod_{i=0}^{n-1} (2^n-2^i)
  = (2^n-1)(2^n-2)(2^n-4) \dots (2^n-2^{n-1}). \]

---

The answer is $(n,k) =(1,1)$ and $(n,k) = (2,3)$ which work.

Let $A = \prod_i (2^n-2^k)$, and assume $A = k!$ for some $k \ge 3$.
Recall by exponent lifting that
\[ \nu_3(2^t-1) = \begin{cases}
    0 & t \text{ odd} \\
    1 + \nu_3(t) & t \text{ even}.
  \end{cases} \]
Consequently, we can compute
\begin{align*}
  k > \nu_2(k!) &= \nu_2(A) = 1 + 2 + \dots + (n-1) = \frac{n(n-1)}{2} \\
  \left\lfloor \frac k3 \right\rfloor
    \le \nu_3(k!) &= \nu_3(A) = \left\lfloor \frac n2 \right\rfloor
    + \left\lfloor \frac n6 \right\rfloor + \dots < \frac 34n.
\end{align*}
where the first inequality can be justified
by Legendre's formula $\nu_2(k!) = k - s_2(k)$.

In this way, we get
\[ \frac 94 n + 3 > k > \frac{n(n-1)}{2} \]
which means $n \le 6$; a manual check then shows the
solutions we claimed earlier are the only ones.

\begin{remark*}
  An amusing corollary of the problem pointed out in the shortlist
  is that the symmetric group $S_k$ cannot be isomorphic to the group $\GL_n(\FF_2)$
  unless $(n,k) = (1,1)$ or $(n,k) = (2,3)$, which indeed produce isomorphisms.
\end{remark*}
