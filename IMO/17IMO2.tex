desc: Reasonably okay FE, cancel trick, triple involution
author: Dorlir Ahmeti (ALB)
source: IMO 2017/2
tags: [FE, 2017-07, cancel, instructive, good, neatness, zayin]
hardness: 40
url: https://aops.com/community/p8633190

---

Solve over $\RR$ the functional equation
\[ f\left( f(x)f(y) \right) + f(x+y) = f(xy). \]

---

The only solutions are $f(x) = 0$, $f(x) = x-1$ and $f(x)=1-x$,
which clearly work.

Note that
\begin{itemize}
  \ii If $f$ is a solution, so is $-f$.
  \ii Moreover, if $f(0)=0$ then setting $y=0$ gives $f\equiv0$.
  So henceforth we assume $f(0)>0$.
\end{itemize}

\begin{claim*}
  We have $f(z) = 0 \iff z =1$.
  Also, $f(0)=1$ and $f(1)=0$.
\end{claim*}
\begin{proof}
  For the forwards direction, if $f(z)=0$ and $z \neq 1$
  one may put $(x,y) = \left( z, z(z-1)\inv \right)$
  (so that $x+y=xy$) we deduce $f(0) = 0$
  which is a contradiction.

  For the reverse, $f(f(0)^2)=0$ by setting $x=y=0$,
  and use the previous part.
  We also conclude $f(1) = 0$, $f(0) = 1$.
\end{proof}

\begin{claim*}
  If $f$ is injective, we are done.
\end{claim*}
\begin{proof}
  Setting $y=0$ in the original equation
  gives $f(f(x)) = 1-f(x)$.
  We apply this three times on the expression $f^3(x)$:
  \[ f(1-f(x)) = f(f(f(x))) = 1 - f(f(x)) = f(x). \]
  Hence $1-f(x) = x$ or $f(x) = 1-x$.
\end{proof}
\begin{remark*}
  The result $f(f(x)) + f(x) = 1$ also implies that surjectivity
  would solve the problem.
\end{remark*}

\begin{claim*}
  $f$ is injective.
\end{claim*}
\begin{proof}
  Setting $y=1$ in the original equation gives
  $f(x+1) = f(x)-1$, and by induction
  \begin{equation}
    f(x+n) = f(x)-n.
    \label{eq:linshift}
  \end{equation}
  Assume now $f(a) = f(b)$.
  By using \eqref{eq:linshift} we may shift $a$ and $b$
  to be large enough that
  we may find $x$ and $y$ obeying $x+y=a+1$, $xy=b$.
  Setting these gives
  \begin{align*}
    f(f(x)f(y)) &= f(xy) - f(x+y) = f(b) - f(a+1) \\
    &= f(b) + 1 - f(a) = 1
  \end{align*}
  from which we conclude
  \[ f\left( f(x)f(y) + 1 \right) = 0. \]
  Hence by the first claim
  we have $f(x)f(y) + 1 = 1$, so $f(x)f(y) = 0$.
  Applying the first claim again gives $1 \in \{x,y\}$.
  But that implies $a=b$.
\end{proof}

\begin{remark*}
  Jessica Wan points out that
  for any $a \neq b$, at least one of $a^2 > 4(b-1)$
  and $b^2 > 4(a-1)$ is true.
  So shifting via \eqref{eq:linshift}
  is actually unnecessary for this proof.
\end{remark*}

\begin{remark*}
  One can solve the problem over $\QQ$
  using only \eqref{eq:linshift} and the easy parts.
  Indeed, that already implies $f(n) = 1-n$ for all $n$.
  Now we induct to show $f(p/q) = 1-p/q$ for all $0 < p < q$ (on $q$).
  By choosing $x = 1+p/q$, $y = 1+q/p$,
  we cause $xy = x+y$,
  and hence $0 = f\left( f(1+p/q)f(1+q/p) \right)$
  or $1 = f(1+p/q)f(1+q/p)$.

  By induction we compute $f(1+q/p)$
  and this gives $f(p/q+1) = f(p/q)-1$.
\end{remark*}

---

This problem is sort of divided into two parts.
One is the ``standard'' part, which is not easy \emph{per se},
but which experienced contestants won't find surprising.
However, the argument in the final part is quite nice and conceptual,
and much less run-of-the-mill.

We begin with some standard plug/chug.
\begin{walk}
  \ii Find all three linear solutions
  and convince yourself there are no other polynomial solutions.
  \ii Check that if $f$ is a solution, then so is $-f$.
  \ii Show there exists $z$ such that $f(z) = 0$.
  (We'll find the exact value later;
  for now just show it exists.)
  \ii Show that if $f(0) = 0$ then $f \equiv 0$.
  So we henceforth assume $f(0) \neq 0$.
  \ii Using the cancellation trick, prove that if $f(z) = 0$
  (and $f(0) \neq 0$) for some $z$, then $z = 1$.
  Then show that $f(0) = \pm 1$.
\end{walk}
From (b) and (e), we assume $f(0) = 1$, $f(1) = 0$ in what follows,
and will try to show $f(x) \equiv 1-x$.
This lets us plug in some more stuff.
\begin{walk}[resume]
  \ii Show that $f(x+1) = f(x)-1$
  and compute $f$ on all integer values.
  \ii Show that $f(f(x)) = 1-f(x)$.
  Thus if $f$ was surjective we would be done.
  However, this seems hard to arrange,
  since the original equation has everything wrapped in $f$'s.
  \ii Using the triple involution trick,
  prove that $f(1-f(x)) = f(x)$.
  Thus if $f$ was injective, we would also be done.
\end{walk}
So we will now prove $f$ is injective: this is the nice part.
Assume $f(a) = f(b)$; we will try to prove $a = b$.
\begin{walk}[resume]
  \ii Show that if $N$ is a sufficiently large integer,
  then we can find $x$ and $y$ such that $x+y = a+N$
  and $xy = b+N$.
  Use this to prove that $f(f(x) f(y)) = 0$ for that
  pair $(x,y)$ and hence
  thus $f(x) f(y) = 1$.
  \ii The previous part shows us how we might think
  about using the cancellation trick.
  However, it is basically useless since $f(x) f(y) = 1$
  is not really a useful condition.

  However, modify the approach of (i) so that
  instead the conclusion
  ends up as $f(x)f(y) = 0$ instead.
  Deduce that $1 \in \{x,y\}$ in that case.
  \ii Using the argument in (j) prove that $a = b$.
\end{walk}
Some historical lore about this problem:
this was shortlisted as A6,
and in my opinion too hard for the P2 position,
despite being nice for a functional equation.
Most countries did poorly, with USA and China having only two solves,
but the Korean team had an incredibly high five solves.
However, an unreasonably generous 4 points was awarded
for progress up to part (h),
thus cancelling a lot of the advantage from the Korean team.
Thus I was relieved that the Korean team still finished first.
