desc: $BC$, $EF$, $O_1O_2$ concurrent
source: IMO 2021/3
url: https://aops.com/community/p22698068
tags: [2023-01, pop, anglechase, zayin]
author: Mykhalio Shtandenko (UKR)
hardness: 45

---

Let $D$ be an interior point of the acute triangle $ABC$
with $AB > AC$ so that $\angle DAB = \angle CAD$.
The point $E$ on the segment $AC$ satisfies $\angle ADE =\angle BCD$,
the point $F$ on the segment $AB$ satisfies $\angle FDA =\angle DBC$,
and the point $X$ on the line $AC$ satisfies $CX = BX$.
Let $O_1$ and $O_2$ be the circumcenters of the triangles
$ADC$ and $EXD$, respectively.
Prove that the lines $BC$, $EF$, and $O_1O_2$ are concurrent.

---

\emph{This problem and solution were contributed by Abdullahil Kafi}.


\begin{claim*}
    Quadrilateral $BCEF$ is cyclic.
\end{claim*}

\begin{proof}
    Let $D'$ be the isogonal conjugate of the point $D$. The
    angle condition implies quadrilateral $CEDD'$ and $BFDD'$
    are cyclic. By power of point we have \[ AE\cdot AC=AD\cdot AD'=AF\cdot AB \]
    So $BCEF$ is cyclic.
\end{proof}

\begin{claim*}
    Line $ZD$ is tangent to the circles $(BCD)$ and $(DEF)$
    where $Z=EF\cap BC$.
\end{claim*}

\begin{proof}
    Let $\angle CAD=\angle BAD=\alpha$, $\angle BCD=\beta$,
    $\angle DBC=\gamma$, $\angle ACD=\phi$,
    $\angle ABD=\epsilon$.
    From $\triangle ABC$ we have
    $2\alpha+\beta+\gamma+\phi+\epsilon=180^\circ$.
    Let $\ell$ be a line tangent to $(BCD)$ and $K$ be a
    point on it in the same side of $AD$ as $C$ and
    $L=AD\cap BC$. From our labeling we have,
    \begin{align*}
	    \angle AFE &= \beta + \phi \qquad \angle BFD =
	    \alpha + \gamma \qquad \angle DFE = \alpha + \phi
	    \qquad \angle CDL = \alpha + \phi
    \end{align*}
    Now $\angle CDJ = 180^\circ - \gamma - \beta - (\alpha + \phi) = \alpha + \epsilon$.
    So $\angle DFE = \angle EDK = \alpha + \epsilon$, which
    means $\ell$ is also tangent to $(DEF)$. Now by the
    radical center theorem we have $\ell$ passes through
    $Z$.
\end{proof}

Let $M$ be the Miquel point of the cyclic quadrilateral
$BCEF$. From the Miquel configuration we have $A$, $M$, $Z$
are collinear and $(AFEM)$, $(ZCEM)$ are cyclic.

\begin{claim*}
    Points $B$, $X$, $M$, $E$ are cyclic.
\end{claim*}

\begin{proof}
    Notice that $\angle EMB = 180^\circ - \angle AMB -\angle EMZ$
    $=$ $180^\circ - 2\angle ACB = \angle EXB$.
\end{proof}

Let $N$ be the other intersection of circles $(ACD)$ and
$(DEX)$ and let $R$ be the intersection of $AC$ and $BM$.

\begin{center}
\begin{asy}
/*
    Converted from GeoGebra by User:Azjps using Evan's magic cleaner
    https://github.com/vEnhance/dotfiles/blob/main/py-scripts/export-ggb-clean-asy.py
*/
import graph;
size(14cm);
pen ffxfqq = rgb(1.,0.49803,0.);
pen qqwuqq = rgb(0.,0.39215,0.);
pen ffdxqq = rgb(1.,0.84313,0.);
pen qqffff = rgb(0.,1.,1.);
draw((12.,-7.5)--(3.88816,0.36923), linewidth(0.4) + red);
draw((3.88816,0.36923)--(0.5,-7.5), linewidth(0.4) + red);
draw((2.73129,-2.31767)--(4.74594,-3.92841), linewidth(0.4));
draw((4.74594,-3.92841)--(5.47993,-1.17493), linewidth(0.4));
draw(circle((6.25,-6.90419), 5.78078), linewidth(0.4) + red);
draw((6.25,5.85474)--(12.,-7.5), linewidth(0.4));
draw((-9.73375,-7.5)--(5.47993,-1.17493), linewidth(0.4) + red);
draw((-9.73375,-7.5)--(12.,-7.5), linewidth(0.4) + red);
draw((-9.73375,-7.5)--(4.74594,-3.92841), linewidth(0.4));
draw(circle((0.03259,-2.63473), 4.88766), linewidth(0.4) + ffxfqq);
draw(circle((6.84056,0.75675), 5.13208), linewidth(0.4) + qqwuqq);
draw(circle((13.09242,3.87122), 11.42357), linewidth(0.4) + ffxfqq);
draw(circle((6.25,-5.31169), 6.15233), linewidth(0.4) + red);
draw(circle((3.88816,-1.22327), 1.59250), linewidth(0.4) + red);
draw((3.88816,0.36923)--(-9.73375,-7.5), linewidth(0.4));
draw(circle((9.81689,-0.52472), 7.30892), linewidth(0.4) + ffdxqq);
draw((2.50861,-0.42772)--(12.,-7.5), linewidth(0.4));
draw((1.83901,1.90685)--(4.74594,-3.92841), linewidth(0.4));
draw((6.25,5.85474)--(3.88816,0.36923), linewidth(0.4));
draw((0.5,-7.5)--(4.74594,-3.92841), linewidth(0.4));
draw((4.74594,-3.92841)--(12.,-7.5), linewidth(0.4));
draw(shift((-9.73375,-7.5))*xscale(14.91368)*yscale(14.91368)*arc((0,0),1,3.18577,50.41705), linewidth(0.4) + dotted + qqffff);

dot((12.,-7.5),linewidth(3.pt));
label("$B$", (12.47142,-8.47771), NE);
dot((0.5,-7.5),linewidth(3.pt));
label("$C$", (-0.29392,-8.47771), NE);
dot((3.88816,0.36923),linewidth(3.pt));
label("$A$", (3.42645,1.11336), NE);
dot((6.25,5.85474),linewidth(3.pt));
label("$X$", (5.78156,6.47208), NE);
dot((4.74594,-3.92841),linewidth(3.pt));
label("$D$", (4.65520,-5.03038), NE);
dot((5.47993,-1.17493),linewidth(3.pt));
label("$F$", (5.64503,-0.86628), NE);
dot((2.73129,-2.31767),linewidth(3.pt));
label("$E$", (1.58333,-2.43635), NE);
dot((-9.73375,-7.5),linewidth(3.pt));
label("$Z$", (-10.77243,-8.20465), NE);
dot((2.50861,-0.42772),linewidth(3.pt));
label("$M$", (1.20788,-0.21777), NE);
dot((1.83901,1.90685),linewidth(3.pt));
label("$N$", (0.79829,2.47864), NE);
dot((3.29328,-1.01240),linewidth(3.pt));
label("$R$", (3.63125,-1.10520), NE);
\end{asy}
\end{center}

\begin{claim*}
    Points $B$, $D$, $M$, $N$ are cyclic.
\end{claim*}

\begin{proof}
  By power of point we have
  \[
    \opname{Pow}(R, (ACD)) = RC \cdot RA = RM \cdot RB
    = RE \cdot RX = \opname{Pow}(R, (DEX)).
  \]
  Hence $R$ lies on the radical axis of $(ACD)$ and
  $(DEX)$, so $N$, $R$, $D$ are collinear. Also
  \[ RN \cdot RD = RA \cdot RC = RM \cdot RB \] So $BDMN$
  is cyclic.
\end{proof}

Notice that $(ACD)$, $(BDMN)$, $(DEX)$ are coaxial so their
centers are collinear. Now we just need to prove the
centers of $(ACD)$, $(BDMN)$ and $Z$ are collinear. To
prove this, take a circle $\omega$ with radius $ZD$
centered at $Z$. Notice that by power of point
\[ ZC \cdot ZB = ZD^2 = ZE \cdot ZF = ZM \cdot ZA \]
which means inversion circle $\omega$ swaps $(ACD)$ and $(BDMN)$.
So the centers of $(ACD)$ and $(BDMN)$ must
have to be collinear with the center of inversion circle, as desired.
