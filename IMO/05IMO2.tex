desc:  Infinitely many pos/neg problem
source:  IMO 2005/2
tags:  [induct, rigid, smallcases, stronger, criticalclaim, reliable, scouting, optimization, good, pigeonhole, 2017-12, aleph]
hardness: 10
url: https://aops.com/community/p281572
author: Nicholas de Bruijn (NLD)

---

Let $a_1$, $a_2$, \dots\ be a sequence of integers
with infinitely many positive and negative terms.
Suppose that for every positive integer $n$
the numbers $a_1$, $a_2$, \dots, $a_n$
leave $n$ different remainders upon division by $n$.
Prove that every integer occurs exactly once in the sequence.

---

Obviously every integer appears at most once
(otherwise take $n$ much larger).
So we will prove every integer appears at least once.

\begin{claim*}
  For any $i < j$ we have $\left\lvert a_i-a_j \right\rvert < j$.
\end{claim*}
\begin{proof}
  Otherwise, let $n = \left\lvert a_i-a_j \right\rvert \neq 0$.
  Then $i,j \in [1,n]$ and $a_i \equiv a_j \pmod n$,
  contradiction.
\end{proof}

\begin{claim*}
  For any $n$, the set $\{a_1, \dots, a_n\}$
  is of the form $\{k+1, \dots, k+n\}$ for some integer $k$.
\end{claim*}
\begin{proof}
  By induction, with the base case $n=1$ being vacuous.
  For the inductive step,
  suppose $\{a_1, \dots, a_n\} = \{k+1, \dots, k+n\}$ are determined.
  Then
  \[ a_{n+1} \equiv k \pmod{n+1}. \]
  Moreover by the earlier claim we have
  \[ \left\lvert a_{n+1}-a_1 \right\rvert < n+1. \]
  From this we deduce $a_{n+1} \in \{k, k+n+1\}$ as desired.
\end{proof}

This gives us actually a complete description
of all possible sequences satisfying the hypothesis:
choose any value of $a_1$ to start.
Then, for the $n$th term,
the set $S = \{a_1, \dots, a_{n-1}\}$
is (in some order) a set of $n-1$ consecutive integers.
We then let $a_n = \max S + 1$ or $a_n = \min S - 1$.
A picture of six possible starting terms is shown below.

\begin{center}
\begin{asy}
size(6cm);
defaultpen(fontsize(9pt));
int[] a = {0, 6, 5, 7, 4, 3, 8};
for (int i=1; i<=9; ++i) {
  draw((0,i)--(7.2,i), grey+dotted);
}
for (int i=1; i<=7; ++i) {
  draw((i,0)--(i,9.2), grey+dotted);
  if (i <= 5) draw((i,a[i])--(i+1,a[i+1]), red+dashed);
  if (i <= 6) dot("$"+(string) a[i] + "$", (i, a[i]), dir(15), red);
  if (i <= 6) label("$a_{" + (string) i + "}$", (i, 0), dir(-90));
}
draw( (-0.2,0)--(7.2,0), black );
draw( (0,-0.2)--(0,9.2), black );
\end{asy}
\end{center}


Finally, we observe that the condition that
the sequence has infinitely many positive and negative terms
(which we have not used until now)
implies it is unbounded above and below.
Thus it must contain every integer.
