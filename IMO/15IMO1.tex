desc:  Build a balanced set, then find all center-free ones
source:  IMO 2015/1
tags:  [adhoc, construct, dumb, parity, global, reliable, greedy, find, good, equalitycase, bet]
author: Merlijn Staps (NLD)
hardness: 15
url: https://aops.com/community/p5079689

---

We say that a finite set $\mathcal{S}$ of points in the plane
is \emph{balanced} if,
for any two different points $A$ and $B$ in $\mathcal{S}$,
there is a point $C$ in $\mathcal{S}$ such that $AC=BC$.
We say that $\mathcal{S}$ is \emph{centre-free} if for
any three different points $A$, $B$ and $C$ in $\mathcal{S}$,
there are no points $P$ in $\mathcal{S}$ such that $PA=PB=PC$.

\begin{enumerate}
\item[(a)] Show that for all integers $n\ge 3$,
  there exists a balanced set consisting of $n$ points.
\item[(b)] Determine all integers $n\ge 3$ for which
  there exists a balanced centre-free set consisting of $n$ points.
\end{enumerate}

---

For part (a), take a circle centered at a point $O$,
and add $n-1$ additional points by adding pairs of points
separated by an arc of $60^{\circ}$ or similar triples.
An example for $n = 6$ is shown below.
\begin{center}
\begin{asy}
  size(4cm);
  draw(unitcircle);
  dotfactor *= 1.5;
  pair O = (0,0);
  dot("$O$", O, dir(-90), blue);
  pair A = dir(37);
  pair B = A*dir(-60);
  pair C = B*dir(-60);
  pair X = dir(117);
  pair Y = X*dir(60);
  draw(O--X--Y--cycle, deepgreen+dashed);
  dot(X, deepgreen);
  dot(Y, deepgreen);
  draw(B--O--A--B--C--O, red+dashed);
  dot(A, red);
  dot(B, red);
  dot(C, red);
\end{asy}
\end{center}

For part (b), the answer is odd $n$, achieved by taking a regular $n$-gon.
To show even $n$ fail, note that some point is on the perpendicular bisector of
\[ \left\lceil \frac 1n \binom n2 \right\rceil = \frac{n}{2} \]
pairs of points, which is enough.
(This is a standard double-counting argument.)

As an aside, there is a funny joke about this problem.
There are two types of people in the world:
those who solve (b) quickly and then take forever to solve (a),
and those who solve (a) quickly and then can't solve (b) at all.
(Empirically true when the Taiwan IMO 2014 team was working on it.)
