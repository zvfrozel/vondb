desc: Hexagon with opposite sides
source: IMO 2003/3
tags: [2019-04, brave, length, complex, instructive, weird, meta, manip, zayin]
hardness: 40
url: https://aops.com/community/p263
author: Waldemar Pompe (POL)

---

Each pair of opposite sides of convex hexagon has the property that
the distance between their midpoints is $\frac{\sqrt3}{2}$
times the sum of their lengths.
Prove that the hexagon is equiangular.

---

Unsurprisingly, this is a geometric inequality.
Denote the hexagon by $ABCDEF$.
Then we have that
\[
  \left\lvert
  \frac{\vec D + \vec E}{2} - \frac{\vec A + \vec B}{2}
  \right\rvert
  = \sqrt3 \cdot \frac{\left\lvert \vec B - \vec A \right\rvert
    + \left\lvert \vec E - \vec D \right\rvert}{2}
  \ge \sqrt 3 \cdot
  \left\lvert \frac{(\vec B - \vec A) - (\vec E - \vec D)}{2}  \right\rvert
\]
and cyclic variations.
Suppose we define the right-hand sides as variables
\begin{align*}
  \vec x = (\vec B - \vec A) - (\vec E - \vec D) \\
  \vec y = (\vec D - \vec C) - (\vec A - \vec F) \\
  \vec z = (\vec F - \vec E) - (\vec C - \vec B).
\end{align*}
Then we now have
\begin{align*}
  \left\lvert \vec y - \vec z \right\rvert
  &\ge \sqrt 3 \left\lvert \vec x  \right\rvert \\
  \left\lvert \vec z - \vec x \right\rvert
  &\ge \sqrt 3 \left\lvert \vec y  \right\rvert \\
  \left\lvert \vec x - \vec y \right\rvert
  &\ge \sqrt 3 \left\lvert \vec z  \right\rvert.
\end{align*}
We square all sides (using
$\left\lvert \vec v \right\rvert^2 = \vec v \cdot \vec v$)
and then sum to get
\[ \sum_{\text{cyc}} (\vec y - \vec z) \cdot (\vec y - \vec z)
  \ge 3 \sum_{\text{cyc}} \vec x \cdot \vec x \]
which rearranges to
\[- \left\lvert \vec x + \vec y + \vec z \right\rvert^2 \ge 0. \]
This can only happen if $\vec x + \vec y + \vec z =0$,
and moreover all the inequalities above were actually equalities.
That means that our triangle inequalities above were actually sharp
(and already we have $\ol{AB} \parallel \ol{DE}$ and so on).

Working with just $x$ and $y$ now we have
\begin{align*}
  3 (\vec x \cdot \vec x) &= (2 \vec y - \vec x) \cdot (2 \vec y - \vec x) \\
  &= \vec x \cdot \vec x - 4 \vec y \cdot \vec x + 4 \vec y \cdot \vec y \\
  \implies
  -\vec x \cdot \vec x + 2 (\vec y \cdot \vec y) &= 2 \vec x \cdot \vec y \\
  2 (\vec x \cdot \vec x) - \vec y \cdot \vec y &= 2 \vec x \cdot \vec y.
\end{align*}
which implies $\vec x \cdot \vec x = \vec y \cdot \vec y$,
that is, $\vec x$ and $\vec y$ have the same magnitude.
In this way we find $\vec x$, $\vec y$, $\vec z$ all
have the same magnitude,
and since $\vec x + \vec y + \vec z = 0$
they are related by $120\dg$ rotations, as desired.

\begin{remark*}
  In fact one can show further that the equiangular hexagons
  which work are exactly those formed by taking an equilateral triangle
  and cutting off equally sized corners.
  This equality case helps motivate the solution.
\end{remark*}
\begin{remark*}
  One can note this ``must'' be an inequality
  because the space of such hexagons is $2$-dimensional,
  even though \emph{a priori} the space of hexagons satisfying
  three given conditions should have dimension $9-3=6$.
\end{remark*}
