desc: Geoff clapping and frogs jumping
source: IMO 2016/6
tags: [2017-02, process, parity, invariant, process, optimization, rushdown, smallcases, scouting, thinkbig, dalet]
author: Josef Tkadlec (CZE)
hardness: 25
url: https://aops.com/community/p6642576

---

There are $n\ge 2$ line segments in the plane such that
every two segments cross and no three segments meet at a point.
Geoff has to choose an endpoint of each segment and place a frog
on it facing the other endpoint. Then he will clap his hands $n-1$ times.
Every time he claps, each frog will immediately jump forward
to the next intersection point on its segment.
Frogs never change the direction of their jumps.
Geoff wishes to place the frogs in such a way that no two of them
will ever occupy the same intersection point at the same time.

\begin{enumerate}[(a)]
  \ii Prove that Geoff can always fulfill his wish if $n$ is odd.
  \ii Prove that Geoff can never fulfill his wish if $n$ is even.
\end{enumerate}

---

The following solution was communicated to me by Yang Liu.

Imagine taking a larger circle $\omega$ encasing
all $\binom{n}{2}$ intersection points.
Denote by $P_1$, $P_2$, \dots, $P_{2n}$ the order of the points on $\omega$
in clockwise order; we imagine placing the frogs on $P_i$ instead.
Observe that, in order for every pair of segments to meet,
each line segment must be of the form $P_i P_{i+n}$.
\begin{center}
\begin{asy}
  draw(unitcircle);
  pair P_1 = dir(100);
  pair P_2 = dir(150);
  pair P_3 = dir(190);
  pair P_4 = dir(210);
  pair P_5 = dir(280);
  pair P_6 = dir(340);
  pair P_7 = dir(2);
  pair P_8 = dir(55);
  dot("$1$", P_1, dir(P_1));
  dot("$2$", P_2, dir(P_2));
  dot("$3$", P_3, dir(P_3));
  dot("$4$", P_4, dir(P_4));
  dot("$5$", P_5, dir(P_5));
  dot("$6$", P_6, dir(P_6));
  dot("$7$", P_7, dir(P_7));
  dot("$8$", P_8, dir(P_8));
  draw(P_1--P_5);
  draw(P_2--P_6);
  draw(P_3--P_7);
  draw(P_4--P_8);
  dot(extension(P_1, P_5, P_2, P_6), red);
  dot(extension(P_1, P_5, P_3, P_7), red);
  dot(extension(P_1, P_5, P_4, P_8), red);
  dot(extension(P_2, P_6, P_3, P_7), red);
  dot(extension(P_2, P_6, P_4, P_8), red);
  dot(extension(P_3, P_7, P_4, P_8), red);
\end{asy}
\end{center}
Then:
\begin{enumerate}[(a)]
  \ii Place the frogs on $P_1$, $P_3$, \dots, $P_{2n-1}$.
  A simple parity arguments shows this works.

  \ii Observe that we cannot place frogs on consecutive $P_i$,
  so the $n$ frogs must be placed on alternating points.
  But since we also are supposed to not place frogs on
  diametrically opposite points,
  for even $n$ we immediately get a contradiction.
\end{enumerate}

\begin{remark*}
Yang says: this is easy to guess if you just do a
few small cases and notice that the pairs of
``violating points'' just forms a large cycle around.
\end{remark*}
