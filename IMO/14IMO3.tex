desc:  Inversion at H, also trig
source:  IMO 2014/3
tags:  [inversion, good, trig, criticalclaim, anglechase, zayin]
author: ALi Zamani (IRN)
hardness: 40
url: https://aops.com/community/p3542092

---

Convex quadrilateral $ABCD$ has $\angle ABC = \angle CDA = 90\dg$.
Point $H$ is the foot of the perpendicular from $A$ to $\ol{BD}$.
Points $S$ and $T$ lie on sides $AB$ and $AD$,
respectively, such that $H$ lies inside triangle $SCT$ and
\[ \angle CHS - \angle CSB = 90^{\circ},
  \quad \angle THC - \angle DTC = 90^{\circ}. \]
Prove that line $BD$ is tangent to the circumcircle of triangle $TSH$.

---

\paragraph{First solution (mine).}
First we rewrite the angle condition in a suitable way.
\begin{claim*}
  We have $\angle ATH = \angle TCH + 90\dg$.
  Thus the circumcenter of $\triangle CTH$ lies on $\ol{AD}$.
  Similarly the circumcenter of $\triangle CSH$ lies on $\ol{AB}$.
\end{claim*}
\begin{proof}
  \begin{align*}
    \dang ATH &= \dang DTH \\
    &= \dang DTC + \dang CTH \\
    &= \dang DTC - \dang THC - \dang HCT \\
    &= 90\dg - \dang HCT = 90\dg + \dang TCH.
  \end{align*}
  which implies conclusion.
\end{proof}

\begin{center}
\begin{asy}
pair A = dir(110);
pair B = dir(200);
pair D = dir(-20);
pair C = -A;
pair H = foot(A, B, D);
pair Cs = -C+2*foot(origin, H, C);
pair As = -A+2*foot(origin, A, H);

filldraw(unitcircle, opacity(0.1)+lightcyan, lightblue);
draw(A--B--C--D--cycle, lightblue);

pair M = midpoint(B--As);
pair N = midpoint(D--As);

pair U = IP(Cs--N, circumcircle(D, H, As));
pair V = IP(Cs--M, circumcircle(B, H, As));
pair Ss = D+As-U;
pair Ts = B+As-V;
pair T = extension(Ts, H, A, D);
pair S = extension(Ss, H, A, B);

pair O_D = circumcenter(C, H, T);

filldraw(circumcircle(C, H, T), opacity(0.1)+yellow, red);
pair P = extension(A, H, O_D, midpoint(T--H));
draw(A--H, lightblue);
draw(B--D, lightblue);

draw(H--C--T--cycle, red);
draw(H--O_D, lightblue);
draw(P--O_D, heavygreen);

dot("$A$", A, dir(A));
dot("$B$", B, dir(B));
dot("$D$", D, dir(D));
dot("$C$", C, dir(C));
dot("$H$", H, dir(H));
dot("$T$", T, dir(80));
dot("$S$", S, dir(S));
dot("$O_D$", O_D, dir(45));
dot("$P$", P, dir(45));

/* TSQ Source:

A = dir 110
B = dir 200
D = dir -20
C = -A
H = foot A B D
C* := -C+2*foot origin H C R80
A* := -A+2*foot origin A H

unitcircle 0.1 lightcyan / lightblue
A--B--C--D--cycle lightblue

M := midpoint B--As
N := midpoint D--As

U := IP Cs--N circumcircle D H As
V := IP Cs--M circumcircle B H As
S* := D+As-U
T* := B+As-V
T = extension Ts H A D R80
S = extension Ss H A B

O_D = circumcenter C H T R45

circumcircle C H T 0.1 yellow / red
P = extension A H O_D midpoint T--H R45
A--H lightblue
B--D lightblue

H--C--T--cycle red
H--O_D lightblue
P--O_D heavygreen

*/
\end{asy}
\end{center}

Let the perpendicular bisector of $TH$ meet $AH$ at $P$ now.
It suffices to show that $\frac{AP}{PH}$ is symmetric in $b = AD$ and $d=AB$,
because then $P$ will be the circumcenter of $\triangle TSH$.
To do this, set $AH = \frac{bd}{2R}$ and $AC=2R$.

Let $O$ denote the circumcenter of $\triangle CHT$.
Use the Law of Cosines on $\triangle ACO$ and $\triangle AHO$,
using variables $x=AO$ and $r=HO$.
We get that
\[ r^2 = x^2 + AH^2 - 2x \cdot AH \cdot \frac{d}{2R} = x^2 + (2R)^2 - 2bx. \]
By the angle bisector theorem,
$\frac{AP}{PH} = \frac{AO}{HO}$.

The rest is computation: notice that
\[ r^2 - x^2 = h^2 - 2xh \cdot \frac{d}{2R} = (2R)^2 - 2bx \]
where $h = AH = \frac{bd}{2R}$, whence
\[ x = \frac{(2R)^2-h^2}{2b - 2h \cdot \frac{d}{2R}}. \]
Moreover,
\[ \frac{1}{2} \left( \frac{r^2}{x^2}-1 \right)
= \frac{1}{x} \left( \frac 2x R^2 - b \right). \]
Now, if we plug in the $x$ in the
right-hand side of the above, we obtain
\[ \frac{2b-2h \cdot \frac{d}{2R}}{4R^2-h^2}
  \left( \frac{2b-2h \cdot \frac{d}{2R}}{4R^2-h^2} \cdot 2R^2 - b\right)
  = \frac{2h}{(4R^2-h^2)^2} \left( b- h \cdot \frac{d}{2R} \right)
  \left( -2hdR + bh^2 \right). \]
Pulling out a factor of $-2Rh$ from the rightmost term,
we get something that is symmetric in $b$ and $d$, as required.

\paragraph{Second solution (Victor Reis).}
Here is the fabled solution using inversion at $H$.
First, we rephrase the angle conditions in the following ways:
\begin{itemize}
  \ii $\ol{AD} \perp (THC)$,
  which is equivalent to the claim from the first solution.
  \ii $\ol{AB} \perp (SHC)$, by symmetry.
  \ii $\ol{AC} \perp (ABCD)$, by definition.
\end{itemize}
Now for concreteness we will use a negative inversion at $H$
which swaps $B$ and $D$ and overlay it on the original diagram.
As usual we denote inverses with stars.

Let us describe the inverted problem.
We let $M$ and $N$ denote the midpoints of $\ol{A^\ast B^\ast}$
and $\ol{A^\ast D^\ast}$, which are the centers of
$(HA^\ast B^\ast)$ and $(HA^\ast D^\ast)$.
From $\ol{T^\ast C^\ast} \perp (HA^\ast D^\ast)$,
we know have $C^\ast$, $M$, $T^\ast$ collinear.
Similarly, $C^\ast$, $N$, $S^\ast$ are collinear.
We have that $(A^\ast H C^\ast)$ is orthogonal to $(ABCD)$ which remains fixed.
We wish to show $\ol{T^\ast S^\ast}$ and $\ol{MN}$ are parallel.

\begin{center}
\begin{asy}
pair A = dir(100);
pair B = dir(200);
pair D = dir(-20);
pair C = -A;
pair H = foot(A, B, D);
pair Cs = -C+2*foot(origin, H, C);
pair As = -A+2*foot(origin, A, H);

filldraw(unitcircle, opacity(0.1)+lightcyan, lightblue);
draw(A--B--C--D--cycle, lightblue);

filldraw(circumcircle(B, H, As), opacity(0.1)+lightgreen, dotted+heavygreen);
filldraw(circumcircle(D, H, As), opacity(0.1)+lightgreen, dotted+heavygreen);

pair M = midpoint(B--As);
pair N = midpoint(D--As);

pair U = IP(Cs--N, circumcircle(D, H, As));
pair V = IP(Cs--M, circumcircle(B, H, As));
pair Ss = D+As-U;
pair Ts = B+As-V;
pair T = extension(Ts, H, A, D);
pair S = extension(Ss, H, A, B);

draw(Ts--Cs--Ss, heavygreen);
draw(Ts--Ss, red);
draw(B--D, lightblue);
draw(A--As, lightblue);
draw(A--B--As--D--cycle, lightblue);
draw(C--Cs--A--cycle, lightblue);

draw(S--Ss, dotted+blue);
draw(T--Ts, dotted+blue);
draw(M--N, red);

filldraw(circumcircle(As, M, N), opacity(0.1)+lightred, red+dashed);
filldraw(circumcircle(As, H, Cs), opacity(0.1)+pink, purple+dashed);
clip(box((-1.4,-1.4),(2,2)));

dot("$A$", A, dir(A));
dot("$B$", B, dir(B));
dot("$D$", D, dir(D));
dot("$C$", C, dir(C));
dot("$H$", H, dir(H));
dot("$C^\ast$", Cs, dir(80));
dot("$A^\ast$", As, dir(As));
dot("$M$", M, dir(M));
dot("$N$", N, dir(N));
dot("$S^\ast$", Ss, dir(Ss));
dot("$T^\ast$", Ts, dir(Ts));
dot("$T$", T, dir(50));
dot("$S$", S, dir(S));

/* TSQ Source:

A = dir 100
B = dir 200
D = dir -20
C = -A
H = foot A B D
C* = -C+2*foot origin H C R80
A* = -A+2*foot origin A H

unitcircle 0.1 lightcyan / lightblue
A--B--C--D--cycle lightblue

circumcircle B H As 0.1 lightgreen / dotted heavygreen
circumcircle D H As 0.1 lightgreen / dotted heavygreen

M = midpoint B--As
N = midpoint D--As

U := IP Cs--N circumcircle D H As
V := IP Cs--M circumcircle B H As
S* = D+As-U
T* = B+As-V
T = extension Ts H A D R50
S = extension Ss H A B

Ts--Cs--Ss heavygreen
Ts--Ss red
B--D lightblue
A--As lightblue
A--B--As--D--cycle lightblue
C--Cs--A--cycle lightblue

S--Ss dotted blue
T--Ts dotted blue
M--N red

circumcircle As M N 0.1 lightred / red dashed
circumcircle As H Cs 0.1 pink / purple dashed
!clip(box((-1.4,-1.4),(2,2)));

*/
\end{asy}
\end{center}

Lot $\omega$ denote the circumcircle of $\triangle A^\ast H C^\ast$,
which is orthogonal to the original circle $(ABCD)$.
It would suffices to show $(A^\ast H C^\ast)$
is an $H$-Apollonius circle with respect to $\ol{MN}$,
from which we would get $C^\ast M / H M = C^\ast N / H N$.

However, $\omega$ through $H$ and $A$,
hence it center lies on line $MN$.
Moreover $\omega$ is orthogonal to $(A^\ast MN)$
(since $(A^\ast MN)$ and $(A^\ast BD)$ are homothetic).
This is enough (for example, if we let $O$ denote the center of $\omega$,
we now have $\mathrm{r}(\omega)^2 = OH^2 = OM \cdot ON$).
(Note in this proof that the fact that $C^\ast$ lies on $(ABCD)$
is not relevant.)


---

We present a direct trig solution.
\begin{walk}
  \ii By angle chasing, show the angle condition
  becomes $\angle ATH = \angle TCH  + 90\dg$.
  \ii Prove that the circumcenter of $\triangle HCT$ lies on line $AD$,
  while the circumcenter of $\triangle CSH$ lies on line $AB$.
\end{walk}
Now let the perpendicular bisector of $\ol{TH}$
meet $\ol{AH}$ at $P$.
We will show $AP/PH$ is symmetric, solving the problem.
Let $O$ denote the circumcenter of $\triangle TCH$.
\begin{walk}[resume]
  \ii Prove that $AH \cdot AC = AB \cdot AD$.
  \ii Show that $AP/PH = AO/OH$.
  This is our last serious synthetic observation.
\end{walk}
We will use $R$, the circumradius,
together with $d = AB$ and $b = AD$ as the variables now.
We should be able to calculate everything now.
\begin{walk}[resume]
  \ii Apply the law of cosines to $\triangle AHO$,
  and eliminate any trig expressions.
  \ii Apply the law of cosines to $\triangle ACO$,
  and eliminate any trig expressions.
  \ii Solve for $AO$ to get that
  \[ AO = \frac{(2R)^2-h^2}{2b - 2h \cdot \frac{d}{2R}}. \]
  \ii Let $x = AO$. Show that
  \[ \half \left( \frac{HO^2}{x^2} - 1 \right)
    = \frac 1x \left( \frac 2x R^2 - b \right). \]
  \ii Simplify the right-hand side and show it is symmetric
  in $b$ and $d$; conclude.
\end{walk}

---

A very long solution through $H$-inversion:

Here is the solution from inversion at $H$.
First we have to use the claim above to deal with the angle condition.
Let $E$ be the second intersection of ray $\ol{AH}$ with $(ABCD)$,
so $\ol{CE} \parallel \ol{BD}$.

\begin{center}
\begin{asy}
size(10cm);
pair A = dir(110);
pair B = dir(190);
pair D = dir(-10);
pair C = -A;
pair H = foot(A, B, D);
pair K = -C+2*foot(origin, H, C);
pair E = -A+2*foot(origin, A, H);

filldraw(unitcircle, opacity(0.1)+lightcyan, lightblue);
draw(A--B--C--D--cycle, lightblue);
draw(A--B--E--D--cycle, lightblue);
draw(C--K--A--cycle, lightblue);

filldraw(circumcircle(B, H, E), opacity(0.1)+lightgreen, heavygreen);
filldraw(circumcircle(D, H, E), opacity(0.1)+lightgreen, heavygreen);

pair M = midpoint(B--E);
pair N = midpoint(D--E);

pair U = IP(K--N, circumcircle(D, H, E));
pair V = IP(K--M, circumcircle(B, H, E));
pair Sp = D+E-U;
pair Tp = B+E-V;
draw(Tp--K--Sp, heavygreen);
draw(Tp--Sp, red);
draw(B--D, lightblue);
draw(A--E, lightblue);

pair T = extension(Tp, H, A, D);
pair S = extension(Sp, H, A, B);
draw(S--Sp, dotted+blue);
draw(T--Tp, dotted+blue);

draw(circumcircle(A, K, H), red);
pair X = foot(H, A, D);
pair Y = foot(H, A, B);
draw(X--M, magenta);
draw(Y--N, magenta);

draw(circumcircle(B, D, X), magenta);
pair O = circumcenter(B, D, Y);

draw(circumcircle(B, K, M), dashed);

dot("$A$", A, dir(A));
dot("$B$", B, dir(B));
dot("$D$", D, dir(D));
dot("$C$", C, dir(C));
dot("$H$", H, dir(H));
dot("$K$", K, dir(K));
dot("$E$", E, dir(E));
dot("$M$", M, dir(M));
dot("$N$", N, dir(N));
dot("$S'$", Sp, dir(Sp));
dot("$T'$", Tp, dir(Tp));
dot("$T$", T, dir(T));
dot("$S$", S, dir(S));
dot("$X$", X, dir(X));
dot("$Y$", Y, dir(Y));
dot("$O$", O, dir(O));

/* TSQ Source:

A = dir 110
B = dir 190
D = dir -10
C = -A
H = foot A B D
K = -C+2*foot origin H C
E = -A+2*foot origin A H

unitcircle 0.1 lightcyan / lightblue
A--B--C--D--cycle lightblue
A--B--E--D--cycle lightblue
C--K--A--cycle lightblue

circumcircle B H E 0.1 lightgreen / heavygreen
circumcircle D H E 0.1 lightgreen / heavygreen

M = midpoint B--E
N = midpoint D--E

U := IP K--N circumcircle D H E
V := IP K--M circumcircle B H E
S' = D+E-U
T' = B+E-V
Tp--K--Sp heavygreen
Tp--Sp red
B--D lightblue
A--E lightblue

T = extension Tp H A D
S = extension Sp H A B
S--Sp dotted blue
T--Tp dotted blue

circumcircle A K H red
X = foot H A D
Y = foot H A B
X--M magenta
Y--N magenta

circumcircle B D X magenta
O = circumcenter B D Y

circumcircle B K M dashed

*/
\end{asy}
\end{center}

Let $K$ be the second point of intersection of $(AH)$ with $(ABCD)$.
Then consider a negative inversion at $H$
swapping $B$ and $D$, $A$ and $C$, $K$ and $E$.
Suppose it sends $S$ to $S'$.
Since $\triangle CSH$ had circumcenter on $\ol{AB}$,
it follows that the reflection of $H$ across $\ol{KS'}$
is supposed to lie on $(DHE)$.
Letting $N$ denote the midpoint of $\ol{ED}$,
this actually means $S'$ is the intersection of ray $KN$ with $(HDE)$.
Point $T'$ is defined similarly
and we wish to prove $\ol{S'T'} \parallel \ol{BD}$.

By similar triangles we wish really to show $KM/KN = BM/DN$.
So by law of sines it would suffice to show $\dang BKM = \dang NKD$.

Let $O$ be the midpoint of $\ol{HC}$.
We contend that $KBMO$ is cyclic, since
\[ \dang BKO = \dang EBD = \dang BMO. \]
Thus $\dang BKM = \dang BOM = \dang OBD$.
By analogy $\dang NKD = \dang BDO$ so we're done.

(Some additional facts about this picture:
if $\ol{MH} \cap \ol{AD}$ is $X$ angle chasing also shows $X$
is the foot from $H$ to $\ol{AD}$.
Also, $\dang KXM = \dang KXH = \dang KAH = \dang KAE = \dang KBE = \dang KBM$
and thus $X$ lies on $(KBMO)$ as well.)
