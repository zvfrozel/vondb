desc:  $x$ and $2-x$ are solutions, fixed point blah
source:  IMO 2015/5
tags:  [FE, pitfall, ugly, grinding, hardanswer, zayin]
author: Dorlir Ahmeti (ALB)
hardness: 35
url: https://aops.com/community/p5083463

---

Solve the functional equation
\[ f(x+f(x+y)) + f(xy) = x + f(x+y) + yf(x) \]
for $f \colon \RR \to \RR$.

---

The answers are $f(x) \equiv x$ and $f(x) \equiv 2-x$.
Obviously, both of them work.

Let $P(x,y)$ be the given assertion.
We also will let $S = \{t \mid f(t) = t\}$
be the set of fixed points of $f$.
\begin{itemize}
  \ii From $P(0,0)$ we get $f(f(0)) = 0$.
  \ii From $P(0,f(0))$ we get $2f(0) = f(0)^2$
  and hence $f(0) \in \{0,2\}$.
  \ii From $P(x,1)$
  we find that $x+f(x+1) \in S$ for all $x$.
\end{itemize}

We now solve the case $f(0) = 2$.
\begin{claim*}
  If $f(0) = 2$ then $f(x) \equiv 2-x$.
\end{claim*}
\begin{proof}
  Let $t \in S$ be any fixed point.
  Then $P(0,t)$ gives $2 = 2t$ or $t = 1$;
  so $S = \{1\}$.
  But we also saw $x+f(x+1) \in S$,
  which implies $f(x) \equiv 2-x$.
\end{proof}

Henceforth, assume $f(0) = 0$.
\begin{claim*}
  If $f(0) = 0$ then $f$ is odd.
\end{claim*}
\begin{proof}
  Note that $P(1,-1) \implies f(1) + f(-1) = 1 - f(1)$
  and $P(-1,1) \implies f(-1) + f(-1) = -1 + f(1)$,
  together giving $f(1) = 1$ and $f(-1) = -1$.
  To prove $f$ odd we now obtain more fixed points:
  \begin{itemize}
    \ii From $P(x,0)$ we find that $x+f(x) \in S$ for all $x \in \RR$.
    \ii From $P(x-1,1)$ we find that $x-1+f(x) \in S$ for all $x \in \RR$.
    \ii From $P(1, f(x)+x-1)$ we find $x+1+f(x) \in S$ for all $x \in \RR$.
  \end{itemize}
  Finally $P(x, -1)$ gives $f$ odd.
\end{proof}
To finish from $f$ odd, notice that
\begin{align*}
  P(x,-x) &\implies f(x) + f(-x^2) = x - xf(x) \\
  P(-x,x) &\implies f(-x) + f(-x^2) = -x + xf(-x)
\end{align*}
which upon subtracting gives $f(x) \equiv x$.
