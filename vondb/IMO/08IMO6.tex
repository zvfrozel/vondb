desc: Super homothety problem post-Pitot
source: IMO 2008/6
tags: [2017-08, homothety, nice, criticalclaim, length, yod]
hardness: 40
url: https://aops.com/community/p1191671
author: Vladimir Shmarov (RUS)

---

Let $ABCD$ be a convex quadrilateral with $BA \neq BC$.
Denote the incircles of triangles $ABC$ and $ADC$
by $\omega_1$ and $\omega_2$ respectively.
Suppose that there exists a circle $\omega$ tangent
to ray $BA$ beyond $A$ and to the ray $BC$ beyond $C$,
which is also tangent to the lines $AD$ and $CD$.
Prove that the common external tangents to
$\omega_1$ and $\omega_2$ intersect on $\omega$.

---

By the external version of Pitot theorem, the existence
of $\omega$ implies that
\[ BA + AD = CB + CD. \]
Let $\ol{PQ}$ and $\ol{ST}$ be diameters of $\omega_1$ and $\omega_2$
with $P, T \in \ol{AC}$.
Then the length relation on $ABCD$ implies that $P$ and $T$
are reflections about the midpoint of $\ol{AC}$.

Now orient $AC$ horizontally and let $K$ be the ``uppermost'' point of $\omega$, as shown.

\begin{center}
\begin{asy}
size(12cm);
pair W = dir(48.4);
pair X = dir(68.4);
pair Y = dir(138.4);
pair Z = dir(173.4);

pair A = 2*X*Z/(X+Z);
pair B = 2*W*Z/(W+Z);
pair C = 2*W*Y/(W+Y);
pair D = 2*X*Y/(X+Y);

draw(arc(origin, 1, 40, 180), dashed+orange);

filldraw(incircle(A, B, C), opacity(0.1)+lightred, red);
filldraw(incircle(A, D, C), opacity(0.1)+lightred, red);
filldraw(A--B--C--D--cycle, opacity(0.1)+yellow, orange);
draw(A--C, red);

pair I_B = incenter(A, B, C);
pair I_D = incenter(A, D, C);
pair P = foot(I_B, A, C);
pair Q = 2*I_B-P;
pair T = foot(I_D, A, C);
pair S = 2*I_D-T;

draw(A--Z, orange);
draw(C--W, orange);
pair E = 2*Y*Z/(Y+Z);
pair F = 2*X*W/(X+W);
draw(E--D--F, orange);

pair K = extension(B, Q, D, P);
draw(B--K--P, dashed+red);
draw(P--Q, red);
draw(S--T, red);

dot("$W$", W, dir(W));
dot("$X$", X, dir(95));
dot("$Y$", Y, dir(Y));
dot("$Z$", Z, dir(Z));
dot("$A$", A, dir(135));
dot("$B$", B, dir(B));
dot("$C$", C, dir(45));
dot("$D$", D, dir(-D));
dot("$P$", P, dir(270));
dot("$Q$", Q, 0.8*dir(95));
dot("$T$", T, dir(45));
dot("$S$", S, dir(225));
dot("$K$", K, dir(315));

/* TSQ Source:

!size(12cm);
W = dir 48.4
X = dir 68.4 R95
Y = dir 138.4
Z = dir 173.4

A = 2*X*Z/(X+Z) R135
B = 2*W*Z/(W+Z)
C = 2*W*Y/(W+Y) R45
D = 2*X*Y/(X+Y) R-D

!draw(arc(origin, 1, 40, 180), dashed+orange);

incircle A B C 0.1 lightred / red
incircle A D C 0.1 lightred / red
A--B--C--D--cycle 0.1 yellow / orange
A--C red

I_B := incenter A B C
I_D := incenter A D C
P = foot I_B A C R270
Q = 2*I_B-P 0.8R95
T = foot I_D A C R45
S = 2*I_D-T R225

A--Z orange
C--W orange
E := 2*Y*Z/(Y+Z)
F := 2*X*W/(X+W)
E--D--F orange

K = extension B Q D P R315
B--K--P dashed red
P--Q red
S--T red

*/
\end{asy}
\end{center}

Consequently, a homothety at $B$ maps $Q$, $T$, $K$ to each other
(since $T$ is the uppermost of the excircle, $Q$ of the incircle).
Similarly, a homothety at $D$ maps $P$, $S$, $K$ to each other.
As $\ol{PQ}$ and $\ol{ST}$ are parallel diameters
it then follows $K$ is the exsimilicenter of $\omega_1$ and $\omega_2$.
