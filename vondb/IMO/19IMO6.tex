desc: Traditional geo for complex numbers
source: IMO 2019/6
tags: [2019-07, moving, complex, inversion, rich, pop, zayin]
author: Anant Mudgal (IND)
hardness: 35
url: https://aops.com/community/p12752769

---

Let $ABC$ be a triangle with incenter $I$ and incircle $\omega$.
Let $D$, $E$, $F$ denote the tangency points of $\omega$
with $\ol{BC}$, $\ol{CA}$, $\ol{AB}$.
The line through $D$ perpendicular to $\ol{EF}$
meets $\omega$ again at $R$ (other than $D$),
and line $AR$ meets $\omega$ again at $P$ (other than $R$).
Suppose the circumcircles of $\triangle PCE$ and $\triangle PBF$
meet again at $Q$ (other than $P$).
Prove that lines $DI$ and $PQ$ meet on the external $\angle A$-bisector.

---

We present three solutions.

\paragraph{First solution by complex numbers (Evan Chen, with Yang Liu).}
We use complex numbers with $D=x$, $E=y$, $F=z$.
\begin{center}
\begin{asy}
pair I = origin;
pair D = dir(110);
pair E = dir(210);
pair F = dir(330);
draw(D--E--F--cycle, blue);
filldraw(unitcircle, opacity(0.1)+lightcyan, blue);
pair A = 2*E*F/(E+F);
pair B = 2*F*D/(F+D);
pair C = 2*D*E/(D+E);
draw(A--B--C--cycle, blue);
pair R = -E*F/D;
pair P = -R+2*foot(I, A, R);
filldraw(circumcircle(P, C, E), opacity(0.1)+lightcyan, deepgreen);
filldraw(circumcircle(P, B, F), opacity(0.1)+lightcyan, deepgreen);
pair Q = -P+2*foot(P, circumcenter(C, P, E), circumcenter(B, P, F));
pair T = extension(P, Q, D, I);
draw(A--T, deepcyan);
draw(D--T--P, deepcyan);
draw(D--R, lightblue);
draw(A--P, lightblue);

dot("$I$", I, dir(45));
dot("$D$", D, dir(D));
dot("$E$", E, dir(E));
dot("$F$", F, dir(F));
dot("$A$", A, dir(A));
dot("$B$", B, dir(B));
dot("$C$", C, dir(C));
dot("$R$", R, dir(R));
dot("$P$", P, dir(200));
dot("$Q$", Q, dir(70));
dot("$T$", T, dir(T));

/* TSQ Source:

I = origin R45
D = dir 110
E = dir 210
F = dir 330
D--E--F--cycle blue
unitcircle 0.1 lightcyan / blue
A = 2*E*F/(E+F)
B = 2*F*D/(F+D)
C = 2*D*E/(D+E)
A--B--C--cycle blue
R = -E*F/D
P = -R+2*foot I A R R200
circumcircle P C E 0.1 lightcyan / deepgreen
circumcircle P B F 0.1 lightcyan / deepgreen
Q = -P+2*foot P circumcenter C P E circumcenter B P F R70
T = extension P Q D I
A--T deepcyan
D--T--P deepcyan
D--R lightblue
A--P lightblue

*/
\end{asy}
\end{center}


Then $A = \frac{2yz}{y+z}$,
$R = \frac{-yz}{x}$ and so
\[ P = \frac{A-R}{1-R\ol{A}}
  = \frac{\frac{2yz}{y+z} + \frac{yz}{x}}
  {1 + \frac{yz}{x} \cdot \frac{2}{y+z}}
  = \frac{yz(2x+y+z)}{2yz+x(y+z)}. \]
We now compute
\begin{align*}
  O_B &= \det \begin{bmatrix}
    P & P \ol P & 1 \\
    F & F \ol F & 1 \\
    B & B \ol B & 1
  \end{bmatrix}
  \div \det \begin{bmatrix}
    P & \ol P & 1 \\
    F & \ol F & 1 \\
    B & \ol B & 1
  \end{bmatrix}
  = \det \begin{bmatrix}
    P & 1 & 1 \\
    z & 1 & 1 \\
    \frac{2xz}{x+z} & \frac{4xz}{(x+z)^2} & 1
  \end{bmatrix}
  \div \det \begin{bmatrix}
    P & 1/P & 1 \\
    z & 1/z & 1 \\
    \frac{2xz}{x+z} & \frac{2}{x+z} & 1
  \end{bmatrix} \\
  &= \frac{1}{x+z} \det \begin{bmatrix}
    P & 0 & 1 \\
    z & 0 & 1 \\
    2xz(x+z) & -(x-z)^2 & (x+z)^2
  \end{bmatrix}
  \div \det \begin{bmatrix}
    P & 1/P & 1 \\
    z & 1/z & 1 \\
    2xz & 2 & x+z
  \end{bmatrix} \\
  &= \frac{(x-z)^2}{x+z} \cdot \frac{P-z}{(x+z)(P/z-z/P)+2z-2x + \frac{2xz}{P}-2P} \\
  &= \frac{(x-z)^2}{x+z} \cdot \frac{P-z}{
    (\frac xz-1) P - 2(x-z) + (xz-z^2) \frac 1P  } \\
  &= \frac{x-z}{x+z} \cdot \frac{P-z}{P/z + z/P - 2}
  = \frac{x-z}{x+z} \cdot \frac{P-z}{\frac{(P-z)^2}{Pz}}
  = \frac{x-z}{x+z} \cdot \frac{1}{\frac 1z - \frac 1P} \\
  &= \frac{x-z}{x+z} \cdot \frac{y(2x+y+z)}{y(2x+y+z) - (2yz+xy+xz)}
  = \frac{x-z}{x+z} \cdot \frac{yz(2x+y+z)}{xy+y^2-yz-xz} \\
  &= \frac{x-z}{x+z} \cdot \frac{yz(2x+y+z)}{(y-z)(x+y)}.
\end{align*}
Similarly
\[ O_C = \frac{x-y}{x+y} \cdot \frac{yz(2x+y+z)}{(z-y)(x+z)}. \]
Therefore, subtraction gives
\[ O_B-O_C
  =
  \frac{yz(2x+y+z)}{(x+y)(x+z)(y-z)}
  \left[ (x-z) + (x-y) \right]
  = \frac{yz(2x+y+z)(2x-y-z)}{(x+y)(x+z)(z-y)}.
\]
It remains to compute $T$.
Since $T \in \ol{ID}$ we have $t/x \in \RR$
so $\ol t = t/x^2$.
Also,
\begin{align*}
  \frac{t - \frac{2yz}{y+z}}{y+z} \in i \RR
  \implies 0 &= \frac{t-\frac{2yz}{y+z}}{y+z}
  + \frac{\frac{t}{x^2}-\frac{2}{y+z}}{\frac1y+\frac1z} \\
  &= \frac{1+\frac{yz}{x^2}}{y+z} t - \frac{2yz}{(y+z)^2} - \frac{2yz}{(y+z)^2} \\
  \implies t &= \frac{x^2}{x^2+yz} \cdot \frac{4yz}{y+z}
\end{align*}
Thus
\begin{align*}
  P-T &= \frac{yz(2x+y+z)}{2yz+x(y+z)} - \frac{4x^2yz}{(x^2+yz)(y+z)} \\
  &= yz \cdot \frac{(2x+y+z)(x^2+yz)(y+z) - 4x^2(2yz+xy+xz)}
    {(y+z)(x^2+yz)(2yz+xy+xz)} \\
  &= -yz \cdot \frac{(2x-y-z)(x^2y+x^2z+4xyz+y^2z+yz^2)}
    {(y+z)(x^2+yz)(2yz+xy+xz)}.
\end{align*}
This gives $\ol{PT} \perp \ol{O_B O_C}$ as needed.

\paragraph{Second solution by tethered moving points, with optimization (Evan Chen).}
Fix $\triangle DEF$ and $\omega$, with $B = \ol{DD} \cap \ol{FF}$
and $C = \ol{DD} \cap \ol{EE}$.
We consider a variable point $M$ on $\omega$
and let $X$, $Y$ be on $\ol{EF}$ with
$\ol{CY} \cap \parallel \ol{ME}$, $\ol{BX} \cap \parallel \ol{MF}$.
We define $W = \ol{CY} \cap \ol{BX}$.
Also, let line $MW$ meet $\omega$ again at $V$.

\begin{center}
\begin{asy}
pair D = dir(105);
pair E = dir(200);
pair F = dir(340);
pair M = dir(275);
pair N = extension(D, M, E, F);
pair Rp = -D;
pair B = 2*D*F/(D+F);
pair C = 2*D*E/(D+E);
pair X = extension(E, F, B, B+M-F);
pair Y = extension(E, F, C, C+M-E);
pair W = extension(B, X, C, Y);
pair V = extension(N, Rp, M, W);

filldraw(unitcircle, opacity(0.1)+lightred, red);
draw(D--E--F--cycle, red);
draw(E--C--D--B--F, orange);
draw(V--M);
draw(X--B, deepgreen);
draw(C--Y, deepgreen);
draw(E--M--F, heavygreen);
draw(D--M, lightred);
draw(V--M, blue);
filldraw(circumcircle(C, W, V), opacity(0.1)+lightcyan, blue);
filldraw(circumcircle(B, W, V), opacity(0.1)+lightcyan, blue);
draw(V--Rp, grey+dotted);

dot("$D$", D, dir(D));
dot("$E$", E, dir(E));
dot("$F$", F, dir(F));
dot("$M$", M, dir(M));
dot("$N$", N, dir(N));
dot("$R'$", Rp, dir(Rp));
dot("$B$", B, dir(B));
dot("$C$", C, dir(C));
dot("$X$", X, dir(270));
dot("$Y$", Y, dir(270));
dot("$W$", W, dir(W));
dot("$V$", V, dir(200));

/* TSQ Source:

D = dir 105
E = dir 200
F = dir 340
M = dir 275
N = extension D M E F
R' = -D
B = 2*D*F/(D+F)
C = 2*D*E/(D+E)
X = extension E F B B+M-F R270
Y = extension E F C C+M-E R270
W = extension B X C Y
V = extension N Rp M W R200

unitcircle 0.1 lightred / red
D--E--F--cycle red
E--C--D--B--F orange
V--M
X--B deepgreen
C--Y deepgreen
E--M--F heavygreen
D--M lightred
V--M blue
circumcircle C W V 0.1 lightcyan / blue
circumcircle B W V 0.1 lightcyan / blue
V--Rp grey dotted

*/
\end{asy}
\begin{asy}
pair D = dir(115);
pair E = dir(210);
pair F = dir(330);
pair N = midpoint(E--F);
pair M = -D+2*foot(origin, D, N);
pair Rp = -D;
pair B = 2*D*F/(D+F);
pair C = 2*D*E/(D+E);
pair X = extension(E, F, B, B+M-F);
pair Y = extension(E, F, C, C+M-E);
pair Q = extension(B, X, C, Y);
pair P = extension(N, Rp, M, Q);

pair R = E*F/Rp;
pair Mp = E*F/M;
pair A = 2*E*F/(E+F);
pair Z = extension(P, M, D, Rp);

filldraw(unitcircle, opacity(0.1)+lightred, red);
draw(D--E--F--cycle, red);
draw(E--A--F, orange);
draw(P--M);
draw(D--M, lightred);
// E--D--C orange
draw(P--Rp, deepgreen);

draw(A--Z, lightblue);
draw(P--Z--D, blue);
draw(P--A--D, deepcyan);

dot("$D$", D, dir(D));
dot("$E$", E, dir(E));
dot("$F$", F, dir(F));
dot("$N$", N, dir(N));
dot("$M$", M, dir(M));
dot("$R'$", Rp, dir(Rp));
dot("$Q$/$W$", Q, dir(Q));
dot("$P$/$V$", P, dir(200));
dot("$R$", R, dir(R));
dot("$M'$", Mp, dir(Mp));
dot("$A$", A, dir(A));

/* TSQ Source:

D = dir 115
E = dir 210
F = dir 330
N = midpoint E--F
M = -D+2*foot origin D N
R' = -D
B := 2*D*F/(D+F)
C := 2*D*E/(D+E)
X := extension E F B B+M-F R270
Y := extension E F C C+M-E R270
Q = extension B X C Y
P = extension N Rp M Q R200

R = E*F/Rp
M' = E*F/M
A = 2*E*F/(E+F)
Z := extension P M D Rp

unitcircle 0.1 lightred / red
D--E--F--cycle red
E--A--F orange
P--M
D--M lightred
// E--D--C orange
P--Rp deepgreen

A--Z lightblue
P--Z--D blue
P--A--D deepcyan

*/
\end{asy}
\end{center}

\begin{claim*}
  [Angle chasing]
  Pentagons $CVWXE$ and $BVWYF$ are cyclic.
\end{claim*}
\begin{proof}
  By $\dang EVW = \dang EVM = \dang EFM = \dang CEM = \dang ECW$
  and $\dang EXW = \dang EFM = \dang CEM = \dang ECW$.
\end{proof}

Let $N = \ol{DM} \cap \ol{EF}$ and $R'$ be the $D$-antipode on $\omega$.
\begin{claim*}
  [Black magic]
  The points $V$, $N$, $R'$ are collinear.
\end{claim*}
\begin{proof}
  We use tethered moving points with $\triangle DEF$ fixed.

  Obviously the map $\omega \mapsto \ol{EF} \mapsto \omega$
  by $M \mapsto N \mapsto \ol{R'N} \cap \omega$ is projective.
  Also, the map $\omega \mapsto \ol{EF} \mapsto \omega$
  by $M \mapsto X \mapsto V$ is also projective
  (the first by projection to the line at infinity at back;
  the second say by inversion at $E$).

  So it suffices to check for three points.
  When $M=E$ we get $N=E$ so $\ol{R'N} \cap \omega = E$,
  while $W=E$ and thus $V=E$.
  The case $M=F$ is similar.
  Finally, if $M = R'$, then $W$ is the center of $\omega$
  and so $V = \ol{R'N} \cap \ol{EF} = D$.
\end{proof}


We now address the original problem by specializing $M$:
choose it so that $N$ is the midpoint of $\ol{EF}$.
Let $M' = \ol{DA} \cap (DEF)$.
\begin{claim*}
  After this specialization, $V=P$ and $W=Q$.
\end{claim*}
\begin{proof}
  Thus $\ol{RR'}$ and $\ol{MM'}$ are parallel to $\ol{EF}$.
  From  $(EF;PR) = -1 = (EF;N\infty) \overset{R'}{=} (EF;NV)$,
  we derive that $P=V$ and $Q=R$, proving (i).
\end{proof}
Finally, the concurrence requested follows
by Pascal theorem on $M'MDR'PR$.

\paragraph{Third solution by power of a point linearity (Luke Robitaille).}

Let us define
\[ f(\bullet) = \opname{Pow}(\bullet, (CPE))
  - \opname{Pow}(\bullet, (BPF)) \]
which is a linear function from the plane to $\RR$.

Define $W = \ol{BA} \cap \ol{PE}$,
$V = \ol{AC} \cap \ol{PF}$.
Also, let $W_1 = \ol{ER} \cap \ol{AB}$,
$V_1 = \ol{FR} \cap \ol{AC}$.
Note that
\[ -1 = (PR;EF) \overset{E}{=} (WA; W_1F) \]
and similarly $(VA; V_1E) = -1$.

\begin{claim*}
  We have
  \begin{align*}
    f(F) &= \frac{|EF| \cdot (s-c) \sin C/2}{\sin B/2} \\
    f(E) &= -\frac{|EF| \cdot (s-b) \sin B/2}{\sin C/2}.
  \end{align*}
\end{claim*}
\begin{proof}
  We have
  \[ f(W) = WF^2 - WB \cdot WF = WF \cdot BF \]
  where lengths are directed.
  Next,
  \begin{align*}
    f(F) &= \frac{AF \cdot f(W) + FW \cdot f(A)}{AW} \\
    &= \frac{AF \cdot WF \cdot BF + FW \cdot
      \left( AE \cdot AC - AF \cdot AB \right)}{AW} \\
    &= \frac{WF(AF \cdot BF + AF \cdot AB) + FW \cdot AE \cdot AC}{AW} \\
    &= \frac{WF \cdot AF^2 - WF \cdot AE \cdot AC}{AW}
      = \frac{WF}{AW} \cdot (AE^2 - AE \cdot AC) \\
    &= \frac{WF}{AW} \cdot AE \cdot CE
      = -\frac{W_1F}{AW_1} \cdot AE \cdot CE.
  \end{align*}
  Since $\triangle DEF$ is acute,
  the point $R$ lies inside $\triangle AEF$.
  Thus $W_1$ lies inside segment $\ol{AF}$
  and the ratio $\frac{W_1F}{AW_1}$ is positive.
  We now determine its value: by the ratio lemma
  \begin{align*}
    \frac{|W_1F|}{|AW_1|}
    &= \frac{|EF| \sin \angle W_1 E F}{|AE| \sin \angle A E W_1} \\
    &= \frac{|EF| \sin \angle REF}{|AE| \sin \angle AER} \\
    &= \frac{|EF| \sin \angle RDF}{|AE| \sin \angle EDR} \\
    &= \frac{|EF| \sin C/2}{|AE| \sin B/2}.
  \end{align*}
  Also, we have $AE \cdot CE < 0$ since $E$ lies inside $\ol{AC}$.
  Hence
  \[ f(F) = -\frac{|EF| \sin C/2}{|AE| \sin B/2}.
    \cdot AE \cdot CE
    = |EF| \cdot \frac{|CE| \sin B/2}{\sin C/2}
    = |EF| \cdot \frac{(s-c) \sin B/2}{\sin C/2}. \]
  The calculation for $f(E)$ is similar,
  (noting the sign flips since $f$ is anti-symmetric
  in terms of $B$ and $C$).
\end{proof}

Let $Z \in \ol{DI}$ with $\angle ZAI = 90\dg$
be the point requested in the problem now.
Our goal is to show $f(Z) = 0$.
We assume WLOG that $AB < AC$, so $\frac{ZA}{EF} > 0$.
Then
\begin{align*}
  |ZA| &= |AI| \cdot \tan \angle AIZ \\
  &= |AI| \cdot \tan \angle(\ol{AI}, \ol{DI}) \\
  &= \frac{s-a}{\cos A/2} \cdot \tan (\ol{BC}, \ol{EF}) \\
  &= \frac{s-a}{\cos A/2} \tan (B/2-C/2).
\end{align*}

To this end we compute
\begin{align*}
  f(Z) &= f(A) + \left[ f(Z) - f(A) \right]
    = f(A) + \frac{ZA}{EF} \left[ f(E)-f(F) \right] \\
  &= f(A) - \frac{ZA}{EF}
  \left[ \frac{|EF| \cdot (s-b) \sin B/2}{\sin C/2}
    + \frac{|EF| \cdot (s-c) \sin C/2}{\sin B/2} \right] \\
  &= f(A) - |ZA| \left[ \frac{(s-b) \sin B/2}{\sin C/2}
    + \frac{(s-c) \sin C/2}{\sin B/2} \right] \\
  &= \left[ b(s-a) - c(s-a) \right]
    - |ZA| \left[ \frac{(s-b) \sin B/2}{\sin C/2}
    + \frac{(s-c) \sin C/2}{\sin B/2} \right] \\
  &= (b-c)(s-a) - \frac{s-a}{\cos A/2} \tan (B/2-C/2)
    \left[ \frac{(s-b) \sin B/2}{\sin C/2}
    + \frac{(s-c) \sin C/2}{\sin B/2} \right].
\end{align*}
Dividing out,
\begin{align*}
  \frac{f(Z)}{s-a}
  &= (b-c) - \frac{1}{\cos A/2} \tan (B/2-C/2)
    \left[ \frac{r \cos B/2}{\sin C/2}
      + \frac{r \cos C/2}{\sin B/2} \right] \\
  &= (b-c) - \frac{r \tan(B/2-C/2)}{\cos A/2}
    \cdot \frac{\cos B/2 \sin B/2 + \cos C/2 \sin C/2}%
    {\sin C/2 \sin B/2} \\
  &= (b-c) - \frac{r \tan(B/2-C/2)}{\cos A/2}
    \cdot \frac{\sin B + \sin C}%
    {2\sin C/2 \sin B/2} \\
  &= (b-c) - \frac{r \tan(B/2-C/2)}{\cos A/2}
    \cdot \frac{\sin(B/2+C/2)\cos(B/2-C/2)}
    {\sin C/2 \sin B/2} \\
  &= (b-c) - r \frac{\sin(B/2-C/2)}{\sin B/2 \sin C/2} \\
  &= (b-c) - r(\cot C/2 - \cot B/2)
    = (b-c) - \left( (s-c) - (s-b) \right) = 0.
\end{align*}
\paragraph{Fourth solution by incircle inversion (USA IMO live stream, led by Andrew Gu).}
Let $T$ be the intersection of line $DI$ and the external $\angle A$-bisector.
Also, let $G$ be the antipode of $D$ on $\omega$.

We perform inversion around $\omega$, using $\bullet^\ast$ for the inverse.
Then $\triangle A^\ast B^\ast C^\ast$
is the medial triangle of $\triangle DEF$,
and $T^\ast$ is the foot from $A^\ast$ on to $\ol{DI}$.
If we denote $Q^\ast$ as the second intersection
of $(PC^\ast E)$ and $(PB^\ast F)$,
then the goal it show that $Q^\ast$ lies on $(PIT^\ast)$.

\begin{center}
\begin{asy}
pair D = dir(115);
pair E = dir(210);
pair F = dir(330);
pair As = midpoint(E--F);
pair G = -D;
pair I = origin;
pair A = 2*E*F/(E+F);

pair R = -E*F/D;
pair P = -R+2*foot(I, R, A);

pair T = extension(D, I, A, E-F+A);

pair Bs = midpoint(D--F);
pair Cs = midpoint(D--E);
pair M = midpoint(As--D);
pair Qs = OP(circumcircle(Cs, P, E), circumcircle(Bs, P, F));

filldraw(unitcircle, opacity(0.1)+lightcyan, deepcyan);
draw(D--E--F--cycle, deepcyan);
draw(E--A--F, lightblue);
draw(P--G, deepgreen);
draw(D--T, lightblue);
draw(A--T, lightblue);
draw(A--P, lightblue);
pair Ts = foot(As, D, T);
draw(As--Ts, lightred);
draw(Qs--Bs, lightred+dashed);
draw(D--As, lightblue);

filldraw(circumcircle(P, Cs, E), opacity(0.05)+yellow, lightred);
filldraw(circumcircle(P, Bs, F), opacity(0.05)+yellow, lightred);

draw(CP(M, D), lightgreen);
draw(M--I, deepgreen);

clip(box((-1.8,-2.2), (2,2)));
filldraw(circumcircle(P, I, Ts), opacity(0.1)+orange, red+dashed);

dot("$D$", D, dir(D));
dot("$E$", E, dir(E));
dot("$F$", F, dir(F));
dot("$A^\ast$", As, dir(225));
dot("$G$", G, dir(245));
dot("$I$", I, dir(45));
dot("$A$", A, dir(A));
dot("$R$", R, dir(R));
dot("$P$", P, dir(P));
dot("$T$", T, dir(T));
dot("$B^\ast$", Bs, dir(Bs));
dot("$C^\ast$", Cs, dir(45));
dot("$M$", M, dir(225));
dot("$Q^\ast$", Qs, dir(Qs));
dot("$T^\ast$", Ts, dir(45));

/* TSQ Source:

D = dir 115
E = dir 210
F = dir 330
A* = midpoint E--F R225
G = -D R245
I = origin R45
A = 2*E*F/(E+F)

R = -E*F/D
P = -R+2*foot I R A

T = extension D I A E-F+A

B* = midpoint D--F
C* = midpoint D--E R45
M = midpoint As--D R225
Q* = OP circumcircle Cs P E circumcircle Bs P F

unitcircle 0.1 lightcyan / deepcyan
D--E--F--cycle deepcyan
E--A--F lightblue
P--G deepgreen
D--T lightblue
A--T lightblue
A--P lightblue
T* = foot As D T R45
As--Ts lightred
Qs--Bs lightred dashed
D--As lightblue

circumcircle P Cs E 0.05 yellow / lightred
circumcircle P Bs F 0.05 yellow / lightred

CP M D lightgreen
M--I deepgreen

!clip(box((-1.8,-2.2), (2,2)));
circumcircle P I Ts 0.1 orange / red dashed

*/
\end{asy}
\end{center}

\begin{claim*}
  Points $Q^\ast$, $B^\ast$, $C^\ast$ are collinear.
\end{claim*}
\begin{proof}
  $\dang PQ^\ast C^\ast = \dang PEC^\ast = \dang PED = \dang PFD
  = \dang PFB^\ast = \dang PQ^\ast B^\ast$.
\end{proof}

\begin{claim*}
  [cf Brazil 2011/5]
  Points $P$, $A^\ast$, $G$ are collinear.
\end{claim*}
\begin{proof}
  Project harmonic quadrilateral $PERF$ through $G$,
  noting $\ol{GR} \parallel \ol{EF}$.
\end{proof}

Denote by $M$ the center of parallelogram $DC^\ast A^\ast B^\ast$.
Note that it is the center of the circle with
diameter $\ol{DA^\ast}$, which passes through $P$ and $T^\ast$.
Also, $\ol{MI} \parallel \ol{PA^\ast G}$.

\begin{claim*}
  Points $P$, $M$, $I$, $T^\ast$ are cyclic.
\end{claim*}
\begin{proof}
  $\dang IT^\ast P = \dang D T^\ast P = \dang DA^\ast P
    = \dang M A^\ast P = \dang A^\ast P M  = \dang IMP$.
\end{proof}

\begin{claim*}
  Points $P$, $M$, $I$, $Q^\ast$ are cyclic.
\end{claim*}
\begin{proof}
  $\dang MQ^\ast P = \dang C^\ast Q^\ast P = \dang C^\ast E P
  = \dang D E P = \dang D G P = \dang GPI = \dang MIP$.
\end{proof}

\paragraph{Fifth solution by double inversion (Brandon Wang, Luke Robitaille, Michael Ren, Evan Chen).}
We outline one final approach.
After inverting about $\omega$ as in the previous approach,
we then apply another inversion around $P$.
Dropping the apostrophes/stars/etc now one can check
that the problem we arrive at becomes the following.
\begin{proposition*}
  [Doubly inverted problem]
  In $\triangle PEF$, the $P$-symmedian meets $\ol{EF}$
  and $(PEF)$ at $K$, $L$.
  Let $D \in \ol{EF}$ with $\angle DPK = 90\dg$,
  and let $T$ be the foot from $K$ to $\ol{DL}$.
  Denote by $I$ the reflection of $P$ about $\ol{EF}$.
  Finally, let $PDNE$ and $PDMF$ be cyclic harmonic quadrilaterals.
  Then lines $EN$, $MF$, $TI$, are concurrent.
\end{proposition*}

The proof proceeds in three steps.
Suppose the line through $L$ perpendicular to $\ol{EF}$
meets $\ol{EF}$ at $W$ and $(PEF)$ at $Z$.
\begin{center}
\begin{asy}
size(10cm);
pair P = dir(190);
pair E = dir(220);
pair F = dir(-40);
pair T = 2*E*F/(E+F);
pair K = extension(E, F, P, T);
pair L = -P+2*foot(origin, P, K);
pair W = foot(L, E, F);
pair Z = -L+2*foot(origin, W, L);
pair D = extension(E, F, P, P+dir(90)*(P-K));
pair T = foot(K, D, L);
pair I = -P+2*foot(P, E, F);
pair H = extension(P, Z, W, I);
pair N = -E+2*foot(circumcenter(P, D, E), H, E);
pair M = -F+2*foot(circumcenter(P, D, F), H, F);

filldraw(unitcircle, opacity(0.1)+lightcyan, blue);
draw(P--E--F--cycle, blue);
draw(D--E, blue);
draw(D--P--L--cycle, deepcyan);
draw(T--K, deepcyan);

draw(E--Z, orange);
draw(F--Z, orange);
draw(H--Z, orange);
draw(L--Z, orange);

filldraw(circumcircle(P, D, E), opacity(0.1)+lightgreen, deepgreen);
filldraw(circumcircle(P, D, F), opacity(0.1)+lightgreen, deepgreen);
draw(E--H, grey);
draw(W--H, grey);
draw(F--M, grey);

clip(box((-3,-2.3), (1.5,1.5)));

dot("$P$", P, dir(110));
dot("$E$", E, dir(135));
dot("$F$", F, dir(350));
dot("$K$", K, dir(70));
dot("$L$", L, dir(270));
dot("$W$", W, dir(45));
dot("$Z$", Z, dir(Z));
dot("$D$", D, dir(D));
dot("$T$", T, dir(270));
dot("$I$", I, dir(315));
dot(H);
dot("$N$", N, dir(100));
dot("$M$", M, dir(M));

/* TSQ Source:

P = dir 190 R110
E = dir 220 R135
F = dir -40 R350
T := 2*E*F/(E+F)
K = extension E F P T R70
L = -P+2*foot origin P K R270
W = foot L E F R45
Z = -L+2*foot origin W L
D = extension E F P P+dir(90)*(P-K)
T = foot K D L R270
I = -P+2*foot P E F R315
H .= extension P Z W I
N = -E+2*foot circumcenter P D E H E R100
M = -F+2*foot circumcenter P D F H F

unitcircle 0.1 lightcyan / blue
P--E--F--cycle blue
D--E blue
D--P--L--cycle deepcyan
T--K deepcyan

E--Z orange
F--Z orange
H--Z orange
L--Z orange

circumcircle P D E 0.1 lightgreen / deepgreen
circumcircle P D F 0.1 lightgreen / deepgreen
E--H grey
W--H grey
F--M grey

! clip(box((-3,-2.3), (1.5,1.5)));

*/
\end{asy}
\end{center}

\begin{enumerate}
  \ii Since $\dang ZEP = \dang WLP = \dang WDP$,
  it follows $\ol{ZE}$ is tangent to $(PDNE)$. \\
  Similarly, $\ol{ZF}$ is tangent to $(PDMF)$.
  \ii $\triangle WTP$ is the orthic triangle of $\triangle DKL$,
  so $\ol{WD}$ bisects $\angle PWT$ and $\ol{WTI}$ collinear.
  \ii $-1 = E(PN;DZ) = F(PM;DZ) = W(PI;DZ)$, so
  $\ol{EN}$, $\ol{FM}$, $\ol{WI}$ meet on $\ol{PZ}$.
\end{enumerate}
