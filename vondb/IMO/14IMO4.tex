desc: Trivial bary but also good harmonic
source: IMO 2014/4
tags: [2016-08, harmonic, bary, nice, instructive, aleph, well]
author: Giorgi Arabidze (GEO)
hardness: 5
url: https://aops.com/community/p3543136

---

Let $P$ and $Q$ be on segment $BC$ of an acute triangle $ABC$
such that $\angle PAB=\angle BCA$ and $\angle CAQ=\angle ABC$.
Let $M$ and $N$ be points on $\ol{AP}$ and $\ol{AQ}$,
respectively, such that $P$ is the midpoint of $\ol{AM}$
and $Q$ is the midpoint of $\ol{AN}$.
Prove that $\ol{BM}$ and $\ol{CN}$ meet on the
circumcircle of $\triangle ABC$.

---

We give three solutions.

\paragraph{First solution by harmonic bundles.}
Let $\ol{BM}$ intersect the circumcircle again at $X$.

\begin{center}
\begin{asy}
pair A = dir(70);
pair B = dir(190);
pair C = dir(350);
filldraw(unitcircle, opacity(0.1)+palecyan, lightblue);
filldraw(A--B--C--cycle, opacity(0.1)+heavycyan, blue);

pair T = 2*B*C/(B+C);
pair P = extension(A, B+A-T, B, C);
pair Q = extension(A, C+A-T, B, C);
pair M = 2*P-A;
pair N = 2*Q-A;
pair X = extension(B, M, C, N);
draw(B--M, lightblue);
draw(C--N, lightblue);

draw(A--M, lightred);
draw(A--N, orange);
real r = 0.4;
draw((B+r*dir(B-T))--(B+r*dir(T-B)), lightred);
draw((C+r*dir(C-T))--(C+r*dir(T-C)), orange);

dot("$A$", A, dir(A));
dot("$B$", B, dir(B));
dot("$C$", C, dir(C));
dot("$P$", P, dir(225));
dot("$Q$", Q, dir(225));
dot("$M$", M, dir(M));
dot("$N$", N, dir(N));
dot("$X$", X, dir(280));

/* TSQ Source:

A = dir 70
B = dir 190
C = dir 350
unitcircle 0.1 palecyan / lightblue
A--B--C--cycle 0.1 heavycyan / blue

T := 2*B*C/(B+C)
P = extension A B+A-T B C R225
Q = extension A C+A-T B C R225
M = 2*P-A
N = 2*Q-A
X = extension B M C N R280
B--M lightblue
C--N lightblue

A--M lightred
A--N orange
! real r = 0.4;
(B+r*dir(B-T))--(B+r*dir(T-B)) lightred
(C+r*dir(C-T))--(C+r*dir(T-C)) orange

*/
\end{asy}
\end{center}

The angle conditions imply that the tangent to $(ABC)$ at $B$
is parallel to $\ol{AP}$.
Let $\infty$ be the point at infinity along line $AP$.
Then \[ -1 = (AM;P\infty) \overset{B}{=} (AX;BC). \]
Similarly, if $\ol{CN}$ meets the circumcircle at $Y$
then $(AY;BC) = -1$ as well.
Hence $X=Y$, which implies the problem.

\paragraph{Second solution by similar triangles.}
Once one observes $\triangle CAQ \sim \triangle CBA$,
one can construct $D$ the reflection of $B$ across $A$,
so that $\triangle CAN \sim \triangle CBD$.
Similarly, letting $E$ be the reflection of $C$ across $A$,
we get $\triangle BAP \sim \triangle BCA
\implies \triangle BAM \sim \triangle BCE$.
Now to show $\angle ABM + \angle ACN = 180\dg$
it suffices to show $\angle EBC + \angle BCD = 180\dg$,
which follows since $BCDE$ is a parallelogram.

\paragraph{Third solution by barycentric coordinates.}
Since $PB = c^2/a$ we have
\[ P = (0 : a^2-c^2 : c^2) \]
so the reflection $\vec M = 2\vec P - \vec A$ has coordinates
\[ M = (-a^2 : 2(a^2-c^2) : 2c^2). \]

Similarly $N = (-a^2 : 2b^2 : 2(b^2-a^2))$. Thus
\[ \ol{BM} \cap \ol{CN} = (-a^2 : 2b^2 : 2c^2) \]
which clearly lies on the circumcircle,
and is in fact the point identified in the first solution.

---

We give a walkthrough for the harmonic bundles solution.

\begin{walk}
  \ii Show that the tangent to $B$
  is parallel to $\ol{APM}$.
  \ii Find a natural harmonic bundle using the answer to (a).
\end{walk}
Let $\ol{BM}$ intersect the circumcircle again at $X$.
\begin{walk}[resume]
  \ii Projecting the answer to (b) onto the circumcircle
  gives a harmonic quadrilateral. Which one?
  \ii Deduce that $\ol{CN}$ passes through $X$ as well.
\end{walk}
