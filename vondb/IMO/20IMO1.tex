desc: Angles in 1:2:3 ratio
hardness: 10
source: IMO 2020/1
tags: [2020-09, anglechase, troll, aleph]
author: Dominik Burek (POL)
url: https://aops.com/community/p17821635

---

Consider the convex quadrilateral $ABCD$.
The point $P$ is in the interior of $ABCD$.
The following ratio equalities hold:
\[\angle PAD:\angle PBA:\angle DPA
  = 1:2:3
  = \angle CBP:\angle BAP:\angle BPC.\]
Prove that the following three lines meet in a point:
the internal bisectors of angles $\angle ADP$ and $\angle PCB$
and the perpendicular bisector of segment $AB$.

---

Let $O$ denote the circumcenter of $\triangle PAB$.
We claim it is the desired concurrency point.
\begin{center}
\begin{asy}
pair O = origin;
pair P = dir(118);
pair A = dir(190);
pair B = dir(350);
pair C = extension(P/dir(64), B, P, B*dir(192));
pair D = extension(P*dir(36), A, P, A/dir(108));

filldraw(unitcircle, opacity(0.1)+lightcyan, blue+dotted);
filldraw(A--B--C--D--cycle, opacity(0.1)+yellow, grey);
filldraw(P--A--B--cycle, opacity(0.1)+lightcyan, blue);
draw(circumcircle(P, O, B), lightred);
draw(circumcircle(P, O, D), lightred);
draw(D--P--C, grey);

draw(D--O--C, dashed+red);

dot("$O$", O, dir(270));
dot("$P$", P, dir(P));
dot("$A$", A, dir(A));
dot("$B$", B, dir(B));
dot("$C$", C, dir(C));
dot("$D$", D, dir(D));

/* TSQ Source:

O = origin R270
P = dir 118
A = dir 190
B = dir 350
C = extension P/dir(64) B P B*dir(192)
D = extension P*dir(36) A P A/dir(108)

unitcircle 0.1 lightcyan / blue dotted
A--B--C--D--cycle 0.1 yellow / grey
P--A--B--cycle 0.1 lightcyan / blue
circumcircle P O B lightred
circumcircle P O D lightred
D--P--C grey

D--O--C dashed red

*/
\end{asy}
\end{center}
Indeed, $O$ obviously lies on the perpendicular bisector of $AB$.
Now
\begin{align*}
  \dang BCP &= \dang CBP + \dang BPC \\
  &= 2\dang BAP = \dang BOP
\end{align*}
it follows $BOPC$ are cyclic.
And since $OP = OB$, it follows that $O$ is on
the bisector of $\angle PCB$, as needed.

\begin{remark*}
  The angle equality is only used insomuch $\angle BAP$
  is the average of $\angle CBP$ and $\angle BPC$,
  i.e.\ only $\frac{1+3}{2} = 2$ matters.
\end{remark*}
