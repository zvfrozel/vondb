desc: Cyclic fractions, just $\nu_p$ it
author: Bayarmagnai Gombodorj (MNG)
source: IMO 2018/5
tags: [2018-07, vp, rushdown, reliable, instructive, good, primes, bet]
hardness: 20
url: https://aops.com/community/p10632353

---

Let $a_1$, $a_2$, \dots\ be an infinite sequence of positive integers,
and $N$ a positive integer.
Suppose that for all integers $n \ge N$, the expression
\[ \frac{a_1}{a_2} + \frac{a_2}{a_3} + \dots
  + \frac{a_{n-1}}{a_n} + \frac{a_n}{a_1} \]
is an integer.
Prove that $(a_n)$ is eventually constant.

---

The condition implies that the difference
\[ S(n) = \frac{a_{n+1} - a_n}{a_1} + \frac{a_n}{a_{n+1}} \]
is an integer for all $n > N$.
We proceed by $p$-adic valuation only henceforth;
fix a prime $p$.
Then analyzing the $\nu_p$, we immediately get that for $n > N$:
\begin{itemize}
\ii If $\nu_p(a_n) < \nu_p(a_{n+1})$, then $\nu_p(a_{n+1}) = \nu_p(a_1)$.
\ii If $\nu_p(a_n) = \nu_p(a_{n+1})$, no conclusion.
\ii If $\nu_p(a_n) > \nu_p(a_{n+1})$,
then $\nu_p(a_{n+1}) \ge \nu_p(a_1)$.
\end{itemize}
In other words:
\begin{claim*}
Let $p$ be a prime.  Consider the sequence
$\nu_p(a_{N+1})$, $\nu_p(a_{N+2})$, \dots.
Then either:
\begin{itemize}
  \ii We have $\nu_p(a_{N+1}) \ge \nu_p(a_{N+2}) \ge \dots$
  and so on, i.e.\ the sequence is weakly decreasing immediately; or
  \ii For some index $K > N$ we have
  $\nu_p(a_K) < \nu_p(a_{K+1}) = \nu_p(a_{K+2}) = \dots = \nu_p(a_1)$,
  i.e.\ the sequence ``jumps'' to $\nu_p(a_1)$
  at some point and then stays there forever after.
  Note this requires $\nu_p(a_1) > 0$.
\end{itemize}
\end{claim*}

A cartoon of the situation is drawn below.
\begin{center}
  \begin{asy}
    label("$\nu_p(a_1)$", (0,3), dir(180), deepcyan);
    draw( (0,3)--(11,3), blue+dotted );
    draw( (0,9)--(0,-1)--(12,-1), black, Arrows );
    dot( (1,6), red );
    dot( (2,5), red );
    dot( (3,5), red );
    dot( (4,5), red );
    dot( (5,4), red );
    dot( (6,4), red );
    dot( (7,4), red );
    dot( (8,3), red );
    dot( (1,9), darkred );
    dot( (2,7), darkred );
    dot( (3,6), darkred );
    dot( (4,6), darkred );
    dot( (5,6), darkred );
    dot( (6,6), darkred );
    dot( (7,6), darkred );
    dot( (8,6), darkred );
    dot( (9,6), darkred );
    draw( (1,6)--(2,5)--(4,5)--(5,4)--(7,4)--(8,3), red );
    draw( (1,9)--(2,7)--(3,6)--(4,6)--(9,6), darkred );
    dot( (1,1), brown );
    dot( (2,1), brown );
    dot( (3,1), brown );
    dot( (4,1), brown );
    dot( (5,1), brown );
    dot( (6,3), brown );
    dot( (6,3), brown );
    draw( (1,1)--(5,1)--(6,3), brown );
    dot( (1,0), orange );
    dot( (2,0), orange );
    dot( (3,0), orange );
    dot( (4,0), orange );
    dot( (5,0), orange );
    dot( (6,0), orange );
    dot( (7,0), orange );
    dot( (8,0), orange );
    dot( (9,0), orange );
    draw( (1,0)--(9,0), orange );
    draw( (1,-0.8)--(1,-1.2) );
    draw("$n > N$", (1,-1.2), dir(-90) );
  \end{asy}
\end{center}

As only finitely many primes $p$ divide $a_1$,
after some time $\nu_p(a_n)$ is fixed for all such $p \mid a_1$.
Afterwards, the sequence satisfies $a_{n+1} \mid a_n$ for each $n$,
and thus must be eventually constant.

\begin{remark*}
  This solution is almost completely $p$-adic,
  in the sense that I think a similar result
  holds if one replaces $a_n \in \ZZ$
  by $a_n \in \ZZ_p$ for any particular prime $p$.
  In other words, the primes almost do not talk to each other.

  There is one caveat: if $x_n$ is an integer sequence
  such that $\nu_p(x_n)$ is eventually constant for each prime
  then $x_n$ may not be constant.
  For example, take $x_n$ to be the $n$th prime!
  That's why in the first claim (applied to co-finitely many of the primes),
  we need the stronger non-decreasing condition,
  rather than just eventually constant.
\end{remark*}

\begin{remark*}
  An alternative approach is to show that, when the fractions $a_n / a_1$
  is written in simplest form for $n = N+1, N+2, \dots$,
  the numerator and denominator are both weakly decreasing.
  Hence it must eventually be constant; in which case it equals $\frac11$.
\end{remark*}

---


\begin{claim*}
  If $p \nmid a_1$, then $\nu_p(a_{n+1}) \le \nu_p(a_n)$ for $n \ge N$.
\end{claim*}
\begin{proof}
  The first two terms of $S(n)$ have nonnegative $\nu_p$,
  so we need $\nu_p(\frac{a_n}{a_{n+1}}) \ge 0$.
\end{proof}

\begin{claim*}
  If $p \mid a_1$, then $\nu_p(a_n)$ is eventually constant.
\end{claim*}
\begin{proof}
  By hypothesis $\nu_p(a_1) > 0$.
  We consider two cases.
  \begin{itemize}
    \ii First assume $\nu_p(a_k) \ge \nu_p(a_1)$ for some $k > N$.
    We claim that for any $n \ge k$ we have:
    \[ \nu_p(a_1) \le \nu_p(a_{n+1}) \le \nu_p(a_n). \]
    This is just by induction on $n$;
    from $\nu(\frac{a_n}{a_1}) \ge 0$, we have
    \[ \nu_p\left( \frac{a_{n+1}}{a_1} + \frac{a_n}{a_{n+1}} \right) \ge 0 \]
    which implies the displayed inequality
    (since otherwise exactly one term of $S(n)$ has nonnegative $\nu_p$).
    Thus once we reach this case, $\nu_p(a_n)$ is monotic but bounded below by
    $\nu_p(a_1)$, and so it is eventually constant.

    \ii Now assume $\nu_p(a_k) < \nu_p(a_1)$ for every $k > N$.
    Take any $n > N$ then.
    We have $\nu_p\left(\frac{a_{n+1}}{a_1}\right) < 0$,
    and also $\nu_p\left(\frac{a_n}{a_1}\right) < 0$,
    so among the three terms of $S(n)$,
    two must have equal $p$-adic valuation.
    We consider all three possibilities:
    \begin{align*}
      \nu_p\left(\frac{a_{n+1}}{a_1}\right) = \nu_p\left(\frac{a_n}{a_1}\right)
      &\implies \boxed{\nu_p(a_{n+1}) = \nu_p(a_{n})} \\
      \nu_p\left(\frac{a_{n+1}}{a_1}\right) = \nu_p\left(\frac{a_n}{a_{n+1}}\right)
      &\implies \boxed{\nu_p(a_{n+1}) = \frac{\nu_p(a_n) + \nu_p(a_1)}{2}} \\
      \nu_p\left(\frac{a_{n}}{a_1}\right) = \nu_p\left(\frac{a_n}{a_{n+1}}\right)
      &\implies \nu_p(a_{n+1}) = \nu_p(a_1),\text{ but this is impossible}.
    \end{align*}
    Thus, $\nu_p(a_{n+1}) \ge \nu_p(a_n)$
    and $\nu_p(a_n)$ is bounded above by $\nu_p(a_1)$.
    So in this case we must also stabilize. \qedhere
  \end{itemize}
\end{proof}

Since the latter claim is applied only to finitely many primes,
after some time $\nu_p(a_n)$ is fixed for all $p \mid a_1$.
Afterwards, the sequence satisfies $a_{n+1} \mid a_n$ for each $n$,
and thus must be eventually constant.
