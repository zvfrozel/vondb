author: Silouanos Brazitikos, Vangelis Psyxas, Michael Sarantis (HEL)
desc: Easy perpendicular bisector geometry
source: IMO 2018/1
tags: [2019-06, manysolutions, nice, anglechase, trig, complex, aleph]
hardness: 10
url: https://aops.com/community/p10626500

---

Let $\Gamma$ be the circumcircle of acute triangle $ABC$.
Points $D$ and $E$ lie on segments $AB$ and $AC$,
respectively, such that $AD = AE$.
The perpendicular bisectors of $\ol{BD}$ and $\ol{CE}$
intersect the minor arcs $AB$ and $AC$ of $\Gamma$
at points $F$ and $G$, respectively.
Prove that the lines $DE$ and $FG$ are parallel.

---

We present a synthetic solution from the IMO shortlist
as well as a complex numbers approach.
We also outline a trig solution (the one I found at IMO),
and a fourth solution from Derek Liu.


\paragraph{Synthetic solution (from Shortlist).}
Construct parallelograms $AXFD$ and $AEGY$,
noting that $X$ and $Y$ lie on $\Gamma$.
As $\ol{XF} \parallel \ol{AB}$ we can let $M$
denote the midpoint of minor arcs $\widehat{XF}$ and $\widehat{AB}$
(which coincide). Define $N$ similarly.

\begin{center}
\begin{asy}
pair A = dir(110);
pair B = dir(210);
pair C = dir(330);
pair F = dir(190);
pair M = dir(160);
pair N = dir(40);
pair G = M*N/F;
pair K = foot(F, A, B);
pair L = foot(G, A, C);
pair D = -B+2*K;
pair E = -C+2*L;
pair X = M*M/F;
pair Y = N*N/G;
filldraw(unitcircle, opacity(0.1)+lightcyan, lightblue);
draw(D--B--C--E--cycle, lightblue);

draw(X--A, red);
draw(A--Y, pink);
draw(F--X, heavygreen);
draw(Y--G, heavygreen);
draw(F--G, dotted+blue);
draw(M--N, dotted+blue);
filldraw(A--X--F--D--cycle, opacity(0.1)+lightgreen, heavygreen);
filldraw(A--E--G--Y--cycle, opacity(0.1)+lightgreen, heavygreen);
draw(B--F, heavygreen);
draw(C--G, heavygreen);

draw(X--F, heavycyan+1);
draw(D--A--E, heavycyan+1);
draw(Y--G, heavycyan+1);

dot("$A$", A, dir(A));
dot("$B$", B, dir(B));
dot("$C$", C, dir(C));
dot("$F$", F, dir(F));
dot("$M$", M, dir(M));
dot("$N$", N, dir(N));
dot("$G$", G, dir(G));
dot("$D$", D, dir(160));
dot("$E$", E, dir(80));
dot("$X$", X, dir(X));
dot("$Y$", Y, dir(Y));

/* TSQ Source:

A = dir 110
B = dir 210
C = dir 330
F = dir 190
M = dir 160
N = dir 40
G = M*N/F
K := foot F A B
L := foot G A C
D = -B+2*K R160
E = -C+2*L R80
X = M*M/F
Y = N*N/G
unitcircle 0.1 lightcyan / lightblue
D--B--C--E--cycle lightblue

X--A red
A--Y pink
F--X heavygreen
Y--G heavygreen
F--G dotted blue
M--N dotted blue
A--X--F--D--cycle 0.1 lightgreen / heavygreen
A--E--G--Y--cycle 0.1 lightgreen / heavygreen
B--F heavygreen
C--G heavygreen

X--F heavycyan+1
D--A--E heavycyan+1
Y--G heavycyan+1

*/
\end{asy}
\end{center}

Observe that $XF = AD = AE = YG$,
so arcs $\widehat{XF}$ and $\widehat{YG}$ have equal measure;
hence arcs $\widehat{MF}$ and $\widehat{NG}$ have equal measure;
therefore $\ol{MN} \parallel \ol{FG}$.

Since $\ol{MN}$ and $\ol{DE}$ are both perpendicular
to the $\angle A$ bisector, so we're done.

\paragraph{Complex numbers solution.}
Let $b$, $c$, $f$, $g$, $a$ be as usual.
Note that
\begin{align*}
  d-a &= \left( 2 \cdot \frac{f+a+b-ab\ol f}{2} -b \right)-a
    = f - \frac{ab}{f} \\
  e-a &= g - \frac{ac}{g}
\end{align*}
We are given $AD = AE$ from which one deduces
\begin{align*}
  \left( \frac{e-a}{d-a} \right)^2 &= \frac cb
  \implies \frac{(g^2-ac)^2}{(f^2-ab)^2} = \frac{g^2 c}{f^2 b} \\
  \implies bc(bg^2-cf^2)a^2 &= g^2f^4c - f^2g^4b = f^2g^2(f^2c-g^2b) \\
  \implies bc \cdot a^2 &= (fg)^2 \implies \left( -\frac{fg}{a} \right)^2 = bc.
\end{align*}
Since $\frac{-fg}{a}$ is the point $X$ on the circle
with $\ol{AX} \perp \ol{FG}$,
we conclude $\ol{FG}$ is either parallel or perpendicular
to the $\angle A$-bisector; it must the latter
since the $\angle A$-bisector separates the two minor arcs.

\paragraph{Trig solution (outline).}
Let $\ell$ denote the $\angle A$ bisector.
Fix $D$ and $F$.
We define the phantom point $G'$ such that $\ol{FG'} \perp \ell$
and $E'$ on side $\ol{AC}$ such that $GE'=GC$.
\begin{claim*}
  [Converse of the IMO problem]
  We have $AD = AE'$, so that $E = E'$.
\end{claim*}
\begin{proof}
  Since $\ol{FG'} \perp \ell$,
  one can deduce $\angle FBD = \half C + x$
  and $\angle GCA = \half B + x$ for some $x$.
  (One fast way to see this is to note that $\ol{FG} \parallel \ol{MN}$
  where $M$ and $N$ are in the first solution.)
  Then $\angle FAB = \half C -x$ and $\angle GAC = \half B - x$.

  Let $R$ be the circumradius.
  Now, by the law of sines,
  \[ BF = 2R \sin\left( \half C - x \right). \]
  From there we get
  \begin{align*}
  BD &= 2 \cdot BF \cos\left(\half C + x\right)
    = 4R \cos\left(\half C+x\right) \sin \left(\half C-x\right) \\
  DA &= AB - BD = 2R\sin C
    - 4R\cos\left(\half C+x\right) \sin\left(\half C -x\right) \\
  &= 2R\left[ \sin C - 2\cos\left( \half C + x \right) \sin \left( \half C -x \right) \right] \\
  &= 2R\left[ \sin C - \left( \sin C - \sin 2x  \right) \right]
  = 2R \sin 2x.
  \end{align*}
  A similar calculation gives $AE' = 2R \sin 2x$ as needed.
\end{proof}
Thus, $\ol{FG'} \parallel \ol{DE}$, so $G = G'$ as well.
This concludes the proof.

\paragraph{Synthetic solution from Derek Liu.}
Let lines $FD$ and $GE$ intersect $\Gamma$ again at $J$ and $K$, respectively.
\begin{center}
\begin{asy}
unitsize(0.5cm);
import graph;
pair A=(-2.5,6), B=(-5.2,-3.9), C=(5.2,-3.9), D=(-4,0.5), E=(1,1.5), F=(-6.39,-1.21), G=(6.36,1.33), J=(3.19,5.66), K=(-6.27,1.72);
draw(Circle((0,0),6.5)); draw(A--B--C--A); draw(B--F--J--A--K--G--C); draw(Circle(A,5.7),dashed);
dot(A); dot(B); dot(C); dot(D); dot(E); dot(F); dot(G); dot(J); dot(K);
label("$A$",A,N);
label("$B$",B,SW);
label("$C$",C,SE);
label("$D$",D,SSE);
label("$E$",E,S);
label("$F$",F,W);
label("$G$",G,E);
label("$J$",J,ENE);
label("$K$",K,WSW);
\end{asy}
\end{center}
Notice that $\triangle BFD\sim\triangle JAD$; as $FB=FD$, it follows that $AJ=AD$.
Likewise, $\triangle CGE\sim\triangle KAE$ and $GC=GE$, so $AK=AE$.
Hence,
\[ AK=AE=AD=AJ, \]
so $DEJK$ is cyclic with center $A$.

It follows that
\[ \dang KED=\dang KJD=\dang KJF=\dang KGF, \]
so we're done.
\begin{remark*}
Note that $K$ and $J$ must be distinct for this solution to work.
Since $G$ and $K$ lie on opposite sides of $AC$, $K$ is on major arc $ABC$.
As $AK=AD=AE\le \min(AB,AC)$, $K$ lies on minor arc $AB$.
Similarly, $J$ lies on minor arc $AC$, so $K\neq J.$
\end{remark*}

---

We will prove the problem in the converse direction:
we show if $\ol{FG}$ is perpendicular to the $\angle A$-bisector
(which we denote by $\ell$),
then $AD = AE$.
\begin{walk}
  \ii Work out a phantom point argument showing this is sufficient.
\end{walk}
It turns out this is pretty easy once you have the idea
to use trigonometry.
Fix $\triangle ABC$.
\begin{walk}[resume]
  \ii Using $\ol{FG} \perp \ell$, show that
  $\angle FBA = \half C + x$ and $\angle GCA = \half B + x$
  for some $x$.
  \ii Calculate $BF$ (in terms of the circumradius $R$ and $A$, $B$, $C$, $x$).
  \ii Calculate $BD$.
  \ii Calculate $AD$.
  \ii Calculate $AE$.
  \ii Show that $AD = AE = 2R\sin 2x$.
\end{walk}
