desc: $(f(x)+f(z))(f(y)+f(t))$
source: IMO 2002/5
tags: [FE, 2019-04, cauchy, size, symmetry, aleph]
hardness: 15
url: https://aops.com/community/p118703
author: Belur Jana Venkatachala (IND)

---

Find all functions $f \colon \RR \to \RR$ such that
\[ \left(f(x)+f(z)\right)\left(f(y)+f(t)\right)
  = f(xy-zt)+f(xt+yz) \]
for all real numbers $x$, $y$, $z$, $t$.

---

The answer is $f(x) \equiv 0$, $f(x) \equiv 1/2$
and $f(x) \equiv x^2$ which are easily seen to work.
Let's prove they are the only ones;
we show two solutions.

\paragraph{First solution (multiplicativity).}
Let $P(x,y,z,t)$ denote the given statement.
\begin{itemize}
  \ii By comparing $P(x,1,0,0)$ and $P(0,0,1,x)$
  we get $\boxed{f \text{ even}}$.
  \ii By $P(0,y,0,t)$ we get for nonconstant $f$
  that $f(0) = 0$.
  If $f$ is constant we get the solutions earlier,
  so in the sequel assume $\boxed{f(0) = 0}$.
  \ii By $P(x,y,0,0)$ we get $\boxed{f(xy) = f(x) f(y)}$.
  Note in particular that for any real number $x$ we now have
  \[ f(x) = f(|x|) = f\left( \sqrt{|x|} \right)^2 \ge 0 \]
  that is, $f \ge 0$.
\end{itemize}

From $P(x,y,y,x)$ we now have
\[ f(x^2 + y^2) = \left( f(x) + f(y) \right)^2
  = f(x^2) + 2f(x)f(y) + f(y^2) \ge f(x^2) \]
so $f$ is weakly increasing.
Combined with $f$ multiplicative and nonconstant,
this implies $f(x) = |x|^r$ for some real number $r$.

Finally, $P(1,1,1,1)$ gives $f(2) = 4f(1)$,
so $f(x) \equiv x^2$.

\paragraph{Second solution (ELMO).}
Let $P(x,y,z,t)$ denote the statement.
Assume $f$ is nonconstant,
as before we derive that $f$ is even, $f(0) = 0$,
and $f(x) \ge 0$ for all $x$.

Now comparing $P(x,y,z,t)$ and $P(z,y,x,t)$ we obtain
\[ f(xy-zt) + f(xt+yz) =
  \left( f(x)+f(z) \right)
  \left( f(y)+f(t) \right)
  = f(xy+zt) + f(xt-yz) \]
which in particular implies that
\[ f(a-d) + f(b+c) = f(a+d) + f(b-c)
\qquad \text{ if } ad=bc \text{ and } a,b,c,d > 0. \]
Thus the restriction of $f$ to $(0,\infty)$ satisfies
\textbf{ELMO 2011, problem 4}
which implies that $f(x) = kx^2+\ell$ for constants $k$ and $\ell$.
From here we recover the original.

(Minor note: technically ELMO 2011/4 is $f \colon (0,\infty) \to (0,\infty)$
but we only have $f \ge 0$,
however the proof for the ELMO problem
works as long as $f$ is bounded below;
we could also just apply the ELMO problem to $f+0.01$ instead.)
