desc: Japanese triangle
hardness: 35
source: IMO 2023/5
url: https://aops.com/community/p28104367
tags: [2023-09, grid, find, bestpossible, gimel]
author: Merlijn Staps and Daniël Kroes (NLD)

---

Let $n$ be a positive integer.
A \emph{Japanese triangle} consists of $1 + 2 + \dots + n$ circles arranged in an
equilateral triangular shape such that for each $1 \le i \le n$,
the $i$\ts{th} row contains exactly $i$ circles, exactly one of which is colored red.
A \emph{ninja path} in a Japanese triangle is a sequence of $n$ circles
obtained by starting in the top row, then repeatedly going from a circle to
one of the two circles immediately below it and finishing in the bottom row.
Here is an example of a Japanese triangle with $n = 6$,
along with a ninja path in that triangle containing two red circles.
\begin{center}
  \begin{asy}
  size(4cm);
  pair X = dir(240); pair Y = dir(0);
  path c = scale(0.5)*unitcircle;
  int[] t = {0,0,2,2,3,0};
  for (int i=0; i<=5; ++i) {
    for (int j=0; j<=i; ++j) {
      filldraw(shift(i*X+j*Y)*c, (t[i]==j) ? lightred : white);
      draw(shift(i*X+j*Y)*c);
    }
  }
  draw((0,0)--(X+Y)--(2*X+Y)--(3*X+2*Y)--(4*X+2*Y)--(5*X+2*Y),linewidth(1.5));
  path q = (3,-3sqrt(3))--(-3,-3sqrt(3));
  draw(q,Arrows(TeXHead, 1));
  label("$n = 6$", q, S);
  \end{asy}
\end{center}
In terms of $n$, find the greatest $k$ such that in each Japanese triangle
there is a ninja path containing at least $k$ red circles.

---

The answer is
\[ k = \left\lfloor \log_2(n) \right\rfloor + 1. \]

\paragraph{Construction.}
It suffices to find a Japanese triangle for $n = 2^e-1$
with the property that at most $e$ red circles in any ninja path.
The construction shown below for $e=4$ obviously generalizes,
and works because in each of the sets $\{1\}$, $\{2,3\}$, $\{4,5,6,7\}$,
\dots, $\{2^{e-1},\dots,2^e-1\}$, at most one red circle can be taken.
(These sets are colored in different shades of red for visual clarity).

\begin{center}
  \begin{asy}
  size(8cm);
  pair X = dir(240); pair Y = dir(0);
  path c = scale(0.5)*unitcircle;

  fill(shift(0*X+0*Y)*c, heavyred);

  fill(shift(1*X+0*Y)*c, red);
  fill(shift(2*X+2*Y)*c, red);

  fill(shift(3*X+0*Y)*c, magenta);
  fill(shift(4*X+2*Y)*c, magenta);
  fill(shift(5*X+4*Y)*c, magenta);
  fill(shift(6*X+6*Y)*c, magenta);

  fill(shift(7*X+0*Y)*c,   lightred);
  fill(shift(8*X+2*Y)*c,   lightred);
  fill(shift(9*X+4*Y)*c,   lightred);
  fill(shift(10*X+6*Y)*c,  lightred);
  fill(shift(11*X+8*Y)*c,  lightred);
  fill(shift(12*X+10*Y)*c, lightred);
  fill(shift(13*X+12*Y)*c, lightred);
  fill(shift(14*X+14*Y)*c, lightred);

  for (int i=0; i<15; ++i) {
    for (int j=0; j<=i; ++j) {
      draw(shift(i*X+j*Y)*c);
    }
  }
  \end{asy}
\end{center}

\paragraph{Bound.}
Conversely, we show that in any Japanese triangle,
one can find a ninja path containing at least
\[ k = \left\lfloor \log_2(n) \right\rfloor + 1. \]
The following short solution was posted at \url{https://aops.com/community/p28134004},
apparently first found by the team leader for Iran.

We construct a rooted binary tree $T_1$ on the set of all circles as follows.
For each row, other than the bottom row:
\begin{itemize}
  \ii Connect the red circle to both circles under it;
  \ii White circles to the left of the red circle in its row are connected to the left;
  \ii White circles to the right of the red circle in its row are connected to the right.
\end{itemize}
The circles in the bottom row are all leaves of this tree.
For example, the $n=6$ construction in the beginning gives the tree
shown on the left half of the figure below:
\begin{center}
  \begin{asy}
  size(11cm);
  pair X = dir(240); pair Y = dir(0);
  path c = scale(0.3)*unitcircle;
  int[] t = {0,0,2,2,3,0};
  for (int i=0; i<5; ++i) {
    for (int j=0; j<=i; ++j) {
      if (j <= t[i]) {
        draw( (i*X+j*Y)--((i+1)*X+j*Y), blue+1.4);
      }
      if (j >= t[i]) {
        draw( (i*X+j*Y)--((i+1)*X+(j+1)*Y), blue+1.4);
      }
    }
  }
  for (int i=0; i<=5; ++i) {
    for (int j=0; j<=i; ++j) {
      filldraw(shift(i*X+j*Y)*c, (t[i]==j) ? lightred : white, black);
    }
  }
  label("$T_1$", 6*X+3*Y, fontsize(14pt));

  transform sh = shift(7,0);
  draw(sh*((5*X+0*Y)--(0*X+0*Y)--(2*X+2*Y)--(3*X+2*Y)--(4*X+3*Y)), blue+1.4);
  for (int i=0; i<=5; ++i) {
    filldraw(sh*shift(i*X+t[i]*Y)*c, lightred, black);
  }
  label("$T_2$", sh*(6*X+3*Y), fontsize(14pt));
  \end{asy}
\end{center}

Now focus on only the red circles, as shown in the right half of the figure.
We build a new rooted tree $T_2$ where each red circle is joined to the
red circle below it if there was a path of (zero or more)
white circles in $T_1$ between them.
Then each red circle has at most $2$ direct descendants in $T_2$.
Hence the depth of the new tree $T_2$ exceeds $\log_2(n)$, which produces the desired path.

\paragraph{Another recursive proof of bound, communicated by Helio Ng.}
We give another proof that $\lfloor \log_2 n\rfloor + 1$ is always achievable.
Define $f(i, j)$ to be the maximum number of red circles contained in
the portion of a ninja path from $(1, 1)$ to $(i, j)$,
including the endpoints $(1, 1)$ and $(i, j)$.
(If $(i,j)$ is not a valid circle in the triangle, define $f(i, j)=0$ for convenience.)
An example is shown below with the values of $f(i,j)$ drawn in the circles.

\begin{center}
\begin{asy}
unitsize(7mm);

pair X = dir(240); pair Y = dir(0);
int[] t = {0,0,2,0,2,4,6};
for (int i=0; i<=6; ++i) {
  for (int j=0; j<=i; ++j) {
    filldraw(shift(i*X+j*Y)*scale(0.5)*unitcircle, (t[i]==j) ? lightred : white, black);
  }
}
label("$1$",0*X+0*Y);
label("$2$",1*X+0*Y);
label("$1$",1*X+1*Y);
label("$2$",2*X+0*Y);
label("$2$",2*X+1*Y);
label("$2$",2*X+2*Y);
label("$3$",3*X+0*Y);
label("$2$",3*X+1*Y);
label("$2$",3*X+2*Y);
label("$2$",3*X+3*Y);
label("$3$",4*X+0*Y);
label("$3$",4*X+1*Y);
label("$3$",4*X+2*Y);
label("$2$",4*X+3*Y);
label("$2$",4*X+4*Y);
label("$3$",5*X+0*Y);
label("$3$",5*X+1*Y);
label("$3$",5*X+2*Y);
label("$3$",5*X+3*Y);
label("$3$",5*X+4*Y);
label("$2$",5*X+5*Y);
label("$3$",6*X+0*Y);
label("$3$",6*X+1*Y);
label("$3$",6*X+2*Y);
label("$3$",6*X+3*Y);
label("$3$",6*X+4*Y);
label("$3$",6*X+5*Y);
label("$3$",6*X+6*Y);
\end{asy}
\end{center}

We have that
\[ f(i, j) = \max \left\{f(i-1,j-1), f(i,j-1)  \right\} +
    \begin{cases}
        1 & \text{if $(i,j)$ is red} \\
        0 & \text{otherwise}
    \end{cases} \]
since every ninja path passing through $(i, j)$ also passes through
either $(i-1,j-1)$ or $(i,j-1)$. Now consider the quantity
$S_j = f(0, j) + \dots + f(j, j)$. We obtain the following recurrence:

\begin{claim*}
$S_{j+1} \geq S_j + \left\lceil \frac{S_j}{j} \right\rceil + 1$.
\end{claim*}
\begin{proof}
Consider a maximal element $f(m, j)$ of  $ \{f(0, j), \dots, f(j, j) \}$.
We perform the following manipulations:
\begin{align*}
    S_{j+1}
    &= \sum_{i=0}^{j+1} \max \left\{f(i-1,j), f(i,j) \right\} + \sum_{i=0}^{j+1}
    \begin{cases}
    1 & \text{if $(i,j+1)$ is red} \\
    0 & \text{otherwise}
    \end{cases} \\
    &= \sum_{i=0}^{m} \max \left\{f(i-1,j), f(i,j) \right\} +
    \sum_{i=m+1}^{j} \max \left\{f(i-1,j), f(i,j) \right\} + 1 \\
    &\geq \sum_{i=0}^{m} f(i,j) +  \sum_{i=m+1}^{j} f(i-1,j) + 1 \\
    &= S_j + f(m, j) + 1 \\
    &\geq S_j + \left\lceil \frac{S_j}{j} \right\rceil + 1
\end{align*}
where the last inequality is due to Pigeonhole.
\end{proof}

This is actually enough to solve the problem.
Write $n = 2^c + r$, where $0 \leq r \leq 2^c - 1$.
\begin{claim*}
$S_n \geq cn + 2r + 1$. In particular, $\left\lceil \frac{S_n}{n} \right\rceil \geq c + 1$.
\end{claim*}

\begin{proof}
First note that $S_n \geq cn + 2r + 1$ implies
$\left\lceil \frac{S_n}{n} \right\rceil \geq c + 1$ because
\[ \left\lceil \frac{S_n}{n} \right\rceil \geq \left\lceil \frac{cn + 2r + 1}{n} \right\rceil
= c + \left\lceil \frac{2r + 1}{n} \right\rceil = c + 1.
\]
We proceed by induction on $n$.
The base case $n = 1$ is clearly true as $S_1 = 1$.
Assuming that the claim holds for some $n=j$, we have
\begin{align*}
    S_{j+1} &\geq S_j + \left\lceil \frac{S_j}{j} \right\rceil + 1 \\
    &\geq cj + 2r + 1 + c + 1 + 1 \\
    &= c(j+1) + 2(r+1) + 1
\end{align*}
so the claim is proved for $n=j+1$ if $j+1$ is not a power of $2$.
If $j+1 = 2^{c+1}$, then by writing
$c(j+1) + 2(r+1) + 1 = c(j+1) + (j+1) + 1 = (c+2)(j+1) + 1$, the claim is also proved.
\end{proof}

Now $\left\lceil \frac{S_n}{n} \right\rceil \geq c + 1$ implies the existence of
some ninja path containing at least $c+1$ red circles, and we are done.
