desc: $\angle BAF = \angle CAE < \frac12\angle BAC$
source: IMO 2010/2
tags: [2016-09, harmonic, simtri, projective, instructive, bet]
hardness: 20
url: https://aops.com/community/p1935927
author: Tai Wai Ming and Wang Chongli (HKG)

---

Let $I$ be the incenter of a triangle $ABC$ and let $\Gamma$ be its circumcircle.
Let line $AI$ intersect $\Gamma$ again at $D$.
Let $E$ be a point on arc $\widehat{BDC}$ and $F$ a point on side $BC$ such that
\[ \angle BAF = \angle CAE < \tfrac12 \angle BAC. \]
Finally, let $G$ be the midpoint of $\ol{IF}$.
Prove that $\ol{DG}$ and $\ol{EI}$ intersect on $\Gamma$.

---

Let $\ol{EI}$ meet $\Gamma$ again at $K$.
Then it suffices to show that $\ol{KD}$ bisects $\ol{IF}$.
Let $\ol{AF}$ meet $\Gamma$ again at $H$, so $\ol{HE} \parallel \ol{BC}$.
By Pascal theorem on \[ AHEKDD \]
we then obtain that $P = \ol{AH} \cap \ol{KD}$ lies on a line through $I$
parallel to $\ol{BC}$.

Let $I_A$ be the $A$-excenter,
and set $Q = \ol{I_AF} \cap \ol{IP}$, and $T = \ol{AIDI_A} \cap \ol{BFC}$.
Then
\[ -1 = (AI;TI_A) \overset{F} = (IQ;\infty P) \]
where $\infty$ is the point at infinity along $\ol{IPQ}$.
Thus $P$ is the midpoint of $\ol{IQ}$.
Since $D$ is the midpoint of $\ol{II_A}$ by ``Fact 5'',
it follows that $\ol{DP}$ bisects $\ol{IF}$.

\begin{center}
\begin{asy}
size(9cm);

pair A = dir(110);
pair B = dir(210);
pair C = dir(330);
pair D = dir(-90);
pair I = incenter(A, B, C);
pair I_A = 2*D-I;

pair E = dir(310);
pair K = -E+2*foot(origin, E, I);
pair H = B*C/E;
pair F = extension(B, C, A, H);
pair G = extension(D, K, I, F);
pair P = extension(A, F, D, G);
pair Q = extension(I, P, I_A, F);

draw(A--B--C--cycle, lightblue);
filldraw(unitcircle, opacity(0.1)+lightcyan, lightblue);

draw(A--I_A, lightblue);
draw(A--H--E--K--D--cycle, orange);
draw((D+0.3)--(D-0.3), orange);
draw(I--Q--I_A, red);
draw(I--F, blue);
pair T = extension(B, C, A, D);

dot("$A$", A, dir(A));
dot("$B$", B, dir(B));
dot("$C$", C, dir(C));
dot("$D$", D, dir(220));
dot("$I$", I, dir(10));
dot("$I_A$", I_A, dir(I_A));
dot("$E$", E, dir(E));
dot("$K$", K, dir(K));
dot("$H$", H, dir(H));
dot("$F$", F, dir(F));
dot("$G$", G, dir(350));
dot("$P$", P, dir(45));
dot("$Q$", Q, dir(Q));
dot("$T$", T, dir(T));

/* TSQ Source:

!size(9cm);

A = dir 110
B = dir 210
C = dir 330
D = dir -90 R220
I = incenter A B C R10
I_A = 2*D-I

E = dir 310
K = -E+2*foot origin E I
H = B*C/E
F = extension B C A H
G = extension D K I F R350
P = extension A F D G R45
Q = extension I P I_A F

A--B--C--cycle lightblue
unitcircle 0.1 lightcyan / lightblue

A--I_A lightblue
A--H--E--K--D--cycle orange
(D+0.3)--(D-0.3) orange
I--Q--I_A red
I--F blue
T = extension B C A D

*/
\end{asy}
\end{center}
