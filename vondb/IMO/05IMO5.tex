desc: Spiral congruence at IMO
source: IMO 2005/5
tags: [2016-09, spiralsim, dalet]
hardness: 25
url: https://aops.com/community/p282140
author: Waldemar Pompe (POL)

---

Let $ABCD$ be a fixed convex quadrilateral
with $BC=DA$ and $\ol{BC} \nparallel \ol{DA}$.
Let two variable points $E$ and $F$ lie on the
sides $BC$ and $DA$, respectively, and satisfy $BE=DF$.
The lines $AC$ and $BD$ meet at $P$,
the lines $BD$ and $EF$ meet at $Q$,
the lines $EF$ and $AC$ meet at $R$.
Prove that the circumcircles of the triangles $PQR$,
as $E$ and $F$ vary, have a common point other than $P$.

---

Let $M$ be the Miquel point of complete quadrilateral $ADBC$;
in other words, let $M$ be the second intersection point
of the circumcircles of $\triangle APD$ and $\triangle BPC$.
(A good diagram should betray this secret;
all the points are given in the picture.)
This makes lots of sense since we know $E$ and $F$
will be sent to each other under the spiral similarity too.

\begin{center}
\begin{asy}
size(10cm);
pair A = dir(120);
pair D = dir(210);
pair C = dir(330);
pair B = dir(80);
filldraw(A--D--B--C--cycle, opacity(0.2)+lightcyan, blue+1.2);

pair E = 0.8*B+0.2*C;
pair F = 0.8*D+0.2*A;

pair P = extension(A, C, B, D);
pair Q = extension(E, F, B, D);
pair R = extension(E, F, A, C);

draw(E--F, lightblue);

pair M = (A*B-C*D)/(A+B-C-D);

filldraw(circumcircle(P, Q, R), opacity(0.2)+yellow, red);
draw(circumcircle(A, P, D), lightblue);
draw(circumcircle(B, P, C), lightblue);
filldraw(circumcircle(F, A, R), opacity(0.1)+lightgreen, dotted+heavygreen);
filldraw(circumcircle(F, D, Q), opacity(0.1)+lightgreen, dotted+heavygreen);
filldraw(circumcircle(E, B, Q), opacity(0.1)+lightgreen, dotted+heavygreen);
filldraw(circumcircle(E, C, R), opacity(0.1)+lightgreen, dotted+heavygreen);

dot("$A$", A, dir(A));
dot("$D$", D, dir(240));
dot("$C$", C, dir(C));
dot("$B$", B, dir(B));
dot("$E$", E, dir(E));
dot("$F$", F, dir(F));
dot("$P$", P, 1.8*dir(95));
dot("$Q$", Q, dir(Q));
dot("$R$", R, dir(R));
dot("$M$", M, 1.4*dir(-30));

/* TSQ Source:

!size(10cm);
A = dir 120
D = dir 210 R240
C = dir 330
B = dir 80
A--D--B--C--cycle 0.2 lightcyan / blue+1.2

E = 0.8*B+0.2*C
F = 0.8*D+0.2*A

P = extension A C B D 1.8R95
Q = extension E F B D
R = extension E F A C

E--F lightblue

M = (A*B-C*D)/(A+B-C-D) 1.4R-30

circumcircle P Q R 0.2 yellow / red
circumcircle A P D lightblue
circumcircle B P C lightblue
circumcircle F A R 0.1 lightgreen / dotted heavygreen
circumcircle F D Q 0.1 lightgreen / dotted heavygreen
circumcircle E B Q 0.1 lightgreen / dotted heavygreen
circumcircle E C R 0.1 lightgreen / dotted heavygreen

*/
\end{asy}
\end{center}

Thus $M$ is the Miquel point of complete quadrilateral $FACE$.
As $R = \ol{FE} \cap \ol{AC}$ we deduce $FARM$ is a cyclic quadrilateral
(among many others, but we'll only need one).

Now look at complete quadrilateral $AFQP$.
Since $M$ lies on $(DFQ)$ and $(RAF)$,
it follows that $M$ is in fact the Miquel point of $AFQP$ as well.
So $M$ lies on $(PQR)$.

Thus $M$ is the fixed point that we wanted.

\begin{remark*}
  Naturally, the congruent length
  condition can be relaxed to $DF/DA = BE/BC$.
\end{remark*}

---

\begin{walk}
  \ii Draw a good diagram and locate the fixed point $M$.
  Find examples of cyclic quadrilaterals it lies on.
  \ii Identify $M$ as the Miquel point of a quadrilateral
  not involving the points $P$, $Q$, $R$.
  \ii Optionally, show that $MF = ME$.
  (This isn't necessary, but is a sanity check to
  make sure you've figured out everything you need to proceed).
  \ii Prove that quadrilateral $FARM$ is cyclic,
  and identify $M$ as the Miquel point
  of a different quadrilateral involving $P$, $Q$, $R$.
  \ii Prove that $M$ lies on the circumcircle of $\triangle PQR$.
\end{walk}
