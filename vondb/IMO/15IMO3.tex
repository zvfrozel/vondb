desc:  The inversion problem
author: Danylo Khilko and Mykhailo Plotnikov (UKR)
source:  IMO 2015/3
tags:  [inversion, rich, good, dalet]
hardness: 25
url: https://aops.com/community/p5079655

---

Let $ABC$ be an acute triangle with $AB > AC$.
Let $\Gamma$ be its circumcircle, $H$ its orthocenter, and $F$ the foot of the altitude from $A$.
Let $M$ be the midpoint of $\ol{BC}$.
Let $Q$ be the point on $\Gamma$ such that $\angle HQA = 90\dg$
and let $K$ be the point on $\Gamma$ such that $\angle HKQ = 90\dg$.
Assume that the points $A$, $B$, $C$, $K$ and $Q$ are all different and lie on $\Gamma$ in this order.
Prove that the circumcircles of triangles $KQH$ and $FKM$ are tangent to each other.

---

Let $L$ be on the nine-point circle with $\angle HML = 90^{\circ}$.
The negative inversion at $H$ swapping $\Gamma$ and nine-point circle maps
\[ A \longleftrightarrow F, \quad
  Q \longleftrightarrow M, \quad
  K \longleftrightarrow L. \]
In the inverted statement, we want line $ML$ to be tangent to $(AQL)$.

\begin{center}
\begin{asy}
size(10cm);
pair A = dir(80);
pair B = dir(220);
pair C = dir(-40);

filldraw(unitcircle, opacity(0.1)+yellow, orange);
pair O = circumcenter(A, B, C);
pair H = orthocenter(A, B, C);
pair Q = IP(CP(midpoint(A--H), H), unitcircle);
pair N = midpoint(H--Q);
pair T = midpoint(A--H);
pair N_9 = midpoint(O--H);
pair M = midpoint(B--C);
pair F = foot(A, B, C);

filldraw(circumcircle(M, F, N), opacity(0.1)+yellow, heavyred);

pair L = M+T-N;
pair K = foot(Q, L, H);

draw(A--B--C--cycle, red+1);
draw(A--F, red+1);
draw(T--M--L--N, red);
draw(M--Q--A, brown);
draw(Q--K--L, brown);

filldraw(A--Q--L--cycle, opacity(0.1)+cyan, cyan);

dot("$A$", A, dir(A));
dot("$B$", B, dir(B));
dot("$C$", C, dir(C));
dot("$O$", O, dir(O));
dot("$H$", H, dir(H));
dot("$Q$", Q, dir(Q));
dot("$N$", N, dir(N));
dot("$T$", T, dir(T));
dot("$N_9$", N_9, dir(N_9));
dot("$M$", M, dir(M));
dot("$F$", F, dir(F));
dot("$L$", L, dir(L));
dot("$K$", K, dir(K));

/* TSQ Source:

!size(10cm);
A = dir 80
B = dir 220
C = dir -40

unitcircle 0.1 yellow / orange
O = circumcenter A B C
H = orthocenter A B C
Q = IP CP midpoint A--H H unitcircle
N = midpoint H--Q
T = midpoint A--H
N_9 = midpoint O--H
M = midpoint B--C
F = foot A B C

circle M F N 0.1 yellow / heavyred

L = M+T-N
K = foot Q L H

A--B--C--cycle red+1
A--F red+1
T--M--L--N red
M--Q--A brown
Q--K--L brown

A--Q--L--cycle 0.1 cyan / cyan

*/
\end{asy}
\end{center}

\begin{claim*}
  $\ol{LM} \parallel \ol{AQ}$.
\end{claim*}
\begin{proof}
  Both are perpendicular to $\ol{MHQ}$.
\end{proof}
\begin{claim*}
  $LA = LQ$.
\end{claim*}
\begin{proof}
  Let $N$ and $T$ be the midpoints of $\ol{HQ}$ and $\ol{AH}$,
  and $O$ the circumcenter.
  As $\ol{MT}$ is a diameter, we know $LTNM$ is a rectangle,
  so $\ol{LT}$ passes through $O$.
  Since $\ol{LOT} \perp \ol{AQ}$ and $OA=OQ$, the proof is complete.
\end{proof}
Together these two claims solve the problem.
