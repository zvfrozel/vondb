desc: Silly social network graph problem
source: IMO 2019/3
tags: [2019-07, graph, instructive, algorithm, adhoc, free, meta, gimel]
author: Adrian Beker (HRV)
hardness: 40
url: https://aops.com/community/p12744851

---

A social network has $2019$ users,
some pairs of which are friends (friendship is symmetric).
If $A$, $B$, $C$ are three users such that
$AB$ are friends and $AC$ are friends but $BC$ is not,
then the administrator may perform the following operation:
change the friendships such that $BC$ are friends,
but $AB$ and $AC$ are no longer friends.

Initially, $1009$ users have $1010$ friends
and $1010$ users have $1009$ friends.
Prove that the administrator can make a sequence of operations
such that all users have at most $1$ friend.

---

We take the obvious graph formulation
and call the move a \emph{toggle}.

\begin{claim*}
  Let $G$ be a connected graph.
  Then one can toggle $G$ without disconnecting the graph,
  unless $G$ is a clique, a cycle, or a tree.
\end{claim*}
\begin{proof}
  Assume $G$ is connected and not a tree, so it has a cycle.
  Take the smallest cycle $C$; by hypothesis $C \neq G$.

  If $C$ is not a triangle (equivalently, $G$ is triangle-free),
  then let $b \notin C$ be a vertex adjacent to $C$, say at $a$.
  Take a vertex $c$ of the cycle adjacent to $a$ (hence not to $b$).
  Then we can toggle $abc$.

  Now assume there exists a triangle; let $K$ be the maximal clique.
  By hypothesis, $K \neq G$.
  We take an edge $e = ab$ dangling off the clique,
  with $a \in K$ and $b \notin K$.
  Note some vertex $c$ of $K$ is not adjacent to $b$; now toggle $abc$.
\end{proof}

Back to the original problem;
let $G_{\text{imo}}$ be the given graph.
The point is that we can apply toggles (by the claim) repeatedly,
without disconnecting the graph, until we get a tree.
This is because
\begin{itemize}
  \ii $G_{\text{imo}}$ is connected,
  since any two vertices which are not adjacent
  have a common neighbor by pigeonhole
  ($1009 + 1009 + 2 > 2019$).
  \ii $G_{\text{imo}}$ cannot become a cycle,
  because it initially has an odd-degree vertex,
  and toggles preserve parity of degree!
  \ii $G_{\text{imo}}$ is obviously not a clique initially
  (and hence not afterwards).
\end{itemize}
So, we can eventually get $G_{\text{imo}}$ to be a tree.

Once $G_{\text{imo}}$ is a tree the problem follows by repeatedly applying
toggles arbitrarily until no more are possible;
the graph (although now disconnected) remains acyclic
(in particular having no triangles)
and therefore can only terminate in the desired situation.

\begin{remark*}
  The above proof in fact shows the following better result:
  \begin{quote}
  The task is possible if and only if
  $G_{\text{imo}}$ is a connected graph which is not a clique
  and has any vertex of odd degree.
  \end{quote}
  The ``only if'' follows from the observation
  that toggles preserve parity of degree.

  Thus the given condition about the degrees of vertices
  being $1009$ and $1010$ is largely a red herring;
  it's a somewhat strange way of masking the correct and more natural
  both-sufficient-and-necessary condition.
\end{remark*}
