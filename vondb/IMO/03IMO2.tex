desc: $2ab^2-b^3+1$ positive divides $a^2$
source: IMO 2003/2
tags: [2017-01, vietajump, grinding, hardanswer, smallcases, scouting, equalitycase, gimel]
hardness: 35
url: https://aops.com/community/p262
author: Aleksander Ivanov (BGR)

---

Determine all pairs of positive integers $(a,b)$ such that
\[ \frac{a^2}{2ab^2-b^3+1} \]
is a positive integer.

---

The answer is $(a,b) = (2\ell, 1)$, $(a,b) = (\ell, 2\ell)$
and $(a,b) = (8\ell^4-\ell, 2\ell)$, for any $\ell$.
Check these work.

In the sequel, assume $b > 1$,
and integers $a$, $b$, $k$ obey $k = \frac{a^2}{2ab^2-b^3+1}$.
Expanding, we have the polynomial
\[ X^2 - 2kb^2 \cdot X + k(b^3-1) = 0 \]
has two integer roots, one of which is $X = a$.
This means solutions to the original problem come in pairs
(even with $k$ fixed):
\[ (a,b) \longleftrightarrow
  \left( 2kb^2 - a, b\right)
  = \left( \frac{k(b^3-1)}{a}, b\right). \]
(Here, the first representation ensures
$2kb^2-a \in \ZZ$,
while the latter representation and the hypothesis $b > 1$ ensures
that $\frac{k(b^3-1)}{a} > 0$.)

On the other hand, we claim that:
\begin{claim*}
For any solution $(a,b)$,
either $2a = b$ or $a > b$.
\end{claim*}
\begin{proof}
  Since the denominator is positive, $a \ge b/2$.
  Now,
  \[ a^2 \ge 2ab^2 - b^3 + 1 \iff a^2 \ge b^2(2a-b) + 1 \]
  and so if $2a - b > 0$ then $a^2 > b^2 \implies a > b$.
\end{proof}

Now assume we have pair $(a_1, b)$ and $(a_2, b)$
of solutions with $b \neq 2a_1, 2a_2$.
Then assume $a_1 > a_2 > b$ and
\begin{align*}
  a_1 + a_2 &= 2k \cdot b^2 \\
  a_1a_2 &= k(b^3-1)
\end{align*}
That's impossible, since then $a_1 > \frac{a_1+a_2}{2} = k b^2$
and hence $a_1a_2 > kb^2 \cdot b = kb^3$.
Thus the only solutions are the ones we claimed at the beginning.

\begin{remark*}
  Important to notice that the problem is \emph{positive divides},
  not just divides.
  There is an implicit inequality built in to the problem
  statement and it is essentially impossible to solve without.
  I would be interested in a pair $(a,b)$
  for which $k < 0$, $k \in \ZZ$ yet $a, b > 0$.
\end{remark*}
