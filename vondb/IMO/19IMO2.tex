desc: Medium geo for the first time in a while, so bary
source: IMO 2019/2
tags: [2019-07, bary, anglechase, pop, bet]
author: Anton Trygub (UKR)
hardness: 25
url: https://aops.com/community/p12744870

---

In triangle $ABC$ point $A_1$ lies on side $BC$
and point $B_1$ lies on side $AC$.
Let $P$ and $Q$ be points on segments $AA_1$ and $BB_1$,
respectively, such that $\ol{PQ} \parallel \ol{AB}$.
Point $P_1$ is chosen on ray $PB_1$ beyond $B_1$
such that $\angle PP_1C = \angle BAC$.
Point $Q_1$ is chosen on ray $QA_1$ beyond $A_1$
such that $\angle CQ_1Q = \angle CBA$.
Prove that points $P_1$, $Q_1$, $P$, $Q$ are cyclic.

---

We present two solutions.

\paragraph{First solution by bary (Evan Chen).}
Let $PB_1$ and $QA_1$ meet line $AB$ at $X$ and $Y$.
Since $\ol{XY} \parallel \ol{PQ}$ it is equivalent
to show $P_1XYQ_1$ is cyclic (Reim's theorem).

Note the angle condition implies $P_1CXA$ and $Q_1CYB$ are cyclic.

Letting $T = \ol{PX} \cap \ol{QY}$ (possibly at infinity),
it suffices to show that the
radical axis of $\triangle CXA$ and $\triangle CYB$ passes through $T$,
because that would imply $P_1XYQ_1$ is cyclic
(by power of a point when $T$ is Euclidean,
and because it is an isosceles trapezoid if $T$ is at infinity).

\begin{center}
\begin{asy}
pair C = dir(110);
pair A = dir(210);
pair B = dir(330);

pair A_1 = 0.45*C+0.55*B;
pair P = 0.53*A+0.47*A_1;
pair B_1 = midpoint(A--C);
pair Q = extension(B, B_1, P, P+A-B);
pair T = extension(A_1, Q, B_1, P);
pair X = extension(A, B, P, T);
pair Y = extension(A, B, Q, T);

filldraw(A--B--C--cycle, opacity(0.1)+lightcyan, lightblue);
filldraw(circumcircle(A, C, X), opacity(0.05)+yellow, lightred);
filldraw(circumcircle(B, C, Y), opacity(0.05)+yellow, lightred);
draw(A--A_1, lightblue);
draw(B--B_1, lightblue);
draw(P--Q, blue);
draw(C--T, deepgreen);
pair P_1 = -X+2*foot(circumcenter(A, X, C), P, X);
pair Q_1 = -Y+2*foot(circumcenter(B, Y, C), Q, Y);
draw(P_1--T--Q_1, orange);

dot("$C$", C, dir(C));
dot("$A$", A, dir(A));
dot("$B$", B, dir(B));
dot("$A_1$", A_1, dir(A_1));
dot("$P$", P, dir(160));
dot("$B_1$", B_1, dir(B_1));
dot("$Q$", Q, dir(20));
dot("$T$", T, dir(T));
dot("$X$", X, dir(X));
dot("$Y$", Y, dir(Y));
dot("$P_1$", P_1, dir(P_1));
dot("$Q_1$", Q_1, dir(Q_1));

/* TSQ Source:

C = dir 110
A = dir 210
B = dir 330

A_1 = 0.45*C+0.55*B
P = 0.53*A+0.47*A_1 R160
B_1 = midpoint A--C
Q = extension B B_1 P P+A-B R20
T = extension A_1 Q B_1 P
X = extension A B P T
Y = extension A B Q T

A--B--C--cycle 0.1 lightcyan / lightblue
circumcircle A C X 0.05 yellow / lightred
circumcircle B C Y 0.05 yellow / lightred
A--A_1 lightblue
B--B_1 lightblue
P--Q blue
C--T deepgreen
P_1 = -X+2*foot circumcenter A X C P X
Q_1 = -Y+2*foot circumcenter B Y C Q Y
P_1--T--Q_1 orange

*/
\end{asy}
\end{center}

To this end we use barycentric coordinates on $\triangle ABC$.
We begin by writing
\[ P = (u+t : s : r), \quad Q = (t : u+s : r) \]
from which it follows that
$A_1 = (0 : s : r)$ and $B_1 = (t : 0 : r)$.

Next, compute $X = \left(
  \det \left[ \begin{smallmatrix} u+t & r \\ t & r \end{smallmatrix} \right]
  : \det \left[ \begin{smallmatrix} s & r \\ 0 & r \end{smallmatrix} \right]
  : 0 \right) = (u : s : 0)$.
Similarly, $Y = (t : u : 0)$.
So we have computed all points.
\begin{claim*}
  Line $B_1X$ has equation
  $-rs \cdot x + ru \cdot y + st \cdot z = 0$,
  while line $C_1 Y$ has equation
  $ru \cdot x - rt \cdot y + st \cdot z = 0$.
\end{claim*}
\begin{proof}
  Line $B_1X$ is $0 = \det(B_1, X, -)
  = \det \left[ \begin{smallmatrix}
    t & 0 & r \\
    u & s & 0 \\
    x & y & z
\end{smallmatrix} \right]$.
  Line $C_1Y$ is analogous.
\end{proof}

\begin{claim*}
  The radical axis $(u+t) y - (u+s) x = 0$.
\end{claim*}
\begin{proof}
Circle $(AXC)$ is given by
$-a^2yz - b^2zx - c^2xy + (x+y+z) \cdot \frac{c^2 \cdot u}{u+s} y = 0$.
Similarly, circle $(BYC)$ has equation
$-a^2yz - b^2zx - c^2xy + (x+y+z) \cdot \frac{c^2 \cdot u}{u+t} x = 0$.
Subtracting gives the radical axis.
\end{proof}

Finally, to see these three lines are concurrent, we now compute
\begin{align*}
  \det \begin{bmatrix}
    -rs & ru & st \\
    ru & -rt & st \\
    -(u+s) & u+t & 0
  \end{bmatrix}
  &= rst \left[ [u(u+t)-t(u+s)] + [s(u+t)-u(u+s)] \right] \\
  &= rst \left[ (u^2-st) + (st-u^2) \right] = 0.
\end{align*}
This completes the proof.

\paragraph{Second official solution by tricky angle chasing.}
Let lines $AA_1$ and $BB_1$ meet at the circumcircle of $\triangle ABC$
again at points $A_2$ and $B_2$.
By Reim's theorem, $PQA_2B_2$ are cyclic.
\begin{center}
\begin{asy}
pair C = dir(110);
pair A = dir(210);
pair B = dir(330);

pair A_1 = 0.45*C+0.55*B;
pair P = 0.53*A+0.47*A_1;
pair B_1 = midpoint(A--C);
pair Q = extension(B, B_1, P, P+A-B);
pair T = extension(A_1, Q, B_1, P);
pair X = extension(A, B, P, T);
pair Y = extension(A, B, Q, T);

filldraw(A--B--C--cycle, opacity(0.1)+lightcyan, lightblue);
pair P_1 = -X+2*foot(circumcenter(A, X, C), P, X);
pair Q_1 = -Y+2*foot(circumcenter(B, Y, C), Q, Y);
pair A_2 = -A+2*foot(origin, A, P);
pair B_2 = -B+2*foot(origin, B, Q);
filldraw(unitcircle, opacity(0.1)+lightcyan, lightblue);

draw(A--A_2, lightblue);
draw(B--B_2, lightblue);

filldraw(circumcircle(C, B_1, B_2), opacity(0.05)+orange, red);
filldraw(circumcircle(C, A_1, A_2), opacity(0.05)+orange, red);
draw(P_1--P--Q--Q_1, blue);
filldraw(circumcircle(P, Q, Q_1), opacity(0.1)+yellow, deepgreen+dashed);

dot("$C$", C, dir(C));
dot("$A$", A, dir(A));
dot("$B$", B, dir(B));
dot("$A_1$", A_1, dir(A_1));
dot("$P$", P, dir(270));
dot("$B_1$", B_1, dir(B_1));
dot("$Q$", Q, dir(270));
dot("$P_1$", P_1, dir(P_1));
dot("$Q_1$", Q_1, dir(Q_1));
dot("$A_2$", A_2, dir(A_2));
dot("$B_2$", B_2, dir(B_2));

/* TSQ Source:

C = dir 110
A = dir 210
B = dir 330

A_1 = 0.45*C+0.55*B
P = 0.53*A+0.47*A_1 R270
B_1 = midpoint A--C
Q = extension B B_1 P P+A-B R270
T := extension A_1 Q B_1 P
X := extension A B P T
Y := extension A B Q T

A--B--C--cycle 0.1 lightcyan / lightblue
P_1 = -X+2*foot circumcenter A X C P X
Q_1 = -Y+2*foot circumcenter B Y C Q Y
A_2 = -A+2*foot origin A P
B_2 = -B+2*foot origin B Q
unitcircle 0.1 lightcyan / lightblue

A--A_2 lightblue
B--B_2 lightblue

circumcircle C B_1 B_2 0.05 orange / red
circumcircle C A_1 A_2 0.05 orange / red
P_1--P--Q--Q_1 blue
circumcircle P Q Q_1 0.1 yellow / deepgreen dashed

*/
\end{asy}
\end{center}

\begin{claim*}
  The points $P$, $Q$, $A_2$, $Q_1$ are cyclic.
  Similarly the points $P$, $Q$, $B_2$, $P_1$ are cyclic.
\end{claim*}
\begin{proof}
  Note that $CA_1A_2Q_1$ is cyclic since
  $\dang CQ_1A_1 = \dang CQ_1Q = \dang CBA = \dang CA_2A = \dang CA_2A_1$.
  Then $\dang QQ_1A_2 = \dang A_1Q_1A_2 = \dang A_1 C A_2
  = \dang B C A_2 = \dang B A A_2 = \dang Q P A_2$.
\end{proof}

This claim obviously solves the problem.
