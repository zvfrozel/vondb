desc: Separating points with line with min distance 1
hardness: 50
source: IMO 2020/6
tags: [2020-09, nice, algorithm, brave, maturity, yod]
author: Ting-Feng Lin, Hung-Hsun Hans Yu (TWN)
url: https://aops.com/community/p17821732

---

Consider an integer $n > 1$, and a set $\mathcal S$ of $n$ points
in the plane such that the distance between any two different points
in $\mathcal S$ is at least $1$.
Prove there is a line $\ell$ separating $\mathcal S$
such that the distance from any point of $\mathcal S$ to $\ell$
is at least $\Omega(n^{-1/3})$.

(A line $\ell$ separates a set of points $S$
if some segment joining two points in $\mathcal S$ crosses $\ell$.)

---

We present the official solution given by the Problem Selection Committee.

Let's suppose that among all projections
of points in $\mathcal S$ onto some line $m$,
the maximum possible distance between two consecutive projections is $\delta$.
We will prove that $\delta \ge \Omega(n^{-1/3})$,
solving the problem.

We make the following the definitions:
\begin{itemize}
  \ii Define $A$ and $B$ as the two points farthest apart in $\mathcal S$.
  This means that all points lie in the intersections
  of the circles centered at $A$ and $B$ with radius $R = AB \ge 1$.
  \ii We pick chord $\ol{XY}$ of $\odot(B)$
  such that $\ol{XY} \perp \ol{AB}$ and the distance
  from $A$ to $\ol{XY}$ is exactly $\half$.
  \ii We denote by $\mathcal T$
  the smaller region bound by $\odot(B)$ and chord $\ol{XY}$.
\end{itemize}
The figure is shown below with $\mathcal T$ drawn in yellow,
and points of $\mathcal S$ drawn in blue.
\begin{center}
\begin{asy}
size(10cm);
pair A = (-2,0);
pair B = (5,0);

pen auxpen = brown+linetype("4 2");
pair X = (0, 24^0.5);
pair Y = (0, -24^0.5);

pair[] S = {
  (-1.3, 2.7),
  (-0.57, -2.61),
  (-0.21, 1.07),
  (0.33,-1.06),
  (0.83, 3.71),
  (1.57, -3.18),
  (2.14, 1.66),
  (2.72, 3.32),
  (3.37, -1.17),
  (3.97, 1.34),
  (4.37, -2.35),
};

fill(Y..A..X--cycle, paleyellow);
for (int i=0; i<S.length; ++i) {
  draw(S[i]--(S[i].x,0), grey);
  dot(S[i], blue);
}
draw(A--B, blue+1.3);
draw(arc(B,abs(B-A), 110, 250), auxpen);
draw(arc(A,abs(B-A), -70, 70), auxpen);
draw(B--X--Y--cycle, auxpen);

dot("$A$", A, dir(180), blue);
dot("$B$", B, dir(0), blue);
dot("$X$", X, dir(90), auxpen);
dot("$Y$", Y, dir(270), auxpen);

draw( (-1.9,-0.1)--(-1.9,-0.2)--(-0.1,-0.2)--(-0.1, -0.1) );
label("$\frac12$", (-1, -0.2), dir(-90));

draw( (2.14,-0.1)--(2.14,-0.2)--(2.72,-0.2)--(2.72,-0.1) );
label("$< \delta$", (2.43, -0.2), dir(-90), fontsize(9pt));

label("$\mathcal T$", (-0.7,3), fontsize(14pt));
\end{asy}
\end{center}

\begin{claim*}
  [Length of $AB$ + Pythagorean theorem]
  We have $XY < 2\sqrt{n\delta}$.
\end{claim*}
\begin{proof}
  First, note that we have $R = AB < (n-1) \cdot \delta$,
  since the $n$ projections of points onto $AB$
  are spaced at most $\delta$ apart.
  The Pythagorean theorem gives
  \[ XY = 2\sqrt{R^2 - \left(R-\half\right)^2}
    = 2\sqrt{R - \frac14} < 2\sqrt{n\delta}. \qedhere \]
\end{proof}
\begin{claim*}
  [$|\mathcal T|$ lower bound + narrowness]
  We have $XY > \frac{\sqrt3}{2} \left( \half \delta\inv - 1 \right)$.
\end{claim*}
\begin{proof}
  Because $\mathcal T$ is so narrow (has width $\half$ only),
  the projections of points in $\mathcal T$ onto line $XY$
  are spaced at least $\frac{\sqrt3}{2}$ apart (more than just $\delta$).
  This means
  \[ XY > \frac{\sqrt3}{2}
    \left( \left\lvert \mathcal T \right\rvert - 1 \right). \]
  But projections of points in $\mathcal T$
  onto the segment of length $\half$ are spaced at most $\delta$ apart,
  so apparently
  \[ \left\lvert \mathcal T \right\rvert > \half \cdot \delta\inv. \]
  This implies the result.
\end{proof}
Combining these two this implies $\delta \ge \Omega(n^{-1/3})$ as needed.

\begin{remark*}
  The constant $1/3$ in the problem is actually optimal
  and cannot be improved;
  the constructions give an example showing $\Theta(n^{-1/3} \log n)$.
\end{remark*}
