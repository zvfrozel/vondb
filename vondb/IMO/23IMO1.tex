desc: $d_i$ divides $d_{i+1}+d_{i+2}$
source: IMO 2023/1
url: https://aops.com/community/p28097575
tags: [2023-09, size, primes, smallestprime, find, divis, aleph]
hardness: 5
author: Santiago Rodriguez (COL)

---

Determine all composite integers $n>1$ that satisfy the following property:
if $d_1 < d_2 < \dots < d_k$ are all the positive divisors of $n$ with
then $d_i$ divides $d_{i+1} + d_{i+2}$ for every $1 \leq i \leq k - 2$.

---

The answer is prime powers.

\paragraph{Verification that these work.}
When $n = p^e$, we get $d_i = p^{i-1}$.
The $i$\ts{th} relationship reads \[ p^{i-1} \mid p^i + p^{i+1} \]
which is obviously true.

\paragraph{Proof that these are the only answers.}
Conversely, suppose $n$ has at least two distinct prime divisors.
Let $p < q$ denote the two smallest ones,
and let $p^e$ be the largest power of $p$ which both divides $n$
and is less than $q$, hence $e \ge 1$.
Then the smallest factors of $n$ are $1$, $p$, \dots, $p^e$, $q$.
So we are supposed to have
\[ \frac{n}{q} \mid \frac{n}{p^e} + \frac{n}{p^{e-1}}
    = \frac{(p+1)n}{p^e} \]
which means that the ratio
\[ \frac{q(p+1)}{p^e} \]
needs to be an integer, which is obviously not possible.
