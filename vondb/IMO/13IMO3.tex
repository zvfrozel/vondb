desc: Circumcenter of extouch on (ABC)
source: IMO 2013/3
tags: [2018-08, spiralsim, anglechase, good, complex, gimel]
author:  Alexander A. Polyansky (RUS)
hardness: 35
url: https://aops.com/community/p5720184

---

Let the excircle of triangle $ABC$ opposite
the vertex $A$ be tangent to the side $BC$ at the point $A_1$.
Define the points $B_1$ on $CA$ and $C_1$ on $AB$ analogously,
using the excircles opposite $B$ and $C$, respectively.
Suppose that the circumcenter of triangle $A_1B_1C_1$ lies
on the circumcircle of triangle $ABC$.
Prove that triangle $ABC$ is right-angled.

---

We ignore for now the given condition
and prove the following important lemma.

\begin{lemma*}
  Let $(AB_1C_1)$ meet $(ABC)$ again at $X$.
  From $BC_1 = B_1C$ follows $XC_1 = XB_1$,
  and $X$ is the midpoint of major arc $\widehat{BC}$.
\end{lemma*}
\begin{proof}
  This follows from the fact that we have
  a spiral similarity $\triangle XBC_1 \sim \triangle XCB_1$
  which must actually be a spiral congruence
  since $BC_1 = B_1C$.
\end{proof}

We define the arc midpoints $Y$ and $Z$ similarly,
which lie on the perpendicular bisectors of
$\ol{A_1 C_1}$, $\ol{A_1 B_1}$.

\begin{center}
\begin{asy}
pair A = dir(110);
pair B = dir(180);
pair C = dir(0);
pair X = dir(90);
pair Y = dir(235);
pair Z = dir(325);

pair A_1 = foot(Y+Z-X, B, C);
pair B_1 = foot(Z+X-Y, C, A);
pair C_1 = foot(X+Y-Z, A, B);

filldraw(unitcircle, opacity(0.1)+grey, grey);
draw(C_1--X--B_1, red);
draw(A--B--C--cycle, grey);
filldraw(A_1--B_1--C_1--cycle, opacity(0.1)+cyan, heavycyan);
draw(Y--X--Z, orange);
draw(circumcircle(X, B_1, C_1), dashed+grey);
draw(X--A_1, red+dashed);

dot("$A$", A, dir(A));
dot("$B$", B, dir(B));
dot("$C$", C, dir(C));
dot("$X$", X, dir(X));
dot("$Y$", Y, dir(Y));
dot("$Z$", Z, dir(Z));
dot("$A_1$", A_1, dir(270));
dot("$B_1$", B_1, dir(B_1));
dot("$C_1$", C_1, dir(C_1));

/* TSQ Source:

A = dir 110
B = dir 180
C = dir 0
X = dir 90
Y = dir 235
Z = dir 325

A_1 = foot Y+Z-X B C R270
B_1 = foot Z+X-Y C A
C_1 = foot X+Y-Z A B

unitcircle 0.1 grey / grey
C_1--X--B_1 red
A--B--C--cycle grey
A_1--B_1--C_1--cycle 0.1 cyan / heavycyan
Y--X--Z orange
circumcircle X B_1 C_1 dashed grey
X--A_1 red dashed

*/
\end{asy}
\end{center}

We now turn to the problem condition
which asserts the circumcenter $W$ of $\triangle A_1B_1C_1$
lies on $(ABC)$.
\begin{claim*}
  We may assume WLOG that $W = X$.
\end{claim*}
\begin{proof}
  This is just configuration analysis,
  since we already knew that the arc midpoints
  both lie on $(ABC)$ and the relevant perpendicular bisectors.

  Point $W$ lies on $(ABC)$ and hence outside $\triangle ABC$,
  hence outside $\triangle A_1 B_1 C_1$.
  Thus we may assume WLOG that $\angle B_1 A_1 C_1 > 90\dg$.
  Then $A$ and $X$ lie on the same side of line $\ol{B_1 C_1}$,
  and since $W$ is supposed to lie both on $(ABC)$
  and the perpendicular bisector of $\ol{B_1C_1}$ it follows $W = X$.
\end{proof}

Consequently, $\ol{XY}$ and $\ol{XZ}$
are exactly the perpendicular bisectors
of $\ol{A_1 C_1}$, $\ol{A_1 B_1}$.
The rest is angle chase, the fastest one is
\begin{align*}
  \angle A &= \angle C_1 X B_1
  = \angle C_1 X A_1 + \angle A_1 X B_1
  = 2 \angle YXA_1 + 2 \angle A_1 X Z \\
  &= 2 \angle YXZ =  180\dg - \angle A
\end{align*}
which solves the problem.

\begin{remark*}
  Angle chasing is also possible even without
  the points $Y$ and $Z$, though it takes much longer.
  Introduce the Bevan point $V$ and use the fact
  that $VA_1B_1C$ is cyclic (with diameter $\ol{VC}$)
  and similarly $VA_1C_1B$ is cyclic;
  a calculation then gives $\angle CVB = 180\dg - \half \angle A$.
  Thus $V$ lies on the circle with diameter $\ol{I_b I_c}$.
\end{remark*}
