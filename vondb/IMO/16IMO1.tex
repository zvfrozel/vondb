desc: Fact 5 in a pentagon $ABCDE$
source: IMO 2016/1
tags: [2018-08, rich, config, anglechase, instructive, bet]
author: Art Waeterschoot (BEL)
hardness: 15
url: https://aops.com/community/p6637656

---

In convex pentagon $ABCDE$ with $\angle B > 90\dg$,
let $F$ be a point on $\ol{AC}$ such that $\angle FBC = 90\dg$.
It is given that $FA=FB$, $DA=DC$, $EA=ED$,
and rays $\ol{AC}$ and $\ol{AD}$ trisect $\angle BAE$.
Let $M$ be the midpoint of $\ol{CF}$.
Let $X$ be the point such that $AMXE$ is a parallelogram.
Show that $\ol{FX}$, $\ol{EM}$, $\ol{BD}$ are concurrent.

---

Here is a ``long'' solution which I think
shows where the ``power'' in the configuration comes from
(it should be possible to come up with shorter solutions
by cutting more directly to the desired conclusion).
Throughout the proof, we let
\[ \theta = \angle FAB = \angle FBA = \angle DAC = \angle DCA
  = \angle EAD = \angle EDA. \]

We begin by focusing just on $ABCD$ with point $F$,
ignoring for now the points $E$ and $X$
(and to some extent even point $M$).
It turns out this is a very familiar configuration.
\begin{lemma*}
  [Central lemma]
  The points $F$ and $C$
  are the incenter and $A$-excenter of $\triangle DAB$.
  Moreover, $\triangle DAB$ is isosceles with $DA = DB$.
\end{lemma*}
\begin{proof}
  The proof uses three observations:
  \begin{itemize}
  \ii We already know that $\ol{FAC}$ is the angle bisector of $\angle ABD$.
  \ii We were given $\angle FBC = 90\dg$.
  \ii Next, note that $\triangle AFB \sim \triangle ADC$
  (they are similar isosceles triangles).
  From this it follows that $AF \cdot AC = AB \cdot AD$.
  \end{itemize}
  These three facts, together with $F$ lying inside $\triangle ABD$,
  are enough to imply the result.
\end{proof}
\begin{center}
  \begin{asy}
  pair F = dir(157);
  pair C = -F;
  pair B = conj(F);
  pair M = origin;
  pair A = dir(F-C)*abs(F-B)+F;

  pair t(pair X) { return A + (X-A) * (F-A) / (B-A); }

  pair D = t(C);

  filldraw(A--F--B--cycle, opacity(0.1)+red, grey);
  filldraw(A--C--D--cycle, opacity(0.1)+orange, grey);
  filldraw(F--B--C--cycle, opacity(0.1)+lightcyan, grey);

  draw(circumcircle(A, B, D), orange+dashed);
  draw(circumcircle(B, C, F), orange+dashed);
  draw(B--D, grey);
  draw(F--D, grey);
  draw(A--B--D--cycle, black+0.8);

  dot("$F$", F, dir(115));
  dot("$C$", C, dir(C));
  dot("$B$", B, dir(B));
  dot("$M$", M, dir(280));
  dot("$A$", A, dir(A));
  dot("$D$", D, dir(60));

  /* TSQ Source:

  F = dir 157 R115
  C = -F
  B = conj(F)
  M = origin R280
  A = dir(F-C)*abs(F-B)+F

  !pair t(pair X) { return A + (X-A) * (F-A) / (B-A); }

  D = t(C) R60

  A--F--B--cycle 0.1 red / grey
  A--C--D--cycle 0.1 orange / grey
  F--B--C--cycle 0.1 lightcyan / grey

  circumcircle A B D orange dashed
  circumcircle B C F orange dashed
  B--D grey
  F--D grey
  A--B--D--cycle black+0.8

  */
  \end{asy}
  \end{center}


\begin{corollary*}
  The point $M$ is the midpoint of arc $\widehat{BD}$ of $(DAB)$,
  and the center of cyclic quadrilateral $FDCB$.
\end{corollary*}
\begin{proof}
  Fact 5.
\end{proof}

Using these observations as the anchor
for everything that follows,
we now prove several claims about $X$ and $E$ in succession.
\begin{center}
\begin{asy}
pair F = dir(157);
pair C = -F;
pair B = conj(F);
pair M = origin;
pair A = dir(F-C)*abs(F-B)+F;

pair t(pair X) { return A + (X-A) * (F-A) / (B-A); }

pair D = t(C);
pair E = t(D);

pair X = E+M-A;

draw(E--F, lightgreen);
draw(D--M, lightgreen);
draw(B--D, lightblue);
draw(F--X, lightblue);
draw(M--E, lightblue);

filldraw(A--F--B--cycle, opacity(0.1)+red, grey);
filldraw(A--C--D--cycle, opacity(0.1)+orange, grey);
filldraw(A--E--D--cycle, opacity(0.1)+yellow, grey);
filldraw(F--B--C--cycle, opacity(0.1)+lightcyan, grey);

draw(A--E--X--M--cycle, heavygreen);

draw(circumcircle(E, F, X), lightgreen+dashed);
draw(circumcircle(A, B, D), orange);
draw(circumcircle(B, C, F), orange);

dot("$F$", F, dir(45));
dot("$C$", C, dir(C));
dot("$B$", B, dir(B));
dot("$M$", M, dir(260));
dot("$A$", A, dir(A));
dot("$D$", D, dir(60));
dot("$E$", E, dir(E));
dot("$X$", X, dir(X));

/* TSQ Source:

F = dir 157 R45
C = -F
B = conj(F)
M = origin R260
A = dir(F-C)*abs(F-B)+F

!pair t(pair X) { return A + (X-A) * (F-A) / (B-A); }

D = t(C) R60
E = t(D)

X = E+M-A

E--F lightgreen
D--M lightgreen
B--D lightblue
F--X lightblue
M--E lightblue

A--F--B--cycle 0.1 red / grey
A--C--D--cycle 0.1 orange / grey
A--E--D--cycle 0.1 yellow / grey
F--B--C--cycle 0.1 lightcyan / grey

A--E--X--M--cycle heavygreen

circumcircle E F X lightgreen dashed
circumcircle A B D orange
circumcircle B C F orange

*/
\end{asy}
\end{center}

\begin{claim*}
  Point $E$ is the midpoint of arc $\widehat{AD}$
  in $(ABMD)$, and hence lies on ray $BF$.
\end{claim*}
\begin{proof}
  This follows from
  $\angle EDA = \theta = \angle EBA$.
\end{proof}


\begin{claim*}
  Points $X$ is the second intersection of ray $\ol{ED}$
  with $(BFDC)$.
\end{claim*}
\begin{proof}
  First, $\ol{ED} \parallel \ol{AC}$ already since
  $\angle AED = 180\dg - 2\theta$
  and $\angle CAE = 2\theta$.

  Now since $DB = DA$, we get $MB = MD = ED = EA$.
  Thus, $MX = AE = MB$,
  so $X$ also lies on the circle $(BFDC)$ centered at $M$.
\end{proof}

\begin{claim*}
  The quadrilateral $EXMF$ is an isosceles trapezoid.
\end{claim*}
\begin{proof}
  We already know $\ol{EX} \parallel \ol{FM}$.
  Since $\angle EFA = 180\dg - \angle AFB = 2\theta = \angle FAE$,
  we have $EF = EA$ as well (and $F \neq A$).
  As $EXMA$ was a parallelogram,
  it follows $EXMF$ is an isosceles trapezoid.
\end{proof}

The problem then follows by radical axis theorem
on the three circles $(AEDMB)$, $(BFDXC)$ and $(EXMF)$.
