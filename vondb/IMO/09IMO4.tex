desc: Find angle $A$ given $\angle BEK = 45$
source: IMO 2009/4
tags: [2019-04, length, trig, find, aleph]
hardness: 15
url: https://aops.com/community/p1562847
author: Hojoo Lee, Peter Vandendriessche, Jan Vonk (BEL)

---

Let $ABC$ be a triangle with $AB = AC$.
The angle bisectors of $\angle CAB$ and $\angle ABC$
meet the sides $BC$ and $CA$ at $D$ and $E$, respectively.
Let $K$ be the incenter of triangle $ADC$.
Suppose that $\angle BEK = 45^\circ$.
Find all possible values of $\angle CAB$.

---

Here is the solution presented in my book \emph{EGMO}.

Let $I$ be the incenter of $ABC$,
and set $\angle DAC = 2x$ (so that $0\dg < x < 45\dg$).
From $\angle AIE = \angle DIC$, it is easy to compute
\[
  \angle KIE = 90\dg - 2x, \;
  \angle ECI = 45\dg -x, \;
  \angle IEK = 45\dg, \;
  \angle KEC = 3x. \]
Having chased all the angles we want, we need a relationship.
We can find it by considering the side ratio $\frac{IK}{KC}$.
Using the angle bisector theorem,
we can express this in terms of triangle $IDC$;
however we can also express it in terms of triangle $IEC$.


\begin{center}
  \begin{asy}
    size(7cm);
    pair A = Drawing("A", (0,7), dir(90));
    pair C = Drawing("C", (3,0), dir(-45));
    pair B = Drawing("B", -C, dir(225));
    draw(A--B--C--cycle);
    pair I = incenter(A,B,C);
    pair D = Drawing("D", origin, dir(-90));
    pair E = Drawing("E", extension(B,I,A,C), dir(45));
    pair K = Drawing("K", incenter(A,D,C), dir(-30));
    draw(A--D--K);
    draw(B--E--K);
    draw(incircle(A,D,C));
    markangle(9.0,B,E,K);
  \end{asy}
  \quad
  \begin{asy}
    pair A = Drawing("A", (0,5), dir(90));
    pair C = Drawing("C", (3,0), dir(-45));
    pair B = -C;
    pair D = Drawing("D", origin, dir(-90));
    pair I = Drawing("I", incenter(A,B,C), dir(180));
    pair E = Drawing("E", extension(B,I,A,C), dir(45));
    pair K = Drawing("K", incenter(A,D,C), 1.3*dir(-90));
    draw(A--D--C--cycle);
    draw(C--I--E--K);
    draw(D--K);
    MP("2x", A, dir(K-A)*8);
    MP(rotate(-35)*"45^{\circ}-x", C+dir(130)*1.2);
    MP(rotate(-10)*"45^{\circ}-x", C+dir(155));
    MP("45^{\circ}", E, dir(240)*6);
    MP("3x", E, dir(-80)*6);
    markangle(9.0,D,I,C);
    markangle(9.0,E,I,A);
  \end{asy}
\end{center}

By the law of sines, we obtain
\[ \frac{IK}{KC} = \frac%
  {\sin 45\dg \cdot \frac{EK}{\sin \left( 90\dg - 2x \right)}}%
  {\sin \left( 3x \right) \cdot \frac{EK}{\sin \left( 45\dg-x \right)}}
  = \frac{\sin 45\dg \sin \left( 45\dg - x \right)}%
  {\sin \left( 3x \right) \sin \left( 90\dg - 2x \right)}. \]
Also, by the angle bisector theorem on $\triangle IDC$,
we have
\[ \frac{IK}{KC} = \frac{ID}{DC}
  = \frac{\sin \left( 45\dg-x \right)}{\sin\left( 45\dg+x \right)}. \]
Equating these and cancelling $\sin \left( 45\dg-x \right) \neq 0$ gives
\[ \sin 45\dg \sin \left( 45\dg + x \right)
  = \sin 3x \sin \left( 90\dg - 2x \right). \]

Applying the product-sum formula
(again, we are just trying to break down things as much as possible),
this just becomes
\[ \cos\left( x \right) - \cos \left( 90\dg + x \right)
  = \cos \left( 5x-90\dg \right) - \cos \left( 90\dg+x \right) \]
or $\cos x = \cos \left( 5x-90\dg \right)$.

At this point we are basically done;
the rest is making sure we do not miss any solutions
and write up the completion nicely.
One nice way to do this is by using product-sum in reverse as
\[ 0 = \cos \left( 5x-90\dg \right) - \cos x
  = 2 \sin \left( 3x - 45\dg \right) \sin \left( 2x-45\dg \right). \]
This way we merely consider the two cases
\[ \sin \left( 3x-45\dg \right) = 0 \text{ and }
  \sin \left( 2x - 45\dg \right) = 0. \]
Notice that $\sin\theta = 0$ if and only $\theta$
is an integer multiple of $180\dg$.
Using the bound $0\dg < x < 45\dg$,
it is easy to see that that the permissible values of $x$
are $x = 15\dg$ and $x = \frac{45}{2}\dg$.
As $\angle A = 4x$, this corresponds to $\angle A = 60\dg$
and $\angle A = 90\dg$, which can be seen to work.
