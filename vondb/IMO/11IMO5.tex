desc: $f(m-n) \mid f(m)-f(n)$ from $\ZZ \to \ZZ_{>0}$
source: IMO 2011/5
tags: [2018-03, divis, euclid, size, rushdown, reliable, bet]
author: Mahyar Sefidgaran (IRN)
hardness: 15
url: https://aops.com/community/p2365041

---

Let $f \colon \ZZ \to \ZZ_{>0}$ be a function such that
$f(m-n) \mid f(m) - f(n)$ for $m,n \in \ZZ$.
Prove that if $m,n \in \ZZ$ satisfy $f(m) \le f(n)$
then $f(m) \mid f(n)$.

---

Let $P(m,n)$ denote the given assertion.
First, we claim $f$ is even.
This is straight calculation:
\begin{itemize}
  \ii $P(x,0) \implies f(x) \mid f(x)-f(0)
  \implies f(x) \mid M \coloneqq f(0)$.
  \ii $P(0,x) \implies f(-x) \mid M-f(x) \implies
  f(-x) \mid f(x)$.
  Analogously, $f(x) \mid f(-x)$.
  So $f(x) = f(-x)$ and $f$ is even.
\end{itemize}

\begin{claim*}
  Let $x$, $y$, $z$ be integers with $x+y+z=0$.
  Then among $f(x)$, $f(y)$, $f(z)$,
  two of them are equal and divide the third.
\end{claim*}
\begin{proof}
  Let $a = f(\pm x)$, $b = f(\pm y)$, $c = f(\pm z)$
  be positive integers.
  Note that
  \begin{align*}
    a &\mid b-c \\
    b &\mid c-a \\
    c &\mid a-b
  \end{align*}
  from $P(y,-z)$ and similarly.
  WLOG $c = \max(a,b,c)$; then $c > |a-b|$
  so $a=b$.  Thus $a=b \mid c$ from the first two.
\end{proof}
This implies the problem,
by taking $x$ and $y$ in the previous claim
to be the integers $m$ and $n$.

\begin{remark*}
  At \url{https://aops.com/community/c6h418981p2381909},
  Davi Medeiros gives the following characterization
  of functions $f$ satisfying the hypothesis.

  Pick $f(0)$, $k$ positive integers,
  a chain $d_1 \mid d_2 \mid \dots \mid d_k$ of divisors of $f(0)$,
  and positive integers $a_1,a_2,\dots,a_{k-1}$,
  greater than $1$ (if $k=1$, $a_i$ doesn't exist, for every $i$).
  We'll define $f$ as follows:
  \begin{itemize}
    \ii $f(n)=d_1$, for every integer $n$ that is not divisible by $a_1$;
    \ii $f(a_1n)=d_2$, for every integer $n$ that is not divisible by $a_2$;
    \ii $f(a_1a_2n)=d_3$, for every integer $n$ that is not divisible by $a_3$;
    \ii $f(a_1a_2a_3n)=d_4$, for every integer $n$ that is not divisible by $a_4$;
    \ii \dots
    \ii $f(a_1a_2 \dots a_{k-1}n)=d_k$, for every integer $n$;
  \end{itemize}
\end{remark*}
