desc: From $SC = SP$ follows $MK = ML$
source: IMO 2010/4
tags: [2018-08, harmonic, nice, projective, anglechase, instructive, bet]
hardness: 10
url: https://aops.com/community/p1936916
author: Marcin Kuczma (POL)

---

Let $P$ be a point interior to triangle $ABC$ (with $CA \neq CB$).
The lines $AP$, $BP$ and $CP$ meet again its circumcircle $\Gamma$
at $K$, $L$, $M$, respectively.
The tangent line at $C$ to $\Gamma$ meets the line $AB$ at $S$.
Show that from $SC = SP$ follows $MK = ML$.

---

We present two solutions using harmonic bundles.

\paragraph{First solution (Evan Chen).}
Let $N$ be the antipode of $M$, and let $NP$ meet $\Gamma$ again at $D$.
Focus only on $CDMN$ for now (ignoring the condition).
Then $C$ and $D$ are feet of altitudes in $\triangle MNP$;
it is well-known that the circumcircle of $\triangle CDP$
is orthogonal to $\Gamma$
(passing through the orthocenter of $\triangle MPN$).
\begin{center}
\begin{asy}
pair N = dir(100);
pair M = -N;
pair L = dir(150);
pair K = N*N/L;
pair C = dir(122);
pair D = conj(C);
pair P = extension(D, N, M, C);
pair A = -K+2*foot(origin, K, P);
pair B = -L+2*foot(origin, L, P);
pair S = circumcenter(C, P, D);
filldraw(unitcircle, opacity(0.1)+lightgreen, heavygreen);
draw(C--S--D, heavygreen);
draw(A--C--B--S, lightblue);
draw(A--D--B, lightblue);
filldraw(L--N--K--M--cycle, opacity(0.1)+lightred, lightred);
draw(M--N, lightred);
draw(K--L, lightred);
draw(A--K, dashed+orange);
draw(B--L, dashed+orange);
draw(C--M, dashed+orange);
draw(D--N, dashed+orange);
draw(S--P, heavygreen);
draw(arc(S,abs(S-D),-60,60), heavygreen);

dot("$N$", N, dir(N));
dot("$M$", M, dir(M));
dot("$L$", L, dir(170));
dot("$K$", K, dir(K));
dot("$C$", C, dir(80));
dot("$D$", D, dir(280));
dot("$P$", P, dir(P));
dot("$A$", A, dir(220));
dot("$B$", B, dir(B));
dot("$S$", S, dir(S));

/* TSQ Source:

N = dir 100
M = -N
L = dir 150 R170
K = N*N/L
C = dir 122 R80
D = conj(C) R280
P = extension D N M C
A = -K+2*foot origin K P R220
B = -L+2*foot origin L P
S = circumcenter C P D
unitcircle 0.1 lightgreen / heavygreen
C--S--D heavygreen
A--C--B--S lightblue
A--D--B lightblue
L--N--K--M--cycle 0.1 lightred / lightred
M--N lightred
K--L lightred
A--K dashed orange
B--L dashed orange
C--M dashed orange
D--N dashed orange
S--P heavygreen
!draw(arc(S,abs(S-D),-60,60), heavygreen);

*/
\end{asy}
\end{center}
Now, we are given that point $S$ is such that $\ol{SC}$
is tangent to $\Gamma$, and $SC = SP$.
It follows that $S$ is the circumcenter of $\triangle CDP$,
and hence $\ol{SC}$ and $\ol{SD}$ are tangents to $\Gamma$.

Then $-1 = (AB;CD) \overset{P}{=} (KL;MN)$.
Since $\ol{MN}$ is a diameter, this implies $MK = ML$.

\begin{remark*}
  I think it's more natural to come up with
  this solution in reverse.
  Namely, suppose we define the points the other way:
  let $\ol{SD}$ be the other tangent, so $(AB;CD) = -1$.
  Then project through $P$ to get $(KL;MN) = -1$,
  where $N$ is the second intersection of $\ol{DP}$.
  However, if $ML = MK$ then $KMLN$ must be a kite.
  Thus one can recover the solution in reverse.
\end{remark*}

\paragraph{Second solution (Sebastian Jeon).}
We have \[ SP^2 = SC^2 = SA \cdot SB
  \implies
  \dang SPA = \dang PBA = \dang LBA = \dang LKA = \dang LKP \]
(the latter half is Reim's theorem).
Therefore $\ol{SP}$ and $\ol{LK}$ are \emph{parallel}.

Now, let $\ol{SP}$ meet $\Gamma$ again at $X$ and $Y$,
and let $Q$ be the antipode of $P$ on $(S)$.
Then
\[ SP^2 = SQ^2 = SX \cdot SY
  \implies (PQ;XY) = -1 \implies \angle QCP = 90\dg \]
that $\ol{CP}$ bisects $\angle XCY$.
Since $\ol{XY} \parallel \ol{KL}$,
it follows $\ol{CP}$ bisects to $\angle LCK$ too.

---

Here is a walkthrough of a projective solution.

Let $D$ denote the other tangency point from $S$.
Let $\ol{DP}$ meet $(ABC)$ again at $N$.
\begin{walk}
  \ii Show that $KMLN$ is a harmonic quadrilateral.
  \ii We want to prove $MK = ML$.
  What does that suggest should be true about $MLNK$?
\end{walk}
In light of (b), our goal is to show that $\ol{MN}$ is a diameter.

It will make more sense actually to
let $N'$ be the antipode of $M$ on $\Gamma$,
and let $D'$ be the second intersection of $N'P$ with $\Gamma$.
We will show $D' = D$.
\begin{walk}[resume]
  \ii Show that $(CD'P)$ is orthogonal to $\Gamma$.
  (Possible hint: ``three tangents'' lemma in \S1 of EGMO.)
  \ii Identify the circumcenter of $\triangle CD'P$.
  \ii Show that $\ol{SD'}$ is tangent to $\Gamma$, so $D' = D$.
\end{walk}

So we now know have everything we need:
we know both that $\ol{MN}$ is a diameter and $\ol{SD}$ is tangent.
\begin{walk}[resume]
  \ii Deduce that $KLNM$ is a kite, and thus $NL = NK$.
\end{walk}
