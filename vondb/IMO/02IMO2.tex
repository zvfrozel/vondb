desc:  Trivial application of Fact 5
source:  IMO 2002/2
tags:  [anglechase, trivial]
hardness: 10
url: https://aops.com/community/p118672
author: Hojoo Lee (KOR)

---

Let $BC$ be a diameter of circle $\omega$ with center $O$. Let $A$ be a point
of circle $\omega$ such that $0^\circ < \angle AOB < 120^\circ$. Let $D$ be the midpoint
of arc $AB$ not containing $C$. Line $\ell$ passes through $O$ and is
parallel to line $AD$. Line $\ell$ intersects line $AC$ at $J$.
The perpendicular bisector of segment $OA$ intersects circle $\omega$ at $E$
and $F$. Prove that $J$ is the incenter of triangle $CEF$.

---

By construction, $AEOF$ is a rhombus with $60\dg$-$120\dg$ angles.
Consequently, we may set $s = AO = AE = AF = EO = EF$.
\begin{center}
\begin{asy}
pair B = dir(0);
pair C = dir(180);
pair D = dir(37);
pair A = D*D;
pair E = dir(60)*A;
pair F = dir(-60)*A;
pair O = origin;
pair J = A+O-D;
filldraw(unitcircle, opacity(0.1)+lightcyan, heavycyan);

filldraw(A--B--C--cycle, opacity(0.1)+lightcyan, heavycyan);
draw(A--D, red);
draw(O--J, red);
filldraw(C--E--F--cycle, opacity(0.1)+green, heavygreen);
filldraw(A--E--O--F--cycle, opacity(0.1)+lightblue, blue);
draw(E--F, blue);
draw(A--O, blue);

dot("$B$", B, dir(B));
dot("$C$", C, dir(C));
dot("$D$", D, dir(D));
dot("$A$", A, dir(A));
dot("$E$", E, dir(E));
dot("$F$", F, dir(F));
dot("$O$", O, dir(-90));
dot("$J$", J, dir(J));

/* TSQ Source:

B = dir 0
C = dir 180
D = dir 37
A = D*D
E = dir(60)*A
F = dir(-60)*A
O = origin R-90
J = A+O-D
unitcircle 0.1 lightcyan / heavycyan

A--B--C--cycle 0.1 lightcyan / heavycyan
A--D red
O--J red
C--E--F--cycle 0.1 green / heavygreen
A--E--O--F--cycle 0.1 lightblue / blue
E--F blue
A--O blue

*/
\end{asy}
\end{center}
\begin{claim*}
  We have $AJ = s$ too.
\end{claim*}
\begin{proof}
  It suffices to show $AJ = AO$ which is angle chasing.
  Let $\theta = \angle BOD = \angle DOA$,
  so $\angle BOA = 2\theta$.
  Thus $\angle CAO = \half \angle BOA = \theta$.
  However $\angle AOJ = \angle OAD = 90\dg - \half \theta$,
  as desired.
\end{proof}
Then, since $AE = AJ = AF$,
we are done by the infamous Fact 5.
