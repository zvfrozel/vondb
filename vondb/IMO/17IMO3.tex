desc: Hunter and rabbit
author: Gerhard Woeginger (AUT)
source: IMO 2017/3
tags: [2017-07, process, adhoc, wishful, rushdown, inefficient, free, design, stronger, dumb, understand, nice, yesno, gimel]
hardness: 40
url: https://aops.com/community/p8633324

---

A hunter and an invisible rabbit play a game in the plane.
The rabbit and hunter start at points $A_0 = B_0$.
In the $n$th round of the game ($n \ge 1$), three things occur in order:
\begin{enumerate}[(i)]
  \ii The rabbit moves invisibly from $A_{n-1}$ to a point $A_n$
  such that $A_{n-1} A_n = 1$.
  \ii The hunter has a tracking device (e.g.\ dog)
  which reports an approximate location $P_n$ of the rabbit,
  such that $P_n A_n \le 1$.
  \ii The hunter moves visibly from $B_{n-1}$ to a point $B_n$
  such that $B_{n-1} B_n = 1$.
\end{enumerate}
Let $N = 10^9$. Can the hunter guarantee that $A_N B_N < 100$?

---

No, the hunter cannot.
We will show how to increase the distance in the following way:

\begin{claim*}
  Suppose the rabbit is at a distance $d \ge 1$ from the hunter
  at some point in time.
  Then it can increase its distance to at least
  $\sqrt{d^2+\half}$ in $4d$ steps
  regardless of what the hunter already knows about the rabbit.
\end{claim*}
\begin{proof}
  Consider a positive integer $n > d$, to be chosen later.
  Let the hunter start at $B$ and the rabbit at $A$, as shown.
  Let $\ell$ denote line $AB$.

  Now, we may assume the rabbit reveals its location $A$,
  so that all previous information becomes irrelevant.

  The rabbit chooses two points $X$ and $Y$ symmetric about $\ell$
  such that $XY = 2$ and $AX = AY = n$, as shown.
  The rabbit can then hop to either $X$ or $Y$,
  pinging the point $P_n$ on the $\ell$ each time.
  This takes $n$ hops.
  \begin{center}
  \begin{asy}
    pair A = 2*dir(0);
    pair B = dir(180);
    pair H = B+4*A;
    pair X = (2+63**0.5,1);
    pair Y = (2+63**0.5,-1);

    draw(B--H, blue);
    draw(A--X, red, EndArrow(TeXHead), Margins);
    draw(A--Y, red, EndArrow(TeXHead), Margins);
    pair M = midpoint(X--Y);
    draw(H--M, dotted);
    draw(X--Y, dotted);
    draw(X--H--Y, dashed+heavygreen);

    dot("$A$", A, dir(270));
    label("rabbit", A, dir(90));
    dot("$B$", B, dir(270));
    label("hunter", B, dir(90));
    dot("$H$", H, dir(270));
    dot("$X$", X, dir(X));
    dot("$Y$", Y, dir(Y));
    dot("$M$", M, dir(M));
    label("$n$", A--X, dir(90), red);
    label("$n$", A--Y, dir(-90), red);

    /* TSQ Source:

    A = 2*dir(0) R270
    B = dir(180) R270
    H = B+4*A R270
    X = (2+63**0.5,1)
    Y = (2+63**0.5,-1)

    B--H blue
    X--A--Y red
    M = midpoint X--Y
    H--M dotted
    X--Y dotted
    X--H--Y dashed heavygreen

    */
  \end{asy}
  \end{center}

  Now among all points $H$ the hunter can go to,
  $\min \max \{HX, HY\}$ is clearly minimized with $H \in \ell$ by symmetry.
  So the hunter moves to a point $H$ such that $BH = n$ as well.
  In that case the new distance is $HX = HY$.

  We now compute
  \begin{align*}
    HX^2 &= 1 + HM^2 = 1 + \left( \sqrt{AX^2-1}-AH \right)^2 \\
    &= 1 + \left( \sqrt{n^2-1}-(n-d) \right)^2 \\
    &\ge 1 + \left( \left( n-\frac1n \right) - (n-d) \right)^2 \\
    &= 1 + (d-1/n)^2
  \end{align*}
  which exceeds $d^2 + \half$ whenever $n \ge 4d$.
\end{proof}

In particular we can always take $n = 400$ even very crudely;
applying the lemma $2 \cdot 100^2$ times,
this gives a bound of $400 \cdot 2 \cdot 100^2 < 10^9$, as desired.

\begin{remark*}
  The step of revealing the location of the rabbit seems
  critical because as far as I am aware it is basically
  impossible to keep track of ping locations in the problem.
\end{remark*}

\begin{remark*}
  Reasons to believe the answer is ``no'':
  the $10^9$ constant,
  and also that ``follow the last ping'' is losing for the hunter.
\end{remark*}

\begin{remark*}
  I think there are roughly two ways you can approach the problem
  once you recognize the answer.

  \begin{enumerate}[(i)]
    \ii Try and control the location of the pings
    \ii Abandon the notion of controlling possible locations,
    and try to increase the distance by a little bit,
    say from $d$ to $\sqrt{d^2+\varepsilon}$.
    This involves revealing the location of the rabbit
    before each iteration of several jumps.
  \end{enumerate}
  I think it's clear that the difficulty of
  my approach is realizing that (ii) is possible;
  once you do, the two-point approach is more or less the only one possible.

  My opinion is that (ii) is not that magical;
  as I said it was the first idea I had.
  But I am biased, because when I test-solved the problem
  at the IMO it was called ``C5'' and not ``IMO3'';
  this effectively told me it was unlikely that the official solution
  was along the lines of (i),
  because otherwise it would have been placed much later in the shortlist.
\end{remark*}
