desc: Alternating numbers
source: IMO 2004/6
tags: [2018-06, construct, adhoc, cases, neatness, rushdown, reliable, mods, CRT, free, gimel]
hardness: 35
url: https://aops.com/community/p99760
author:  Mohsen Jamaali and Armin Morabbi (IRN)

---

We call a positive integer \emph{alternating} if every two consecutive digits
in its decimal representation are of different parity.
Find all positive integers $n$ which have an alternating multiple.

---

If $20 \mid n$, then clearly $n$ has no alternating
multiple since the last two digits are both even.
We will show the other values of $n$ all work.

The construction is just rush-down do-it.
The meat of the solution is the two following steps.
\begin{claim*}
  [Tail construction]
  For every even integer $w \ge 2$,
  \begin{itemize}
    \ii there exists an even alternating multiple $g(w)$ of $2^{w+1}$
    with exactly $w$ digits, and
    \ii there exists an even alternating multiple $h(w)$ of $5^{w}$
    with exactly $w$ digits.
  \end{itemize}
\end{claim*}
(One might note this claim is implied by the problem, too.)
\begin{proof}
  We prove the first point by induction on $w$.
  If $w = 2$, take $g(2) = 32$.
  In general, we can construct $g(w+2)$ from $g(w)$
  by adding some element in
  \[ 10^w \cdot \{10, 12, 14, 16, 18, 30, \dots, 98\} \]
  to $g(w)$, corresponding to the digits
  we want to append to the start.
  This multiple is automatically divisible by $2^{w+1}$,
  and also can take any of the four possible values modulo $2^{w+3}$.

  The second point is a similar induction,
  with base case $h(2) = 50$.
  The same set above consists of numbers divisible by $5^w$,
  and covers all residues modulo $5^{w+2}$.
  Careful readers might recognize the second part
  as essentially USAMO 2003/1.
\end{proof}

\begin{claim*}
  [Head construction]
  If $\gcd(n,10) = 1$, then for any $b$,
  there exists an even alternating number $f(b \bmod n)$ which is
  $b \pmod n$.
\end{claim*}
\begin{proof}
  A standard argument shows that
  \[ 10 \cdot \frac{100^m-1}{99}
    = \underbrace{1010\dots10}_{m\ 10\text{'s}}
    \equiv 0 \pmod n \]
  for any $m$ divisible by $\varphi(99n)$.
  Take a very large such $m$,
  and then add on $b$ distinct numbers of the form $10^{\varphi(n)r}$
  for various even values of $r$; these all are $1 \pmod n$
  and change some of the $1$'s to $3$'s.
\end{proof}


Now, we can solve the problem.
Consider three cases:
\begin{itemize}
  \ii If $n = 2^k m$ where $\gcd(m,10) = 1$ and $k \ge 2$ is even,
  then the concatenated number
  \[ 10^k f\left( -\frac{g(k)}{10^k} \bmod m \right) + g(k) \]
  works fine.

  \ii If $n = 5^k m$ where $\gcd(m,10) = 1$ and $k \ge 2$ is even,
  then the concatenated number
  \[ 10^k f\left( -\frac{h(k)}{10^k} \bmod m \right) + h(k) \]
  works fine.

  \ii If $n = 50m$ where $\gcd(m,10) = 1$,
  then the concatenated number
  \[ 100 f\left( -\frac{1}{2} \bmod m \right) + 50 \]
  works fine.
\end{itemize}
Since every non-multiple of $20$ divides such a number, we are done.
