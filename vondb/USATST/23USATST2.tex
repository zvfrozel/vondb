author: Kevin Cong
desc: Exsim $K$ of $(BME)$ and $(CMF)$ on $ABC$; $AK \perp BC$
source: USA TST 2023/2
url: https://aops.com/community/p26685484
tags: [2022-12, inversion, anglechase, pop, config, good, gimel]

---

Let $ABC$ be an acute triangle.
Let $M$ be the midpoint of side $BC$,
and let $E$ and $F$ be the feet of the altitudes from $B$ and $C$, respectively.
Suppose that the common external tangents
to the circumcircles of triangles $BME$ and $CMF$ intersect at a point $K$,
and that $K$ lies on the circumcircle of $ABC$.
Prove that line $AK$ is perpendicular to line $BC$.

---

We present several distinct approaches.

\paragraph{Inversion solution submitted by Ankan Bhattacharya and Nikolai Beluhov.}
Let $H$ be the orthocenter of $\triangle ABC$.
We use inversion in the circle with diameter $\ol{BC}$.
We identify a few images:
\begin{itemize}
  \ii The circumcircles of $\triangle BME$ and $\triangle CMF$ are mapped to lines $BE$ and $CF$.
  \ii The common external tangents are mapped to the two circles through $M$
  which are tangent to lines $BE$ and $CF$.
  \ii The image of $K$, denoted $K^\ast$, is the second intersection of these circles.
  \ii The assertion that $K$ lies on $(ABC)$ is equivalent to $K^\ast$ lying on $(BHC)$.
\end{itemize}
However, now $K^\ast$ is simple to identify directly:
it's just the reflection of $M$ in the bisector of $\angle BHC$.

\begin{center}
\begin{asy}
size(12cm);

pair B = dir(202.82);
pair C = dir(337.18);
pair A = dir(136);

pair K = -B*C/A;
pair E = foot(B, C, A);
pair F = foot(C, A, B);
draw(B--E, lightred);
draw(C--F, lightred);
draw(A--K, lightred);

pair M = midpoint(B--C);
filldraw(unitcircle, opacity(0.1)+yellow, red);
filldraw(A--B--C--cycle, opacity(0.1)+yellow, red);
filldraw(circumcircle(B, M, E), opacity(0.1)+cyan, grey);
filldraw(circumcircle(C, M, F), opacity(0.1)+cyan, grey);

pair O_2 = circumcenter(C, M, F);
pair K_u = IP(CP(midpoint(K--O_2), K), CP(O_2, M));
pair K_v = OP(CP(midpoint(K--O_2), K), CP(O_2, M));
draw(K_u--K--K_v, grey);

pair H = orthocenter(A, B, C);
pair Q = foot(A, H, M);
filldraw(circumcircle(B, H, C), opacity(0.1)+green, deepgreen);
pair K_asterisk = ((2*foot(Q, B, C))-Q);

draw(circumcircle(K_asterisk, B, M), dotted);
draw(circumcircle(K_asterisk, C, M), dotted);

clip(box((-2,-2),(2,2)));


dot("$B$", B, dir(B));
dot("$C$", C, dir(C));
dot("$A$", A, dir(A));
dot("$K$", K, dir(K));
dot("$E$", E, dir(70));
dot("$F$", F, dir(55));
dot("$M$", M, dir(285));
dot("$H$", H, dir(315));
dot("$K^{\ast}$", K_asterisk, dir(K_asterisk));

/* -----------------------------------------------------------------+
|                 TSQX: by CJ Quines and Evan Chen                  |
| https://github.com/vEnhance/dotfiles/blob/main/py-scripts/tsqx.py |
+-------------------------------------------------------------------+
!size(12cm);
B = dir 202.82
C = dir 337.18
A = dir 136
K = -B*C/A
E 70= foot B C A
F 55 = foot C A B
B--E lightred
C--F lightred
A--K lightred
M 285 = midpoint B--C
unitcircle / 0.1 yellow / red
A--B--C--cycle / 0.1 yellow / red
circumcircle B M E / 0.1 cyan / grey
circumcircle C M F / 0.1 cyan / grey
O_2 := circumcenter C M F
K_u := IP (CP (midpoint K--O_2) K) (CP O_2 M)
K_v := OP (CP (midpoint K--O_2) K) (CP O_2 M)
K_u--K--K_v / grey
H 315 = orthocenter A B C
Q := foot A H M
circumcircle B H C / 0.1 green / deepgreen
K& = (minus (mult 2 (foot Q B C)) Q)
circumcircle K& B M / dotted
circumcircle K& C M / dotted
!clip(box((-2,-2),(2,2)));
*/
\end{asy}
\end{center}

In particular, $\ol{HK^\ast}$ is a symmedian of $\triangle BHC$.
However, since $K^\ast$ lies on $(BHC)$,
this means $(HK^\ast; BC) = -1$.

Then, we obtain that $\ol{BC}$ bisects $\angle HMK^\ast \equiv \angle HMK$.
However, $K$ also lies on $(ABC)$,
which forces $K$ to be the reflection of $H$ in $\ol{BC}$.
Thus $\ol{AK} \perp \ol{BC}$, as wanted.

\paragraph{Solution with coaxial circles (Pitchayut Saengrungkongka).}

Let $H$ be the orthocenter of $\triangle ABC$. Let $Q$ be the second intersection of $\odot(BME)$ and $\odot(CMF)$.
We first prove the following well-known properties of $Q$.
\begin{claim*}
$Q$ is the Miquel point of $BCEF$. In particular, $Q$ lies on both $\odot(AEF)$ and $\odot(ABC)$.
\end{claim*}

\begin{proof}
  Follows since $BCEF$ is cyclic with $M$ being the circumcenter.
\end{proof}

\begin{claim*}
  $A(Q,H;B, C) = -1$.
\end{claim*}

\begin{proof}
By the radical center theorem on $\odot(AEF)$, $\odot(ABC)$, and $\odot(BCEF)$,
we get that $AQ$, $EF$, and $BC$ are concurrent. Now, the result follows from a
well-known harmonic property.
\end{proof}

Now, we get to the meat of the solution. Let the circumcircle of
$\odot(QMK)$ meet $BC$ again at $T\neq M$. The key claim is the following.
\begin{claim*}
$QT$ is tangent to $\odot(BQC)$.
\end{claim*}
\begin{proof}
    We use the ``forgotten coaxiality lemma".
    \begin{align*}
	\frac{BT}{TC} &= \frac{TB\cdot TM}{TC\cdot TM} \\
		      &= \frac{\operatorname{pow}(T, \odot(BME))}
		      {\operatorname{pow}(T, \odot(CMF))} \\
		      &= \frac{\operatorname{pow}(K, \odot(BME))}
		      {\operatorname{pow}(K, \odot(CMF))} \\
      &= \left(\frac{r_{\odot(BME)}}{r_{\odot(CMF)}}\right)^2 \\
	&= \left(\frac{BQ/\sin\angle QMB}{CQ/\sin\angle QMC}\right)^2 \\
	&= \frac{BQ^2}{CQ^2},
\end{align*}
implying the result.
\end{proof}

To finish, let $O$ be the center of $\odot(ABC)$. Then, from the claim, $\angle OQT = 90^\circ = \angle OMT$, so $O$ also lies on $\odot(QMTK)$.
Thus, $\angle OKT=90^\circ$, so $KT$ is also tangent to $\odot(ABC)$ as well. This implies that $QBKC$ is harmonic quadrilateral, and the result follows from the second claim.

\paragraph{Solution by Luke Robitaille.}
Let $Q$ be the second intersection of $\odot(BME)$ and
$\odot(CMF)$. We use the first two claims of the previous solution.
In particular, $Q\in\odot(ABC)$. We have the following claim.
\begin{claim*}[Also appeared in ISL 2017 G7]
We have $\measuredangle QKM = \measuredangle QBM + \measuredangle QCM$.
\end{claim*}
\begin{proof}
    Let $KQ$ and $KM$ meet $\odot(BME)$ again at $Q'$ and $M'$. Then, by homothety, $\measuredangle Q'QM' = \measuredangle QCM$, so
    \begin{align*}
        \measuredangle QKM
	&= \measuredangle Q'QM' + \measuredangle QM'M \\
	&= \measuredangle QCM + \measuredangle QBM,
    \end{align*}
    as desired.
\end{proof}
Now, we extend $KM$ to meet $\odot(ABC)$ again at $Q_1$. We have
\begin{align*}
    \measuredangle Q_1QB
    = \measuredangle Q_1KB
    &= \measuredangle Q_1KQ + \measuredangle QCB \\
    &= \measuredangle MKQ + \measuredangle QKB \\
    &= (\measuredangle MBQ + \measuredangle MCQ)
	+ \measuredangle QCB \\
    &= \measuredangle CBQ,
\end{align*}
implying that $QQ_1\parallel BC$. This implies that $QBKC$ is harmonic quadrilateral, so we are done.

\paragraph{Synthetic solution due to Andrew Gu (Harvard 2026).}
Define $O_1$ and $O_2$ as the circumcenters of $(BME)$ and $(CMF)$.
Let $T$ be the point on $(ABC)$ such that $\ol{AT} \perp \ol{BC}$.
Denote by $L$ the midpoint of minor arc $\arc{BC}$.

We are going to ignore the condition that $K$ lies on the circumcircle of $ABC$,
and prove the following unconditional result:
\begin{proposition*}
  The points $T$, $L$, $K$ are collinear.
\end{proposition*}
This will solve the problem because if $K$ is on the circumcircle of $ABC$,
it follows $K = T$ or $K = L$; but $K = L$ can never occur
since $O_1$ and $O_2$ are obviously on different sides of line $LM$
so line $LM$ must meet $O_1 O_2$ inside segment $O_1 O_2$,
and $K$ lies outside this segment.

\begin{center}
\begin{asy}
size(14cm);

pair B = dir(200);
pair C = dir(340);
pair A = dir(113);
pair L = dir(270);

pair T = -B*C/A;
pair E = foot(B, C, A);
pair F = foot(C, A, B);
draw(B--E, lightred);
draw(C--F, lightred);
draw(A--T, lightred);

pair M = midpoint(B--C);
filldraw(unitcircle, opacity(0.1)+yellow, red);
filldraw(A--B--C--cycle, opacity(0.1)+yellow, red);
pair Q = foot(A, M, -A);
filldraw(circumcircle(B, M, Q), opacity(0.1)+cyan, grey);
filldraw(circumcircle(C, M, Q), opacity(0.1)+cyan, grey);

pair O_2 = circumcenter(C, M, F);
pair O_1 = circumcenter(B, M, E);
pair P_2 = 2*O_2-M;
pair P_1 = 2*O_1-M;
draw(P_2--M--P_1, lightcyan);

pair O = circumcenter(A, B, C);
pair Q_2 = extension(A, B, M, O);
pair Q_1 = extension(A, C, M, O);
draw(L--Q_2--A, lightred);

draw(B--Q_1, lightcyan);
draw(C--Q_2, lightcyan);
draw(B--P_1--Q_1, blue);
draw(C--P_2--Q_2, blue);

pair K = extension(L, T, O_1, O_2);
pair K_u = IP(CP(midpoint(K--O_2), K), CP(O_2, M));
pair K_v = OP(CP(midpoint(K--O_2), K), CP(O_2, M));
draw(K_u--K--K_v, grey);
draw(K--L, deepgreen);
draw(K--O_2, grey);

pair N = dir(90);
draw(K--M, deepgreen);
draw(A--N, deepgreen);

dot("$B$", B, dir(B));
dot("$C$", C, dir(C));
dot("$A$", A, dir(A));
dot("$L$", L, dir(L));
dot("$T$", T, dir(T));
dot("$E$", E, dir(80));
dot("$F$", F, dir(35));
dot("$M$", M, dir(285));
dot("$Q$", Q, dir(150));
dot("$O_2$", O_2, dir(350));
dot("$O_1$", O_1, dir(315));
dot("$P_2$", P_2, dir(P_2));
dot("$P_1$", P_1, dir(130));
dot("$Q_2$", Q_2, dir(Q_2));
dot("$Q_1$", Q_1, dir(45));
dot("$K$", K, dir(K));
dot("$N$", N, dir(45));

/* -----------------------------------------------------------------+
|                 TSQX: by CJ Quines and Evan Chen                  |
| https://github.com/vEnhance/dotfiles/blob/main/py-scripts/tsqx.py |
+-------------------------------------------------------------------+
!size(16cm);
B = dir 200
C = dir 340
A = dir 113
L = dir 270
T = -B*C/A
E 80 = foot B C A
F 35 = foot C A B
B--E lightred
C--F lightred
A--T lightred
M 285 = midpoint B--C
unitcircle / 0.1 yellow / red
A--B--C--cycle / 0.1 yellow / red
Q 150 = foot A M -A
circumcircle B M Q / 0.1 cyan / grey
circumcircle C M Q / 0.1 cyan / grey
O_2 350 = circumcenter C M F
O_1 315 = circumcenter B M E
P_2 = 2*O_2-M
P_1 130 = 2*O_1-M
P_2--M--P_1 / lightcyan
O := circumcenter A B C
Q_2 = extension A B M O
Q_1 45 = extension A C M O
L--Q_2--A / lightred
B--Q_1 / lightcyan
C--Q_2 / lightcyan
B--P_1--Q_1 / blue
C--P_2--Q_2 / blue
K = extension L T O_1 O_2
K_u := IP (CP (midpoint K--O_2) K) (CP O_2 M)
K_v := OP (CP (midpoint K--O_2) K) (CP O_2 M)
K_u--K--K_v / grey
K--L deepgreen
K--O_2 / grey
N 45 = dir 90
K--M deepgreen
A--N deepgreen
*/
\end{asy}
\end{center}

We now turn to the proof of the main lemma.
Let $P_1$ and $P_2$ be the antipodes of $M$ on these circles.
\begin{claim*}
  Lines $AC$ and $LM$ meet at the antipode $Q_1$ of $B$ on $(BME)$,
  so that $BP_1Q_1M$ is a rectangle.
  Similarly,
  lines $AB$ and $LM$ meet at the antipode $Q_2$ of $C$ on $(CMF)$,
  so that $CP_2Q_2M$ is a rectangle.
\end{claim*}
\begin{proof}
  Let $Q_1' = \omega_1 \cap AC \neq E$.
  Then $\dang BMQ_1' = \dang BEQ_1' = 90^\circ$ hence $Q_1' \in LM$.
  The other half of the lemma follows similarly.
\end{proof}
From this, it follows that $P_1Q_1 = BM = \half BC = MC = P_2Q_2$.
Letting $r_1$ denote the radius of $\omega_1$
(and similarly for $\omega_2$), we deduce that $CQ_1 = BQ_1 =  2r_1$.

\begin{claim*}
  $KM = KL$.
\end{claim*}
\begin{proof}
  I first claim that $\ol{CL}$ is the external bisector of $\angle Q_1 C Q_2$;
  this follows from
  \[ \dang Q_1 C L = \dang ACL = \dang ABL = \dang Q_2 B L = \dang Q_2 C L. \]
  The external angle bisector theorem then gives an equality of directed ratios
  \[ \frac{LQ_1}{LQ_2} = \frac{|CQ_1|}{|CQ_2|} = \frac{|BQ_1|}{|CQ_2|}
    = \frac{2r_1}{2r_2} = \frac{r_1}{r_2} \]
  Let the reflection of $M$ over $K$ be $P$; then $P$ lies on $\ol{P_1P_2}$ and
  \[ \frac{PP_1}{PP_2} = \frac{2KO_1}{2KO_2} = \frac{KO_1}{KO_2}
    = \frac{r_1}{r_2} = \frac{LQ_1}{LQ_2} \]
  where again the ratios are directed.
  Projecting everything onto line $LM$, so that $P_1$ lands at $Q_1$
  and $P_2$ lands at $Q_2$,
  we find that the projection of $P$ must land exactly at $L$.
\end{proof}

\begin{claim*}
  Line $KM$ is an external angle bisector of $\angle O_1MO_2$.
\end{claim*}
\begin{proof}
  Because $\frac{KO_1}{KO_2} = \frac{r_1}{r_2} = \frac{MO_1}{MO_2}$.
\end{proof}

To finish, note that we know that
$\ol{MP_1} \parallel \ol{CQ_1} \equiv \ol{AC}$
and $\ol{MP_2} \parallel \ol{BQ_2} \equiv \ol{AB}$,
meaning the angles $\angle O_1 M O_2$ and $\angle CAB$ have parallel legs.
Hence, if $N$ is the antipode of $L$,
it follows that $\ol{MK} \parallel \ol{AN}$.
Now from $MK = KL$ and the fact that $ANLT$ is an isosceles trapezoid,
we deduce that $\ol{LT}$ and $\ol{LK}$ are lines in the same direction
(namely, the reflection of $MK \parallel AN$ across $\ol{BC}$), as needed.

\paragraph{Complex numbers approach with Apollonian circles, by Carl Schildkraut.}
We use complex numbers.
As in the first approach, we will ignore the hypothesis that $K$ lies on $(ABC)$.

Let $Q \coloneqq (AH) \cap (ABC) \cap (AEF) \neq A$
be the Miquel point of $BFEC$ again.
Construct the point $T$ on $(ABC)$ for which $AT\perp BC$;
note that $T=-\frac{bc}a$.
This time the unconditional result is:
\begin{proposition*}
  We have $Q$, $M$, $T$, $K$ are concyclic (or collinear)
  on an Apollonian circle of $\ol{O_1O_2}$.
\end{proposition*}
This will solve the original problem since once $K$ lies on $(ABC)$
it must be either $Q$ or $T$.
But since $K$ is not on $(BME)$, $K\neq Q$, it will have to be $T$.

We now prove the proposition.
Suppose $(ABC)$ is the unit circle and let $A=a$, $B=b$, $C=c$.
Let $H=a+b+c$ be the orthocenter of $\triangle ABC$.
By the usual formulas,
\[ E \coloneqq \half\left(a+b+c-\frac{bc}{a}\right). \]
Let $O_1$ be the center of $(BME)$ and $O_2$ be the center of $(CMF)$.

\begin{claim*}
  [Calculation of the Miquel point]
  We have $Q = \frac{2a+b+c}{a\left( \frac 1a + \frac 1b + \frac 1c \right)+1}$.
\end{claim*}
\begin{proof}
  We now compute that $Q=q$ satisfies $\ol q=1/q$
  (since $Q$ is on the unit circle)
  and $\frac{q-h}{q-a}\in i\RR$ (since $AQ\perp QH$),
  which expands to
  \[0=\frac{q-h}{q-a}+\frac{1/q-\ol h}{1/q-1/a}
    =\frac{q-h}{q-a}-\frac{a(1-q\ol h)}{q-a}.\]
  This solves to $q=\frac{h+a}{a\ol h+1}=\frac{2a+b+c}{a\ol h+1}$.
\end{proof}

\begin{claim*}
  [Calculation of $O_1$ and $O_2$]
  We have $O_1 = \frac{b(2a+b+c)}{2(a+b)}$
  and $O_2 = \frac{c(2a+b+c)}{2(a+c)}$.
\end{claim*}
\begin{proof}
  We now compute $O_1$ and $O_2$.
  For $x,y,z\in\CC$, let $\opname{Circum}(x,y,z)$
  denote the circumcenter of the triangle
  defined by vertices $x$, $y$, and $z$ in $\CC$. We have
  \begin{align*}
  O_1&=\opname{Circum}(B,M,E)\\
  &=b+\frac12\opname{Circum}\left(0,c-b,\frac{(a-b)(b-c)}b\right)\\
  &=b-\frac{b-c}{2b}\opname{Circum}\left(0,b,b-a\right)\\
  &=b-\frac{b-c}{2b}\left(b-\opname{Circum}\left(0,b,a\right)\right)\\
  &=b-\frac{b-c}{2b}\left(b-\frac{ab}{a+b}\right)=b-\frac{b(b-c)}{2(a+b)}=\frac{b(2a+b+c)}{2(a+b)}.
  \end{align*}
  Similarly, $O_2=\frac{c(2a+b+c)}{2(a+c)}$.
\end{proof}

We are now going to prove the following:
\begin{claim*}
  We have
  \[\frac{TO_1}{TO_2}=\frac{MO_1}{MO_2}=\frac{QO_1}{QO_2}.\]
\end{claim*}
\begin{proof}
  We now compute
  \[MO_1=BO_1=\left|b-\frac{b(2a+b+c)}{2(a+b)}\right|
    =\left|\frac{b(b-c)}{2(a+b)}\right|=\frac12\left|\frac{b-c}{a+b}\right|\]
  and
  \[QO_1=\left|r-\frac{b(2a+b+c)}{2(a+b)}\right|
    =\left|1-\frac{b(a+h)}{2(a+b)r}\right|=\left|1-\frac{b(a\ol h+1)}{2(a+b)}\right|
    =\left|\frac{a-\frac{ab}c}{2(a+b)}\right|=\frac12\left|\frac{b-c}{a+b}\right|.\]
  This implies both (by symmetry) that
  $\frac{MO_1}{MO_2}=\frac{QO_1}{QO_2}=\big|\frac{a+c}{a+b}\big|$
  and that $Q$ is on $(BME)$ and $(CMF)$.
  Also,
  \[\frac{TO_1}{TO_2}
    =\frac{\left|\frac{b(2a+b+c)}{2(a+b)}+\frac{bc}a\right|}%
    {\left|\frac{c(2a+b+c)}{2(a+c)}+\frac{bc}a\right|}
    =\left|\frac{\frac{b(2a^2+ab+ac+2ac+2bc)}{2a(a+b)}}{\frac{c(2a^2+ab+ac+2ab+2bc)}{2a(a+c)}}\right|
    =\left|\frac{a+c}{a+b}\right|\cdot\left|\frac{2a^2+2bc+ab+3ac}{2a^2+2bc+3ab+ac}\right|;\]
  if $z=2a^2+2bc+ab+3ac$, then $a^2bc\ol z=2a^2+2bc+3ab+ac$, so the second
  term has magnitude $1$.
  This means $\frac{TO_1}{TO_2}=\frac{MO_1}{MO_2}=\frac{QO_1}{QO_2}$, as desired.
\end{proof}

To finish, note that this common ratio is the ratio between the radii
of these two circles, so it is also $\frac{KO_1}{KO_2}$.
By Apollonian circles the points $\{Q,M,T,K\}$ lie on a circle or a line.

---

%\paragraph{Sketch of approach by author.}
%It is not difficult to prove that $(BME)$ and $(CMF)$
%intersect at $M$ and the Miquel Point $Q$ of $BCEF$.
%Use the same point definitions as in the first solutions and set $QL$ to
%intersect $(BME)$, $(CMF)$ at $U,V \neq Q$. With some work, one can show
%that $BU \parallel CV$ and $\frac{BU}{CV} = \frac{r_1}{r_2}$. It follows via an
%easy configuration analysis that in fact $BU$ maps to $CV$ under the natural
%homothety centered at $K$. The desired statement then follows from a combination
%of phantom points and ratio calculations via parallel lines and Menelaus (note
%that $BU \parallel CV \parallel LT$)
%which proves both the main claims as in the first solution.
