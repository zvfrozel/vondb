author: Merlijn Staps
desc: Double midpoint tangency
source: USA TST 2019/1
tags: [2018-12, harmonic, simtri, nice, anglechase, aleph]
hardness: 20
url: https://aops.com/community/p11419585

---

Let $ABC$ be a triangle and let $M$ and $N$
denote the midpoints of $\ol{AB}$ and $\ol{AC}$, respectively.
Let $X$ be a point such that $\ol{AX}$
is tangent to the circumcircle of triangle $ABC$.
Denote by $\omega_B$ the circle through $M$ and $B$ tangent to $\ol{MX}$,
and by $\omega_C$ the circle through $N$ and $C$ tangent to $\ol{NX}$.
Show that $\omega_B$ and $\omega_C$ intersect on line $BC$.

---

We present four solutions,
the second of which shows that $M$ and $N$
can be replaced by any two points on $AB$ and $AC$
satisfying $AM/AB + AN/AC = 1$.

\paragraph{First solution using symmedians (Merlijn Staps).}
Let $\ol{XY}$ be the other tangent from $X$ to $(AMN)$.
\begin{claim*}
  Line $\ol{XM}$ is tangent to $(BMY)$;
  hence $Y$ lies on $\omega_B$.
\end{claim*}
\begin{center}
\begin{asy}
size(11cm);

pair A = dir(130); pair B = dir(220); pair C = dir(320);
draw(B--A--C--B);

pair M = 0.5*A + 0.5*B;
pair N = 0.5*A + 0.5*C;
pair O = circumcenter(A,M,N);
pair Q = rotate(-90,A)*O;
pair X = 3*Q - 2*A;
pair Y = intersectionpoints(circle(X,abs(X-A)),circumcircle(A,M,N))[0];
pair Z = 0.5*A + 0.5*Y;

path anglemark(pair A, pair B, pair C, real t=8) {
  pair M,N,P[],Q[];
  path mark;
  M=t*0.03*unit(A-B)+B;
  N=t*0.03*unit(C-B)+B;
  mark=arc(B,M,N);
  mark=(mark--B--cycle);
  return mark;
}

fill(anglemark(A,M,X,t=4),rgb(0,1,0.5)+opacity(0.7));
fill(anglemark(Y,M,Z,t=4),rgb(0,1,0.5)+opacity(0.7));
fill(anglemark(M,Y,B,t=4),rgb(0,1,0.5)+opacity(0.7));

filldraw(circumcircle(B,Y,M),lightcyan+opacity(0.1), blue+opacity(0.5));
filldraw(circumcircle(C,Y,N),lightcyan+opacity(0.1), blue+opacity(0.5));
filldraw(circumcircle(A,M,N),lightred+opacity(0.1), red);

draw(A--X,red);
draw(A--Y,dashed);
draw(X--Y,red);
draw(X--M--Z,gray);
draw(B--Y--M,gray);

dot("$A$", A, dir(A)); dot("$B$", B, dir(B)); dot("$C$", C, dir(0));
dot("$M$", M, dir(220)); dot("$N$", N, dir(10));
dot("$X$", X, dir(X));
dot("$Y$", Y, dir(-40));
dot("$Z$", Z, dir(0));
\end{asy}
\end{center}

\begin{proof}
  Let $Z$ be the midpoint of $\ol{AY}$.
  Then $\ol{MX}$ is the $M$-symmedian in triangle $AMY$.
  Since $\ol{MZ} \parallel \ol{BY}$,
  it follows that $\dang AMX = \dang ZMY = \dang BYM$.
  We conclude that $\ol{XM}$ is tangent
  to the circumcircle of triangle $BMY$.
\end{proof}

Similarly, $\omega_C$ is the circumcircle of triangle $CNY$.
As $AMYN$ is cyclic too,
it follows that $\omega_B$ and $\omega_C$ intersect on $\ol{BC}$,
by Miquel's theorem.

\begin{remark*}
  The converse of Miquel's theorem is true,
  which means the problem is equivalent
  to showing that the second intersection
  of the $\omega_B$ and $\omega_C$ moves along $(AMN)$.
  Thus the construction of $Y$ above is not so unnatural.
\end{remark*}

\paragraph{Second solution (Jetze Zoethout).}
Let $\omega_B$ intersect $\ol{BC}$ again at $S$
and let $\ol{MS}$ intersect $\ol{AC}$ again at $Y$.
Angle chasing gives
$\dang XMY = \dang XMS = \dang MBS = \dang ABC = \dang XAC = \dang XAY$,
so $Y$ is on the circumcircle of triangle $AMX$.
Furthermore, from $\dang XMY = \dang ABC$ and
$\dang ACB = \dang XAB = \dang XYM$ it follows that
$\triangle ABC \sim \triangle XMY$ and from $\dang XAY = \dang MBS$
and $\dang YXA = \dang YMA = \dang BMS$ it follows that
$\triangle AXY \sim \triangle BMS$.

\begin{center}
\begin{asy}
size(10cm);

pair A = dir(130); pair B = dir(220); pair C = dir(320);
draw(B--A--C--B);

pair M = 0.5*A + 0.5*B;
pair N = 0.5*A + 0.5*C;
pair O = circumcenter(A,M,N);
pair Q = rotate(-90,A)*O;
pair X = 3*Q - 2*A;

pair T = intersectionpoints(circle(X,abs(X-A)),circumcircle(A,M,N))[0];
pair S = intersectionpoints(circumcircle(B,M,T),circumcircle(C,N,T))[0];
pair Y = extension(M,S,A,C);
fill(X--M--S--cycle,green+opacity(0.1));
fill(X--A--N--cycle,green+opacity(0.1));

draw(S--Y--A);
draw(X--A,blue);
draw(X--M,red);
draw(X--N--S--X,gray);

draw(circumcircle(B,M,S),red);
draw(circumcircle(A,B,C),blue);
draw(circumcircle(X,M,A),gray+dashed);
draw(circumcircle(X,N,S),gray+dashed);

dot("$A$", A, dir(70)); dot("$B$", B, dir(B)); dot("$C$", C, dir(0));
dot("$M$", M, dir(50)); dot("$N$", N, dir(10));
dot("$X$", X, dir(X));
dot("$Y$", Y, dir(Y));
dot("$S$", S, dir(-60));
\end{asy}
\end{center}

We now find
\[
\frac{AN}{AX} = \frac{AN/BM}{AX/BM} = \frac{AC/AB}{MS/XY} = \frac{AB/AB}{MS/XM} = \frac{XM}{MS},
\]
which together with $\angle XMS = \angle XAN$ yields
$\triangle XMS \sim \triangle XAN$.
From $\dang XSY = \dang XSM = \dang XNA = \dang XNY$
we now have that $S$ is on the circumcircle of triangle $XNY$.
Finally, we have $\dang XNS = \dang XYS = \dang XYM = \dang ACB = \dang NCS$
so $\ol{XN}$ is tangent to the circle through $C$, $N$, and $S$, as desired.

\paragraph{Third solution by moving points method.}
Fix triangle $ABC$ and animate $X$ along the tangent at $A$.
We let $D$ denote the second intersection point of $\omega_C$
with line $\ol{BC}$.

\begin{claim*}
  The composed map $X \mapsto D$ is a fractional linear transformation
  (i.e.\ a projective map) in terms of
  a real coordinate on line $\ol{AA}$, $\ol{BC}$.
\end{claim*}
\begin{proof}
  Let $\ell$ denote the perpendicular bisector of $\ol{CN}$,
  also equipped with a real coordinate.
  We let $P$ denote the intersection of $\ol{XM}$ with $\ell$,
  $S$ the circumcenter of $\triangle CMD$.
  Let $T$ denote the midpoint of $\ol{BD}$.

  We claim that the composed map
  \begin{align*}
    \ol{AA} &\to \ell \to \ell \to \ol{BC} \to \ol{BC} \\
    \text{by} \quad
      X &\mapsto P \mapsto S \mapsto T \mapsto D
  \end{align*}
  is projective, by showing each individual map is projective.
  \begin{center}
  \begin{asy}
  size(8cm);

  pair A = dir(129); pair B = dir(220); pair C = dir(320);
  filldraw(A--B--C--cycle, opacity(0.1)+yellow, black);

  pair M = 0.5*A + 0.5*B;
  pair N = 0.5*A + 0.5*C;
  pair O = circumcenter(A,M,N);
  pair Q = rotate(-90,A)*O;
  pair X = 4.2*Q - 3.2*A;
  pair Y = intersectionpoints(circle(X,abs(X-A)),circumcircle(A,M,N))[0];
  pair Z = 0.5*A + 0.5*Y;

  // filldraw(circumcircle(B,Y,M),lightcyan+opacity(0.1), blue+opacity(0.5));
  filldraw(circumcircle(C,Y,N),lightcyan+opacity(0.1), blue+opacity(0.5));
  // filldraw(circumcircle(A,M,N),lightred+opacity(0.1), red);
  draw(A--X,red);
  pair S = circumcenter(C, N, Y);
  pair T = foot(S, B, C);
  pair D = 2*T - C;
  pair P = extension(S, midpoint(N--C), X, N);
  draw(P--S, deepgreen);
  draw(X--P, lightblue);
  draw(S--T, black);

  dot("$A$", A, dir(A)); dot("$B$", B, dir(B)); dot("$C$", C, dir(0));
  dot("$M$", M, dir(135)); dot("$N$", N, dir(80));
  dot("$X$", X, dir(X));
  dot("$S$", S, dir(135));
  dot("$T$", T, dir(-90));
  dot("$D$", D, dir(225));
  dot("$P$", P, dir(45));
  dot(extension(S,P,N,C));
  \end{asy}
  \end{center}
  \begin{itemize}
    \ii The map $X \mapsto P$ is projective since it is
    a perspectivity through $N$ from $\ol{AA}$ to $\ell$.
    \ii The map $P \mapsto S$ is projective
    since it is equivalent to a negative inversion on $\ell$
    at the midpoint of $\ol{NC}$ with radius $\half NC$.
    (Note $\angle PNS = 90\dg$ is fixed.)
    \ii The map $S \mapsto T$ is projective
    since it is a perspectivity $\ell \to \ol{BC}$
    through the point at infinity perpendicular to $\ol{BC}$
    (in fact, it is linear).
    \ii The map $T \mapsto D$ is projective (in fact, linear)
    since it is a homothety through $C$ with fixed ratio $2$.
  \end{itemize}
  Thus the composed map is projective as well.
\end{proof}

Similarly, if we define $D'$ so that $\ol{XM}$
is tangent to $(BMD')$, the map $X \mapsto D'$ is projective as well.
We aim to show $D = D'$, and since the maps
correspond to fractional linear transformations
in projective coordinates,
it suffices to verify it for three distinct choices of $X$.
We do so:
\begin{itemize}
  \ii If $X = \ol{AA} \cap \ol{MN}$,
  then $D$ and $D'$ satisfy $MB = MD'$, $NC = ND$.
  This means they are the feet of the $A$-altitude on $\ol{BC}$.
  \ii As $X$ approaches $A$ the points $D$ and $D'$
  approach the infinity point along $\ol{BC}$.
  \ii If $X$ is a point at infinity along $\ol{AA}$,
  then $D$ and $D'$ coincide with the midpoint of $\ol{BC}$.
\end{itemize}
This completes the solution.

\begin{remark*}
  [Anant Mudgal]
  An alternative (shorter) way to show $X \mapsto D$ is projective
  is to notice $\dang XND$ is a constant angle.
  I left the longer ``original'' proof for instructional reasons.
\end{remark*}

\paragraph{Fourth solution by isogonal conjugates (Anant Mudgal).}
Let $Y$ be the isogonal conjugate of $X$ in $\triangle AMN$
and $Z$ be the reflection of $Y$ in $\ol{MN}$.
As $\ol{AX}$ is tangent to the circumcircle of $\triangle AMN$,
it follows that $\ol{AY} \parallel \ol{MN}$.
Thus $Z$ lies on $\ol{BC}$ since $\ol{MN}$
bisects the strip made by $\ol{AY}$ and $\ol{BC}$.

\begin{asy}
size(11cm);

pair A = dir(130); pair B = dir(220); pair C = dir(320);
draw(B--A--C--B);

pair M = 0.5*A + 0.5*B;
pair N = 0.5*A + 0.5*C;
pair O = circumcenter(A,M,N);
pair Q = rotate(-90,A)*O;
pair X = 3*Q - 2*A;
pair V = intersectionpoints(circle(X,abs(X-A)),circumcircle(A,M,N))[0];
pair Z = OP(circumcircle(B,V,M), circumcircle(C,V,N));
pair Y = -Z+2*foot(Z,M,N);

filldraw(circumcircle(B,V,M),lightcyan+opacity(0.1), blue+opacity(0.5));
filldraw(circumcircle(C,V,N),lightcyan+opacity(0.1), blue+opacity(0.5));
filldraw(circumcircle(A,M,N),lightred+opacity(0.1), red);

// draw(A--X--P);
// draw(X--B);
draw(A--X,red);
draw(B--V--M,gray);
dot(B*C/A);

filldraw(unitcircle, opacity(0.1)+yellow, black);
draw(M--Z--N, blue);
draw(N--X--M, lightblue);
draw(M--N, red);
draw(A--(B*C/A), red);

dot("$A$", A, dir(A)); dot("$B$", B, dir(B)); dot("$C$", C, dir(0));
dot("$M$", M, dir(220)); dot("$N$", N, dir(10));
dot("$X$", X, dir(X));
dot("$Y$", Y, dir(-45));
dot("$Z$", Z, dir(Z));
\end{asy}

Finally, \[ \dang ZMX=\dang ZMN+\dang NMX=\dang NMY+\dang YMA
  = \dang NMA=\dang ZBM \]
so $\ol{XM}$ is tangent to the circumcircle of
$\triangle ZMB$, hence $Z$ lies on $\omega_B$.
Similarly, $Z$ lies on $\omega_C$ and we're done.
