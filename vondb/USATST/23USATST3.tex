author: Sean Li
desc: Young diagrams inequality
source: USA TST 2023/3
url: https://aops.com/community/p26685437
tags: [2022-12, find, bestpossible, blind, yod]

---

Consider pairs $(f,g)$ of functions
from the set of nonnegative integers to itself such that
\begin{itemize}
		\ii $f(0) \ge f(1) \ge f(2) \ge \dots \ge f(300) \ge 0$;
		\ii $f(0) + f(1) + f(2) + \dots + f(300) \leq 300$;
		\ii for any $20$ nonnegative integers $n_1$, $n_2$, \dots, $n_{20}$,
		not necessarily distinct, we have
		\[ g(n_1 + n_2 + \dots + n_{20}) \leq f(n_1) + f(n_2) + \dots + f(n_{20}). \]
\end{itemize}
Determine the maximum possible value of $g(0) + g(1) + \dots + g(6000)$
over all such pairs of functions.

---

Replace $300 = \frac{24 \cdot 25}{2}$ with $\frac{s(s+1)}{2}$ where $s=24$, and
$20$ with $k$. The answer is $115440 = \frac{ks(ks+1)}{2}$. Equality is achieved
at $f(n) = \max (s-n, 0)$ and $g(n) = \max (ks-n,0)$. To prove
\[g(n_1+\dotsb+n_k)\leq f(n_1)+\dotsb+f(n_k),\]
write it as
\[\max(x_1+\dotsb+x_k, 0)\leq \max(x_1, 0)+\dotsb+\max(x_k, 0)\]
with $x_i=s-n_i$. This can be proven from the $k=2$ case and induction.

\medskip

It remains to show the upper bound. For this problem, define a \emph{partition}
to be a nonincreasing function $p \colon \ZZ_{\geq 0}\to \ZZ_{\geq 0}$
such that $p(n)=0$ for some $n$. The sum of $p$ is defined to be
$\sum_{n=0}^{\infty}p(n)$, which is finite under the previous assumption. Let
$L=\ZZ_{\geq 0}^2$. The
\emph{Young diagram} of the partition is the set of points
\[\mathcal{P} := \{(x, y)\in L: y < p(x)\}.\]
The number of points in $\mathcal{P}$ is equal to the sum of $p$.
The \emph{conjugate} of a partition defined as
\[p_*(n) = \text{the number of $i$ for which $p(i) > n$}.\]
This is a partition with the same sum as $p$. Geometrically, the Young diagrams
of $p$ and $p_*$ are reflections about $x=y$.

Since each $g(n)$ is independent, we may
maximize each one separately for all $n$ and assume that
\[
  g(n) = \min_{n_1+\dotsb+n_k=n}(f(n_1)+\dotsb+f(n_k)) \tag{*}.
\]
The conditions of the problem statement imply that $f\big(\tfrac{s(s+1)}{2}\big)=0$.
Then, for any $n\leq k\frac{s(s+1)}{2}$, there exists an optimal combination
$(n_1, \dots, n_k)$ in (*) where all $n_i$ are at most $\frac{s(s+1)}{2}$, by
replacing any term in an optimum greater than $\frac{s(s+1)}{2}$ by
$\frac{s(s+1)}{2}$ and shifting the excess to smaller terms (because $f$ is
nonincreasing). Therefore we may extend $f$ to a partition by letting $f(n) = 0$
for $n > \frac{s(s+1)}{2}$ without affecting the relevant values of $g$. Then
(*) implies that $g$ is a partition as well.

The problem can be restated as follows: $f$ is a partition with sum
$\frac{s(s+1)}{2}$, and $g$ is a partition defined by (*). Find the maximum
possible sum of $g$. The key claim is that the problem is the
same under conjugation.
\begin{claim*}
  Under these conditions, we have
  \[
    g_*(n) = \min_{n_1+\dotsb+n_k=n}(f_*(n_1)+\dotsb+f_*(n_k)).
  \]
\end{claim*}

\begin{proof}
  Let $\mathcal{F}$ and $\mathcal{G}$ be the Young diagrams of $f$ and $g$
  respectively, and $\ol{\mathcal{F}} = L\setminus \mathcal{F}$ and
  $\ol{\mathcal{G}} = L\setminus \mathcal{G}$ be their complements. The
  lower boundary of $\ol{\mathcal{F}}$ is formed by the points $(n, f(n))$
  for $i\in \ZZ_{\geq 0}$. By the definition of $g$, the lower boundary
  of $\ol{\mathcal{G}}$
  consists of points $(n, g(n))$ which are formed by adding $k$ points of
  $\ol{\mathcal{F}}$. This means
  \[
    \ol{\mathcal{G}} = \underbrace{\ol{\mathcal{F}} + \dots +
    \ol{\mathcal{F}}}_{k \ \ol{\mathcal{F}}\text{'s}}
  \]
  where $+$ denotes set addition. This definition remains invariant under
  reflection about $x=y$, which swaps $f$ and $g$ with their conjugates.
\end{proof}

Let $A$ be the sum of $g$. We now derive different bounds on $A$. First, by
Hermite's identity
\[n = \sum_{i=0}^{k-1}\left\lfloor\tfrac{n+i}{k}\right\rfloor\]
we have
\begin{align*}
  A &= \sum_{n=0}^{\infty} g(n) \\
    & \leq \sum_{n=0}^{\infty} \sum_{i=0}^{k-1} f\left( \left\lfloor
    \tfrac{n+i}{k} \right\rfloor \right) \\
    &= k^2 \sum_{n=0}^{\infty}f(n) - \frac{k(k-1)}{2}f(0) \\
    &= k^2\frac{s(s+1)}{2}-\frac{k(k-1)}{2}f(0).
\end{align*}
By the claim, we also get the second bound $A \leq k^2 \frac{s(s+1)}{2} -
\frac{k(k-1)}{2}f_*(0)$.

For the third bound, note that $f(f_*(0)) = 0$ and thus $g(k f_*(0)) = 0$. Moreover,
\[g(q f_*(0) + r) \le q \cdot f(f_*(0)) + (k-q-1) f(0) + f(r) = (k-q-1)f(0) +
f(r),\]
so we have
\begin{align*}
A &= \sum_{\substack{0 \leq q < k \\ 0 \leq r < f_*(0)}} g(qf_*(0) + r) \\
& \le \frac{k(k-1)}{2}f_*(0)f(0) + k \sum_{0 \leq r < f_*(0)} f(r) \\
&= \frac{k(k-1)}{2} f_*(0)f(0) + k\frac{s(s+1)}{2}.
\end{align*}

Now we have three cases:
\begin{itemize}
  \item If $f(0)\geq s$ then
    \[
      A\leq k^2 \frac{s(s+1)}{2} - \frac{k(k-1)}{2}f(0) \leq \frac{ks(ks+1)}{2}.
    \]
  \item If $f_*(0)\geq s$ then
    \[
      A\leq k^2 \frac{s(s+1)}{2} - \frac{k(k-1)}{2}f_*(0) \leq \frac{ks(ks+1)}{2}.
    \]
  \item Otherwise, $f(0)f_*(0) \leq s^2$ and
    \[
      A\leq \frac{k(k-1)}{2} f_*(0)f(0) + k\frac{s(s+1)}{2} \leq \frac{ks(ks+1)}{2}.
    \]
\end{itemize}

\medskip

In all cases, $A \leq \frac{ks(ks+1)}{2}$, as desired.

\begin{remark*}
  One can estimate the answer to be around $k^2 \frac{s(s+1)}{2}$ by
  observing the set addition operation ``dilates'' $\mathcal{F}$ by a factor of
  $k$, but significant care is needed to sharpen the bound.
\end{remark*}
