author: Carl Schildkraut
desc: Iterated polynomial orbits mod n
hardness: 25
source: USA TST 2020/5
tags: [2020-01, mods, find, hardanswer, polynomial, arrows, dalet]
url: https://aops.com/community/p13913769

---

Find all integers $n \ge 2$ for which there exists an integer $m$ and
a polynomial $P(x)$ with integer coefficients satisfying the following three conditions:
\begin{itemize}
  \item $m > 1$ and $\gcd(m,n) = 1$;
  \item the numbers $P(0)$, $P^2(0)$, \dots, $P^{m-1}(0)$ are not divisible by $n$; and
  \item $P^m(0)$ is divisible by $n$.
\end{itemize}
Here $P^k$ means $P$ applied $k$ times, so $P^1(0) = P(0)$, $P^2(0) = P(P(0))$, etc.

---

The answer is that this is possible if and only if
there exists primes $p' < p$ such that $p \mid n$ and $p' \nmid n$.
(Equivalently, the radical $\opname{rad}(n)$
must not be the product of the first several primes.)

For a polynomial $P$, and an integer $N$, we introduce the notation
\[ \mathbf{zord}(P \bmod N)
  \coloneqq \min \left\{ e > 0 \mid P^e(0) \equiv 0 \bmod N \right\} \]
where we set $\min\varnothing=0$ by convention.
Note that in general we have
\[ \mathbf{zord}(P \bmod N) = \opname*{lcm}_{q \mid N}
  \left( \mathbf{zord}(P \bmod q)  \right) \qquad (\dagger) \]
where the index runs over all prime powers
$q \mid N$ (by Chinese remainder theorem).
This will be used in both directions.

\textbf{Construction}:
First, we begin by giving a construction.
The idea is to first use the following prime power case.
\begin{claim*}
  Let $p^e$ be a prime power, and $1 \le k < p$.
  Then
  \[ f(X) = X+1 - k \cdot \frac{X(X-1)(X-2) \dots (X-(k-2))}{(k-1)!} \]
  viewed as a polynomial in $(\ZZ/p^e)[X]$ satisfies $\mathbf{zord}(f \bmod{p^e}) = k$.
\end{claim*}
\begin{proof}
  Note $f(0) = 1$, $f(1) = 2$, \dots, $f(k-2) = k-1$, $f(k-1) = 0$ as needed.
\end{proof}
This gives us a way to do the construction now.
For the prime power $p^e \mid n$, we choose $1 \le p' < p$ and require
$\mathbf{zord}(P \bmod{p^e}) = p'$ and
$\mathbf{zord}(P \bmod{q}) = 1$ for every other prime power $q$ dividing $n$.
Then by $(\dagger)$ we are done.
\begin{remark*}
  The claim can be viewed as a special case of Lagrange interpolation
  adapted to work over $\ZZ/p^e$ rather than $\ZZ/p$.
  Thus the construction of the polynomial $f$ above is quite natural.
\end{remark*}

\textbf{Necessity}: by $(\dagger)$ again,
it will be sufficient to prove the following claim.
\begin{claim*}
  For any prime power $q = p^e$, and any polynomial $f(x) \in \ZZ[x]$,
  if the quantity $\mathbf{zord}(f \bmod q)$ is nonzero
  then it has all prime factors at most $p$.
\end{claim*}
\begin{proof}
  This is by induction on $e \ge 1$.
  For $e=1$, the pigeonhole principle immediately
  implies that $\mathbf{zord}(P \bmod p) \le p$.

  Now assume $e \ge 2$.
  Let us define
  \[ k \coloneqq \mathbf{zord}(P \bmod{p^{e-1}}),
    \qquad Q \coloneqq P^k. \]
  Since being periodic modulo $p^e$ requires periodic modulo $p^{e-1}$,
  it follows that
  \[ \mathbf{zord}(P \bmod{p^e}) = k \cdot \mathbf{zord}(Q \bmod{p^e}). \]
  However since $Q(0) \equiv 0\bmod p^{e-1}$, it follows $\{Q(0), Q^2(0), \dots\}$
  are actually all multiples of $p^{e-1}$.
  There are only $p$ residues modulo $p^e$ which are also multiples of $p^{e-1}$,
  so $\mathbf{zord}(Q \bmod{p^e}) \le p$, as needed.
\end{proof}

\begin{remark*}
  One reviewer submitted the following test-solving comments:

  This is one of these problems where you can make many useful natural observations,
  and if you make enough of them eventually they cohere into a solution.
  For example, here are some things I noticed while solving:
  \begin{itemize}
  \ii The polynomial $1-x$ shows that $m=2$ works for any odd $n$.
  \ii In general, if $\zeta$ is a primitive $m$th root of unity
  modulo $n$, then $\zeta(x+1)-1$ has the desired property (assuming $\gcd(m,n) = 1$).
  We can extend this using the Chinese remainder theorem
  to find that if $p \mid n$, $m \mid p-1$, and $\gcd(m,n)=1$, then $n$ works.
  So by this point I already have something
  about the prime factors of $n$ being sort-of closed downwards.
  \ii By iterating $P$ we see it is enough to consider $m$ prime.
  \ii In the case where $n=2^k$,
  it is not too difficult to show that no odd prime $m$ works.
  \end{itemize}
\end{remark*}
