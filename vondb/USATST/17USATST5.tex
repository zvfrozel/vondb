desc: Rhombus EVAN
author: Danielle Wang, Evan Chen
source: USA TST 2017/5
tags: [2016-12, mine, nice, rich, harmonic, polar, config, gimel]
hardness: 30
url: https://aops.com/community/p7732197

---

Let $ABC$ be a triangle with altitude $\ol{AE}$.
The $A$-excircle touches $\ol{BC}$ at $D$,
and intersects the circumcircle at two points $F$ and $G$.
Prove that one can select points $V$ and $N$
on lines $DG$ and $DF$ such that quadrilateral $EVAN$ is a rhombus.

---

Let $I$ denote the incenter, $J$ the $A$-excenter,
and $L$ the midpoint of $\ol{AE}$.
Denote by $\ol{IY}$, $\ol{IZ}$ the tangents
from $I$ to the $A$-excircle.
Note that lines $\ol{BC}$, $\ol{GF}$, $\ol{YZ}$ then concur at $H$
(unless $AB=AC$, but this case is obvious),
as it's the radical center of cyclic hexagon $BICYJZ$,
the circumcircle and the $A$-excircle.

\begin{center}
\begin{asy}
size(12cm);
pair A = dir(110);
pair B = dir(210);
pair C = dir(330);
pair I = incenter(A, B, C);
pair M = dir(270);
pair J = 2*M-I;
pair D = foot(J, B, C);
pair E = foot(A, B, C);

pair F = IP(CP(J, D), unitcircle);
pair G = OP(CP(J, D), unitcircle);

filldraw(unitcircle, opacity(0.1)+lightblue, lightblue);
draw(arc(J,abs(D-J),-30,210), lightblue);

pair L = midpoint(A--E);
pair V = extension(G, D, L, L+B-C);
pair N = extension(F, D, L, L+B-C);
draw(A--B--C--cycle, lightblue);
draw(A--E, lightblue);
draw(G--V, lightblue);
draw(N--F, lightblue);
filldraw(E--V--A--N--cycle, opacity(0.1)+lightgreen, heavygreen);
draw(V--N, heavygreen);
pair H = extension(G, F, B, C);
draw(C--H--G, lightblue);

filldraw(circumcircle(B, I, C), opacity(0.1)+lightred, lightred);
pair Y = IP(circumcircle(B, I, C), circumcircle(D, F, G));
pair Z = OP(circumcircle(B, I, C), circumcircle(D, F, G));
pair T = -D+2*foot(J, D, I);
draw(T--H, blue);
draw(Z--H, blue);
draw(L--T, blue);
draw(Y--I--Z, blue);

dot("$A$", A, dir(A));
dot("$B$", B, dir(B));
dot("$C$", C, dir(C));
dot("$I$", I, dir(I));
dot("$M$", M, dir(M));
dot("$J$", J, dir(J));
dot("$D$", D, dir(D));
dot("$E$", E, dir(E));
dot("$F$", F, dir(270));
dot("$G$", G, dir(270));
dot("$L$", L, dir(L));
dot("$V$", V, dir(V));
dot("$N$", N, dir(N));
dot("$H$", H, dir(H));
dot("$Y$", Y, dir(Y));
dot("$Z$", Z, dir(Z));
dot("$T$", T, dir(T));

/* TSQ Source:

!size(12cm);
A = dir 110
B = dir 210
C = dir 330
I = incenter A B C
M = dir 270
J = 2*M-I
D = foot J B C
E = foot A B C

F = IP CP J D unitcircle R270
G = OP CP J D unitcircle R270

unitcircle 0.1 lightblue / lightblue
!draw(arc(J,abs(D-J),-30,210), lightblue);

L = midpoint A--E
V = extension G D L L+B-C
N = extension F D L L+B-C
A--B--C--cycle lightblue
A--E lightblue
G--V lightblue
N--F lightblue
E--V--A--N--cycle 0.1 lightgreen / heavygreen
V--N heavygreen
H = extension G F B C
C--H--G lightblue

circumcircle B I C 0.1 lightred / lightred
Y = IP circumcircle B I C circumcircle D F G
Z = OP circumcircle B I C circumcircle D F G
T = -D+2*foot J D I
T--H blue
Z--H blue
L--T blue
Y--I--Z blue

*/
\end{asy}
\end{center}

Now let $\ol{HD}$ and $\ol{HT}$ be the tangents
from $H$ to the $A$-excircle.
It follows that $\ol{DT}$ is the symmedian of $\triangle DZY$,
hence passes through $I = \ol{YY} \cap \ol{ZZ}$.
Moreover, it's well known that $\ol{DI}$ passes through $L$,
the midpoint of the $A$-altitude
(for example by homothety).
Finally, $(DT;FG) = -1$,
hence project through $D$ onto the line through $L$
parallel to $\ol{BC}$ to obtain $(\infty L; VN) = -1$ as desired.

\paragraph{Authorship comments.}
This is a joint proposal with Danielle Wang (mostly by her).
The formulation given was that the tangents to the $A$-excircle
at $F$ and $G$ was on line $\ol{DI}$;
I solved this formulation using the radical axis argument above.
I then got the idea to involve the point $L$,
already knowing it was on $\ol{DI}$.
Observing the harmonic quadrilateral,
I took perspectivity through $M$ onto the line through $L$ parallel to $\ol{BC}$
(before this I had tried to use the $A$-altitude with little luck).
This yields the rhombus in the problem.

---

Alternative ending:
Now pick $V$ and $N$ such that points $V$, $L$, $N$ are collinear
parallel to $\ol{BC}$ (with $V \in \ol{DG}$, $N \in \ol{DF}$).
Since $BCGF$ is cyclic, so is $GFNV$,
and so the symmedian $\ol{DI}$ bisects $\ol{VN}$.
Hence $L$ is the midpoint of $\ol{VN}$, as desired.
