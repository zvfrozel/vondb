
desc: Bunbun hopping
source: USA TST 2023/1
url: https://aops.com/community/p26685816
tags: [2022-12, find, bestpossible, global, equalitycase, aleph]
author: Kevin Cong

---

There are $2022$ equally spaced points
on a circular track $\gamma$ of circumference $2022$.
The points are labeled $A_1$, $A_2$, \dots, $A_{2022}$ in some order, each label used once.
Initially, Bunbun the Bunny begins at $A_1$.
She hops along $\gamma$ from $A_1$ to $A_2$, then from $A_2$ to $A_3$,
until she reaches $A_{2022}$, after which she hops back to $A_1$.
When hopping from $P$ to $Q$, she always hops along the
shorter of the two arcs $\arc{PQ}$ of $\gamma$;
if $\ol{PQ}$ is a diameter of $\gamma$, she moves along either semicircle.

Determine the maximal possible sum of the lengths of the $2022$ arcs
which Bunbun traveled, over all possible labellings of the $2022$ points.

---

Replacing $2022$ with $2n$, the answer is $2n^2 - 2n + 2$.
(When $n=1011$, the number is $2042222$.)

\begin{center}
  \begin{asy}
    unitsize(48);
    int n = 5;
    int m = 2*n;
    int[] a = {};
    a.push(0);
    for (int i = 1; i < m; ++i)
      a.push(a[i-1] + (i % 2 > 0 ? n : n-1));

    int[] b = new int[m];
    pair[] B = new pair[m];
    b.cyclic = true;
    B.cyclic = true;
    for (int i = 0; i < m; ++i) {
      b[a[i]] = i+1;
      B[i] = dir(a[i]*360/m);
    }

    draw(unitcircle);

    for (int i = 0; i < m; ++i) {
      draw(B[i]--B[i+1], gray(0.7));
    }
    for (int i = 0; i < m; ++i) {
      pair P = dir(i*360/m);
      dot((string) b[i], P, P);
    }
  \end{asy}
  \hspace{3em}
  \begin{asy}
    unitsize(48);
    int n = 5;
    int m = 2*n;

    draw(unitcircle);

    for (int i = 0; i < m; ++i)
      dot(dir(i*360/m));
    for (int i = 0; i < n; ++i) {
      pair P = dir(i*360/m + 180/m);
      draw(scale(1.15) * (P--(-P)), red);
    }
  \end{asy}
\end{center}

\paragraph{Construction.}
The construction for $n = 5$ shown on the left half of the figure
easily generalizes for all $n$.
\begin{remark*}
  The validity of this construction
  can also be seen from the below proof.
\end{remark*}

\paragraph{First proof of bound.}
Let $d_i$ be the shorter distance from $A_{2i-1}$ to $A_{2i+1}$.
\begin{claim*}
The distance of the leg of the journey $A_{2i-1} \to A_{2i} \to A_{2i+1}$
is at most $2n - d_i$.
\end{claim*}
\begin{proof}
  Of the two arcs from $A_{2i-1}$ to $A_{2i+1}$,
  Bunbun will travel either $d_i$ or $2n-d_i$.
  One of those arcs contains $A_{2i}$ along the way.
  So we get a bound of $\max(d_i, 2n-d_i) = 2n - d_i$.
\end{proof}
That means the total distance is at most
\[ \sum_{i=1}^n \left( 2n - d_i \right)
  = 2n^2 - (d_1 + d_2 + \dots + d_n). \]
\begin{claim*}
  We have \[ d_1 + d_2 + \dots + d_n \ge 2n-2. \]
\end{claim*}
\begin{proof}
  The left-hand side is the sum of the walk
  $A_1 \to A_3 \to \dots \to A_{2n-1} \to A_1$.
  Among the $n$ points here, two of them must have distance at least $n-1$ apart;
  the other $d_i$'s contribute at least $1$ each.
  So the bound is $(n-1) + (n-1) \cdot 1 = 2n-2$.
\end{proof}

\paragraph{Second proof of bound.}
Draw the $n$ diameters through the $2n$ arc midpoints,
as shown on the right half of the figure for $n = 5$ in red.
\begin{claim*}[Interpretation of distances]
  The distance between any two points equals the number of diameters crossed to travel between the points.
\end{claim*}
\begin{proof}
    Clear.
\end{proof}

With this in mind, call a diameter \emph{critical} if it is crossed by all $2n$ arcs.
\begin{claim*}
  At most one diameter is critical.
\end{claim*}
\begin{proof}
  Suppose there were two critical diameters; these divide the circle into four arcs.
  Then all $2n$ arcs cross both diameters, and so travel between opposite arcs.
  But this means that points in two of the four arcs are never accessed --- contradiction.
\end{proof}

\begin{claim*}
  Every diameter is crossed an even number of times.
\end{claim*}
\begin{proof}
  Clear: the diameter needs to be crossed an even number of times for the loop to return to its origin.
\end{proof}

This immediately implies that the maximum possible total distance
is achieved when one diameter is crossed all $2n$ times,
and every other diameter is crossed $2n-2$ times,
for a total distance of at most
\[ n \cdot (2n-2) + 2 = 2n^2 - 2n + 2. \]
