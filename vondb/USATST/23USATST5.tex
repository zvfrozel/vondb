author: Nikolai Beluhov
desc: Linear algebra guessing game
source: USA TST 2023/5
url: https://aops.com/community/p26896130
tags: [2023-01, grid, linalg, nice, find, bestpossible, cyclotomic, gimel]

---

Let $m$ and $n$ be fixed positive integers.
Tsvety and Freyja play a game on an infinite grid of unit square cells.
Tsvety has secretly written a real number inside of each cell so that
the sum of the numbers within every rectangle of size
either $m \times n$ or $n \times m$ is zero.
Freyja wants to learn all of these numbers.

One by one, Freyja asks Tsvety about some cell in the grid,
and Tsvety truthfully reveals what number is written in it.
Freyja wins if, at any point, Freyja can simultaneously deduce
the number written in every cell of the entire infinite grid.
(If this never occurs, Freyja has lost the game and Tsvety wins.)

In terms of $m$ and $n$, find the smallest number of
questions that Freyja must ask to win,
or show that no finite number of questions can suffice.

---

The answer is the following:
\begin{itemize}
  \ii If $\gcd(m, n) > 1$, then Freyja cannot win.
  \ii If $\gcd(m, n) = 1$, then Freyja
  can win in a minimum of $(m-1)^2 + (n-1)^2$ questions.
\end{itemize}

First, we dispose of the case where $\gcd(m, n) > 1$.
Write $d = \gcd(m, n)$.
The idea is that any labeling where each $1 \times d$ rectangle has sum zero is valid.
Thus, to learn the labeling, Freyja must ask at least one question in every row,
which is clearly not possible in a finite number of questions.

Now suppose $\gcd(m, n) = 1$.
We split the proof into two halves.

\paragraph{Lower bound.}
Clearly, any labeling where each $m \times 1$ and $1 \times m$ rectangle has sum zero is valid.
These labelings form a vector space with dimension $(m-1)^2$, by inspection.
(Set the values in an $(m-1)\times (m-1)$ square arbitrarily and every other
value is uniquely determined.)

Similarly, labelings where each $n \times 1$ and $1 \times n$ rectangle have sum
zero are also valid, and have dimension $(n-1)^2$.

It is also easy to see that no labeling other than the all-zero labeling
belongs to both categories;
labelings in the first space are periodic in both directions with period $m$,
while labelings in the second space are periodic in both directions with period $n$;
and hence any labeling in both categories must be constant, ergo all-zero.

Taking sums of these labelings gives a space of valid labelings
of dimension $(m-1)^2 + (n-1)^2$.
Thus, Freyja needs at least $(m-1)^2 + (n-1)^2$ questions to win.

\paragraph{Proof of upper bound using generating functions, by Ankan Bhattacharya.}
We prove:
\begin{claim*}[Periodicity]
  Any valid labeling is doubly periodic with period $mn$.
\end{claim*}
\begin{proof}
  By Chicken McNugget, there exists $N$ such that $N$ and $N+1$
  are both nonnegative integer linear combinations of $m$ and $n$.

  Then both $mn \times N$ and $mn \times (N+1)$ rectangles have zero sum,
  so $mn \times 1$ rectangles have zero sum.
  This implies that any two cells with a vertical displacement of $mn$ are equal;
  similarly for horizontal displacements.
\end{proof}

With that in mind, consider a valid labeling.
It naturally corresponds to a generating function
\[ f(x, y) = \sum_{a=0}^{mn-1} \sum_{b=0}^{mn-1} c_{a, b} x^a y^b \]
where $c_{a, b}$ is the number in $(a, b)$.

The generating function corresponding to sums over $n \times m$ rectangles is
\[
  f(x, y) (1 + x + \dots + x^{m-1}) (1 + y + \dots + y^{n-1})
  = f(x, y) \cdot \frac{x^m - 1}{x - 1} \cdot \frac{y^n - 1}{y - 1}.
\]
Similarly, the one for $m \times n$ rectangles is
\[ f(x, y) \cdot \frac{x^n - 1}{x - 1} \cdot \frac{y^m - 1}{y - 1}. \]

Thus, the constraints for $f$ to be valid are equivalent to
\[
  f(x, y) \cdot \frac{x^m - 1}{x - 1} \cdot \frac{y^n - 1}{y - 1}
  \quad \text{and} \quad
  f(x, y) \cdot \frac{x^n - 1}{x - 1} \cdot \frac{y^m - 1}{y - 1}
\]
being zero when reduced modulo $x^{mn} - 1$ and $y^{mn} - 1$,
or, letting $\omega = \exp(2\pi i/mn)$,
both terms being zero when powers of $\omega$ are plugged in.

To restate the constraints one final time, we need
\[
  f(\omega^a, \omega^b) \cdot \frac{\omega^{am} - 1}{\omega^a - 1} \cdot \frac{\omega^{bn} - 1}{\omega^b - 1}
  = f(\omega^a, \omega^b) \cdot \frac{\omega^{an} - 1}{\omega^a - 1} \cdot \frac{\omega^{bm} - 1}{\omega^b - 1} = 0
\]
for all $a, b \in \{0, \dots, mn-1\}$.

\begin{claim*}
  This implies that $f(\omega^a, \omega^b) = 0$
  for all but at most $(m-1)^2 + (n-1)^2$ values of $(a,b) \in \{0, \dots, mn-1\}^2$.
\end{claim*}
\begin{proof}
  Consider a pair $(a,b)$ such that $f(\omega^a, \omega^b) \neq 0$.
  Then we need
  \[
    \frac{\omega^{am} - 1}{\omega^a - 1} \cdot \frac{\omega^{bn} - 1}{\omega^b - 1}
    = \frac{\omega^{an} - 1}{\omega^a - 1} \cdot \frac{\omega^{bm} - 1}{\omega^b - 1} = 0.
  \]
  This happens when (at least) one fraction in either product is zero.
  \begin{itemize}
    \ii If the first fraction is zero,
    then either $n \mid a$ and $a > 0$, or $m \mid b$ and $b > 0$.
    \ii If the second fraction is zero,
    then either $m \mid a$ and $a > 0$, or $n \mid b$ and $b > 0$.
  \end{itemize}
  If the first condition holds in both cases, then $mn \mid a$, but $0 < a < mn$,
  a contradiction. Thus if $n \mid a$, then we must have $n \mid b$, and similarly
  if $m \mid a$ then $m \mid b$.

  The former case happens $(m-1)^2$ times, and the latter case happens $(n-1)^2$ times.
  Thus, at most $(m-1)^2 + (n-1)^2$ values of $f(\omega^a, \omega^b)$ are nonzero.
\end{proof}

\begin{claim*}
  The $(mn)^2$ equations $f(\omega^a, \omega^b) = 0$ are linearly independent
  when viewed as linear equations in $(mn)^2$ variables $c_{a,b}$.
  Hence, any subset of these equations is also linearly independent.
\end{claim*}
\begin{proof}
  In general, the equation $f(x,y) = 0$ is a polynomial relation with
  $\deg_x f(x,y) = \deg_y f(x,y) < mn$.
  However, if we let $S = \{\omega^0, \omega^1, \dots, \omega^{mn-1}\}$, then
  $|S| = mn$ and we are given $f(s,s') = 0$ for all $s,s' \in S$.
  This can only happen if $f$ is the zero polynomial,
  that is, $c_{a,b} = 0$ for all $a$ and $b$.
\end{proof}

It follows that the dimension of the space of valid labelings is at
most $(m-1)^2 + (n-1)^2$, as desired.


\paragraph{Explicit version of winning algorithm by Freyja, from author.}
Suppose that $\gcd(m, n) = 1$ and $m \le n$. Let $[a, b]$ denote the set of integers
between $a$ and $b$ inclusive.

Let Freyja ask about all cells $(x, y)$ in the two squares
\begin{align*}
  S_\text{1} &= [1, m - 1] \times [1, m - 1] \\
  S_\text{2} &= [m, m + n - 2] \times [1, n - 1].
\end{align*}
In the beginning, one by one,
Freyja determines all values inside of the rectangle
$Q \coloneqq [1, m - 1] \times [m, n - 1]$.
To that end, on each step she considers some
rectangle with $m$ rows and $n$ columns such that its top left corner is in $Q$
and all of the other values in it have been determined already.
In this way, Freyja uncovers all of $Q$, starting with its lower right corner and
then proceeding upwards and to the left.

Thus Freyja can learn all numbers inside of the rectangle
\[ R \coloneqq [1, m + n - 2]
  \times [1, n - 1] = Q \cup S_\text{1} \cup S_\text{2}.  \]
See the figure below for an illustration for $(m,n) = (5,8)$.
The first cell of $Q$ is uncovered using the dotted green rectangle.
\begin{center}
\begin{asy}
  defaultpen(fontsize(18pt));
  size(9cm);
  for (int i=-1; i<=9; ++i) {
    draw( (-1,i)--(12,i), mediumgrey );
  }
  for (int i=-1; i<=12; ++i) {
    draw( (i,-1)--(i,9), mediumgrey );
  }
  filldraw(box((0,0),(4,4)), opacity(0.1)+lightred, black+1.5);
  filldraw(box((4,0),(11,7)), opacity(0.1)+yellow, black+1.5);
  filldraw(box((0,4),(4,7)), opacity(0.1)+cyan, black+1.5);
  filldraw(box((3,0),(11,5)), opacity(0.1)+green, deepgreen+dotted+1.1);
  filldraw(box((0,7),(11,8)), opacity(0.1)+purple, purple+dashed+1.1);
  label("$S_1$", (2,2), red);
  label("$S_2$", (7.5,3.5), brown);
  label("$Q$", (2,5.5), blue);
  label("$T$", (5.5,7.5), purple);
  label("$(m,n)=(5,8)$", (5.5,-1), dir(-90));
\end{asy}
\end{center}


We need one lemma:
\begin{lemma*}
  Let $m$ and $n$ be positive integers with $\gcd(m, n) = 1$.
  Consider an unknown sequence of real numbers $z_1$, $z_2$, $\dots$, $z_s$
  with $s \ge m + n - 2$. Suppose that we know the sums of all contiguous blocks
  of size either $m$ or $n$ in this sequence. Then we can determine all
  individual entries in the sequence as well.
\end{lemma*}

\begin{proof}
  By induction on $m + n$. Suppose, without loss of generality, that $m \le n$.
  Our base case is $m = 1$, which is clear. For the induction step, set $\ell =
  n - m$. Each contiguous block of size $\ell$ within $z_1$, $z_2$, $\dots$,
  $z_{s - n}$ is the difference of two contiguous blocks of sizes $m$ and $n$
  within the original sequence. By the induction hypothesis for $\ell$ and $m$,
  it follows that we can determine all of $z_1$, $z_2$, $\dots$, $z_{s - n}$.
  Then we determine the remaining $z_i$ as well, one by one, in order from left
  to right, by examining on each step an appropriate contiguous block of size $m$.
\end{proof}

Let $T$ be the rectangle $[1, m + n - 2] \times \{n\}$. By looking at
appropriate rectangles of sizes $m \times n$ and $n \times m$ such that
their top row is contained within $T$ and all of their other rows are
contained within $R$, Freyja can learn the sums of all contiguous blocks of
values of sizes $m$ and $n$ within $T$. By the Lemma, it follows that Freyja
can uncover all of $T$.

In this way, with the help of the Lemma, Freyja can extend her rectangular area
of knowledge both upwards and downwards. Once its height reaches $m + n -
2$, by the same method she will be able to extend it to the left and right as
well. This allows Freyja to determine all values in the grid. Therefore, $(m -
1)^2 + (n - 1)^2$ questions are indeed sufficient.

\begin{remark*}
The ideas in the solution also yield a proof of the following result:
\begin{quote}
Let $m$ and $n$ be relatively prime positive integers.
Consider an infinite grid of unit square cells coloured in
such a way that every rectangle of size either $m \times n$ or $n \times m$
contains the same multiset of colours.
Then the colouring is either doubly periodic with period length $m$
or doubly periodic with period length $n$.
\end{quote}

(Here, ``doubly periodic with period length $s$''
means ``both horizontally and vertically periodic with period length $s$''.)

Here is a quick sketch of the proof. Given two positive integers $i$ and $j$
with $1 \le i, j \le m - 1$, we define
\[
  f_{ij}(x, y) \coloneqq
  \begin{cases}
    +1 & \text{ when } (x, y) \equiv (0, 0) \text{ or } (i, j) \pmod{m}; \\
    -1 & \text{ when } (x, y) \equiv (0, j) \text{ or } (i, 0) \pmod{m};
      \text{ and } \\
    \phantom{+}0 & \text{ otherwise.}
  \end{cases}
\]
Define $g_{i, j}$ similarly, but with $1 \le i, j \le n - 1$ and ``mod $m$''
everywhere replaced by ``mod $n$''. First we show that if a linear combination
$h := \sum \alpha_{i, j}f_{i, j}+\sum\beta_{i, j}g_{i, j}$ of the $f_{i, j}$ and
$g_{i, j}$ contains only two distinct values, then either all of the $\alpha_{i,
j}$ vanish or all of the $\beta_{i, j}$ do. It follows that each colour,
considered in isolation, is either doubly periodic with period length $m$ or
doubly periodic with period length $n$. Finally, we check that different period
lengths cannot mix.

On the other hand, if $m$ and $n$ are not relatively prime,
then there exist infinitely many non-isomorphic valid colourings.
Furthermore, when $\gcd(m, n) = 2$, there exist valid colourings
which are not horizontally periodic; and, when $\gcd(m, n) \ge 3$,
there exist valid colourings which are neither horizontally nor vertically periodic.
\end{remark*}
