author: Maxim Li
desc: Iterated FE jumpscare
source: USA TST 2023/6
url: https://aops.com/community/p26896222
tags: [2023-01, arrows, yod]

---

Fix a function $f \colon \NN \to \NN$ and for any $m,n \in \NN$ define
\[
  \Delta(m,n) = \underbrace{f(f(\dots f}_{f(n)\text{ times}} (m)\dots)) %chktex 9
  - \underbrace{f(f(\dots f}_{f(m)\text{ times}} (n)\dots)). %chktex 9
\]
Suppose $\Delta(m,n) \neq 0$ for any distinct $m,n \in \NN$.
Show that $\Delta$ is unbounded, meaning that for any constant $C$
there exist $m,n \in \NN$ with $\left\lvert \Delta(m,n) \right\rvert > C$.

---

Suppose for the sake of contradiction that $|\Delta(m,n)|\le N$ for all $m$, $n$.
Note that $f$ is injective, as
\[ f(m)=f(n) \implies \Delta(m,n) = 0 \implies m=n, \]
as desired.

Let $G$ be the ``arrow graph'' of $f$,
which is the directed graph with vertex set $\NN$ and edges $n\to f(n)$.
The first step in the solution is to classify the structure of $G$.
Injectivity implies that $G$ is a disjoint collection of chains
(infinite and half-infinite) and cycles.
We have the following sequence of claims that further refine the structure.

\begin{claim*}
  The graph $G$ has no cycles.
\end{claim*}
\begin{proof}
  Suppose for the sake of contradiction that $f^k(n)=n$
  for some $k\ge 2$ and $n\in\NN$.
  As $m$ varies over $\NN$, we have $|\Delta(m,n)|\le N$,
  so $f^{f(n)}(m)$ can only take on some finite set of values.
  In particular, this means that
  \[f^{f(n)}(m_1)=f^{f(n)}(m_2)\]
  for some $m_1\ne m_2$, which contradicts injectivity.
\end{proof}
\begin{claim*}
  The graph $G$ has at most $2N+1$ chains.
\end{claim*}
\begin{proof}
  Suppose we have numbers $m_1$, \dots, $m_k$ in distinct chains.
  Select a positive integer $B > \max\{f(m_1), \dots,f(m_k)\}$.
  Now,
  \[ \left|\Delta\left(m_i,f^{B-f(m_i)}(1)\right)\right|\le N
    \implies \left|f^B(1)-f^{f^{B-f(m_i)+1}(1)}(m_i)\right|\le N.\]
  Since the $m_i$s are in different chains,
  we have that $f^{f^{B-f(m_i)+1}(1)}(m_i)$ are distinct for each $i$,
  which implies that $k\le 2N+1$, as desired.
\end{proof}

\begin{claim*}
  The graph $G$ consists of exactly one half-infinite chain.
\end{claim*}
\begin{proof}
  Fix some $c\in\NN$.
  Call an element of $\NN$ \textit{bad} if
  it is not of the form $f^k(c)$ for some $k\ge 0$.
  It suffices to show that there are only finitely many bad numbers.

  Since there are only finitely many chains, $f^{f(c)}(n)$ achieves all
  sufficiently large positive integers, say all positive integers at least $M$.
  Fix $A$ and $B$ such that $B>A\ge M$.
  If $f^{f(c)}(n)\in[A,B]$, then $f^{f(n)}(c)\in[A-N,B+N]$,
  and distinct $n$ generate distinct $f^{f(n)}(c)$ due to the structure of $G$.
  Therefore, we have at least $B-A+1$ good numbers in $[A-N,B+N]$,
  so there are at most $2N$ bad numbers in $[A-N,B+N]$.

  Varying $B$, this shows there are at most $2N$ bad numbers at least $A-N$.
\end{proof}

Let $c$ be the starting point of the chain,
so every integer is of the form $f^k(c)$, where $k\geq 0$.
Define a function $g \colon \ZZ_{\ge 0}\to\NN$
by \[ g(k) \coloneqq f^k(c). \]
Due to the structure of $G$, $g$ is a bijection.
Define
\[\delta(a,b) \coloneqq \Delta(f^a(c),f^b(c)) = g(g(b+1)+a)-g(g(a+1)+b),\]
so the conditions are equivalent to $|\delta(a,b)|\le N$
for all $a,b\in\ZZ_{\ge 0}$ and $\delta(a,b)\ne 0$ for $a\ne b$,
which is equivalent to $g(a+1)-a\ne g(b+1)-b$ for $a\ne b$.
This tells us that $g(x)-x$ is injective for $x\ge 1$.

\begin{lemma*}
  For all $M$, there exists a nonnegative integer $x$ with $g(x)\leq x-M$.
\end{lemma*}
\begin{proof}
  Assume for the sake of contradiction that $g(x)-x$ is bounded below.
  Fix some large positive $K$.
  Since $g(x)-x$ is injective,
  there exists $B$ such that $g(x)-x\geq K$ for all $x\geq B$.
  Then $\min \{g(B+1),g(B+2),\dots\} \ge B+K$,
  while $\{g(0),\dots,g(B)\}$ only achieve $B+1$ values.
  Thus, at least $K-1$ values are not achieved by $g$, which is a contradiction.
\end{proof}

Now pick $B$ such that $g(B)+N\le B$ and $g(B) > N$.
Note that infinitely many such $B$ exist,
since we can take $M$ to be arbitrarily small in the above lemma.
Let
\[ t = \max \{g^{-1}(g(B)-N), g^{-1}(g(B)-N+1), \dots, g^{-1}(g(B)+N)\}. \]
Note that $g(t)\le g(B)+N\le B$, so we have
\[ |\delta(t-1,B-g(t))| = |g(B)-g(t-1+g(B+1-g(t)))| \le N, \]
so
\[ t-1+g(B+1-g(t)) \in
  \{g^{-1}(g(B)-N),g^{-1}(g(B)-N+1),\dots,g^{-1}(g(B)+N)\},\]
so by the maximality of $t$, we must have $g(B+1-g(t))=1$,
so $B+1-g(t)=g^{-1}(1)$.
We have $|g(t)-g(B)|\le N$, so
\[|(B-g(B))+1-g^{-1}(1)|\le N.\]
This is true for infinitely many values of $B$,
so infinitely many values of $B-g(B)$ (by injectivity of $g(x)-x$),
which is a contradiction.
This completes the proof.
