desc:  Combinatorial nonstandard FE with $W$
author: Zilin Jiang
source:  USA TST 2016/5
tags:  [FE, linalg, adhoc, wishful, meta, 2016-04, well, dalet]
hardness: 25
url: https://aops.com/community/p6368185

---

Let $n \ge 4$ be an integer.
Find all functions $W \colon \{1, \dots, n\}^2 \to \RR$ such that
for every partition $[n] = A \cup B \cup C$ into disjoint sets,
\[ \sum_{a \in A} \sum_{b \in B} \sum_{c \in C} W(a,b) W(b,c)
= |A| |B| |C|. \]

---

Of course, $W(k,k)$ is arbitrary for $k \in [n]$.
We claim that $W(a,b) = \pm 1$ for any $a \neq b$, with the sign fixed.
(These evidently work.)

First, let $X_{abc} = W(a,b)W(b,c)$ for all distinct $a$, $b$, $c$,
so the given condition is
\[ \sum_{a,b,c \in A \times B \times C} X_{abc} = |A| |B| |C|. \]
Consider the given equation with the particular choices
\begin{itemize}
  \ii $A = \{1\}$, $B = \{2\}$, $C = \{3,4,\dots,n\}$.
  \ii $A = \{1\}$, $B = \{3\}$, $C = \{2,4,\dots,n\}$.
  \ii $A = \{1\}$, $B = \{2,3\}$, $C = \{4,\dots,n\}$.
\end{itemize}
This gives
\begin{align*}
  X_{123} + X_{124} + \dots + X_{12n} &= n-2 \\
  X_{132} + X_{134} + \dots + X_{13n} &= n-2 \\
  (X_{124} + \dots + X_{12n})
  + (X_{134} + \dots + X_{13n}) &= 2(n-3).
\end{align*}
Adding the first two and
subtracting the last one gives $X_{123} + X_{132} = 2$.
Similarly, $X_{123} + X_{321} = 2$,
and in this way we have $X_{321} = X_{132}$.
Thus $W(3,2)W(2,1) = W(1,3)W(3,2)$,
and since $W(3,2) \neq 0$ (clearly) we get $W(2,1) = W(3,2)$.

Analogously, for any distinct $a$, $b$, $c$ we have $W(a,b) = W(b,c)$.
For $n \ge 4$ this is enough to imply $W(a,b) = \pm 1$ for $a \neq b$
where the choice of sign is the same for all $a$ and $b$.

\begin{remark*}
Surprisingly, the $n = 3$ case has ``extra'' solutions for
$W(1,2) = W(2,3) = W(3,1) = \pm1$,
$W(2,1) = W(3,2) = W(1,3) = \mp1$.
\end{remark*}
\begin{remark*}
[Intuition]
It should still be possible to solve the problem
with $X_{abc}$ in place of $W(a,b) W(b,c)$,
because we have about far more equations than variables $X_{a,b,c}$
so linear algebra assures us we almost certainly have a unique solution.
\end{remark*}

---

As written, this is a system of equations with $O(n^2)$ variables,
but it is quadratic.
The idea I want to push is that linear systems are much better,
and we can act accordingly.
To that end, define \[ X_{abc} = W(a,b)W(b,c). \]
We'll essentially be solving the resulting linear system.
\begin{walk}
  \ii Observe that the values of $W(n,n)$ are unused.
  \ii How many $X$ variables are in consideration?
  \ii Asymptotically, how many equations are given in the linear system?
  The fact that this is so much larger than the answer to (b)
  suggests that the equations should really be solvable.
  \ii Pick some choices of $(A,B,C)$ and add/subtract the givens
  to get an equation with only two terms on the left-hand side.
  \ii Show that $X_{123} = X_{231} = X_{312}$.
  \ii Extract the answer for $W$.
  \ii Does the answer change if $n=3$?
\end{walk}
