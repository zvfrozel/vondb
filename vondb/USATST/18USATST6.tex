author: Mark Sellke
desc: Annulus guessing
source: USA TST 2018/6
tags: [2017-11, nice, criticalclaim, global, magic, zayin]
hardness: 45
url: https://aops.com/community/p9735613

---

Alice and Bob play a game.
First, Alice secretly picks a finite set $S$
of lattice points in the Cartesian plane.
Then, for every line $\ell$ in the plane
which is horizontal, vertical, or has slope $+1$ or $-1$,
she tells Bob the number of points of $S$ that lie on $\ell$.
Bob wins if he can then determine the set $S$.

Prove that if Alice picks $S$ to be of the form
\[ S = \left\{ (x,y) \in \ZZ^2 \mid m \le x^2+y^2 \le n \right\} \]
for some positive integers $m$ and $n$, then Bob can win.
(Bob does not know in advance that $S$ is of this form.)

---

Clearly Bob can compute the number $N$ of points.

The main claim is that:
\begin{claim*}
  Fix $m$ and $n$ as in the problem statement.
  Among all sets $T \subseteq \ZZ^2$ with $N$ points,
  the set $S$ is the \emph{unique} one which maximizes the value of
  \[ F(T):=\sum_{(x,y)\in T} (x^2+y^2)(m+n-(x^2+y^2)).  \]
\end{claim*}
\begin{proof}
  Indeed, the different points in $T$ do not interact in this sum,
  so we simply want the points $(x,y)$ with $x^2+y^2$
  as close as possible to $\frac{m+n}{2}$ which is exactly what $S$ does.
\end{proof}
As a result of this observation,
it suffices to show that Bob has enough information to
compute $F(S)$ from the data given.
(There is no issue with fixing $m$ and $n$,
since Bob can find an upper bound on the magnitude
of the points and then check all pairs $(m,n)$ smaller than that.)
The idea is that he knows the full distribution of each of
$X$, $Y$, $X+Y$, $X-Y$ and hence can compute sums
over $T$ of any power of a single one of those linear functions.
By taking linear combinations we can hence compute $F(S)$.

Let us make the relations explicit.
For ease of exposition we take $Z=(X,Y)$ to be a
uniformly random point from the set $S$.
The information is precisely the individual distributions
of $X$, $Y$, $X+Y$, and $X-Y$.
Now compute
\begin{align*}
  \frac{F(S)}{N} &= \mathbb E\left[ (m+n)(X^2+Y^2) - (X^2+Y^2)^2 \right] \\
  &= (m+n) \left( \mathbb E[X^2] + \mathbb E[Y^2] \right)
  - \mathbb E[X^4] - \mathbb E[Y^4] - 2 \mathbb E[X^2 Y^2].
\end{align*}
On the other hand,
\[ \mathbb E[X^2Y^2]
  = \frac{\mathbb E[(X+Y)^4] + \mathbb E[(X-Y)^4]
  - 2\mathbb E[X^4] - 2\mathbb E[Y^4]}{12}. \]
Thus we have written $F(S)$ in terms of the distributions
of $X$, $Y$, $X-Y$, $X+Y$ which completes the proof.

\begin{remark*}[Mark Sellke]
  \begin{itemize}
  \ii This proof would have worked just as well if we
  allowed arbitrary $[0,1]$-valued weights on points
  with finitely many weights non-zero.
  There is an obvious continuum generalization one
  can make concerning the indicator function for an annulus.
  It's a simpler but fun problem to characterize
  when just the vertical/horizontal directions determine the distribution.
  \ii An obstruction to purely combinatorial arguments
  is that if you take an octagon with points $(\pm a,\pm b)$
  and $(\pm b,\pm a)$ then the two ways to pick every other point
  (going around clockwise) are indistinguishable by Bob.
  This at least shows that Bob's task is far from possible in general,
  and hints at proving an inequality.
  \ii A related and more standard fact
  (among a certain type of person)
  is that given a probability distribution $\mu$ on $\RR^n$,
  if I tell you the distribution of \emph{all} $1$-dimensional
  projections of $\mu$, that determines $\mu$ uniquely.
  This works because this information gives me the Fourier transform $\hat{\mu}$,
  and Fourier transforms are injective.

  For the continuum version of this problem,
  this connection gives a much larger family of counterexamples
  to any proposed extension to arbitrary non-annular shapes.
  Indeed, take a fast-decaying smooth function
  $f \colon \RR^2 \to \RR$ which vanishes on the four lines
  \[ x=0, \; y=0, \; x+y=0, \; x-y=0.\]
  Then the Fourier transform $\hat f$ will have mean $0$
  on each line $\ell$ as in the problem statement.
  Hence the positive and negative parts of $\hat f$
  will not be distinguishable by Bob.
  \end{itemize}
\end{remark*}
