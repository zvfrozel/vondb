author: Ashwin Sah, Yang Liu
desc: Valiant functions
source: USA TST 2019/2
tags: [2018-12, find, mods, nice, global, manip, polynomial, exactsum, zayin]
hardness: 40
url: https://aops.com/community/p11419598

---

Let $\ZZ/n\ZZ$ denote the set of integers considered modulo $n$
(hence $\ZZ/n\ZZ$ has $n$ elements).
Find all positive integers $n$ for which there exists a bijective function
$g \colon \ZZ/n\ZZ \to \ZZ/n\ZZ$,
such that the $101$ functions
\[ g(x), \quad g(x)+x, \quad g(x)+2x, \quad \dots, \quad g(x)+100x \]
are all bijections on $\ZZ/n\ZZ$.

---

Call a function $g$ \emph{valiant} if it obeys this condition.
We claim the answer is all numbers relatively prime to $101!$.

The construction is to just let $g$ be the identity function.

Before proceeding to the converse solution,
we make a long motivational remark.
\begin{remark*}
  [Motivation for both parts]
  The following solution is dense,
  and it is easier to think about some small cases first,
  to motivate the ideas.
  We consider the result
  where $101$ is replaced by $2$ or $3$.
  \begin{itemize}
    \ii If we replaced $101$ with $2$,
    you can show $2 \nmid n$ easily: write
    \[ \sum_x x \equiv \sum_x g(x) \equiv \sum_x (g(x) + x ) \pmod n \]
    which implies
    \[ 0 \equiv \sum_x x = \half n (n+1) \pmod n \]
    which means $\half n(n+1) \equiv 0 \pmod n$, hence $n$ odd.
    \ii If we replaced $101$ with $3$,
    then you can try a similar approach using squares, since
    \begin{align*}
      0 &\equiv \sum_x \left[ \left( g(x)+2x \right)^2
      - 2\left( g(x)+x \right)^2
      + g(x)^2 \right] \pmod n\\
      &= \sum_x 2x^2 = 2 \cdot \frac{n(n+1)(2n+1)}{6}
    \end{align*}
    which is enough to force $3 \nmid n$.
  \end{itemize}
\end{remark*}
We now present several different proofs of the converse,
all of which generalize the ideas contained here.
In everything that follows we assume $n > 1$ for convenience.

\paragraph{First solution (original one).}
The proof is split into two essentially orthogonal claims,
which we state as lemmas.
\begin{lemma*}
  [Lemma I: elimination of $g$]
  Assume valiant $g \colon \ZZ/n\ZZ \to \ZZ/n\ZZ$ exists.
  Then \[ k! \sum_{x \in \ZZ/n\ZZ} x^k \equiv 0 \pmod{n} \]
  for $k = 0, 1, \dots, 100$.
\end{lemma*}

\begin{proof}
  Define $g_x(T) = g(x) + Tx$ for any integer $T$.
  If we view $g_x(T)^k$
  as a polynomial in $\ZZ[T]$ of degree $k$
  with leading coefficient $x^k$,
  then taking the $k$th finite difference implies that,
  for any $x$,
  \[ k! x^k = \binom k0 g_x(k)^k
    - \binom k1 g_x(k-1)^k  + \binom k2 g_x(k-2)^k - \dots
    + (-1)^k \binom kk g_x(0)^k.  \]
  On the other hand, for any $1 \le k \le 100$
  we should have
  \begin{align*}
    \sum_x g_x(0)^k \equiv \sum_x g_x(1)^k
    &\equiv \dots \equiv \sum_x g_x(k)^k \\
    &\equiv S_k \coloneqq 0^k + \dots + (n-1)^k \pmod n
  \end{align*}
  by the hypothesis.
  Thus we find
  \[ k! \sum_x x^k
    \equiv \left[ \binom k0 - \binom k1 + \binom k2 - \dotsb \right] S_k
    \equiv 0 \pmod n \]
  for any $1 \le k \le 100$, but also obviously for $k = 0$.
\end{proof}

We now prove the following self-contained lemma.
\begin{lemma*}
  [Lemma II: power sum calculation]
  Let $p$ be a prime, and let $n$, $M$ be positive integers such that
  \[ M \quad\text{ divides }\quad 1^k + 2^k + \dots + n^k \]
  for $k = 0, 1, \dots, p-1$.
  If $p \mid n$ then $\nu_p(M) < \nu_p(n)$.
\end{lemma*}

\begin{proof}
  The hypothesis means that
  that any polynomial $f(T) \in \ZZ[T]$
  with $\deg f \le p-1$ will have $\sum_{x=1}^n f(x) \equiv 0 \pmod{M}$.
  In particular, we have
  \begin{align*}
    0 &\equiv \sum_{x=1}^n (x-1)(x-2) \dotsm (x-(p-1)) \\
    &= (p-1)! \sum_{x=1}^{n} \binom{x-1}{p-1}
    = (p-1)! \binom np \pmod{M}.
  \end{align*}
  But now $\nu_p(M) \le \nu_p(\binom np) = \nu_p(n) - 1$.
\end{proof}

Now assume for contradiction that valiant
$g \colon \ZZ/n\ZZ \to \ZZ/n\ZZ$ exists,
and $p \le 101$ is the \emph{smallest} prime dividing $n$.
Lemma I implies that $k! \sum_x x^k \equiv 0 \pmod n$
for $k = 1, \dots, p-1$
and hence $\sum_x x^k \equiv 0 \pmod n$ too.
Thus $M = n$ holds in the previous lemma, impossible.

\paragraph{A second solution.}
Both lemmas above admit variations
where we focus on working modulo $p^e$ rather than working modulo $n$.
\begin{lemma*}
  [Lemma I']
  Assume valiant $g \colon \ZZ/n\ZZ \to \ZZ/n\ZZ$ exists.
  Let $p \le 101$ be a prime, and $e = \nu_p(n)$.
  Then \[ \sum_{x \in \ZZ/n\ZZ} x^k \equiv 0 \pmod{p^e} \]
  for $k = 0, 1, \dots, p-1$.
\end{lemma*}
\begin{proof}
  This is weaker than Lemma I,
  but we give an independent specialized proof.
  Begin by writing
  \[ \sum_x \left( g(x) + Tx \right)^k \equiv \sum_x x^k \pmod{p^e}. \]
  Both sides are integer polynomials in $T$,
  which vanish at $T = 0, 1, \dots, p-1$ by hypothesis
  (since $p-1 \le 100$).

  We now prove the following more general fact:
  if $f(T) \in \ZZ[T]$ is an integer polynomial with $\deg f \le p-1$,
  such that $f(0) \equiv \dots \equiv f(p-1) \equiv 0 \pmod{p^e}$,
  then all coefficients of $f$ are divisible by $p^e$.
  The proof is by induction on $e \ge 1$.
  When $e = 1$, this is just the assertion that
  the polynomial has at most $\deg f$ roots modulo $p$.
  When $e \ge 2$, we note that the previous result
  implies all coefficients are divisible by $p$,
  and then we divide all coefficients by $p$.

  Applied here, we have that all coefficients of
  \[ f(T) \coloneqq \sum_x \left( g(x) + Tx \right)^k - \sum_x x^k \]
  are divisible by $p^e$.
  The leading $T^k$ coefficient is $\sum_k x^k$ as desired.
\end{proof}

\begin{lemma*}
  [Lemma II']
  If $e \ge 1$ is an integer,
  and $p$ is a prime, then
  \[ \nu_p\left( 1^{p-1} + 2^{p-1} + \dots + (p^e-1)^{p-1}  \right)
    = e-1. \]
\end{lemma*}
\begin{proof}
  First, note that the cases where $p = 2$ or $e = 1$
  are easy; since if $p = 2$ we have
  $\sum_{x = 0}^{2^e - 1} x \equiv 2^{e - 1}(2^e - 1)
    \equiv -2^{e - 1}\pmod{2^e}$,
  while if $e = 1$ we have $1^{p-1} + \dots + (p-1)^{p-1} \equiv -1 \pmod p$.
  Henceforth assume that $p > 2$, $e > 1$.

  Let $g$ be an integer which is a primitive root modulo $p^e$.
  Then, we can sum the terms which are relatively prime to $p$ as
  \[ S_0 \coloneqq \sum_{\gcd(x,p) = 1} x^{p-1}
    \equiv \sum_{i=1}^{\varphi(p^e)} g^{(p-1) \cdot i}
    \equiv \frac{g^{p^{e-1}(p-1)^2} - 1}{g^{p-1}-1}
    \pmod{p^e}
  \]
  which implies $\nu_p(S_0) = e-1$, by lifting the exponent.
  More generally, for $r \ge 1$ we may set
  \[ S_r \coloneqq \sum_{\nu_p(x) = r} x^{p-1}
    \equiv (p^r)^{p-1} \sum_{i=1}^{\varphi(p^{e-r})}
      g_r^{(p-1) \cdot i}
    \pmod{p^e}
  \]
  where $g_r$ is a primitive root modulo $p^{e-r}$.
  Repeating the exponent-lifting calculation
  shows that $\nu_p(S_r) = r(p-1) + \left( (e-r)-1 \right) > e$,
  as needed.
\end{proof}

Assume to the contrary that $p \le 101$ is a prime dividing
$n$, and a valiant $g \colon \ZZ / n \ZZ \to \ZZ / n \ZZ$ exists.
Take $k = p-1$ in Lemma I' to contradict Lemma II'

\paragraph{A third remixed solution.}
We use Lemma I and Lemma II' from before.
As before, assume $g \colon \ZZ / n \ZZ \to \ZZ / n \ZZ$ is valiant,
and $n$ has a prime divisor $p \le 101$.
Also, let $e = \nu_p(n)$.

Then $(p-1)! \sum_x x^{p-1} \equiv 0 \pmod{n}$ by Lemma I,
and now
\begin{align*}
  0 & \equiv \sum_x x^{p-1} \pmod{p^e} \\
  &\equiv \frac{n}{p^e} \sum_{x=1}^{p^e-1} x^{p-1} \not\equiv 0 \pmod{p^e}
\end{align*}
by Lemma II', contradiction.

\paragraph{A fourth remixed solution.}
We also can combine Lemma I' and Lemma II.
As before, assume $g \colon \ZZ / n \ZZ \to \ZZ / n \ZZ$ is valiant,
and let $p$ be the smallest prime divisor of $n$.

Assume for contradiction $p \le 101$.
By Lemma I' we have
\[ \sum_x x^k \equiv 0 \pmod{p^e} \]
for $k = 0, \dots, p-1$.
This directly contradicts Lemma II with $M = p^e$.

---


\begin{remark*}
  [Possible mistake]
  It is tempting to try to take $k = \varphi(p^e)$,
  but then $k > p-1$ and indeed we could have $k > 100$,
  so this would not imply anything as the first lemma fails.
\end{remark*}


Finally, let $\alpha = \nu_p(n)$.
Then taking $k = \varphi(p^\alpha)$ gives
\[ \sum_{x=0}^{n-1} x^k
  \equiv  \frac{n}{p^\alpha} \cdot \varphi(p^\alpha)
  = n(1-1/p) \pmod{p^\alpha} \]
which is not divisible by $p^\alpha$, contradiction.
