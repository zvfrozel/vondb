desc:  Bezout's Theorem
author: Ivan Borsenco
source:  USA TST 2016/6
tags:  [complex, polynomial, highermath, instructive, find, favorite, 2016-04, well, zayin]
hardness: 40
url: https://aops.com/community/p6368189

---

Let $ABC$ be an acute scalene triangle
and let $P$ be a point in its interior.
Let $A_1$, $B_1$, $C_1$ be projections of $P$ onto
triangle sides $BC$, $CA$, $AB$, respectively.
Find the locus of points $P$ such that
$AA_1$, $BB_1$, $CC_1$ are concurrent
and $\angle PAB + \angle PBC + \angle PCA = 90\dg$.

---

In complex numbers with $ABC$ the unit circle,
it is equivalent to solving the following two cubic equations
in $p$ and $q = \ol p$:
\begin{align*}
 (p-a)(p-b)(p-c) &= (abc)^2 (q -1/a)(q - 1/b)(q - 1/c) \\
 0 &= \prod_{\text{cyc}} (p+c-b-bcq) + \prod_{\text{cyc}} (p+b-c-bcq).
\end{align*}
Viewing this as two cubic curves in $(p,q) \in \CC^2$,
by \emph{B\'ezout's Theorem} it follows there are at most nine solutions
(unless both curves are not irreducible,
but it's easy to check the first one cannot be factored).
Moreover it is easy to name nine solutions (for $ABC$ scalene):
the three vertices, the three excenters, and $I$, $O$, $H$.
Hence the answer is just those three triangle centers $I$, $O$ and $H$.

\begin{remark*}
  On the other hand it is not easy to solve the cubics by hand;
  I tried for an hour without success.
  So I think this solution is only feasible
  with knowledge of algebraic geometry.
\end{remark*}
\begin{remark*}
  These two cubics have names:
  \begin{itemize}
    \ii The locus of $\angle PAB + \angle PBC + \angle PCA = 90\dg$
    is the \textbf{McCay cubic},
    which is the locus of points $P$
    for which $P$, $P^\ast$, $O$ are collinear.
    \ii The locus of the pedal condition
    is the \textbf{Darboux cubic},
    which is the locus of points $P$
    for which $P$, $P^\ast$, $L$ are collinear,
    $L$ denoting the de Longchamps point.
  \end{itemize}
  Assuming $P \neq P^\ast$,
  this implies $P$ and $P^\ast$
  both lie on the Euler line of $\triangle ABC$,
  which is possible only if $P=O$ or $P=H$.

  Some other synthetic solutions are posted at
  \url{https://aops.com/community/c6h1243902p6368189}.
\end{remark*}

---

We will use complex numbers with $a$, $b$, $c$ the unit circle.
\begin{walk}
  \ii Find three distinct examples of points $P$ that work.
  \ii Show that the locus of $P$ ``should'' have $0$ degrees of freedom,
  i.e.\ it should be a finite set of points.
  (This is again dimension counting.)
  \ii Write down an equation in $\CC$ for $p$
  corresponding to the condition that $\angle PAB + \angle PBC + \angle PCA = 90\dg$.
  You don't have to worry about configuration issues:
  just ensure that the equation you have is valid
  for all points $P$ in the interior of $ABC$.
  \ii Write down an equation in $\CC$ for $p$
  corresponding to $AA_1$, $BB_1$, $CC_1$ are concurrent.
  \ii Rewrite (or redo) both equations
  so that they are degree $3$ polynomials in $p$ and $\ol p$.
\end{walk}
The curves you found in (c) and (d)
are called the \textbf{McCay cubic}
and \textbf{Darboux cubic}, respectively.
You can take for granted they are distinct and nondegenerate
(proving this is surprisingly obnoxious;
I've been putting off doing it for a few years now).

Once you have this, we can view the curves
as two equations in $\CC^2$ in two variables $p$ and $q = \ol p$.
After this, it follows from \textbf{B\'{e}zout theorem}
that the curves intersect in exactly $3 \cdot 3 = 9$ points,
with multiplicity.
\begin{walk}[resume]
  \ii If the incenter $I$ lies on both curves,
  what other three points must also lie on both curves?
  \ii Find three more points which
  satisfy the equations (algebraically;
  the geometric interpretation won't make much sense here).
  \ii Conclude that the nine points you found in (a), (f), (g)
  are the only solutions,
  and identify the ones that do lie inside $ABC$.
\end{walk}
