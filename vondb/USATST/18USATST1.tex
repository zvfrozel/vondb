author: Ashwin Sah
desc: nth smallest relatively prime to n is $\sigma(n)$
source: USA TST 2018/1
tags: [2017-11, grinding, reliable, size, manip, nice, primes, bet]
hardness: 25
url: https://aops.com/community/p9513094

---

Let $n \ge 2$ be a positive integer,
and let $\sigma(n)$ denote the sum of the positive divisors of $n$.
Prove that the $n$\textsuperscript{th} smallest positive integer
relatively prime to $n$ is at least $\sigma(n)$,
and determine for which $n$ equality holds.

---

The equality case is $n = p^e$ for
$p$ prime and a positive integer $e$.
It is easy to check that this works.

\bigskip

\paragraph{First solution.}
In what follows, by $[a,b]$ we mean $\{a,a+1,\dots,b\}$.
First, we make the following easy observation.
\begin{claim*}
  If $a$ and $d$ are positive integers,
  then precisely $\varphi(d)$ elements
  of $[a, a + d - 1]$ are relatively prime to $d$.
\end{claim*}

Let $d_1$, $d_2$, \dots, $d_k$ denote
denote the divisors of $n$ in some order.
Consider the intervals
\begin{align*}
  I_1 &= [1, d_1], \\
  I_2 &= [d_1+1, d_1+d_2] \\
  &\vdotswithin= \\
  I_k &= [d_1+\dots+d_{k-1}+1, d_1+\dots+d_k].
\end{align*}
of length $d_1, \dots, d_k$ respectively.
The $j$th interval will have exactly $\varphi(d_j)$ elements
which are relatively prime $d_j$,
hence at most $\varphi(d_j)$ which are relatively prime to $n$.
Consequently, in $I = \bigcup_{j=1}^k I_k$
there are at most
\[ \sum_{j=1}^k \varphi(d_j) = \sum_{d \mid n} \varphi(d) = n \]
integers relatively prime to $n$.
On the other hand $I = [1,\sigma(n)]$ so this implies the inequality.

We see that the equality holds for $n = p^e$.
Assume now $p < q$ are distinct primes dividing $n$.
Reorder the divisors $d_i$ so that $d_1 = q$.
Then $p,q \in I_1$, and so $I_1$ should contain
strictly fewer than $\varphi(d_1)=q-1$ elements
relatively prime to $n$, hence the inequality is strict.

\paragraph{Second solution (Ivan Borsenco and Evan Chen).}
Let $n = p_1^{e_1} \dots p_k^{e_k}$,
where $p_1 < p_2 < \dots$.
We are going to assume $k \ge 2$,
since the $k=1$ case was resolved in the very beginning,
and prove the strict inequality.

For a general $N$, the number of relatively prime integers in $[1,N]$ is
given exactly by
\[ f(N) = N - \sum_i \left\lfloor \frac{N}{p_i} \right\rfloor
  + \sum_{i<j} \left\lfloor \frac{N}{p_i p_j} \right\rfloor - \dots \]
according to the inclusion-exclusion principle.
So, we wish to show that $f(\sigma(n)) < n$ (as $k \ge 2$).
Discarding the error terms from the floors
(noting that we get at most $1$ from the negative floors)
gives
\begin{align*}
  f(N) &< 2^{k-1} + N - \sum_i \frac{N}{p_i}
    + \sum_{i<j} \frac{N}{p_i p_j} - \dots \\
  &= 2^{k-1} + N \prod_i \left( 1 - p_i\inv \right) \\
  &= 2^{k-1} + \prod_i \left( 1 - p_i\inv \right)
    \left( 1 + p_i + p_i^2 + \dots + p_i^{e_i} \right) \\
  &= 2^{k-1} + \prod_i \left( p_i^{e_i} - p_i\inv \right).
\end{align*}
The proof is now divided into two cases.
If $k=2$ we have
\begin{align*}
  f(N) &< 2 + \left( p_1^{e_1} - p_1\inv \right)\left( p_2^{e_2} -
  p_2\inv \right) \\
  &= 2 + n - \frac{p_2^{e_2}}{p_1} - \frac{p_1^{e_1}}{p_2}
    + \frac{1}{p_1p_2} \\
  &\le 2 + n - \frac{p_2}{p_1} - \frac{p_1}{p_2}
    + \frac{1}{p_1p_2} \\
  &= n + \frac{1-(p_1-p_2)^2}{p_1p_2} \le n.
\end{align*}
On the other hand if $k \ge 3$ we may now write
\begin{align*}
  f(N) &< 2^{k-1} + \left[ \prod_{i=2}^{k-1} \left( p_i^{e_i} \right)
    \right] \left( p_1^{e_1} - p_1\inv \right) \\
  &= 2^{k-1} + n - \frac{p_2^{e_2} \dots p_k^{e_k}}{p_1} \\
  &\le 2^{k-1} + n - \frac{p_2 p_3 \dots p_k}{p_1}.
\end{align*}
If $p_1 = 2$, then one can show
by induction that $p_2 p_3 \dots p_k \ge 2^{k+1}-1$,
which implies the result.
If $p_1 > 2$, then one can again show by induction
$p_3 \dots p_k \ge 2^k-1$ (since $p_3 \ge 7$),
which also implies the result.
