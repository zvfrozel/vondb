author: Nikolai Beluhov
desc: Decimal digits of $P(n)$ always Fibonacci
source: TSTST 2019/6
tags: [2019-07, yesno, polynomial, wishful, nice, mods, gimel]
hardness: 50
url: https://aops.com/community/p12608536

---

Suppose $P$ is a polynomial with integer coefficients
such that for every positive integer $n$,
the sum of the decimal digits of $|P(n)|$
is not a Fibonacci number.
Must $P$ be constant?

---

The answer is yes, $P$ must be constant.
By $S(n)$ we mean the sum of the decimal digits of $|n|$.

We need two claims.
\begin{claim*}
  If $P(x) \in \ZZ[x]$ is nonconstant with
  positive leading coefficient,
  then there exists an integer polynomial $F(x)$
  such that all coefficients of $P \circ F$
  are positive except for the second one, which is negative.
\end{claim*}
\begin{proof}
  We will actually construct a cubic $F$.
  We call a polynomial \emph{good} if it has the property.

  First, consider $T_0(x) = x^3+x+1$.
  Observe that in $T_0^{\deg P}$,
  every coefficient is strictly positive,
  except for the second one, which is zero.

  Then, let $T_1(x) = x^3 - \frac1D x^2 + x + 1$.
  Using continuity as $D \to \infty$,
  it follows that if $D$ is large enough (in terms of $\deg P$),
  then $T_1^{\deg P}$ is good,
  with $-\frac3D x^{3\deg P-1}$ being the only negative coefficient.

  Finally, we can let $F(x) = CT_1(x)$
  where $C$ is a sufficiently large multiple of $D$
  (in terms of the coefficients of $P$);
  thus the coefficients of $(CT_1(x))^{\deg P}$ dominate
  (and are integers), as needed.
\end{proof}
\begin{claim*}
  There are infinitely many Fibonacci numbers
  in each residue class modulo $9$.
\end{claim*}
\begin{proof}
  Note the Fibonacci sequence is periodic modulo $9$
  (indeed it is periodic modulo any integer).
  Moreover (allowing negative indices),
  \begin{align*}
    F_{0} = 0 &\equiv 0 \pmod 9 \\
    F_{1} = 1 &\equiv 1 \pmod 9 \\
    F_{3} = 2 &\equiv 2 \pmod 9 \\
    F_{4} = 3 &\equiv 3 \pmod 9 \\
    F_{7} = 13 &\equiv 4 \pmod 9 \\
    F_{5} = 5 &\equiv 5 \pmod 9 \\
    F_{-4} = -3 &\equiv 6 \pmod 9 \\
    F_{9} = 34 &\equiv 7 \pmod 9 \\
    F_{6} = 8 &\equiv 8 \pmod 9. \qedhere
  \end{align*}
\end{proof}

We now show how to solve the problem with the two claims.
WLOG $P$ satisfies the conditions of the first claim,
and choose $F$ as above.
Let
\[ P(F(x)) = c_N x^N - c_{N-1} x^{N-1} + c_{N-2} x^{N-2} + \dots + c_0 \]
where $c_i > 0$ (and $N = 3 \deg P$).
Then if we select $x = 10^e$ for $e$ large enough
(say $x > 10\max_i c_i$),
the decimal representation $P(F(10^e))$ consists of the concatenation of
\begin{itemize}
  \ii the decimal representation of $c_N-1$,
  \ii the decimal representation of $10^e-c_{N-1}$
  \ii the decimal representation of $c_{N-2}$,
  with several leading zeros,
  \ii the decimal representation of $c_{N-3}$,
  with several leading zeros,
  \ii \dots
  \ii the decimal representation of $c_0$, with several leading zeros.
\end{itemize}
(For example,
if $P(F(x)) = 15x^3 - 7x^2 + 4x + 19$,
then $P(F(1000)) = 14{,}993{,}004{,}019$.)
Thus, the sum of the digits of this expression
is equal to
\[ S(P(F(10^e))) = 9e + k \]
for some constant $k$ depending only on $P$ and $F$,
independent of $e$.
But this will eventually hit a Fibonacci number by the second claim,
contradiction.

\begin{remark*}
  It is important to control the number of negative coefficients
  in the created polynomial.
  If one tries to use this approach on a polynomial $P$
  with $m > 0$ negative coefficients,
  then one would require that the Fibonacci sequence
  is surjective modulo $9m$ for any $m > 1$,
  which is not true:
  for example the Fibonacci sequence avoids all numbers
  congruent to $4 \mod{11}$ (and thus $4 \mod{99}$).

  In bases $b$ for which surjectivity modulo $b-1$ fails,
  the problem is false.
  For example, $P(x) = 11x+4$ will avoid all Fibonacci
  numbers if we take sum of digits in base $12$,
  since that base-$12$ sum is necessarily $4 \pmod{11}$,
  hence not a Fibonacci number.
\end{remark*}


---


(A \emph{Fibonacci number} is an element of the sequence
$F_0$, $F_1$, \dots\ defined recursively by
$F_0 = 0$, $F_1 = 1$, and $F_{k+2} = F_{k+1} + F_k$ for $k \ge 0$.)
