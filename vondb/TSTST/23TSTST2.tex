author: Raymond Feng, Luke Robitaille
desc: Zeta-like Cauchy inequality
hardness: 30
source: TSTST 2023/2
url: https://aops.com/community/p28015692
tags: [2023-06, magic, holder, gimel]

---

Let $n \ge m \ge 1$ be integers.
Prove that
\[ \sum_{k=m}^n \left( \frac{1}{k^2} + \frac{1}{k^3} \right)
  \ge m \cdot \left( \sum_{k=m}^n \frac{1}{k^2} \right)^2. \]

---

We show several approaches.

\paragraph{First solution (authors).}
By Cauchy-Schwarz, we have
\begin{align*}
  \sum_{k=m}^n\frac{k+1}{k^3}
  &= \sum_{k=m}^n\frac{\left(\frac1{k^2}\right)^2}{\frac1{k(k+1)}} \\
  &\geq
  \frac{
  \left( \frac{1}{m^2} + \frac{1}{(m+1)^2} + \dots + \frac{1}{n^2} \right)^2
  }
  {
    \frac{1}{m(m+1)} + \frac{1}{(m+1)(m+2)} + \dots + \frac{1}{n(n+1)}
  } \\
  &=
  \frac{
  \left( \frac{1}{m^2} + \frac{1}{(m+1)^2} + \dots + \frac{1}{n^2} \right)^2
  }
  { \frac 1m - \frac{1}{n+1} } \\
  &> \frac{\left(\sum_{k=m}^n\frac1{k^2}\right)^2}{\frac1m}
\end{align*}
as desired.

\begin{remark*}[Bound on error]
  Let $A = \sum_{k=m}^n k^{-2}$ and $B = \sum_{k=m}^n k^{-3}$. The inequality
  above becomes tighter for large $m$ and $n \gg m$. If we use Lagrange's
  identity in place of Cauchy-Schwarz,
  we get \[ A+B-mA^2=m\cdot\sum_{m\leq a<b} \frac{(a-b)^2}{a^3b^3(a+1)(b+1)}.\]
  We can upper bound this error by
  \[
    \leq m\cdot\sum_{m\leq a<b}\frac1{a^3(a+1)b(b+1)}
    =m\cdot\sum_{m\leq a}\frac1{a^3(a+1)^2}
    \approx m\cdot\frac1{m^4} = \frac1{m^3},
  \]
  which is still generous as $(a-b)^2 \ll b^2$
  for $b$ not much larger than $a$,
  so the real error is probably around $\frac1{10m^3}$.
  This exhibits the tightness of the inequality since it implies
  \[ mA^2+O(B/m) > A+B. \]
\end{remark*}

\begin{remark*}[Construction commentary, from author]
  My motivation was to write an inequality where
  Titu could be applied creatively to yield a telescoping sum.
  This can be difficult because most of the time,
  such a reverse-engineered inequality will be so loose it's trivial anyways.
  My first attempt was the not-so-amazing inequality
  \[\frac{n^2+3n}2=\sum_1^n i+1=\sum_1^n \frac{\frac1i}{\frac1{i(i+1)}}
    >\left( \sum_1^n \frac1{\sqrt i} \right)^2,\]
  which is really not surprising given that
  $\sum\frac1{\sqrt i}\ll \frac n{\sqrt2}$.
  The key here is that we need ``near-equality''
  as dictated by the Cauchy-Schwarz equality case,
  i.e.\ the square root of the numerators should be
  approximately proportional to the denominators.

  This motivates using $\frac1{i^4}$ as the numerator,
  which works like a charm. After working out the resulting statement,
  the LHS and RHS even share a sum,
  which adds to the simplicity of the problem.

  The final touch was to unrestrict the starting value of the sum,
  since this allows the strength of the estimate
  $\frac1{i^2}\approx\frac1{i(i+1)}$ to be fully exploited.
\end{remark*}

\paragraph{Second approach by inducting down, Luke Robitaille and Carl Schildkraut.}
Fix $n$; we'll induct downwards on $m$.
For the base case of $n=m$ the result is easy,
since the left side is $\frac{m+1}{m^3}$ and the right side is $\frac m{m^4}=\frac1{m^3}$.

For the inductive step, suppose we have shown the result for $m+1$.
Let
\[A=\sum_{k=m+1}^n\frac1{k^2}\qquad \text{and}\qquad B=\sum_{k=m+1}^n\frac1{k^3}.\]
We know $A+B\geq (m+1)A^2$, and we want to show
\[\left(A+\frac1{m^2}\right)+\left(B+\frac1{m^3}\right)\geq m\left(A+\frac1{m^2}\right)^2.\]
Indeed,
\begin{align*}
  \left(A+\frac1{m^2}\right)+\left(B+\frac1{m^3}\right)-m\left(A+\frac1{m^2}\right)^2
  &=A+B+\frac{m+1}{m^3}-mA^2-\frac{2A}m-\frac1{m^3}\\
  &=\left(A+B-(m+1)A^2\right)+\left(A-\frac1m\right)^2\geq 0,
\end{align*}
and we are done.


\paragraph{Third approach by reducing $n \to \infty$, Michael Ren and Carl Schildkraut.}
First, we give:
\begin{claim*}
  [Reduction to $n \to \infty$]
  If the problem is true when $n \to \infty$, it is true for all $n$.
\end{claim*}
\begin{proof}
  Let $A = \sum_{k=m}^n k^{-2}$ and $B = \sum_{k=m}^n k^{-3}$.
  Consider the region of the $xy$-plane defined by $y > mx^2 - x$.
  We are interested in whether $(A,B)$ lies in this region.

  However, the region is bounded by a convex curve,
  and the sequence of points $(0,0)$,
  $\left( \frac1{m^2},\frac1{m^3} \right)$,
  $\left( \frac1{m^2}+\frac1{(m+1)^2},\frac1{m^3}+\frac1{(m+1)^3} \right)$,
  $\dots$ has successively decreasing slopes between consecutive points.
  Thus it suffices to check that the inequality is true when $n\to\infty$.
\end{proof}

Set $n=\infty$ henceforth.
Let
\[ A=\sum_{k=m}^\infty\frac1{k^2}\text{ and }B=\sum_{k=m}^\infty \frac1{k^3}; \]
we want to show $B\geq mA^2-A$, which rearranges to
\[ 1+4mB\geq (2mA-1)^2. \]
Write
\[ C=\sum_{k=m}^\infty\frac1{k^2(2k-1)(2k+1)}
  \text{ and }D=\sum_{k=m}^\infty \frac{8k^2-1}{k^3(2k-1)^2(2k+1)^2}. \]
Then
\[ \frac2{2k-1}-\frac2{2k+1}=\frac1{k^2}+\frac1{k^2(2k-1)(2k+1)}, \]
and
\[ \frac2{(2k-1)^2}-\frac2{(2k+1)^2}
  = \frac1{k^3}+\frac{8k^2-1}{k^3(2k-1)^2(2k+1)^2}, \]
so that
\[ A=\frac2{2m-1}-C \text{ and } B=\frac2{(2m-1)^2}-D. \]
Our inequality we wish to show becomes
\[ \frac{2m+1}{2m-1}C\geq D+mC^2. \]

We in fact show two claims:
\begin{claim*}
  We have \[ \frac{2m+1/2}{2m-1}C\geq D. \]
\end{claim*}
\begin{proof}
  We compare termwise; we need
  \[ \frac{2m+1/2}{2m-1}\cdot \frac1{k^2(2k-1)(2k+1)}
    \geq \frac{8k^2-1}{k^3(2k-1)^2(2k+1)^2} \]
  for $k\geq m$. It suffices to show
  \[ \frac{2k+1/2}{2k-1}\cdot \frac1{k^2(2k-1)(2k+1)}
    \geq \frac{8k^2-1}{k^3(2k-1)^2(2k+1)^2}, \]
  which is equivalent to $k(2k+1/2)(2k+1)\geq 8k^2-1$.
  This holds for all $k\geq 1$.
\end{proof}
\begin{claim*}
  We have \[ \frac{1/2}{2m-1}C\geq mC^2. \]
\end{claim*}
\begin{proof}
  We need $C\leq 1/(2m(2m-1))$; indeed,
  \[ \frac1{2m(2m-1)}
    = \sum_{k=m}^\infty\left(\frac1{2k(2k-1)} - \frac1{2(k+1)(2k+1)}\right)
    = \sum_{k=m}^\infty\frac{4k+1}{2k(2k-1)(k+1)(2k+1)}; \]
  comparing term-wise with the definition of $C$ and
  using the inequality $k(4k+1)\geq 2(k+1)$ for $k\geq 1$
  gives the desired result.
\end{proof}
Combining the two claims finishes the solution.

\paragraph{Fourth approach by bashing, Carl Schildkraut.}
With a bit more work, the third approach can be adapted to avoid the $n\to\infty$ reduction.
Similarly to before, define
\[ A=\sum_{k=m}^n\frac1{k^2}\text{ and }B=\sum_{k=m}^n \frac1{k^3}; \]
we want to show $1+4mB\geq (2mA-1)^2$. Writing
\[ C=\sum_{k=m}^n\frac1{k^2(2k-1)(2k+1)}
  \text{ and }D=\sum_{k=m}^n \frac{8k^2-1}{k^3(2k-1)^2(2k+1)^2}. \]
We compute
\[ A=\frac2{2m-1}-\frac2{2n+1}-C \text{ and } B=\frac2{(2m-1)^2}-\frac2{(2n+1)^2}-D. \]
Then, the inequality we wish to show reduces (as in the previous solution) to
\[
  \frac{2m+1}{2m-1}C+\frac{2(2m+1)}{(2m-1)(2n+1)}
  \geq D+mC^2+\frac{2(2m+1)}{(2n+1)^2}+\frac{4m}{2n+1}C.
\]

We deal first with the terms not containing the variable $n$, i.e.\ we show that
\[ \frac{2m+1}{2m-1}C\geq D+mC^2. \]
For this part, the two claims from the previous solution go through exactly
as written above, and we have $C\leq 1/(2m(2m-1))$.
We now need to show
\[ \frac{2(2m+1)}{(2m-1)(2n+1)} \geq \frac{2(2m+1)}{(2n+1)^2}+\frac{4m}{2n+1}C \]
(this is just the inequality between the remaining terms);
our bound on $C$ reduces this to proving
\[\frac{4(2m+1)(n-m+1)}{(2m-1)(2n+1)^2}\geq \frac2{(2m-1)(2n+1)}.\]
Expanding and writing in terms of $n$, this is equivalent to
\[n\geq \frac{1+2(m-1)(2m+1)}{4m}=m-\frac{2m+1}{4m},\]
which holds for all $n\geq m$.
