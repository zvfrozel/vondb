desc: $x^2-cx+1 = f/g$ with nonnegative, trig
author: Calvin Deng, Linus Hamilton
source: TSTST 2017/3
tags: [2017-06, polynomial, adhoc, instructive, intuitive, manip, trig, find, bestpossible,
  hardanswer, smallcases, equalitycase, rushdown, reliable, gimel]
hardness: 25
url: https://aops.com/community/p8526130

---

Consider solutions to the equation
\[ x^2 - cx + 1 = \frac{f(x)}{g(x)} \]
where $f$ and $g$ are nonzero polynomials with nonnegative real coefficients.
For each $c > 0$, determine the minimum possible degree of $f$, or show
that no such $f$, $g$ exist.

---

First, if $c \ge 2$ then we claim no such $f$ and $g$ exist.
Indeed, one simply takes $x=1$ to get $f(1)/g(1) \le 0$, impossible.

For $c < 2$, let $c = 2 \cos \theta$, where $0 < \theta < \pi$.
We claim that $f$ exists and has minimum degree equal to $n$,
where $n$ is defined as the smallest integer
satisfying $\sin n\theta \le 0$.
In other words
\[ n = \left\lceil \frac{\pi}{\arccos(c/2)} \right\rceil. \]

First we show that this is necessary.
To see it, write explicitly
\[ g(x) = a_0 + a_1 x + a_2 x^2 + \dots + a_{n-2} x^{n-2} \]
with each $a_i \ge 0$, and $a_{n-2} \neq 0$.
Assume that $n$ is such that $\sin(k\theta) \ge 0$ for $k=1,\dots,n-1$.
Then, we have the following system of inequalities:
\begin{align*}
  a_1 &\ge 2 \cos \theta \cdot a_0 \\
  a_0 + a_2 &\ge 2 \cos \theta \cdot a_1 \\
  a_1 + a_3 &\ge 2 \cos \theta \cdot a_2 \\
  &\vdots \\
  a_{n-5} + a_{n-3} &\ge 2 \cos \theta \cdot a_{n-4} \\
  a_{n-4} + a_{n-2} &\ge 2 \cos \theta \cdot a_{n-3} \\
  a_{n-3} &\ge 2 \cos \theta \cdot a_{n-2}.
\end{align*}
Now, multiply the first equation by $\sin\theta$,
the second equation by $\sin2\theta$, et cetera,
up to $\sin\left( (n-1)\theta \right)$.
This choice of weights is selected since we have
\[ \sin \left( k\theta \right) + \sin\left( (k+2)\theta \right)
  = 2 \sin\left( (k+1)\theta \right) \cos \theta \]
so that summing the entire expression
cancels nearly all terms and leaves only
\[ \sin\left( (n-2)\theta \right) a_{n-2}
  \ge \sin\left( (n-1)\theta  \right) \cdot 2\cos\theta \cdot a_{n-2} \]
and so by dividing by $a_{n-2}$ and using the same identity
gives us $\sin(n\theta) \le 0$, as claimed.

This bound is best possible, because the example
\[ a_k = \sin\left( (k+1)\theta \right) \ge 0 \]
makes all inequalities above sharp, hence giving
a working pair $(f,g)$.

\begin{remark*}
Calvin Deng points out that a cleaner proof of the lower bound
is to take $\alpha = \cos \theta + i \sin \theta$.
Then $f(\alpha) = 0$, but by condition the imaginary part of $f(\alpha)$
is apparently strictly positive, contradiction.
\end{remark*}


\begin{remark*}
  Guessing that $c < 2$ works at all (and realizing $c \ge 2$ fails)
  is the first part of the problem.

  The introduction of trigonometry into the solution may seem magical,
  but is motivated in one of two ways:
  \begin{itemize}
    \ii Calvin Deng points out that it's possible to
    guess the answer from small cases:
    For $c \le 1$ we have $n = 3$, tight at $\frac{x^3+1}{x+1} = x^2-x+1$,
    and essentially the ``sharpest $n=3$ example''.
    A similar example exists at $n = 4$ with
    $\frac{x^4+1}{x^2+\sqrt 2 x+1} = x^2-\sqrt2x+1$
    by the Sophie-Germain identity.
    In general, one can do long division to extract
    an optimal value of $c$ for any given $n$,
    although $c$ will be the root of some polynomial.

    The thresholds $c \le 1$ for $n = 3$, $c \le \sqrt 2$ for $n = 4$,
    $c \le \frac{1+\sqrt5}{2}$ for $n = 5$,
    and $c \le 2$ for $n < \infty$ suggest the unusual form of the
    answer via trigonometry.

    \ii One may imagine trying to construct a polynomial
    recursively / greedily by making all inequalities above hold
    (again the ``sharpest situation'' in which $f$ has few coefficients).
    If one sets $c = 2t$, then we have
    \[ a_0 = 1, \quad a_1 = 2t, \quad a_2 = 4t^2-1, \quad
      a_3 = 8t^3-4t, \quad \dots \]
    which are the Chebyshev polynomials of the second type.
    This means that trigonometry is essentially mandatory.
    (One may also run into this when by using standard linear recursion techniques,
    and noting that the characteristic polynomial
    has two conjugate complex roots.)
  \end{itemize}
\end{remark*}

\begin{remark*}
  Mitchell Lee notes that an IMO longlist problem from 1997 shows that
  if $P(x)$ is any polynomial satisfying $P(x) > 0$ for $x > 0$,
  then $(x+1)^n P(x)$ has nonnegative coefficients
  for large enough $n$.
  This show that $f$ and $g$ at least exist for $c \le 2$,
  but provides no way of finding the best possible $\deg f$.

  Meghal Gupta also points out that showing $f$ and $g$ exist
  is possible in the following way:
  \[ \left( x^2-1.99x+1 \right) \left( x^2+1.99x+1 \right)
    = \left( x^4 - 1.9601x^2 + 1 \right) \]
  and so on, repeatedly multiplying by the ``conjugate''
  until all coefficients become positive.
  To my best knowledge, this also does not give any way
  of actually minimizing $\deg f$,
  although Ankan Bhattacharya points out that this construction
  is actually optimal in the case where $n$ is a power of $2$.
\end{remark*}

\begin{remark*}
  It's pointed out that Matematicheskoe Prosveshchenie, issue 1, 1997, page 194
  contains a nearly analogous result,
  available at \url{https://mccme.ru/free-books/matpros/pdf/mp-01.pdf}
  with solutions presented in \url{https://mccme.ru/free-books/matpros/pdf/mp-05.pdf},
  pages 221--223;
  and \url{https://mccme.ru/free-books/matpros/pdf/mp-10.pdf}, page 274.
\end{remark*}
