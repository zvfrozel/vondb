author: Holden Mui
desc: Double quadratic
hardness: 25
source: TSTST 2021/4
tags: [2021-12, manip, pigeonhole, dalet]
url: https://aops.com/community/p23864177

---

Let $a$ and $b$ be positive integers.
Suppose that there are infinitely many pairs of positive integers $(m, n)$
for which $m^2+an+b$ and $n^2+am+b$ are both perfect squares.
Prove that $a$ divides $2b$.

---

Treating $a$ and $b$ as fixed,
we are given that there are infinitely many quadrpules $(m,n,r,s)$
which satisfy the system
\begin{gather*}
  m^2+an+b=(m+r)^2 \\
  n^2+am+b=(n+s)^2
\end{gather*}
We say that $(r,s)$ is \emph{exceptional}
if there exists infinitely many $(m,n)$ that satisfy.

\begin{claim*}
  If $(r,s)$ is exceptional, then either
  \begin{itemize}
    \ii $0 < r < a/2$, and $0 < s < \frac14 a^2$; or
    \ii $0 < s < a/2$, and $0 < r < \frac14 a^2$; or
    \ii $r^2 + s^2 \le 2b$.
  \end{itemize}
  In particular, finitely many pairs $(r,s)$ can be exceptional.
\end{claim*}

\begin{proof}
  Sum the two equations to get:
  \[ r^2+s^2-2b = (a-2r)m + (a-2s)n. \qquad (\dagger) \]
  If $0 < r < a/2$, then the idea is to use the bound
  $an+b \ge 2m+1$ to get $m \le \frac{an+b-1}{2}$.
  Consequently,
  \[ (n+s)^2 = n^2+am+b \le n^2 + a \cdot \frac{an+b-1}{2} + b \]
  For this to hold for infinitely many integers $n$,
  we need $2s \le \frac{a^2}{2}$, by comparing coefficients.

  A similar case occurs when $0 < s < a/2$.

  If $\min(r,s) > a/2$, then $(\dagger)$ forces $r^2+s^2 \le 2b$,
  giving the last case.
\end{proof}

Hence, there exists some particular pair $(r,s)$
for which there are infinitely many solutions
$(m,n)$. Simplifying the system gives
\begin{gather*}
  an = 2rm + r^2-b \\
  2sn = am + b-s^2
\end{gather*}
Since the system is linear,
for there to be infinitely many solutions $(m, n)$
the system must be dependent.
This gives \[\frac{a}{2s}=\frac{2r}{a}=\frac{r^2-b}{b-s^2}\]
so $a = 2\sqrt{rs}$ and $b = \frac{s^2\sqrt{r}+r^2\sqrt{s}}{\sqrt{r}+\sqrt{s}}$.
Since $rs$ must be square, we can reparametrize as $r=kx^2$, $s=ky^2$, and $\gcd(x, y)=1$.
This gives
\begin{align*}
    a &= 2kxy \\
    b &= k^2xy(x^2-xy+y^2).
\end{align*}
Thus, $a \mid 2b$, as desired.
