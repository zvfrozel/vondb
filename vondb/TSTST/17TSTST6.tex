desc: Fibonacci sequence partition
source: TSTST 2017/6
tags: [2017-06, greedy, justdoit, rigid, free, inefficient, size, analysis, grinding, gimel]
hardness: 30
author: Ivan Borsenco
url: https://aops.com/community/p8526142

---

A sequence of positive integers $(a_{n})_{n \geq 1}$ is of
\emph{Fibonacci type} if it satisfies the recursive relation
$a_{n+2}=a_{n+1}+a_{n}$ for all $n \geq 1$.
Is it possible to partition the set of positive integers
into an infinite number of Fibonacci type sequences?

---

Yes, it is possible.
The following solutions were written for me by Kevin Sun and Mark Sellke.
We let $F_1 = F_2 = 1$, $F_3 = 2$, $F_4 = 3$, $F_5 = 5$, \dots
denote the Fibonacci numbers.

\paragraph{First solution (Kevin Sun).}
We are going to appeal to the so-called Zeckendorf theorem:
\begin{theorem*}
  [Zeckendorf]
  Every positive integer can be uniquely expressed
  as the sum of nonconsecutive Fibonacci numbers.
\end{theorem*}
This means every positive integer has a
Zeckendorf (``Fibonacci-binary'') representation
where we put $1$ in the $i$th digit from the right
if $F_{i+1}$ is used.
The idea is then to take the following so-called \emph{Wythoff array}:
\begin{itemize}
  \ii \textbf{Row 1}: $1$, $2$, $3$, $5$, \dots
  \ii \textbf{Row 101}: $1+3$, $2+5$, $3+8$, \dots
  \ii \textbf{Row 1001}: $1+5$, $2+8$, $3+13$, \dots
  \ii \textbf{Row 10001}: $1+8$, $2+13$, $3+21$, \dots
  \ii \textbf{Row 10101}: $1+3+8$, $2+5+13$, $3+8+21$, \dots
  \ii \dots et cetera.
\end{itemize}
More concretely, the array has the following rows to start:
\[
\begin{matrix}
1&2&3&5&8&13&21&\dotsb\\
4&7&11&18&29&47&76&\dotsb\\
6&10&16&26&42&68&110&\dotsb\\
9&15&24&39&63&102&165&\dotsb\\
12&20&32&52&84&136&220&\dotsb\\
14&23&37&60&97&157&254&\dotsb\\
17&28&45&73&118&191&309&\dotsb\\
\vdots&\vdots&\vdots&\vdots&\vdots&\vdots&\vdots&\ddots\\
\end{matrix}.
\]

Here are the full details.

We begin by outlining a proof of Zeckendorf's theorem,
which implies the representation above is unique.
Note that if $F_k$ is the greatest Fibonacci number at most $n$,
then \[n - F_k < F_{k+1} - F_{k} = F_{k-1}.\]
In particular, repeatedly subtracting off the largest $F_k$
from $n$ will produce one such representation with
no two consecutive Fibonacci numbers.
On the other hand, this $F_k$ must be used, as
\[ n \ge F_k > F_{k-1} + F_{k-3} + F_{k-5} + \dotsb \]
This shows, by a simple inductive argument,
that such a representation exists and unique.

We write $n = \ol{a_k\dotso a_1}_{\text{Fib}}$
for the Zeckendorf representation as we described
(where $a_i = 1$ if $F_{i+1}$ is used).
Now for each $\ol{a_k\dotso a_1}_{\text{Fib}}$
with $a_1 = 1$, consider the sequence
\[ \ol{a_k\dotso a_1}_{\text{Fib}}, \;
  \ol{a_k\dotso a_1 0}_{\text{Fib}}, \;
  \ol{a_k\dotso a_1 00}_{\text{Fib}}, \; \dots \]
These sequences are Fibonacci-type by definition,
and partition the positive integers since
each positive integer has exactly one Fibonacci base representation.

\paragraph{Second solution.} Call an infinite set of integers $S$ \textit{sandwiched} if there exist increasing sequences $\{a_i\}_{i=0}^{\infty}, \{b_i\}_{i=0}^{\infty}$ such that the following are true:
\begin{itemize}
\item $a_i + a_{i+1} = a_{i+2}$ and $b_i + b_{i+1} = b_{i+2}$.
\item The intervals $[a_i+1,b_i-1]$ are disjoint and are nondecreasing in length.
\item $\displaystyle S = \bigcup_{i=0}^{\infty} [a_i+1,b_i-1]$.
\end{itemize}

We claim that if $S$ is any nonempty sandwiched set, then $S$ can be partitioned into a Fibonacci-type sequence (involving the smallest element of $S$) and two smaller sandwiched sets. If this claim is proven, then we can start with $\NN \setminus \{1,2,3,5,\dots\}$, which is a sandwiched set, and repeatedly perform this partition, which will eventually sort each natural number into a Fibonacci-type sequence.

Let $S$ be a sandwiched set given by $\{a_i\}_{i=0}^{\infty}, \{b_i\}_{i=0}^{\infty}$, so the smallest element in $S$ is $x = a_0 + 1$. Note that $y = a_1+1$ is also in $S$ and $x < y$. Then consider the Fibonacci-type sequence given by $f_0 = x, f_1 = y$, and $f_{k+2} = f_{k+1} + f_k$. We can then see that $f_i \in [a_i+1,b_i-1]$, as the sum of numbers in the intervals $[a_{k}+1,b_{k}-1]$, $[a_{k+1}+1,b_{k+1}-1]$ lies in the interval \[[a_{k} + a_{k+1} + 2,b_k + b_{k+1}-2] = [a_{k+2}+2,b_{k+2}-2] \subset [a_{k+2}+1,b_{k+2}-1].\]

Therefore, this gives a natural partition of $S$ into this sequence and two sets: \begin{align*} S_1 &= \bigcup_{i=0}^{\infty} [a_i+1,f_i-1] \\ \text{and} \,\, S_2 &= \bigcup_{i=0}^{\infty} [f_i+1,b_i-1].\end{align*} (For convenience, $[x,x-1]$ will be treated as the empty set.)

We now show that $S_1$ and $S_2$ are sandwiched. Since $\{a_i\}$, $\{f_i\}$, and $\{b_i\}$ satisfy the Fibonacci recurrence, it is enough to check that the intervals have nondecreasing lengths. For $S_1$, that is equivalent to $f_{k+1} - a_{k+1} \ge f_k - a_k$ for each $k$. Fortunately, for $k \ge 1$, the difference is $f_{k-1} - a_{k-1} \ge 0$, and for $k = 0$, $f_1 - a_1 = 1 = f_0 - a_0$. Similarly for $S_2$, checking $b_{k+1} - f_{k+1} \ge b_k - f_k$ is easy for $k \ge 1$ as $b_{k-1} - f_{k-1} \ge 0$, and \[(b_{1} - f_1) - (b_0 - f_0) = (b_1 - a_1) - (b_0 - a_0),\] which is nonnegative since the lengths of intervals in $S$ are nondecreasing.

Therefore we have shown that $S_1$ and $S_2$ are sandwiched. (Note that some of the $[a_i+1,f_i-1]$ may be empty, which would shift some indices back.) Since this gives us a procedure to take a set $S$ and produce a Fibonacci-type sequence with its smallest element, along which two other sandwiched types, we can partition $\NN$ into an infinite number of Fibonacci-type sequences.

\paragraph{Third solution.} We add Fibonacci-type sequences one-by-one. At each step, let $x$ be the smallest number that has not been used in any previous sequence. We generate a new Fibonacci-type sequence as follows. Set $a_0 = x$ and for $i \ge 1$, set \[a_i = \left \lfloor \varphi a_{i-1} + \frac{1}{2} \right \rfloor.\] Equivalently, $a_{i}$ is the closest integer to $\varphi a_{i-1}$. \\
It suffices to show that this sequence is Fibonacci-type and that no two sequences generated in this way overlap.
We first show that for a positive integer $n$, \[\left \lfloor \varphi \left \lfloor \varphi n + \frac{1}{2} \right \rfloor + \frac{1}{2} \right \rfloor = n + \left \lfloor \varphi n + \frac{1}{2} \right \rfloor . \]
Indeed, \begin{align*} \left \lfloor \varphi \left \lfloor \varphi n + \frac{1}{2} \right \rfloor + \frac{1}{2} \right \rfloor &= \left \lfloor (1 + \varphi^{-1}) \left \lfloor \varphi n + \frac{1}{2} \right \rfloor + \frac{1}{2}\right \rfloor \\
&= \left \lfloor \varphi n + \frac{1}{2} \right \rfloor + \left \lfloor \varphi^{-1} \left \lfloor \varphi n + \frac{1}{2}\right \rfloor + \frac{1}{2} \right \rfloor.
\end{align*}
Note that $\left \lfloor \varphi n + \frac{1}{2} \right \rfloor = \varphi n + c$ for some $|c| \le \frac{1}{2}$; this implies that $\varphi^{-1}\left \lfloor \varphi n + \frac{1}{2} \right \rfloor$ is within $\varphi^{-1}\cdot \frac{1}{2} < \frac{1}{2}$ of $n$, so its closest integer is $n$, proving the claim.

Therefore these sequences are Fibonacci-type. Additionally, if $a \neq b$, then $|\varphi a - \varphi b| \ge \varphi > 1$. Then \[a \neq b \implies \left \lfloor \varphi a + \frac{1}{2} \right \rfloor \neq \left \lfloor \varphi b + \frac{1}{2} \right \rfloor,\]
and since the first term of each sequence is chosen to not overlap with any previous sequences, these sequences are disjoint.

\begin{remark*}
  Ankan Bhattacharya points out that the same sequence
  essentially appears in IMO 1993, Problem 5 --- in other words,
  a strictly increasing function $f \colon \ZZ_{>0} \to \ZZ_{>0}$ with $f(1) = 2$,
  and $f(f(n)) = f(n) + n$.

  Nikolai Beluhov sent us an older reference from March 1977,
  where Martin Gardner wrote in his column about Wythoff's Nim.
  The relevant excerpt goes:
  \begin{quote}
    ``Imagine that we go through the infinite sequence of safe pairs (in
    the manner of Eratosthenes' sieve for sifting out primes) and cross
    out the infinite set of all safe pairs that are pairs in the Fibonacci
    sequence. The smallest pair that is not crossed out is 4/7. We can now
    cross out a second infinite set of safe pairs, starting with 4/7, that
    are pairs in the Lucas sequence. An infinite number of safe pairs, of
    which the lowest is now 6/10, remain. This pair too begins another
    infinite Fibonacci sequence, all of whose pairs are safe. The process
    continues forever. Robert Silber, a mathematician at North Carolina
    State University, calls a safe pair ``primitive'' if it is the first
    safe pair that generates a Fibonacci sequence.''
  \end{quote}
  The relevant article by Robert Silber is
  \emph{A Fibonacci Property of Wythoff Pairs},
  from The Fibonacci Quarterly 11/1976.
\end{remark*}


\paragraph{Fourth solution (Mark Sellke).}
For later reference let \[ f_1=0, f_2=1, f_3=1, \dots \]
denote the ordinary Fibonacci numbers.
We will denote the Fibonacci-like sequences by $F^i$ and the elements with subscripts;
hence $F^2_1$ is the first element of the second sequence.
Our construction amounts to just iteratively add new sequences;
hence the following claim is the whole problem.

\begin{lemma*}
  For any disjoint collection of Fibonacci-like sequences
  $F^1,\dots,F^k$ and any integer $m$ contained in none of them,
  there is a new Fibonacci-like sequence $F^{k+1}$ beginning
  with $F^{k+1}_1=m$ which is disjoint from the previous sequences.
\end{lemma*}

Observe first that for each sequence $F^j$ there is $c^j\in \RR^n$ such that
\[ F^j_n=c^j\phi^n+o(1)  \]
where \[\phi=\frac{1+\sqrt{5}}{2}.\]

Collapse the group $(\RR^+,\times)$ into the half-open interval
$J = \left\{ x \mid 1 \le x < \phi \right\}$
by defining $T(x)=y$ for the unique $y\in J$ with $x=y\phi^n$ for some integer $n$.

Fix an interval $I=[a,b]\subseteq [1.2,1.3]$ (the last condition is to
avoid wrap-around issues) which contains none of the $c^j$, and take
$\varepsilon<0.001$ to be small enough that in fact each $c^j$ has
distance at least $10\varepsilon$ from $I$; this means any $c_j$ and
element of $I$ differ by at least a $(1+10\varepsilon)$ factor. The idea
will be to take $F^{k+1}_1=m$ and $F^{k+1}_2$ to be a large such that
the induced values of $F^{k+1}_j$ grow like $k\phi^j$ for $j\in
T^{-1}(I)$, so that $F^{k+1}_n$ is separated from the $c^j$ after
applying $T$. What's left to check is the convergence.

Now let \[c=\lim_{n\to\infty} \frac{f_n}{\phi^n}\] and take
$M$ large enough that for $n>M$ we have
\[\left|\frac{f_n}{c\phi^n}-1\right|<\varepsilon.\]

Now $\frac{T^{-1}(I)}{c}$ contains arbitrarily large integers,
so there are infinitely many $N$ with $cN\in T^{-1}(I)$ with $N>\frac{10m}{\varepsilon}$.
We claim that for any such $N$, the sequence $F^{(N)}$ defined by
\[F^{(N)}_1=m,F^{(N)}_2=N\] will be very multiplicatively similar to the
normal Fibonacci numbers up to rescaling;
indeed for $j=2,j=3$ we have $\frac{F_2^{(N)}}{f_2}=N,\frac{F_3^{(N)}}{f_3}=N+m$
and so by induction we will have
\[ \frac{F^{(N)}_j}{f_j} \in [N,N+m] \subseteq [N,N(1+\varepsilon)] \]
for $j\geq 2$.
Therefore, up to small multiplicative errors, we have
\[F_j^{(N)}\approx Nf_j\approx cN\phi^j.\]
From this we see that for $j>M$ we have
\[T(F_j^{(N)})\in  T(cN)\cdot [1-2\varepsilon,1+2\varepsilon].\]
In particular, since $T(cN)\in I$ and $I$ is separated from each $c_j$
by a factor of $(1+10\varepsilon)$,
we get that $F_j^{(N)}$ is not in any of $F^1,F^2,\dots,F^k$.

Finishing is easy, since we now have a uniform estimate on how many
terms we need to check for a new element before the exponential growth
takes over. We will just use pigeonhole to argue that there are few
possible collisions among those early terms, so we can easily pick a
value of $N$ which avoids them all. We write it out below.

For large $L$, the set \[S_L=(I\cdot \phi^L)\cap\ZZ\] contains at
least $k_I\phi^L$ elements. As $N$ ranges over $S_L$, for each fixed
$j$, the value of $F^{(N)}_j$ varies by at most a factor of $1.1$
because we imposed $I\subseteq [1.2,1.3]$ and so this is true for the
first two terms, hence for all subsequent terms by induction. Now
suppose $L$ is very large, and consider a fixed pair $(i,j)$ with $i\leq
k$ and $j\leq M$. We claim there is at most $1$ possible value $k$ such
that the term $F^i_k$ could equal $F^{(N)}_j$ for some $N\in S_L$;
indeed, the terms of $F^i$ are growing at exponential rate with factor
$\phi>1.1$, so at most one will be in a given interval of multiplicative
width at most $1.1$.

Hence, of these $k_I\phi^L$ values of $N$, at most $kM$ could cause
problems, one for each pair $(i,j)$. However by monotonicity of
$F^{(N)}_j$ in $N$, at most $1$ value of $N$ causes a collision for each
pair $(i,j)$. Hence for large $L$ so that $k_I\phi^L>10kM$ we can find a
suitable $N\in S_L$ by pigeonhole and the sequence $F^{(N)}$ defined by
$(m,N,N+m,\dots)$ works.
