desc: Radical axis trivia
author: Ray Li
source: TSTST 2017/1
tags: [2017-06, pop, bary, complex, rich, aleph]
hardness: 10
url: https://aops.com/community/p8526098

---

Let $ABC$ be a triangle with circumcircle $\Gamma$,
circumcenter $O$, and orthocenter $H$.
Assume that $AB \neq AC$ and $\angle A \neq 90\dg$.
Let $M$ and $N$ be the midpoints of $\ol{AB}$ and $\ol{AC}$,
respectively, and let $E$ and $F$ be the feet of the altitudes
from $B$ and $C$ in $\triangle ABC$, respectively.
Let $P$ be the intersection point of line $MN$
with the tangent line to $\Gamma$ at $A$.
Let $Q$ be the intersection point,
other than $A$, of $\Gamma$ with the circumcircle of $\triangle AEF$.
Let $R$ be the intersection point of lines $AQ$ and $EF$.
Prove that $\ol{PR} \perp \ol{OH}$.

---

\paragraph{First solution (power of a point).}
Let $\gamma$ denote the nine-point circle of $ABC$.
\begin{center}
\begin{asy}
pair A = dir(125);
pair B = dir(210);
pair C = dir(330);
pair M = midpoint(A--B);
pair N = midpoint(A--C);
pair O = origin;
pair H = A+B+C;

draw(A--B--C--cycle, blue);
pair E = foot(B, A, C);
pair F = foot(C, A, B);
draw(B--E, lightblue);
draw(C--F, lightblue);
pair R = extension(E, F, B, C);
pair Q = -A+2*foot(O, A, R);

filldraw(unitcircle, opacity(0.1)+lightcyan, blue);
draw(A--R--E, heavygreen);
pair P = extension(M, N, A, A+dir(90)*A);
draw(A--P--N, red);
filldraw(circumcircle(A, M, N), opacity(0.1)+lightred, red);
filldraw(circumcircle(A, E, F), opacity(0.1)+lightgreen, heavygreen);

filldraw(circumcircle(M, N, E), opacity(0.1)+lightcyan, heavycyan);

dot("$A$", A, dir(A));
dot("$B$", B, dir(B));
dot("$C$", C, dir(C));
dot("$M$", M, dir(145));
dot("$N$", N, dir(20));
dot("$O$", O, dir(315));
dot("$H$", H, dir(H));
dot("$E$", E, dir(40));
dot("$F$", F, dir(F));
dot("$R$", R, dir(R));
dot("$Q$", Q, dir(Q));
dot("$P$", P, dir(P));

/* TSQ Source:

A = dir 125
B = dir 210
C = dir 330
M = midpoint A--B R145
N = midpoint A--C R20
O = origin R315
H = A+B+C

A--B--C--cycle blue
E = foot B A C R40
F = foot C A B
B--E lightblue
C--F lightblue
R = extension E F B C
Q = -A+2*foot O A R

unitcircle 0.1 lightcyan / blue
A--R--E heavygreen
P = extension M N A A+dir(90)*A
A--P--N red
circumcircle A M N 0.1 lightred / red
circumcircle A E F 0.1 lightgreen / heavygreen

circumcircle M N E 0.1 lightcyan / heavycyan

*/
\end{asy}
\end{center}
Note that
\begin{itemize}
  \ii $PA^2 = PM \cdot PN$,
  so $P$ lies on the radical axis of $\Gamma$ and $\gamma$.
  \ii $RA \cdot RQ = RE \cdot RF$,
  so $R$ lies on the radical axis of $\Gamma$ and $\gamma$.
\end{itemize}
Thus $\ol{PR}$ is the radical axis of $\Gamma$ and $\gamma$,
which is evidently perpendicular to $\ol{OH}$.

\begin{remark*}
  In fact, by power of a point one may also observe
  that $R$ lies on $\ol{BC}$,
  since it is on the radical axis of $(AQFHE)$, $(BFEC)$, $(ABC)$.
  Ironically, this fact is not used in the solution.
\end{remark*}

\paragraph{Second solution (barycentric coordinates).}
Again note first $R \in \ol{BC}$ (although this can be avoided too).
We compute the points in much the same way as before.
Since $\ol{AP} \cap \ol{BC} = (0 : b^2 : -c^2)$
we have \[ P = \left( b^2-c^2 : b^2 : -c^2  \right) \]
(since $x=y+z$ is the equation of line $\ol{MN}$).
Now in Conway notation we have
\[ R = \ol{EF} \cap \ol{BC} = (0 : S_C : -S_B)
  =\left( 0 : a^2+b^2-c^2 : -a^2+b^2-c^2 \right). \]
Hence
\[ \overrightarrow{PR}
  = \frac{1}{2(b^2-c^2)} \left( b^2-c^2 , c^2-a^2 , a^2-b^2 \right).  \]
On the other hand, we have
$\overrightarrow{OH} = \vec A + \vec B + \vec C$.
So it suffices to check that
\[ \sum_{\text{cyc}} a^2\left( (a^2-b^2) + (c^2-a^2) \right) = 0 \]
which is immediate.

\paragraph{Third solution (complex numbers).}
Let $ABC$ be the unit circle.
We first compute $P$ as the midpoint of $A$ and $\ol{AA} \cap \ol{BC}$:
\begin{align*}
  p &= \half \left( a + \frac{a^2(b+c)-bc\cdot2a}{a^2-bc} \right) \\
  &= \frac{a(a^2-bc)+a^2(b+c)-2abc}{2(a^2-bc)}.
\end{align*}
Using the remark above, $R$ is the inverse of $D$ with
respect to the circle with diameter $\ol{BC}$,
which has radius $\left\lvert \half(b-c) \right\rvert$.
Thus
\begin{align*}
  r - \frac{b+c}{2}
  &= \frac{\frac14(b-c)\left( \frac1b-\frac1c \right)}%
  {\ol{\half\left( a-\frac{bc}{a} \right)}} \\
  r &= \frac{b+c}{2}
  + \frac{-\half \frac{(b-c)^2}{bc}}{\frac1a-\frac{a}{bc}} \\
  &= \frac{b+c}{2} + \frac{a(b-c)^2}{2(a^2-bc)} \\
  &= \frac{a(b-c)^2+(b+c)(a^2-bc)}{2(a^2-bc)}.
\end{align*}
Expanding and subtracting gives
\[ p-r = \frac{a^3-abc-ab^2-ac^2+b^2c+bc^2}{2(a^2-bc)}
  = \frac{(a+b+c)(a-b)(a-c)}{2(a^2-bc)} \]
which is visibly equal to the negation of its conjugate
once the factor of $a+b+c$ is deleted.

(Actually, one can guess this factorization ahead of time
by noting that if $A=B$, then $P=B=R$, so $a-b$ must be a factor;
analogously $a-c$ must be as well.)

---

We present a barycentric solution in this walkthrough.
(Complex numbers, and fairly simple synthetic arguments,
are also canonical routes.)
Although there is a low-hanging synthetic observations
that can be made to simplify the calculation,
we'll pretend we didn't notice them.
\begin{walk}
  \ii Show that the tangent to $A$ meets $BC$ at $(0 : b^2 : -c^2)$.
  (Hint for projective lovers: harmonic conjugate of symmedian foot.)
  \ii Find the coordinates of $P$,
  using the fact it lies on the midline $x = y+z$.
  \ii We don't like the point $Q$ because it's annoying to calculate.
  Thankfully we don't need it:
  find the radical axis of $(AEF)$ and $(ABC)$.
  Then conclude that the line $AQ$ consists
  of the points of the form $({\ast} : S_C : -S_B)$.
  \ii Use (c) together with collinearity of $\ol{EF}$
  to find the coordinates of $R$.
  (If you find $R$ correctly, you might be a bit surprised
  until you realize the low-hanging synthetic observation
  that I was talking about.)
  \ii Show that
  \[ \overrightarrow{PR} = \frac{(b^2-c^2) \vec A
    + (c^2-a^2) \vec B + (a^2-b^2) \vec C}{2(b^2-c^2)}. \]
  \ii Use the strong perpendicularity criterion
  to show it is perpendicular to $\overrightarrow{OH}$.
\end{walk}
