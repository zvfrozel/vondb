desc:  $0.499n$ residues modulo $n$
author: Yang Liu
source:  TSTST 2016/3
tags:  [QR, mods, troll, hardanswer, construct, explicit, primes, nice, yesno, 2016-07, well, brave, zayin]
hardness: 40
url: https://aops.com/community/p6575217

---

Decide whether or not there exists a nonconstant polynomial $Q(x)$
with integer coefficients with the following property:
for every positive integer $n > 2$, the numbers
\[ Q(0), \; Q(1), Q(2),  \; \dots, \; Q(n-1) \]
produce at most $0.499n$ distinct residues when taken modulo $n$.

---

We claim that
\[ Q(x) = 420(x^2-1)^2 \]
works.
Clearly, it suffices to prove the result when $n=4$ and when $n$ is an odd prime $p$.
The case $n=4$ is trivial, so assume now $n=p$ is an odd prime.

First, we prove the following easy claim.
\begin{claim*}
  For any odd prime $p$, there are at least $\frac12(p-3)$
  values of $a$ for which $\left( \frac{1-a^2}{p} \right) = +1$.
\end{claim*}
\begin{proof}
  Note that if $k \neq 0$, $k \neq \pm 1$, $k^2 \neq -1$, then $a = 2(k+k\inv)$ works.
  Also $a=0$ works.
\end{proof}

Let $F(x) = (x^2-1)^2$.
The range of $F$ modulo $p$ is contained within the $\half(p+1)$ quadratic residues modulo $p$.
On the other hand, if for some $t$ neither of $1 \pm t$ is a quadratic residue,
then $t^2$ is omitted from the range of $F$ as well.
Call such a value of $t$ \emph{useful}, and let $N$ be the number of useful residues.
We aim to show $N \ge \frac14 p - 2$.

We compute a lower bound on the number $N$ of useful $t$ by writing
\begin{align*}
  N &= \frac{1}{4} \left( \sum_t \left[ \left(1 - \left(\frac{1-t}{p} \right) \right)
    \left(1 - \left(\frac{1+t}{p} \right) \right) \right]
    - \left( 1- \left( \frac2p \right) \right)
    - \left( 1- \left( \frac{-2}p \right) \right)
  \right) \\
  &\ge \frac{1}{4} \sum_t \left[ \left(1 - \left(\frac{1-t}{p} \right) \right)
    \left(1 - \left(\frac{1+t}{p} \right) \right) \right] -1 \\
  &= \frac{1}{4} \left(p + \sum_t \left(\frac{1-t^2}{p} \right) \right) -1 \\
  &\ge \frac14 \left( p + (+1) \cdot \tfrac12(p-3) + 0 \cdot 2
    + (-1) \cdot ( (p-2) - \tfrac12(p-3)) \right) - 1 \\
  &\ge \frac14 \left( p - 5 \right).
\end{align*}

Thus, the range of $F$ has size at most
\[ \half(p+1) - \half N \le \frac38(p+3). \]
This is less than $0.499p$ for any $p \ge 11$.

\begin{remark*}
  In fact, the computation above is essentially an equality.
  There are only two points where terms are dropped:
  one, when $p \equiv 3 \pmod 4$ there are no $k^2 = -1$ in the lemma,
  and secondly, the terms $1-\left( 2/p \right)$ and $1-\left( -2/p \right)$
  are dropped in the initial estimate for $N$.
  With suitable modifications, one can show that in fact,
  the range of $F$ is exactly equal to
  \[
    \half(p+1) - \half N =
    \begin{cases}
      \frac18(3p+5) & p \equiv 1 \pmod 8 \\
      \frac18(3p+7) & p \equiv 3 \pmod 8 \\
      \frac18(3p+9) & p \equiv 5 \pmod 8 \\
      \frac18(3p+3) & p \equiv 7 \pmod 8.
    \end{cases}
  \]
\end{remark*}
