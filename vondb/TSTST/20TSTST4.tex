desc: $ab \mid (a-b)^4+1$
hardness: 15
source: TSTST 2020/4
tags: [2020-12, find, size, mods, order, aleph]
author: Yang Liu
url: https://aops.com/community/p19444614

---

Find all pairs of positive integers $(a, b)$ satisfying
the following conditions:
\begin{enumerate}[(i)]
  \ii $a$ divides $b^4+1$,
  \ii $b$ divides $a^4+1$,
  \ii $\lfloor \sqrt{a} \rfloor = \lfloor \sqrt{b} \rfloor$.
\end{enumerate}

---

The only solutions are $(1, 1)$, $(1, 2)$, and $(2, 1)$,
which clearly work. Now we show there are no others.

Obviously, $\gcd(a,b) = 1$, so the problem conditions imply
\[ ab \mid (a-b)^4+1 \]
since each of $a$ and $b$ divide the right-hand side.
We define
\[ k \stackrel{\text{def}}{=} \frac{(b-a)^4 + 1}{ab}. \]

\begin{claim*}
  [Size estimate]
  We must have $k \le 16$.
\end{claim*}
\begin{proof}
  Let $n = \lfloor \sqrt{a} \rfloor = \lfloor \sqrt{b} \rfloor$,
  so that $a,b \in [n^2, n^2+2n]$.
  We have that
  \begin{align*}
    ab &\ge n^2(n^2 + 1) \ge n^4 + 1 \\
    (b-a)^4+1 &\le (2n)^4+1 = 16n^4+1
  \end{align*}
  which shows $k \le 16$.
\end{proof}

\begin{claim*}
  [Orders argument]
  In fact, $k = 1$.
\end{claim*}

\begin{proof}
  First of all, note that $k$ cannot be even:
  if it was, then $a$, $b$ have opposite parity,
  but then $4 \mid (b-a)^4 + 1$, contradiction.

  Thus $k$ is odd. However, every odd prime divisor of $(b-a)^4 + 1$
  is congruent to $1 \pmod{8}$ and is thus at least $17$,
  so $k = 1$ or $k \ge 17$.
  It follows that $k = 1$.
\end{proof}

At this point, we have reduced to solving
\[ ab = (b-a)^4+1 \]
and we need to prove the claimed solutions are the only ones.
Write $b = a+d$, and assume WLOG that $d \ge 0$:
then we have $a(a+d) = d^4 + 1$, or
\[ a^2 - da - (d^4 + 1) = 0. \]
The discriminant $d^2 + 4(d^4 + 1)
= 4d^4 + d^2 + 4$ must be a perfect square.
\begin{itemize}
\ii The cases $d = 0$ and $d = 1$ lead to pairs $(1, 1)$ and $(1, 2)$.
\ii If $d \ge 2$, then we can sandwich
\[
  (2d^2)^2 < 4d^4 + d^2 + 4
  < 4d^4 + 4d^2 + 1 = (2d^2 + 1)^2,
\]
so the discriminant is not a square.
\end{itemize}
The solution is complete.

\begin{remark*}[Author remarks on origin]
  This comes from the problem of the existence of
  a pair of elliptic curves over $\FF_a$, $\FF_b$ respectively,
  such that the number of points on one is the field size of the other.
  The bound $n^2 \le a,b < (n+1)^2$ is the Hasse bound.
  The divisibility conditions correspond to asserting that the embedding degree of each curve is $8$,
  so that they are \emph{pairing friendly}.
  In this way, the problem is essentially the key result of \url{https://arxiv.org/pdf/1803.02067.pdf},
  shown in Proposition 3.
\end{remark*}
