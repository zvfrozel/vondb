author: Hongzhou Lin
desc: Triple $K_A K_B K_C$ construction passes through $O$
source: TSTST 2022/6
url: https://aops.com/community/p25516957
tags: [2023-08, todo]

---

Let $O$ and $H$ be the circumcenter and orthocenter,
respectively, of an acute scalene triangle $ABC$.
The perpendicular bisector of $\ol{AH}$ intersects
$\ol{AB}$ and $\ol{AC}$ at $X_A$ and $Y_A$ respectively.
Let $K_A$ denote the intersection of
the circumcircles of triangles $OX_AY_A$ and $BOC$ other than $O$.

Define $K_B$ and $K_C$ analogously by repeating this construction two more times.
Prove that $K_A$, $K_B$, $K_C$, and $O$ are concyclic.

---

We present several approaches.

\paragraph{First solution, by author.}
Let $\odot OX_AY_A$ intersects $AB$, $AC$ again at $U$, $V$.
Then by Reim's theorem $UVCB$ are concyclic.
Hence the radical axis of $\odot OX_AY_A$, $\odot OBC$ and $\odot (UVCB)$
are concurrent, i.e.\ $OK_A$, $BC$,  $UV$ are concurrent,
Denote the intersection as $K_A^\ast$,
which is indeed the inversion of $K_A$ with respect to $\odot O$.
(The inversion sends $\odot OBC$ to the line $BC$).

Let $P_A$, $P_B$, $P_C$ be the circumcenters of $\triangle OBC$,
$\triangle OCA$, $\triangle OAB$ respectively.

\begin{claim*}
  $K_A^\ast$ coincides with the intersection of $P_BP_C$ and $BC$.
\end{claim*}
\begin{proof}
  Note that $d(O, BC) = 1/2 AH = d(A,X_AY_A)$.
  This means the midpoint $M_C$ of $AB$ is equal distance
  to $X_AY_A$ and the line through $O$ parallel to $BC$.
  Together with $OM_C \perp AB$ implies that $\angle M_CX_AO = \angle B$.
  Hence $\angle UVO = \angle B = \angle AVU$.
  Similarly $\angle VUO = \angle AUV$,
  hence $\triangle AUV \simeq \triangle OUV$.
  In other words, $UV$ is the perpendicular bisector of $AO$,
  which pass through $P_B$, $P_C$.
  Hence $K_A^\ast$ is indeed $P_BP_C \cap BC$.
\end{proof}

Finally by Desargue's theorem,
it suffices to show that $AP_A$, $BP_B$, $CP_C$ are concurrent.
Note that
\begin{align*}
  d(P_A, AB) &= P_AB \sin (90\dg + \angle C - \angle A), \\
  d(P_A, AC) &= P_AC \sin (90\dg + \angle B - \angle A).
\end{align*}
Hence the symmetric product and trig Ceva finishes the proof.

\begin{center}
\begin{asy}
size(12cm);
pair A = dir(50);
pair B = dir(220);
pair C = dir(320);
filldraw(A--B--C--cycle, opacity(0.1)+lightcyan, blue);
pair O = origin;
draw(O--A, lightblue);
draw(O--B, lightblue);
draw(O--C, lightblue);

pair P_A = circumcenter(O, B, C);
pair P_B = circumcenter(O, C, A);
pair P_C = circumcenter(O, A, B);
pair H = orthocenter(A, B, C);
pair M = midpoint(A--H);
pair X_A = extension(A, B, M, B-C+M);
pair Y_A = extension(A, C, M, B-C+M);
pair X = X_A;
pair Y = Y_A;

draw(A--H, blue);
pair K_A = -O+2*foot(O, circumcenter(O, B, C), circumcenter(O, X_A, Y_A));
pair K_As = extension(P_B, P_C, B, C);
draw(circumcircle(O, B, C), grey);
draw(circumcircle(O, X_A, Y_A), grey);

draw(O--K_As, deepgreen);
draw(K_As--B, blue);

pair U_1 = -X+2*foot(circumcenter(O, X_A, Y_A), A, B);
pair U_2 = -Y+2*foot(circumcenter(O, X_A, Y_A), A, C);
draw(U_1--K_As, deepgreen);
pair M_C = midpoint(A--B);

dot("$A$", A, dir(A));
dot("$B$", B, dir(B));
dot("$C$", C, dir(C));
dot("$O$", O, 1.4*dir(250));
dot("$P_A$", P_A, dir(P_A));
dot("$H$", H, dir(225));
dot(M);
dot("$X_A$", X_A, dir(115));
dot("$Y_A$", Y_A, dir(40));
dot("$K_A$", K_A, 1.4*dir(90));
dot("$K_A^\ast$", K_As, dir(K_As));
dot("$U_1$", U_1, dir(130));
dot("$U_2$", U_2, dir(0));
dot("$M_C$", M_C, dir(M_C));

/* TSQ Source:

A = dir 50
B = dir 220
C = dir 320
A--B--C--cycle 0.1 lightcyan / blue
O = origin 1.4R250
O--A lightblue
O--B lightblue
O--C lightblue

P_A = circumcenter O B C
P_B := circumcenter O C A
P_C := circumcenter O A B
H = orthocenter A B C R225
M .= midpoint A--H
X_A = extension A B M B-C+M R115
Y_A = extension A C M B-C+M R40
X := X_A
Y := Y_A

A--H blue
K_A = -O+2*foot O circumcenter O B C circumcenter O X_A Y_A 1.4R90
K_A* = extension P_B P_C B C
circumcircle O B C grey
circumcircle O X_A Y_A grey

O--K_As deepgreen
K_As--B blue

U_1 = -X+2*foot circumcenter O X_A Y_A A B R130
U_2 = -Y+2*foot circumcenter O X_A Y_A A C R0
U_1--K_As deepgreen
M_C = midpoint A--B

*/
\end{asy}
\end{center}

\paragraph{Second solution, from Jeffrey Kwan.}
Let $O_A$ be the circumcenter of $\triangle AX_AY_A$.
The key claim is that:

\begin{claim*}
$O_AX_AY_AO$ is cyclic.
\end{claim*}

\begin{proof}
Let $DEF$ be the orthic triangle;
we will show that $\triangle OX_AY_A\sim \triangle DEF$.
Indeed, since $AO$ and $AD$ are isogonal, it suffices to note that
\[ \frac{AX_A}{AB} = \frac{AH/2}{AD} = \frac{R\cos A}{AD}, \]
and so
\[ \frac{AO}{AD} = R\cdot \frac{AX_A}{AB\cdot R\cos A}
  = \frac{AX_A}{AE} = \frac{AY_A}{AF}. \]
Hence $\angle X_AOY_A = 180\dg - 2\angle A = 180\dg - \angle X_AO_AY_A$,
which proves the claim.
\end{proof}

Let $P_A$ be the circumcenter of $\triangle OBC$,
and define $P_B$, $P_C$ similarly.
By the claim, $A$ is the exsimilicenter of $(OX_AY_A)$ and $(OBC)$,
so $AP_A$ is the line between their two centers.
In particular, $AP_A$ is the perpendicular bisector of $OK_A$.


\begin{center}
\begin{asy}
defaultpen(fontsize(10pt));
size(13cm);
pair A, B, C, O, H, N, D, OA, PA, PB, PC, X1, X2, KA;
A = dir(110);
B = dir(220);
C = dir(320);
O = (0,0);
H = orthocenter(A, B, C);
N = (A+H)/2;
X1 = extension(N, rotate(90, N)*A, A, B);
X2 = extension(N, rotate(90, N)*A, A, C);
KA = -O+2*foot(O, circumcenter(O, X1, X2), circumcenter(B, O, C));
OA = circumcenter(A, X1, X2);
PA = circumcenter(O, B, C);
PB = circumcenter(O, C, A);
PC = circumcenter(O, A, B);
draw(A--B--C--cycle, orange);
draw(circumcircle(A, B, C), red);
draw(A--foot(A, B, C)^^B--foot(B, C, A)^^C--foot(C, A, B), heavygreen+dotted);
draw(X1--X2, dashed+lightblue);
draw(circumcircle(O, B, C), heavycyan);
draw(circumcircle(O, X1, X2), lightblue);
draw(A--PA, magenta);
dot("$A$", A, dir(A));
dot("$B$", B, dir(B));
dot("$C$", C, dir(C));
dot("$O$", O, dir(270));
dot("$H$", H, dir(300));
dot("$X_A$", X1, dir(160));
dot("$Y_A$", X2, dir(45));
dot("$K_A$", KA, dir(270));
dot("$O_A$", OA, dir(90));
dot("$P_A$", PA, dir(270));
\end{asy}
\end{center}

\begin{claim*}
$AP_A$, $BP_B$, $CP_C$ concur at $T$.
\end{claim*}

\begin{proof}
The key observation is that $O$ is the incenter of $\triangle P_AP_BP_C$,
and that $A$, $B$, $C$ are the reflections of $O$ across the sides of $\triangle P_AP_BP_C$.
Hence $P_AA$, $P_BB$, $P_CC$ concur by Jacobi.
\end{proof}

Now $T$ lies on the perpendicular bisectors of $OK_A$, $OK_B$, and $OK_C$.
Hence $OK_AK_BK_C$ is cyclic with center $T$, as desired.
