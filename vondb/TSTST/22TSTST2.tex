author: Hongzhou Lin
desc: Fixed angle, foot of altitudes
source: TSTST 2022/2
url: https://aops.com/community/p25516988
tags: [2023-08, pop, anglechase, todo]

---

Let $ABC$ be a triangle.
Let $\theta$ be a fixed angle for which
\[ \theta < \frac12 \min(\angle A, \angle B, \angle C). \]
Points $S_A$ and $T_A$ lie on segment $BC$
such that $\angle BAS_A = \angle T_AAC = \theta$.
Let $P_A$ and $Q_A$ be the feet from $B$ and $C$
to $\ol{AS_A}$ and $\ol{AT_A}$ respectively.
Then $\ell_A$ is defined as the perpendicular bisector of $\ol{P_A Q_A}$.

Define $\ell_B$ and $\ell_C$ analogously by repeating this construction
two more times (using the same value of $\theta$).
Prove that $\ell_A$, $\ell_B$, and $\ell_C$ are concurrent or all parallel.

---

We discard the points $S_A$ and $T_A$ since they are only there
to direct the angles correctly in the problem statement.

\paragraph{First solution, by author.}
Let $X$ be the projection from $C$ to $AP_A$,
$Y$ be the projection from $B$ to $AQ_A$.
\begin{center}
\begin{asy}
pair A = dir(110);
pair B = dir(200);
pair C = dir(340);

pair Z_1 = dir(215);
pair Z_2 = B*C/Z_1;
pair P_A = foot(B, A, Z_1);
pair Q_A = foot(C, A, Z_2);
filldraw(A--B--C--cycle, opacity(0.1)+lightcyan, blue);
draw(A--P_A, red);
draw(A--Q_A, red);
pair X = foot(C, A, P_A);
pair Y = foot(B, A, Q_A);

draw(B--P_A, orange);
draw(C--Q_A, orange);
draw(B--Y, orange);
draw(C--X, orange);

pair M_A = midpoint(B--C);
pair M_B = midpoint(C--A);
pair M_C = midpoint(A--B);

draw(M_B--M_A--M_C, blue);

draw(P_A--Q_A, deepgreen);
draw(P_A--M_A--Q_A, deepgreen);

draw(circumcircle(P_A, X, Y), grey);

dot("$A$", A, dir(A));
dot("$B$", B, dir(B));
dot("$C$", C, dir(C));
dot("$P_A$", P_A, dir(P_A));
dot("$Q_A$", Q_A, dir(Q_A));
dot("$X$", X, dir(X));
dot("$Y$", Y, dir(70));
dot("$M_A$", M_A, dir(M_A));
dot("$M_B$", M_B, dir(M_B));
dot("$M_C$", M_C, dir(M_C));

/* TSQ Source:

A = dir 110
B = dir 200
C = dir 340

Z_1 := dir 215
Z_2 := B*C/Z_1
P_A = foot B A Z_1
Q_A = foot C A Z_2
A--B--C--cycle 0.1 lightcyan / blue
A--P_A red
A--Q_A red
X = foot C A P_A
Y = foot B A Q_A R70

B--P_A orange
C--Q_A orange
B--Y orange
C--X orange

M_A = midpoint B--C
M_B = midpoint C--A
M_C = midpoint A--B

M_B--M_A--M_C blue

P_A--Q_A deepgreen
P_A--M_A--Q_A deepgreen

circle P_A X Y grey

*/
\end{asy}
\end{center}

\begin{claim*}
  Line $\ell_A$ passes through $M_A$, the midpoint of $BC$.
  Also, quadrilateral $P_AQ_AYX$ is cyclic with circumcenter $M_A$.
\end{claim*}
\begin{proof}
  Since
  \[ AP_A\cdot AX = AB\cdot AC\cdot\cos\theta\cos(\angle A - \theta)
    = AQ_A\cdot AY,\]
  it follows that $P_A$, $Q_A$, $Y$, $X$
  are concyclic by power of a point.
  Moreover, by projection,
  the perpendicular bisector of $P_AX$ passes through $M_A$,
  similar for $Q_AY$, implying that $M_A$ is the center of $P_AQ_AYX$.
  Hence $\ell_A$ passes through $M_A$.
\end{proof}

\begin{claim*}
  $\dang (M_AM_C, \ell_A) = \dang YP_AQ_A$.
\end{claim*}
\begin{proof}
  Indeed, $\ell_A \perp P_AQ_A$, and $M_AM_C \perp P_AY$
  (since $M_AP_A = M_AY$ from $(P_AQ_AY_AX)$
  and $M_CP_A = M_CM_A = M_CY$ from the circle with diameter $AB$).
  Hence $\dang (M_AM_C, \ell_A) = \dang (P_AY, P_AQ_A) = \dang YP_AQ_A$.
\end{proof}
Therefore,
\[
  \frac{\sin \angle (M_AM_C, \ell_A)}{\sin \angle (\ell_A, M_AM_B)}
  = \frac{\sin \angle YP_AQ_A}{\sin \angle P_AQ_AX}
  = \frac{YQ_A}{XP_A}
  = \frac{BC \sin (\angle C+ \theta) }{BC \sin (\angle B+ \theta)}
  = \frac{\sin (\angle C+ \theta)}{\sin (\angle B+ \theta)},
\]
and we conclude by trig Ceva theorem.

\paragraph{Second solution via Jacobi, by Maxim Li.}
Let $D$ be the foot of the $A$-altitude.
Note that line $BC$ is the external angle bisector of $\angle P_ADQ_A$.
\begin{claim*}
  $(DP_AQ_A)$ passes through the midpoint $M_A$ of $BC$.
\end{claim*}
\begin{proof}
  Perform $\sqrt{bc}$ inversion.
  Then the intersection of $BC$ and $(DP_AQ_A)$
  maps to the second intersection of $(ABC)$ and $(A'P_AQ_A)$,
  where $A'$ is the antipode to $A$ on $(ABC)$,
  i.e.\ the center of spiral similarity from $BC$ to $P_AQ_A$.
  Since $BP_A:CQ_A = AB:AC$, we see the center of spiral similarity
  is the intersection of the $A$-symmedian with $(ABC)$,
  which is the image of $M_A$ in the inversion.
\end{proof}

It follows that $M_A$ lies on $\ell_A$;
we need to identify a second point.
We'll use the circumcenter $O_A$ of $(DP_AQ_A)$.
The perpendicular bisector of $DP_A$ passes through $M_C$;
indeed, we can easily show the angle it makes with $M_CM_A$
is $90^\circ - \theta - C$, so $\angle O_AM_CM_A = 90 - \theta - C$,
and then by analogous angle-chasing
we can finish with Jacobi's theorem on $\triangle M_AM_BM_C$.
