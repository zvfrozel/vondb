author: Merlijn Staps
desc: Exactly one of $f(m+1)$ to $f(m+f(n))$ is divisible by $n$
source: TSTST 2022/4
url: https://aops.com/community/p25517031
tags: [2023-08, todo]

---

A function $f \colon \NN \to \NN$ has the property that
for all positive integers $m$ and $n$, exactly one of the $f(n)$ numbers
\[ f(m+1), f(m+2), \dots, f(m+f(n)) \]
is divisible by $n$.
Prove that $f(n)=n$ for infinitely many positive integers $n$.

---

We start with the following claim:

\begin{claim*}
  If $a \mid b$ then $f(a) \mid f(b)$.
\end{claim*}
\begin{proof}
  From applying the condition with $n=a$,
  we find that the set $S_a = \{n \ge 2: a \mid f(n)\}$
  is an arithmetic progression with common difference $f(a)$.
  Similarly, the set $S_b = \{n \ge 2: b \mid f(n)\}$
  is an arithmetic progression with common difference $f(b)$.
  From $a \mid b$ it follows that $S_b \subseteq S_a$.
  Because an arithmetic progression with common difference $x$ can only be
  contained in an arithmetic progression with common difference $y$
  if $y \mid x$, we conclude $f(a) \mid f(b)$.
\end{proof}

In what follows, let $a \ge 2$ be any positive integer.
Because $f(a)$ and $f(2a)$ are both divisible by $f(a)$,
there are $a+1$ consecutive values of $f$ of which
at least two divisible by $f(a)$. It follows that $f(f(a)) \le a$.

However, we also know that exactly one of
$f(2)$, $f(3)$, \dots, $f(1+f(a))$ is divisible by $a$; let this be $f(t)$.
Then we have $S_a = \{t, t+f(a), t+2f(a), \dots\}$.
Because $a \mid f(t) \mid f(2t)$,
we know that $2t \in S_a$, so $t$ is a multiple of $f(a)$.
Because $2 \le t \le 1+f(a)$, and $f(a) \ge 2$ for $a \ge 2$,
we conclude that we must have $t=f(a)$,
so $f(f(a))$ is a multiple of $a$.
Together with $f(f(a)) \le a$, this yields $f(f(a)) = a$.
Because $f(f(a)) = a$ also holds for $a=1$
(from the given condition for $n=1$ it immediately follows that $f(1)=1$),
we conclude that $f(f(a))=a$ for all $a$, and hence $f$ is a bijection.

Moreover, we now have that $f(a) \mid f(b)$ implies $f(f(a)) \mid f(f(b))$,
i.e.\ $a \mid b$, so $a \mid b$ if and only if $f(a) \mid f(b)$.
Together with the fact that $f$ is a bijection,
this implies that $f(n)$ has the same number of divisors of $n$.
Let $p$ be a prime. Then $f(p)=q$ must be a prime as well.
If $q \neq p$, then from $f(p) \mid f(pq)$
and $f(q) \mid f(pq)$ it follows that $pq \mid f(pq)$, so $f(pq) = pq$
because $f(pq)$ and $pq$ must have the same number of divisors.
Therefore, for every prime number $p$ we either have that $f(p)=p$
or $f(pf(p)) = pf(p)$.
From here, it is easy to see that $f(n)=n$ for infinitely many $n$.
