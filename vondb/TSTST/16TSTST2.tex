desc: Orthobrokard, radical axis geometry
author: Evan Chen
source: TSTST 2016/2
tags: [well, 2016-08, pop, rich, harmonic, mine, nice, dalet]
hardness: 30
url: https://aops.com/community/p6575204

---

Let $ABC$ be a scalene triangle with orthocenter $H$ and circumcenter $O$
and denote by $M$, $N$ the midpoints of $\ol{AH}$, $\ol{BC}$.
Suppose the circle $\gamma$ with diameter $\ol{AH}$ meets
the circumcircle of $ABC$ at $G \neq A$,
and meets line $\ol{AN}$ at $Q \neq A$.
The tangent to $\gamma$ at $G$ meets line $OM$ at $P$.
Show that the circumcircles of $\triangle GNQ$
and $\triangle MBC$ intersect on $\ol{PN}$.

---

We present two solutions,
one using essentially only power of a point,
and the other more involved.

\paragraph{First solution (found by contestants).}
Denote by $\triangle DEF$ the orthic triangle.
Observe $\ol{PA}$ and $\ol{PG}$ are tangents to $\gamma$,
since $\ol{OM}$ is the perpendicular bisector of $\ol{AG}$.
Also note that $\ol{AG}$, $\ol{EF}$, $\ol{BC}$ are concurrent at some point $R$
by radical axis on $(ABC)$, $\gamma$, $(BFEC)$.

Now, consider circles $(PAGM)$, $(MFDNE)$, and $(MBC)$.
We already saw the point $R$ satisfies
\[ RA \cdot RG = RE \cdot RF = RB \cdot RC \]
and hence has equal powers to all three circles;
but since the circles at $M$ already, they must actually be coaxial.
Assume they meet again at $T \in \ol{RM}$, say.
Then $\angle PTM$ and $\angle MTN$ are both right angles, hence $T$ lies on $\ol{PN}$.

Finally $H$ is the orthocenter of $\triangle ARN$,
and thus the circle with diameter $\ol{RN}$ passes through $G$, $Q$, $N$.

\begin{center}
\begin{asy}
size(11cm);
pair A = dir(120);
pair B = dir(210);
pair C = dir(330);
pair N = midpoint(B--C);
pair H = orthocenter(A, B, C);
pair M = midpoint(A--H);
pair O = circumcenter(A, B, C);
pair P = extension(O, M, A, B-C+A);
pair T = foot(M, P, N);
pair D = foot(A, B, C);
pair E = foot(B, C, A);
pair F = foot(C, A, B);

filldraw(A--B--C--cycle, opacity(0.1)+lightblue, blue);
draw(unitcircle, blue);

draw(A--D, heavygreen);
draw(B--E, heavygreen);
draw(C--F, heavygreen);
draw(E--F, heavygreen);
draw(B--M--C, lightred);
// filldraw(circumcircle(M, B, C), opacity(0.05)+lightred, red);

filldraw(circumcircle(D, E, F), opacity(0.05)+lightred, red);
draw(O--P--N, heavycyan);
draw(A--P, heavycyan);

filldraw(circumcircle(A, E, F), opacity(0.1)+lightgreen, heavygreen);
pair Q = foot(H, A, N);
pair G = -A+2*foot(A, M, P);

pair R = extension(E, F, B, C);
draw(R--F, heavygreen);
draw(R--B, blue);
draw(R--Q, heavygreen);
draw(A--R, heavygreen);
draw(A--N, blue);
draw(P--G, heavycyan);
draw(G--N, blue);
draw(M--R, heavygreen);
filldraw(circumcircle(R, G, T), opacity(0.05)+lightred, red);

dot("$A$", A, dir(A));
dot("$B$", B, dir(B));
dot("$C$", C, dir(C));
dot("$N$", N, dir(N));
dot("$H$", H, dir(H));
dot("$M$", M, dir(M));
dot("$O$", O, dir(225));
dot("$P$", P, dir(P));
dot("$T$", T, dir(T));
dot("$D$", D, dir(D));
dot("$E$", E, dir(E));
dot("$F$", F, dir(F));
dot("$Q$", Q, dir(Q));
dot("$G$", G, dir(G));
dot("$R$", R, dir(R));

/* TSQ Source:

!size(11cm);
A = dir 120
B = dir 210
C = dir 330
N = midpoint B--C
H = orthocenter A B C
M = midpoint A--H
O = circumcenter A B C R225
P = extension O M A B-C+A
T = foot M P N
D = foot A B C
E = foot B C A
F = foot C A B

A--B--C--cycle 0.1 lightblue / blue
unitcircle blue

A--D heavygreen
B--E heavygreen
C--F heavygreen
E--F heavygreen
B--M--C lightred
// circumcircle M B C 0.05 lightred / red

circumcircle D E F 0.1 lightred / red
O--P--N heavycyan
A--P heavycyan

circumcircle A E F 0.1 lightgreen / heavygreen
Q = foot H A N
G = -A+2*foot A M P

R = extension E F B C
R--F heavygreen
R--B blue
R--Q heavygreen
A--R heavygreen
A--N blue
P--G heavycyan
G--N blue
M--R heavygreen
circumcircle R G T 0.1 lightred / red

*/
\end{asy}
\end{center}

\paragraph{Alternate solution (by proposer).}
Let $L$ be diametrically opposite $A$ on the circumcircle.
Denote by $\triangle DEF$ the orthic triangle.
Let $X = \ol{AH} \cap \ol{EF}$.
Finally, let $T$ be the second intersection of $(MFDNE)$ and $(MBC)$.

\begin{center}
\begin{asy}
size(11cm);
pair A = dir(120);
pair B = dir(210);
pair C = dir(330);
pair N = midpoint(B--C);
pair H = orthocenter(A, B, C);
pair M = midpoint(A--H);
pair O = circumcenter(A, B, C);
pair P = extension(O, M, A, B-C+A);
pair X = extension(P, N, A, H);
pair T = foot(M, P, N);
pair D = foot(A, B, C);
pair E = foot(B, C, A);
pair F = foot(C, A, B);
pair L = -A;
pair K = extension(H, N, A, P);

filldraw(A--B--C--cycle, opacity(0.1)+lightblue, blue);
filldraw(B--L--C--cycle, opacity(0.1)+lightblue, blue);
draw(unitcircle, blue);
draw(A--L, blue);

draw(A--D, heavygreen);
draw(B--E, heavygreen);
draw(C--F, heavygreen);
draw(E--F, heavygreen);
draw(B--M--C, lightred);
filldraw(circumcircle(M, B, C), opacity(0.05)+lightred, red);

filldraw(circumcircle(D, E, F), opacity(0.05)+lightred, red);
draw(A--K--L, heavycyan);
draw(O--P--N, heavycyan);

filldraw(circumcircle(A, E, F), opacity(0.1)+lightgreen, heavygreen);
pair Q = foot(H, A, N);
pair G = -A+2*foot(A, M, P);

pair R = extension(E, F, B, C);
draw(R--F, heavygreen);
draw(R--B, blue);
draw(R--Q, heavygreen);
draw(A--R, heavygreen);
draw(A--N, blue);
draw(P--G, heavycyan);
draw(M--R, heavygreen);
filldraw(circumcircle(R, G, T), opacity(0.05)+lightred, red);

dot("$A$", A, dir(A));
dot("$B$", B, dir(B));
dot("$C$", C, dir(C));
dot("$N$", N, dir(N));
dot("$H$", H, dir(H));
dot("$M$", M, dir(M));
dot("$O$", O, dir(225));
dot("$P$", P, dir(P));
dot("$X$", X, dir(X));
dot("$T$", T, dir(T));
dot("$D$", D, dir(D));
dot("$E$", E, dir(E));
dot("$F$", F, dir(F));
dot("$L$", L, dir(L));
dot("$K$", K, dir(K));
dot("$Q$", Q, dir(Q));
dot("$G$", G, dir(G));
dot("$R$", R, dir(R));

/* TSQ Source:

!size(11cm);
A = dir 120
B = dir 210
C = dir 330
N = midpoint B--C
H = orthocenter A B C
M = midpoint A--H
O = circumcenter A B C R225
P = extension O M A B-C+A
X = extension P N A H
T = foot M P N
D = foot A B C
E = foot B C A
F = foot C A B
L = -A
K = extension H N A P

A--B--C--cycle 0.1 lightblue / blue
B--L--C--cycle 0.1 lightblue / blue
unitcircle blue
A--L blue

A--D heavygreen
B--E heavygreen
C--F heavygreen
E--F heavygreen
B--M--C lightred
circumcircle M B C 0.05 lightred / red

circumcircle D E F 0.05 lightred / red
A--K--L heavycyan
O--P--N heavycyan

circumcircle A E F 0.1 lightgreen / heavygreen
Q = foot H A N
G = -A+2*foot A M P

R = extension E F B C
R--F heavygreen
R--B blue
R--Q heavygreen
A--R heavygreen
A--N blue
P--G heavycyan
M--R heavygreen
circumcircle R G T 0.05 lightred / red

*/
\end{asy}
\end{center}

We begin with a few easy observations.
First, points $H$, $G$, $N$, $L$ are collinear and $\angle AGL = 90\dg$.
Also, $Q$ is the foot from $H$ to $\ol{AN}$.
Consequently, lines $AG$, $EF$, $HQ$, $BC$, $TM$
concur at a point $R$ (radical axis).
Moreover, we already know $\angle MTN = 90\dg$.
This implies $T$ lies on the circle with diameter $\ol{RN}$,
which is exactly the circumcircle of $\triangle GQN$.

Note by Brokard's Theorem on $AFHE$, the point $X$ is the orthocenter of $\triangle MBC$.
But $\angle MTN = 90\dg$ already, and $N$ is the midpoint of $\ol{BC}$.
Consequently, points $T$, $X$, $N$ are collinear.

Finally, we claim $P$, $X$, $N$ are collinear, which solves the problem.
Note $P = \ol{GG} \cap \ol{AA}$.
Set $K = \ol{HNL} \cap \ol{AP}$.
Then by noting \[ -1 = (D,X;A,H) \overset{N}{=} (\infty, \ol{NX} \cap \ol{AK}; A, K) \]
we see that $\ol{NX}$ bisects segment $\ol{AK}$, as desired.
(A more projective finish is to show that $\ol{PXN}$ is the polar of $R$ to $\gamma$).

\begin{remark*}
The original problem proposal reads as follows:
\begin{quote}
Let $ABC$ be a triangle with orthocenter $H$ and circumcenter $O$
and denote by $M$, $N$ the midpoints of $\ol{AH}$, $\ol{BC}$.
Suppose ray $OM$ meets the line parallel to $\ol{BC}$ through $A$ at $P$.
Prove that the line through the circumcenter of $\triangle MBC$
and the midpoint of $\ol{OH}$ is parallel to $\ol{NP}$.
\end{quote}
The points $G$ and $Q$ were added to the picture later
to prevent the problem from being immediate by coordinates.
\end{remark*}

---

%%fakesection Walkthrough
Let $DEF$ be the orthic triangle of $ABC$.
\begin{walk}
  \ii Show that $P$ is really just the intersection
  of the tangents to $\gamma$ at $A$ and $G$
  (and thus the line $\ol{OM}$ is just a distraction).
  \ii Show that lines $\ol{AG}$, $\ol{EF}$, $\ol{BC}$ are concurrent, say at $R$.
  \ii Prove that $(PAMG)$, $(MBC)$, $(MFDNE)$ are concurrent at a point $T \neq M$.
  \ii Show that $T = \ol{PN} \cap \ol{MR}$.
  \ii Show that $R \in \ol{HQ}$.
  \ii Show that $R$, $G$, $T$, $Q$, $N$ are concyclic, completing the proof.
\end{walk}


---


!size(11cm);
A = dir 120
B = dir 210
C = dir 330
N = midpoint B--C
H = orthocenter A B C
M = midpoint A--H
O = circumcenter A B C R225
P = extension O M A B-C+A
X = extension P N A H
T = foot M P N
D = foot A B C
E = foot B C A
F = foot C A B
L = -A
K = extension H N A P

A--B--C--cycle 0.1 lightblue / blue
B--L--C--cycle 0.1 lightblue / blue
unitcircle blue
A--L blue

A--D heavygreen
B--E heavygreen
C--F heavygreen
E--F heavygreen
B--M--C lightred
circumcircle M B C 0.05 lightred / red

circumcircle D E F 0.05 lightred / red
A--K--L heavycyan
O--P--N heavycyan

circumcircle A E F 0.1 lightgreen / heavygreen
Q = foot H A N
G = -A+2*foot A M P

R = extension E F B C
R--F heavygreen
R--B blue
R--Q heavygreen
A--R heavygreen
A--N blue
P--G heavycyan
M--R heavygreen
circumcircle R G T 0.05 lightred / red
