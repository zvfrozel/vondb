author: Ray Li
desc: Red/blue query in the plane
source: TSTST 2022/5
url: https://aops.com/community/p25517063
tags: [2023-08, process, find, bestpossible, todo]

---

Let $A_1$, \dots, $A_{2022}$ be the vertices of a regular $2022$-gon in the plane.
Alice and Bob play a game.
Alice secretly chooses a line and colors all points in the plane
on one side of the line blue, and all points on the other side of the line red.
Points on the line are colored blue,
so every point in the plane is either red or blue.
(Bob cannot see the colors of the points.)

In each round, Bob chooses a point in the
plane (not necessarily among $A_1$, \dots, $A_{2022}$)
and Alice responds truthfully with the color of that point.
What is the smallest number $Q$ for which Bob has a strategy to
always determine the colors of points $A_1$, \dots, $A_{2022}$ in $Q$ rounds?

---

The answer is $22$.
To prove the lower bound, note that there are
$2022\cdot 2021 + 2 > 2^{21}$ possible colorings.
If Bob makes less than $22$ queries,
then he can only output $2^{21}$ possible colorings,
which means he is wrong on some coloring.

Now we show Bob can always win in $22$ queries.
A key observation is that the set of red points is convex,
as is the set of blue points, so if a set of points is all the same color,
then their convex hull is all the same color.
\begin{lemma*}
  Let $B_0$, \dots, $B_{k+1}$ be equally spaced points on a circular arc
  such that colors of $B_0$ and $B_{k+1}$ differ and are known.
  Then it is possible to determine the colors of $B_1$, \dots, $B_k$
	in $\left\lceil \log_2k \right\rceil$ queries.
\end{lemma*}
\begin{proof}
  There exists some $0\le i\le k$ such that $B_0$, \dots, $B_{i}$
  are the same color and $B_{i+1}$, \dots, $B_{k+1}$ are the same color.
  (If $i<j$ and $B_0$ and $B_j$ were red and $B_i$ and $B_{k+1}$ were blue,
  then segment $B_0B_j$ is red and segment $B_iB_{k+1}$ is blue,
  but they intersect).
  Therefore we can binary search.
\end{proof}
\begin{lemma*}
  Let $B_0$, \dots, $B_{k+1}$ be equally spaced points on a circular arc
  such that colors of $B_0$, $B_{\left\lceil k/2 \right\rceil}$, $B_{k+1}$
  are both red and are known.
  Then at least one of the following holds:
  all of $B_1$ ,\dots, $B_{\left\lceil k/2 \right\rceil}$ are red
  or all of $B_{\left\lceil k/2 \right\rceil}$,\dots,$B_k$ are red.
  Furthermore, in one query we can determine which one of the cases holds.
\end{lemma*}
\begin{proof}
  The existence part holds for similar reason to previous lemma.
  To figure out which case, choose a point $P$ such that
  all of $B_0$, \dots, $B_{k+1}$
  lie between rays $PB_0$ and $PB_{\left\lceil k/2 \right\rceil}$,
  and such that $B_1$, \dots, $B_{\left\lceil k/2 \right\rceil-1}$ lie inside
  triangle $PB_0B_{\left\lceil k/2 \right\rceil}$ and such that
  $B_{\left\lceil k/2 \right\rceil+1}$, \dots, $B_{k+1}$ lie outside
  (this point should always exist by looking around the intersections of lines $B_0B_1$ and
  $B_{\left\lceil k/2 \right\rceil-1}B_{\left\lceil k/2 \right\rceil}$).
  Then if $P$ is red, all the inside points are red
	because they lie in the convex hull of red points $P$, $B_0$, $B_{\left\lceil k/2 \right\rceil}$.
  If $P$ is blue, then all the outside points are red:
  if $B_i$ were blue for $i > \left\lceil k/2 \right\rceil$,
  then the segment $PB_i$ is blue and intersect the segment
  $B_0B_{\left\lceil k/2 \right\rceil}$, which is red, contradiction.
\end{proof}
Now the strategy is: Bob picks $A_1$. WLOG it is red.
Now suppose Bob does not know the colors of $\le 2^k-1$ points
$A_i$, \dots, $A_j$ with $j-i+1\le 2^k-1$ and knows the rest are red.
I claim Bob can win in $2k-1$ queries.
First, if $k=1$, there is one point and he wins by querying the point,
so the base case holds, so assume $k>1$.
Bob queries $A_{i+\left\lceil (j-i+1)/2 \right\rceil}$.
If it is blue, he finishes in $2\log_2{\left\lceil (j-i+1)/2 \right\rceil} \le 2(k-1)$
queries by the first lemma, for a total of $2k-1$ queries.
If it is red, he can query one more point and learn some half
of $A_i$, \dots, $A_j$ that are red by the second lemma,
and then he has reduced it to the case with $\le 2^{k-1}-1$ points
in two queries, at which point we induct.
