desc: Monge troll with incircles
author: Ray Li
source: TSTST 2017/5
tags: [2017-06, harmonic, nice, bary, homothety, criticalclaim, instructive, aleph]
hardness: 15
url: https://aops.com/community/p8526136

---

Let $ABC$ be a triangle with incenter $I$.
Let $D$ be a point on side $BC$ and let $\omega_B$ and $\omega_C$
be the incircles of $\triangle ABD$ and $\triangle ACD$, respectively.
Suppose that $\omega_B$ and $\omega_C$ are tangent to segment $BC$
at points $E$ and $F$, respectively.
Let $P$ be the intersection of segment $AD$ with the
line joining the centers of $\omega_B$ and $\omega_C$.
Let $X$ be the intersection point of lines $BI$ and $CP$
and let $Y$ be the intersection point of lines $CI$ and $BP$.
Prove that lines $EX$ and $FY$ meet on the incircle of $\triangle ABC$.

---

\paragraph{First solution (homothety).}
Let $Z$ be the diametrically opposite point on the incircle.
We claim this is the desired intersection.

\begin{center}
\begin{asy}
pair A = dir(110);
pair B = dir(210);
pair C = dir(330);

filldraw(A--B--C--cycle, opacity(0.1)+lightred, red);
filldraw(incircle(A, B, C), opacity(0.1)+lightcyan, lightblue);

pair D = 0.35*B+0.65*C;
draw(A--D, red);

pair I_B = incenter(A, B, D);
pair I_C = incenter(A, C, D);
pair E = foot(I_B, B, C);
pair F = foot(I_C, B, C);

filldraw(incircle(A, B, D), opacity(0.1)+green, heavygreen);
filldraw(incircle(A, C, D), opacity(0.1)+green, heavygreen);
pair I = incenter(A, B, C);

pair P = extension(I_B, I_C, A, D);
draw(I_B--I_C, heavygreen);

pair X = extension(B, I, C, P);
pair Y = extension(C, I, B, P);
pair Z = extension(E, X, F, Y);
draw(B--I--C, red);
draw(X--C, dotted+heavygreen);
draw(Y--B, dotted+heavygreen);
draw(E--Z--F, dashed+heavygreen);

pair T = extension(B, C, I_B, I_C);
draw(I_C--T, dotted+heavygreen);
draw(C--T, dotted+red);
pair W = foot(I, B, C);

dot("$A$", A, dir(A));
dot("$B$", B, dir(270));
dot("$C$", C, dir(270));
dot("$D$", D, dir(D));
dot("$I_B$", I_B, dir(170));
dot("$I_C$", I_C, dir(30));
dot("$E$", E, dir(E));
dot("$F$", F, dir(F));
dot("$I$", I, dir(90));
dot("$P$", P, dir(250));
dot("$X$", X, dir(160));
dot("$Y$", Y, dir(20));
dot("$Z$", Z, dir(140));
dot("$T$", T, dir(T));
dot("$W$", W, dir(W));

/* TSQ Source:

A = dir 110
B = dir 210 R270
C = dir 330 R270

A--B--C--cycle 0.1 lightred / red
incircle A B C 0.1 lightcyan / lightblue

D = 0.35*B+0.65*C
A--D red

I_B = incenter A B D R170
I_C = incenter A C D R30
E = foot I_B B C
F = foot I_C B C

incircle A B D 0.1 green / heavygreen
incircle A C D 0.1 green / heavygreen
I = incenter A B C R90

P = extension I_B I_C A D R250
I_B--I_C heavygreen

X = extension B I C P R160
Y = extension C I B P R20
Z = extension E X F Y R140
B--I--C red
X--C dotted heavygreen
Y--B dotted heavygreen
E--Z--F dashed heavygreen

T = extension B C I_B I_C
I_C--T dotted heavygreen
C--T dotted red
W = foot I B C

*/
\end{asy}
\end{center}

Note that:
\begin{itemize}
  \ii $P$ is the insimilicenter of $\omega_B$ and $\omega_C$
  \ii $C$ is the exsimilicenter of $\omega$ and $\omega_C$.
\end{itemize}
Thus by Monge theorem, the insimilicenter of $\omega_B$ and $\omega$
lies on line $CP$.

This insimilicenter should also lie on the line joining the
centers of $\omega$ and $\omega_B$, which is $\ol{BI}$,
hence it coincides with the point $X$.
So $X \in \ol{EZ}$ as desired.

\paragraph{Second solution (harmonic).}
Let $T = \ol{I_B I_C} \cap \ol{BC}$, and $W$ the foot from $I$ to $\ol{BC}$.
Define $Z = \ol{FY} \cap \ol{IW}$.
Because $\angle I_B D I_C = 90\dg$, we have
\[ -1 = (I_B I_C; PT) \overset{B}{=} (I I_C; YC)
  \overset{F}{=} (I\infty; ZW) \]
So $I$ is the midpoint of $\ol{ZW}$ as desired.

\paragraph{Third solution (outline, barycentric, Andrew Gu).}
Let $AD = t$, $BD = x$, $CD = y$, so $a=x+y$ and by Stewart's theorem we have
\begin{equation}
  (x+y)(xy+t^2) = b^2x+c^2y.
  \label{eq:17TSTST5stewart}
\end{equation}
We then have $D = (0:y:x)$ and so
\[ \ol{AI_B} \cap \ol{BC} = \left( 0 : y + \frac{tx}{c+t} : \frac{cx}{c+t} \right) \]
hence intersection with $BI$ gives
\begin{align*}
  I_B &= (ax : cy+at : cx).
  \intertext{Similarly,}
  I_C &= (ay : by : bx+at).
\end{align*}

Then, we can compute
\[ P = \left( 2axy : y(at+bx+cy) : x(at+bx+cy) \right) \]
since $P \in \ol{I_B I_C}$, and clearly $P \in \ol{AD}$.
Intersection now gives
\begin{align*}
  X &= \left( 2ax : at+bx+cy : 2cx \right) \\
  Y &= \left( 2ay : 2by : at+bx+cy \right).
\end{align*}

Finally, we have $BE = \half(c+x-t)$, and similarly for $CF$.
Now if we reflect
$D = (0, \frac{s-c}{a}, \frac{s-b}{a})$ over
$I = (\frac{a}{2s}, \frac{b}{2s}, \frac{c}{2s})$, we get the antipode
\[ Q \coloneqq \left( 4a^2 : -a^2+2ab-b^2+c^2 : -a^2+2ac-c^2+b^2 \right). \]
We may then check $Q$ lies on each of lines $EX$ and $FY$
(by checking $\det(Q,E,X)=0$ using the equation \eqref{eq:17TSTST5stewart}).

---

%%fakesection Walkthrough
Let $\omega$ denote the incircle of $\triangle ABC$.
\begin{walk}
  \ii Identify the point $Z = \ol{EX} \cap \ol{FY}$ in a good diagram.
  (This was worth a point! Despite this, many contestants were unable to find it.)
  \ii Consider the positive homothety sending $\omega$ to $\omega_C$.
  Determine its center.
  \ii Consider the negative homothety sending $\omega_C$ to $\omega_B$.
  Determine its center.
  \ii The composition of the previous two homotheties
  in (b) and (c) is a negative homothety
  sending $\omega$ to $\omega_B$.
  Determine with proof the center of this homothety.

  This is not as simple as the previous two parts;
  you will need to \emph{use} (b) and (c) to do this part,
  as well as the simple observation that the center
  should lie on the $\angle B$ bisector.
  \ii Conclude that $\ol{FY}$ passes through the point you claimed in (a).
\end{walk}
Experts may notice that this walkthrough gives what is essentially
a proof of Monge d'Alembert theorem.
