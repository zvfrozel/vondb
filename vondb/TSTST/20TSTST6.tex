author: Andrew Gu
desc: Not the orthocenter
hardness: 25
source: TSTST 2020/6
tags: [2020-12, anglechase, length, bet]
url: https://aops.com/community/p19444197

---

Let $A$, $B$, $C$, $D$ be four points
such that no three are collinear
and $D$ is not the orthocenter of triangle $ABC$.
Let $P$, $Q$, $R$ be the orthocenters of
$\triangle BCD$, $\triangle CAD$, $\triangle ABD$, respectively.
Suppose that lines $AP$, $BQ$, $CR$ are pairwise distinct
and are concurrent.
Show that the four points $A$, $B$, $C$, $D$ lie on a circle.

---

Let $T$ be the concurrency point,
and let $H$ be the orthocenter of $\triangle ABC$.
\begin{center}
\begin{asy}
pair A = dir(130);
pair B = dir(210);
pair C = dir(330);
pair D = dir(97);
pair P = B+C+D;
pair Q = C+A+D;
pair R = A+B+D;

filldraw(R--B--D--cycle, opacity(0.1)+lightcyan, palecyan);
filldraw(Q--A--C--cycle, opacity(0.1)+lightred, palered);
filldraw(C--B--D--cycle, opacity(0.1)+lightgreen, palegreen);

draw(R--foot(R,B,D), lightgrey+dashed);
draw(B--foot(B,D,R), lightgrey+dashed);
draw(D--foot(D,R,B), lightgrey+dashed);

draw(A--foot(A,C,Q), lightgrey+dashed);
draw(C--foot(C,Q,A), lightgrey+dashed);
draw(Q--foot(Q,A,C), lightgrey+dashed);

draw(C--foot(C,B,D), lightgrey+dashed);
draw(B--foot(B,D,C), lightgrey+dashed);
draw(D--foot(D,C,B), lightgrey+dashed);

draw(A--foot(A,B,C), lightgrey+dashed);
draw(B--foot(B,C,A), lightgrey+dashed);
draw(C--foot(C,A,B), lightgrey+dashed);

pair T = midpoint(A--P);
pair S = (A+B+C+D)/4;
pair O = 2*S-T;
/*
A' = foot A B C R270
T' = foot T B C R270
D' = foot D B C R300
O' = foot O B C R300
S' = foot S B C R270
T--Tp lightgrey dashed
S--Sp lightgrey dashed
O--Op lightgrey dashed
*/

draw(T--O, lightblue);

filldraw(A--B--C--cycle, opacity(0.1)+yellow, lightgrey);
draw(A--C, palered);

draw(A--Q, red+1);
draw(B--P, red+1);

draw(A--P, blue);
draw(B--Q, blue);
draw(C--R, blue);
pair H = A+B+C;
draw(D--H, lightblue);

dot("$A$", A, dir(A));
dot("$B$", B, dir(B));
dot("$C$", C, dir(C));
dot("$D$", D, dir(80));
dot("$P$", P, dir(285));
dot("$Q$", Q, dir(Q));
dot("$R$", R, dir(R));
dot("$T$", T, 1.8*dir(90));
dot("$S$", S, 1.8*dir(75));
dot("$O$", O, dir(315));
dot("$H$", H, dir(45));

/* TSQ Source:

A = dir 130
B = dir 210
C = dir 330
D = dir 97 R80
P = B+C+D R285
Q = C+A+D
R = A+B+D

R--B--D--cycle 0.1 lightcyan / palecyan
Q--A--C--cycle 0.1 lightred / palered
C--B--D--cycle 0.1 lightgreen / palegreen

R--foot(R,B,D) lightgrey dashed
B--foot(B,D,R) lightgrey dashed
D--foot(D,R,B) lightgrey dashed

A--foot(A,C,Q) lightgrey dashed
C--foot(C,Q,A) lightgrey dashed
Q--foot(Q,A,C) lightgrey dashed

C--foot(C,B,D) lightgrey dashed
B--foot(B,D,C) lightgrey dashed
D--foot(D,C,B) lightgrey dashed

A--foot(A,B,C) lightgrey dashed
B--foot(B,C,A) lightgrey dashed
C--foot(C,A,B) lightgrey dashed

T = midpoint A--P 1.8R90
S = (A+B+C+D)/4 1.8R75
O = 2*S-T R315

T--O lightblue

A--B--C--cycle 0.1 yellow / lightgrey
A--C palered

A--Q red+1
B--P red+1

A--P blue
B--Q blue
C--R blue
H = A+B+C R45
D--H lightblue

*/
\end{asy}
\end{center}

\begin{claim*}[Key claim]
  $T$ is the midpoint of $\ol{AP}$, $\ol{BQ}$, $\ol{CR}$, $\ol{DH}$,
  and $D$ is the orthocenter of $\triangle PQR$.
\end{claim*}
\begin{proof}
  Note that $\ol{AQ} \parallel \ol{BP}$,
  as both are perpendicular to $\ol{CD}$.
  Since lines $AP$ and $BQ$ are distinct,
  lines $AQ$ and $BP$ are distinct.

  By symmetric reasoning, we get that $AQCPBR$
  is a hexagon with \emph{opposite sides parallel}
  and \emph{concurrent diagonals} as $\ol{AP}$, $\ol{BQ}$, $\ol{CR}$ meet at $T$.
  This implies that the \emph{hexagon is centrally symmetric} about $T$;
  indeed \[ \frac{AT}{TP} = \frac{TQ}{BT} = \frac{CT}{TR} = \frac{TP}{AT} \]
  so all the ratios are equal to $+1$.

  % Thus, there exists a homothety $\Psi$ with center $T$ satisfying
  % $\psi(A) = P$ and $\psi(Q) = B$.
  % By repeating this with $\ol{BR}\parallel\ol{CQ}$
  % and $\ol{CP}\parallel\ol{AR}$,
  % we get $\psi(C) = R$ and $\psi(P) = A$.

  % As $A \neq P$, it follows that $\psi$ is reflection about $T$.
  Next, $\ol{PD} \perp \ol{BC} \parallel \ol{QR}$,
  so by symmetry we get $D$ is the orthocenter of $\triangle PQR$.
  This means that $T$ is the midpoint of $\ol{DH}$ as well.
\end{proof}
\begin{corollary*}
  The configuration is now symmetric:
  we have four points $A$, $B$, $C$, $D$,
  and their reflections in $T$ are
  four orthocenters $P$, $Q$, $R$, $H$.
\end{corollary*}

Let $S$ be the centroid of $\{A, B, C, D\}$,
and let $O$ be the reflection of $T$ in $S$.
We are ready to conclude:
\begin{claim*}
  $A$, $B$, $C$, $D$ are equidistant from $O$.
\end{claim*}
\begin{proof}
  Let $A'$, $O'$, $S'$, $T'$, $D'$
  be the projections of $A$, $O$, $S$, $T$, $D$
  onto line $BC$.

  Then $T'$ is the midpoint of $\ol{A'D'}$,
  so $S' = \tfrac14(A'+D'+B+C)$
  gives that $O'$ is the midpoint of $\ol{BC}$.

  Thus $OB = OC$ and we're done.
\end{proof}
