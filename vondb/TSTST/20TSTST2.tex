desc: Incenter of $GQM$
hardness: 25
source: TSTST 2020/2
tags: [2020-11, config, inversion, harmonic, good, gimel]
author: Zack Chroman, Daniel Liu
url: https://aops.com/community/p18933557

---

Let $ABC$ be a scalene triangle with incenter $I$.
The incircle of $ABC$ touches $\ol{BC}$, $\ol{CA}$, $\ol{AB}$
at points $D$, $E$, $F$, respectively.
Let $P$ be the foot of the altitude from $D$ to $\ol{EF}$,
and let $M$ be the midpoint of $\ol{BC}$.
The rays $AP$ and $IP$ intersect the circumcircle
of triangle $ABC$ again at points $G$ and $Q$, respectively.
Show that the incenter of triangle $GQM$ coincides with $D$.

---

Refer to the figure below.
\begin{center}
\begin{asy}
pair A = dir(130);
pair B = dir(210);
pair C = dir(330);
pair I = incenter(A, B, C);
pair D = foot(I, B, C);
pair E = foot(I, C, A);
pair F = foot(I, A, B);
pair P = foot(D, E, F);
pair Y = dir(270);
pair Q = extension(Y, D, I, P);
draw(A--B--C--cycle, lightblue);
pair T = extension(E, F, B, C);
filldraw(unitcircle, opacity(0.1)+lightcyan, blue);
draw(D--E--F--cycle, blue);
draw(circumcircle(A, E, F), deepgreen);
pair M = midpoint(B--C);
pair X = dir(90);
pair G = extension(T, Y, X, D);
draw(A--G, dotted);
draw(D--P, deepgreen);
// Q--Y red
// G--X red
// circumcircle Q P T 0.1 lightred / orange
// F--T--B lightblue
// X--Y dotted blue
filldraw(G--Q--M--cycle, opacity(0.1)+lightred, red);
draw(I--Q, deepgreen);

dot("$A$", A, dir(A));
dot("$B$", B, dir(B));
dot("$C$", C, dir(C));
dot("$I$", I, dir(270));
dot("$D$", D, dir(315));
dot("$E$", E, dir(20));
dot("$F$", F, dir(200));
dot("$P$", P, dir(60));
dot("$Q$", Q, dir(140));
dot("$M$", M, dir(300));
dot("$G$", G, dir(G));

/* TSQ Source:

A = dir 130
B = dir 210
C = dir 330
I = incenter A B C R270
D = foot I B C R315
E = foot I C A R20
F = foot I A B R200
P = foot D E F R60
Y := dir 270
Q = extension Y D I P R140
A--B--C--cycle lightblue
T := extension E F B C
unitcircle 0.1 lightcyan / blue
D--E--F--cycle blue
circumcircle A E F deepgreen
M = midpoint B--C R300
X := dir 90
G = extension T Y X D
A--G dotted
D--P deepgreen
// Q--Y red
// G--X red
// circumcircle Q P T 0.1 lightred / orange
// F--T--B lightblue
// X--Y dotted blue
G--Q--M--cycle 0.1 lightred / red
I--Q deepgreen

*/
\end{asy}
\end{center}

\begin{claim*}
  The point $Q$ is the Miquel point of $BFEC$.
  Also, $\ol{QD}$ bisects $\angle BQC$.
\end{claim*}
\begin{proof}
  Inversion around the incircle maps line $EF$ to $(AIEF)$
  and the nine-point circle of $\triangle DEF$
  to the circumcircle of $\triangle ABC$
  (as the midpoint of $\ol{EF}$ maps to $A$, etc.).
  This implies $P$ maps to $Q$; that is, $Q$ coincides with
  the second intersection of $(AFIE)$ with $(ABC)$.
  This is the claimed Miquel point.

  The spiral similarity mentioned then gives
  $\frac{QB}{BF} = \frac{QC}{CE}$, so $\ol{QD}$ bisects $\angle BQC$.
\end{proof}

\begin{remark*}
  The point $Q$ and its properties mentioned in the first claim
  have appeared in other references.
  See for example Canada 2007/5, ELMO 2010/6,
  HMMT 2016 T-10, USA TST 2017/2, USA TST 2019/6 for a few examples.
\end{remark*}

\begin{claim*}
  We have $(QG;BC) = -1$, so in particular $\ol{GD}$ bisects $\angle BGC$.
\end{claim*}
\begin{proof}
  Note that
  \[ -1 = (AI;EF) \overset{Q}{=} (\ol{AQ} \cap \ol{EF}, P; E, F)
    \overset{A}{=} (QG;BC). \]
  The last statement follows from Apollonian circle,
  or more bluntly $\frac{GB}{GC} = \frac{QB}{QC} = \frac{BD}{DC}$.
\end{proof}

Hence $\ol{QD}$ and $\ol{GD}$ are angle bisectors of $\angle BQC$ and $\angle BGC$.
However, $\ol{QM}$ and $\ol{QG}$ are isogonal in $\angle BQC$
(as median and symmedian), and similarly for $\angle BGC$, as desired.

---

\begin{remark*}
  [Alternate proof of claim avoiding harmonic bundles]
  Redefine $G'$ to be the other point on $(ABC)$
  satisfying $\frac{GB}{GC} = \frac{BD}{DC}$,
  so that $G'$ lies on line $DX$.
  In other words $G'$ is the other intersection
  of the Apollonian circle with $(ABC)$.
  Then \[ \dang DG'P = \dang DTP
    = \dang \left( \ol{BC}, \ol{EF} \right)
    = \dang \left( \ol{XY}, \ol{AIY} \right)
    = \dang XYA = \dang XG'A = \dang DG'A \]
  meaning $A$, $P$, $G'$ collinear, so $G' = G$.
\end{remark*}

To finish, since $QDMX$ and $GDMY$ are cyclic, we have
\begin{align*}
  \dang GQD &= \dang GQY = \dang GXY = \dang DXM = \dang DQM \\
  \dang MGD &= \dang MYD = \dang XYQ = \dang XGQ = \dang DGQ.
\end{align*}
