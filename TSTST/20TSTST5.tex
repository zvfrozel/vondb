desc: Stable sets
hardness: 25
source: TSTST 2020/5
tags: [2020-12, induct, grid, bet]
author: Ashwin Sah, Mehtaab Sawhney
url: https://aops.com/community/p19444403

---

Let $\NN^2$ denote the set of ordered pairs of positive integers.
A finite subset $S$ of $\NN^2$ is \emph{stable}
if whenever $(x,y)$ is in $S$,
then so are all points $(x',y')$ of $\NN^2$
with both $x' \leq x$ and $y' \leq y$.

Prove that if $S$ is a stable set,
then among all stable subsets of $S$
(including the empty set and $S$ itself),
at least half of them have an even number of elements.

---

The following inductive solution was given by Nikolai Beluhov.
We proceed by induction on $|S|$, with $|S| \le 1$ clear.

Suppose $|S| \ge 2$.
For any $p \in S$, let $R(p)$ denote
the stable rectangle with upper-right corner $p$.
We say such $p$ is \emph{pivotal} if $p + (1, 1) \notin S$
and $|R(p)|$ is even.

\begin{center}
\begin{asy}
  path R = box( (0,0), (6,3) );
  fill(R, palered);
  fill(shift(5,2)*unitsquare, lightcyan);
  label("$p$", (5.5,2.5));
  for (int i=0; i<=0; ++i) {
    draw(shift(i,5)*unitsquare);
  }
  for (int i=0; i<=1; ++i) {
    draw(shift(i,4)*unitsquare);
  }
  for (int i=0; i<=3; ++i) {
    draw(shift(i,3)*unitsquare);
  }
  for (int i=0; i<=7; ++i) {
    draw(shift(i,2)*unitsquare);
  }
  for (int i=0; i<=10; ++i) {
    draw(shift(i,1)*unitsquare);
  }
  for (int i=0; i<=10; ++i) {
    draw(shift(i,0)*unitsquare);
  }
  draw(R, heavyred+1.5);
\end{asy}
\end{center}

\begin{claim*}
  If $|S| \ge 2$, then a pivotal $p$ always exists.
\end{claim*}

\begin{proof}
  Consider the top row of $S$.
  \begin{itemize}
    \ii If it has length at least $2$,
    one of the two rightmost points in it is pivotal.
    \ii Otherwise, the top row has length $1$.
    Now either the top point or the point below it
    (which exists as $|S| \ge 2$) is pivotal.
    \qedhere
  \end{itemize}
\end{proof}

We describe how to complete the induction,
given some pivotal $p \in S$.
There is a partition
\[ S = R(p) \sqcup S_1 \sqcup S_2 \]
where $S_1$ and $S_2$ are the sets of points in $S$
above and to the right of $p$
(possibly empty).

\begin{claim*}
  The desired inequality holds for stable subsets containing $p$.
\end{claim*}
\begin{proof}
  Let $E_1$ denote the number of even stable subsets of $S_1$;
  denote $E_2$, $O_1$, $O_2$ analogously.
  The stable subsets containing $p$ are exactly $R(p) \sqcup T_1 \sqcup T_2$,
  where $T_1 \subseteq S_1$ and $T_2 \subseteq S_2$ are stable.

  Since $|R(p)|$ is even,
  exactly $E_1E_2 + O_1O_2$ stable subsets containing $p$ are even,
  and exactly $E_1O_2 + E_2O_1$ are odd.
  As $E_1 \ge O_1$ and $E_2 \ge O_2$ by inductive hypothesis,
  we obtain $E_1E_2 + O_1O_2 \ge E_1O_2 + E_2O_1$ as desired.
\end{proof}

By the inductive hypothesis, the desired inequality also holds
for stable subsets not containing $p$, so we are done.
