author: Ray Li
desc: Red and blue integers
source: TSTST 2018/6
tags: [2018-06, nice, find, genfunc, neatness, induct, parity, gimel]
hardness: 30
url: https://aops.com/community/p10570994

---

Let $S = \left\{ 1, \dots, 100 \right\}$,
and for every positive integer $n$ define
\[
  T_n = \left\{ (a_1, \dots, a_n) \in S^n
    \mid a_1 + \dots + a_n \equiv 0 \pmod{100} \right\}.
\]
Determine which $n$ have the following property:
if we color any $75$ elements of $S$ red,
then at least half of the $n$-tuples in $T_n$
have an even number of coordinates with red elements.

---

We claim this holds exactly for $n$ even.

\paragraph{First solution by generating functions.}
Define
\[
  R(x) = \sum_{s \text{ red}} x^s, \qquad
  B(x) = \sum_{s \text{ blue}} x^s.
\]
(Here ``blue'' means ``not-red'', as always.)
Then, the number of tuples in $T_n$ with exactly
$k$ red coordinates is exactly equal to
\[ \binom nk \cdot \frac 1{100} \sum_{\omega} R(\omega)^k B(\omega)^{n-k} \]
where the sum is over all $100$th roots of unity.
So, we conclude the number of tuples in $T_n$
with an even (resp odd) number of red elements is exactly
\begin{align*}
  X &= \frac1{100} \sum_{\omega} \sum_{k \text{ even}}
    \binom nk R(\omega)^k B(\omega)^{n-k} \\
  Y &= \frac1{100} \sum_{\omega} \sum_{k \text{ odd}}
    \binom nk R(\omega)^k B(\omega)^{n-k} \\
  \implies X-Y &= \frac 1{100} \sum_{\omega}
    \left( B(\omega)-R(\omega) \right)^n \\
  &= \frac 1{100} \left[
     \left( B(1)-R(1) \right)^n
     + \sum_{\omega \neq 1} (2B(\omega))^n \right] \\
  &= \frac 1{100} \left[ \left( B(1)-R(1) \right)^n - (2B(1))^n
    + 2^n \sum_{\omega} B(\omega)^n \right] \\
  &= \frac 1{100} \left[ \left( B(1)-R(1) \right)^n - (2B(1))^n \right]
    + 2^n Z \\
  &= \frac 1{100} \left[ \left( -50 \right)^n - 50^n \right] + 2^n Z.
\end{align*}
where
\[ Z \coloneqq \frac 1{100} \sum_{\omega} B(\omega)^n \ge 0 \]
counts the number of tuples in $T_n$ which are all blue.
Here we have used the fact that $B(\omega)+R(\omega)=0$
for $\omega \neq 1$.

We wish to show $X-Y \ge 0$ holds for $n$ even,
but may fail when $n$ is odd.
This follows from two remarks:
\begin{itemize}
  \item If $n$ is even, then $X-Y = 2^n Z \ge 0$.
  \item If $n$ is odd,
  then if we choose the coloring for which
  $s$ is red if and only if $s \not\equiv 2 \pmod 4$;
  we thus get $Z = 0$.
  Then $X-Y = -\frac 2{100} \cdot 50^n < 0$.
\end{itemize}

\paragraph{Second solution by strengthened induction and random coloring.}
We again prove that $n$ even work.
Let us define
\[
  T_n(a) = \left\{ (a_1, \dots, a_n) \in S^n
    \mid a_1 + \dots + a_n \equiv a \pmod{100} \right\}.
\]
Also, call an $n$-tuple good if it has an even number of red elements.
We claim that $T_n(a)$ also has at least 50\% good tuples, by induction.

This follows by induction on $n \ge 2$.
Indeed, the base case $n = 2$ can be checked by hand,
since $T_2(a) = \{ (x, a-x) \mid x \in S \}$.
With the stronger claim, one can check the case $n=2$ manually
and proceed by induction to go from $n-2$ to $n$, noting that
\[ T_{n}(a) =
  \bigsqcup_{b+c=a} T_{n-2}(b) \oplus T_2(c) \]
where $\oplus$ denotes concatenation of tuples, applied set-wise.
The concatenation of an $(n-2)$-tuple and $2$-tuple is good
if and only if both or neither are good.
Thus for each $b$ and $c$,
if the proportion of $T_{n-2}(b)$ which is good is $p \ge \half$
and the proportion of $T_2(c)$ which is good is $q \ge \half$,
then the proportion of $T_{n-2}(b) \oplus T_2(c)$ which is good
is $pq + (1-p)(1-q) \ge \half$, as desired.
Since each term in the union has at least half the tuples good,
all of $T_n(a)$ has at least half the tuples good, as desired.

It remains to fail all odd $n$.
We proceed by a suggestion of Yang Liu and Ankan Bhattacharya
by showing that if we pick the $75$ elements \emph{randomly},
then any particular tuple in $S^n$ has strictly less than 50\% chance of being good.
This will imply (by linearity of expectation)
that $T_n$ (or indeed any subset of $S^n$) will, for some coloring,
have less than half good tuples.

Let $(a_1, \dots, a_n)$ be such an $n$-tuple.
If any element appears in the tuple more than once,
keep \emph{discarding pairs} of that element until there are zero or one;
this has no effect on the good-ness of the tuple.
If we do this, we obtain an $m$-tuple $(b_1, \dots, b_m)$
with no duplicated elements where $m \equiv n \equiv 1 \pmod 2$.
Now, the probability that any element is red is $\frac34$,
so the probability of being good is
\begin{align*}
  \sum_{k \text{ even}}^m \binom{m}{k} \left( \frac 34 \right)^k
  \left( -\frac 14 \right)^{m-k}
  &= \half\left[ \left( \frac34 + \frac 14 \right)^m
    - \left( \frac 34 - \frac 14 \right)^m \right] \\
  &= \half \left[ 1 - \left( \frac12 \right)^m \right] < \half.
\end{align*}

\begin{remark*}
  [Adam Hesterberg]
  Here is yet another proof that $n$ even works.
  Group elements of $T_n$ into equivalence classes
  according to the $n/2$ sums of pairs of consecutive elements
  (first and second, third and fourth, \dots).
  For each such pair sum, there are at least as many monochrome pairs
  with that sum as nonmonochrome ones, since every nonmonochrome pair uses one of the 25 non-reds.
  The monochromaticity of the pairs is independent.

  If $p_i \le \frac12$ is the probability that the $i$th pair is nonmonochrome,
  then the probability that $k$ pairs are nonmonochrome
  is the coefficient of $x^k$ in $f(x) = \prod_i(xp_i+(1-p_i))$.
  Then the probability that evenly many pairs are nonmonochrome
  (and hence that evenly many coordinates are red)
  is the sum of the coefficients of even powers of $x$ in $f$,
  which is $(f(1) + f(-1))/2 = (1 + \prod_i(1-2p_i))/2 \ge \frac12$,
  as desired.
\end{remark*}
