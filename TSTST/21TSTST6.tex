author: Nikolai Beluhov
desc: Bicentric triangle pair
hardness: 50
source: TSTST 2021/6
tags: [2021-12, yod]
url: https://aops.com/community/p23864189

---

Triangles $ABC$ and $DEF$ share circumcircle $\Omega$ and incircle $\omega$
so that points $A$, $F$, $B$, $D$, $C$, and $E$ occur in this order along $\Omega$.
Let $\Delta_A$ be the triangle formed by lines $AB$, $AC$, and $EF$,
and define triangles $\Delta_B$, $\Delta_C, \dots, \Delta_F$ similarly.
Furthermore, let $\Omega_A$ and $\omega_A$ be the circumcircle and incircle
of triangle $\Delta_A$, respectively, and define circles
$\Omega_B$, $\omega_B, \dots, \Omega_F$, $\omega_F$ similarly.
\begin{enumerate}[(a)]
  \item Prove that the two common external tangents to circles $\Omega_A$ and $\Omega_D$
    and the two common external tangents to circles $\omega_A$ and $\omega_D$
    are either concurrent or pairwise parallel.

  \item Suppose that these four lines meet at point $T_A$,
    and define points $T_B$ and $T_C$ similarly.
    Prove that points $T_A$, $T_B$, and $T_C$ are collinear.
\end{enumerate}

---

\begin{center}
\begin{asy}
import geometry;

linemargin = 0;

real R = 120;
pair O = (0, 0);

pair A = rotate(35, O)*(R, 0);
pair B = rotate(210, O)*(R, 0);
pair C = rotate(330, O)*(R, 0);

circle omega = incircle(A, B, C);
circle k = circle((point) O, R);

pair D = rotate(255, O)*(R, 0);
line[] ts = tangents(omega, (point) D);
pair E = 2*(projection(ts[0])*O) - D;
pair F = 2*(projection(ts[1])*O) - D;

pair K = extension(A, B, E, F);
pair L = extension(A, B, F, D);
pair M = extension(B, C, F, D);
pair N = extension(B, C, D, E);
pair P = extension(C, A, D, E);
pair Q = extension(C, A, E, F);

triangle dA = triangle(Q, A, K);
triangle dB = triangle(L, B, M);
triangle dC = triangle(N, C, P);
triangle dD = triangle(M, D, N);
triangle dE = triangle(P, E, Q);
triangle dF = triangle(K, F, L);

circle oA = incircle(dA);
circle oB = incircle(dB);
circle oC = incircle(dC);
circle oD = incircle(dD);
circle oE = incircle(dE);
circle oF = incircle(dF);

circle kA = circumcircle(dA);
circle kB = circumcircle(dB);
circle kC = circumcircle(dC);
circle kD = circumcircle(dD);
circle kE = circumcircle(dE);
circle kF = circumcircle(dF);

point similitude(circle a, circle b) {
  return extension(a.C, b.C, a.C + (0, a.r), b.C + (0, b.r));
}

point TA = similitude(oA, oD);
point TB = similitude(oB, oE);
point TC = similitude(oC, oF);



draw(A--B--C..cycle);
draw(D--E--F..cycle);
draw(omega);
draw(k);

pen otp = red+opacity(0.5);

draw(TA--projection(tangents(oD, TA)[0])*oD.C, otp);
draw(TA--projection(tangents(oD, TA)[1])*oD.C, otp);
draw(TB--projection(tangents(oB, TB)[0])*oB.C, otp);
draw(TB--projection(tangents(oB, TB)[1])*oB.C, otp);
draw(TC--projection(tangents(oF, TC)[0])*oF.C, otp);
draw(TC--projection(tangents(oF, TC)[1])*oF.C, otp);
//draw(tangents(oA, TA), otp);
//draw(tangents(oB, TB), otp);
//draw(tangents(oC, TC), otp);

pen ktp = lightolive+opacity(0.5);

draw(TA--projection(tangents(kD, TA)[0])*kD.C, ktp);
draw(TA--projection(tangents(kD, TA)[1])*kD.C, ktp);
draw(TB--projection(tangents(kB, TB)[0])*kB.C, ktp);
draw(TB--projection(tangents(kB, TB)[1])*kB.C, ktp);
draw(TC--projection(tangents(kF, TC)[0])*kF.C, ktp);
draw(TC--projection(tangents(kF, TC)[1])*kF.C, ktp);
//draw(tangents(kA, TA), ktp);
//draw(tangents(kB, TB), ktp);
//draw(tangents(kC, TC), ktp);

pen op = blue;

draw(oA, op);
draw(oB, op);
draw(oC, op);
draw(oD, op);
draw(oE, op);
draw(oF, op);

pen kp = heavygreen;

draw(kA, kp);
draw(kB, kp);
draw(kC, kp);
draw(kD, kp);
draw(kE, kp);
draw(kF, kp);

draw(line(TA, TB), 1.0 + heavymagenta);



dot(A);
dot(B);
dot(C);
dot(D);
dot(E);
dot(F);
dot(TA);
dot(TB);
dot(TC);



label("$A$", A, dir(A));
label("$B$", B, dir(B));
label("$C$", C, dir(C));
label("$D$", D, dir(D));
label("$E$", E, dir(E));
label("$F$", F, dir(F));
label("$T_A$", TA, dir((1, 0)));
label("$T_B$", TB, dir((1, 0)));
label("$T_C$", TC, dir((1, 0)));

\end{asy}
\end{center}

Let $I$ and $r$ be the center and radius of $\omega$, and let $O$ and $R$ be the
center and radius of $\Omega$. Let $O_A$ and $I_A$ be the circumcenter and
incenter of triangle $\Delta_A$, and define $O_B$, $I_B, \dots, I_F$
similarly. Let $\omega$ touch $EF$ at $A_1$, and define $B_1$, $C_1, \dots,
F_1$ similarly.

\paragraph{Part (a).} All solutions to part (a) will prove the stronger claim
that
\[(\Omega_A\cup \omega_A)\sim (\Omega_D\cup \omega_D).\]
The four lines will concur at the homothetic center of these figures (possibly
at infinity).

\subparagraph{Solution 1 (author)} Let the second tangent to $\omega$ parallel to $EF$ meet lines $AB$ and $AC$ at $P$ and $Q$, respectively, and let the second tangent to $\omega$ parallel to $BC$ meet lines $DE$ and $DF$ at $R$ and $S$, respectively. Furthermore, let $\omega$ touch $PQ$ and $RS$ at $U$ and $V$, respectively.

Let $h$ be inversion with respect to $\omega$. Then $h$ maps $A$, $B$, and $C$ onto the midpoints of the sides of triangle $D_1E_1F_1$. So $h$ maps $k$ onto the Euler circle of triangle $D_1E_1F_1$.

Similarly, $h$ maps $k$ onto the Euler circle of triangle $A_1B_1C_1$.
Therefore, triangles $A_1B_1C_1$ and $D_1E_1F_1$ share a common nine-point
circle $\gamma$. Let $K$ be its center; its radius equals $\frac{1}{2}r$.

Let $H$ be the reflection of $I$ in $K$. Then $H$ is the common orthocenter of triangles $A_1B_1C_1$ and $D_1E_1F_1$.

Let $\gamma_U$ of center $K_U$ and radius $\frac{1}{2}r$ be the Euler circle of
triangle $UE_1F_1$, and let $\gamma_V$ of center $K_V$ and radius $\frac{1}{2}r$ be
the Euler circle of triangle $VB_1C_1$.

Let $H_U$ be the orthocenter of triangle $UE_1F_1$. Since quadrilateral
$D_1E_1F_1U$ is cyclic, vectors $\overrightarrow{HH_U}$ and
$\overrightarrow{D_1U}$ are equal. Consequently,
$\overrightarrow{KK_U}=\frac{1}{2}\overrightarrow{D_1U}$. Similarly,
$\overrightarrow{KK_V}=\frac{1}{2}\overrightarrow{A_1V}$.

Since both of $A_1U$ and $D_1V$ are diameters in $\omega$, vectors
$\overrightarrow{D_1U}$ and $\overrightarrow{A_1V}$ are equal. Therefore, $K_U$
and $K_V$ coincide, and so do $\gamma_U$ and $\gamma_V$.

As above, $h$ maps $\gamma_U$ onto the circumcircle of triangle $APQ$ and
$\gamma_V$ onto the circumcircle of triangle $DRS$. Therefore, triangles $APQ$
and $DRS$ are inscribed inside the same circle $\Omega_{AD}$.

Since $EF$ and $PQ$ are parallel, triangles $\Delta_A$ and $APQ$ are homothetic,
and so are figures $\Omega_A \cup \omega_A$ and $\Omega_{AD}\cup \omega$.
Consequently, we have
\[(\Omega_A\cup \omega_A)\sim (\Omega_{AD}\cup \omega)\sim (\Omega_D\cup
\omega_D),\]
which solves part (a).

%If figures $\omega_A \cup k_A$ and $\omega_D \cup k_D$ are noncongruent, then the two common external tangents to $\omega_A$ and $\omega_D$ and the two common external tangents to $k_A$ and $k_D$ meet at their homothetic center. Otherwise, the two common external tangents to $\omega_A$ and $\omega_D$ and the two common external tangents to $k_A$ and $k_D$ are pairwise parallel.

\subparagraph{Solution 2 (Michael Ren)} As in the previous solution, let the
second tangent to $\omega$ parallel to $EF$ meet $AB$ and $AC$ at $P$ and $Q$,
respectively. Let $(APQ)$ meet $\Omega$ again at $D'$, so that $D'$ is the
Miquel point of $\{AB, AC, BC, PQ\}$. Since the quadrilateral formed by these
lines has incircle $\omega$, it is classical that $D'I$ bisects $\angle PD'C$
and $BD'Q$ (e.g. by DDIT).

Let $\ell$ be the tangent to $\Omega$ at $D'$ and $D'I$ meet $\Omega$ again at
$M$. We have
\[
  \measuredangle(\ell, D'B) = \measuredangle D'CB = \measuredangle D'QP =
  \measuredangle(D'Q, EF).
\]
Therefore $D'I$ also bisects the angle between $\ell$ and the line parallel to
$EF$ through $D'$. This means that $M$ is one of the arc midpoints of $EF$.
Additionally, $D'$ lies on arc $BC$ not containing $A$, so $D'=D$.

Similarly, letting the second tangent to $\omega$ parallel to $BC$ meet $DE$ and
$DF$ again at $R$ and $S$, we have $ADRS$ cyclic.

\begin{lemma*}
  There exists a circle $\Omega_{AD}$ tangent to $\Omega_A$ and $\Omega_D$ at
  $A$ and $D$, respectively.
\end{lemma*}
\begin{proof}
  (This step is due to Ankan Bhattacharya.) It is equivalent to have
  $\measuredangle OAO_A = \measuredangle O_DDO$. Taking isogonals with respect
  to the shared angle of $\triangle ABC$ and $\Delta_A$, we see that
  \[\measuredangle OAO_A = \measuredangle(\perp EF, \perp BC) = \measuredangle
  (EF, BC).\]
  (Here, $\perp EF$ means the direction perpendicular to $EF$.) By symmetry,
  this is equal to $\measuredangle O_DDO$.
\end{proof}

The circle $(ADPQ)$ passes through $A$ and $D$, and is tangent to $\Omega_A$ by
homothety. Therefore it coincides with $\Omega_{AD}$, as does $(ADRS)$. Like the
previous solution, we finish with
\[(\Omega_A\cup \omega_A)\sim (\Omega_{AD}\cup \omega)\sim (\Omega_D\cup
\omega_D).\]

\subparagraph{Solution 3 (Andrew Gu)}
Construct triangles homothetic to $\Delta_A$ and $\Delta_D$ (with positive ratio) which
have the same circumcircle; it suffices to show that these copies have the same
incircle as well. Let $O$ be the center of this common circumcircle, which we
take to be the origin, and $M_{XY}$ denote the point on the circle such that the
tangent at that point is parallel to line $XY$ (from the two possible choices,
we make the choice that corresponds to the arc midpoint on $\Omega$, e.g. $M_{AB}$
should correspond to the arc midpoint on the internal angle bisector of $ACB$).
The left diagram below shows the original triangles $ABC$ and $DEF$, while the
right diagram shows the triangles homothetic to $\Delta_A$ and $\Delta_D$.

\begin{center}
  \begin{asy}
    size(6cm);
    pair O = origin;
    pair Mbc = dir(-90);
    pair Mca = dir(40);
    pair Mab = dir(160);
    pair A = -Mca*Mab/Mbc;
    pair B = -Mab*Mbc/Mca;
    pair C = -Mbc*Mca/Mab;
    pair I = Mbc+Mca+Mab;
    pair Mef = dir(75);
    pair D = 2*foot(O, I, Mef)-Mef;
    pair X = midpoint(D--I);
    pair Y = X+5*dir(90)*(X-D);
    pair Mfd = intersectionpoints(X--Y, unitcircle)[0];
    pair Mde = 2*foot(O, X, Mfd)-Mfd;
    pair E = -Mde*Mef/Mfd;
    pair F = -Mef*Mfd/Mde;

    draw(unitcircle);
    draw(A--B--C--cycle, red);
    draw(D--E--F--cycle, blue);
    draw(incircle(A, B, C));

    string[] names = {"$A$", "$B$", "$C$", "$M_{BC}$", "$M_{CA}$", "$M_{AB}$", "$I$", "$D$", "$E$", "$F$", "$M_{FD}$", "$M_{DE}$", "$M_{EF}$"};
    pair[] pts = {A, B, C, Mbc, Mca, Mab, I, D, E, F, Mfd, Mde, Mef};
    pair[] labels = {A, B, C, Mbc, Mca, Mab, I, D, E, F, Mfd, Mde, Mef};
    for(int i=0; i<names.length; ++i){
      dot(names[i], pts[i], dir(labels[i]));
    }
  \end{asy}
  \begin{asy}
    size(6cm);
    pair O = origin;
    pair Mbc = dir(-90);
    pair Mca = dir(40);
    pair Mab = dir(160);
    pair Mef = dir(75);
    pair oldI = Mbc+Mca+Mab;
    pair I = -Mef+Mca+Mab;
    pair oldD = 2*foot(O, oldI, Mef)-Mef;
    pair X = midpoint(oldD--oldI);
    pair Y = X+5*dir(90)*(X-oldD);
    pair Mfd = intersectionpoints(X--Y, unitcircle)[0];
    pair Mde = 2*foot(O, X, Mfd)-Mfd;
    pair A = Mca*Mab/Mef;
    pair B = Mab*Mef/Mca;
    pair C = Mef*Mca/Mab;
    pair D = Mfd*Mde/Mbc;
    pair E = Mde*Mbc/Mfd;
    pair F = Mbc*Mfd/Mde;

    draw(unitcircle);
    draw(C--A--B, red);
    draw(B--C, blue);
    draw(F--D--E, blue);
    draw(E--F, red);
    draw(incircle(A, B, C));

    string[] names = {" ", " ", " ", "$M_{BC}$", "$M_{CA}$", "$M_{AB}$", "$I$", " ", " ", " ", "$M_{FD}$", "$M_{DE}$", "$M_{EF}$"};
    pair[] pts = {A, B, C, Mbc, Mca, Mab, I, D, E, F, Mfd, Mde, Mef};
    pair[] labels = {A, B, C, Mbc, Mca, Mab, I, D, E, F, Mfd, Mde, Mef};
    for(int i=0; i<names.length; ++i){
      dot(names[i], pts[i], dir(labels[i]));
    }
  \end{asy}
\end{center}

Using the fact that the incenter is the orthocenter of the arc midpoints, the
incenter of $\Delta_A$ in this reference frame is $M_{AB}+M_{AC}-M_{EF}$ and the
incenter of $\Delta_D$ in this reference frame is $M_{DE}+M_{DF}-M_{BC}$. Since
$ABC$ and $DEF$ share a common incenter, we have
\[M_{AB}+M_{BC}+M_{CA}=M_{DE}+M_{EF}+M_{FD}.\]
Thus the copies of $\Delta_A$ and $\Delta_D$ have the same incenter, and
therefore the same incircle as well (Euler's theorem determines the inradius).

\paragraph{Part (b).}
We present several solutions for this part of the problem. Solutions 3 and 4
require solving part (a) first, while the others do not. Solutions 1, 4, and 5
define $T_A$ solely as the exsimilicenter of $\omega_A$ and $\omega_D$, whereas
solutions 2 and 3 define $T_A$ solely as the exsimilicenter of $\Omega_A$ and
$\Omega_D$.

\subparagraph{Solution 1 (author)}

By Monge's theorem applied to $\omega$, $\omega_A$, and $\omega_D$,
points $A$, $D$, and $T_A$ are collinear. Therefore, $T_A= AD\cap I_AI_D$.

Let $p$ be pole-and-polar correspondence with respect to $\omega$. Then $p$ maps
$A$ onto line $E_1F_1$ and $D$ onto line $B_1C_1$. Consequently, $p$ maps line
$AD$ onto $X_A=B_1C_1\cap E_1F_1$.

Furthermore, $p$ maps the line that bisects the angle formed by lines $AB$ and
$EF$ and does not contain $I$ onto the midpoint of segment $A_1F_1$. Similarly,
$p$ maps the line that bisects the angle formed by lines $AC$ and $EF$ and does
not contain $I$ onto the midpoint of segment $A_1E_1$. So $p$ maps $I_A$ onto
the midline of triangle $A_1E_1F_1$ opposite $A_1$. Similarly, $p$ maps $I_D$
onto the midline of triangle $D_1B_1C_1$ opposite $D_1$. Consequently, $p$ maps
line $I_AI_D$ onto the intersection point $Y_A$ of this pair of midlines, and
$p$ maps $T_A$ onto line $X_AY_A$.

As in the solution to part (a), let $H$ be the common orthocenter of triangles
$A_1B_1C_1$ and $D_1E_1F_1$. Let $H_A$ be the foot of the altitude from $A_1$ in
triangle $A_1B_1C_1$ and let $H_D$ be the foot of the altitude from $D_1$ in
triangle $D_1E_1F_1$. Furthermore, let $L_A=HA_1\cap E_1F_1$ and $L_D=HD_1\cap
B_1C_1$.

Since the reflection of $H$ in line $B_1C_1$ lies on $\omega$, $A_1H \cdot HH_A$ equals half the power of $H$ with respect to $\omega$. Similarly, $D_1H \cdot HH_D$ equals half the power of $H$ with respect to $\omega$.

Then $A_1H \cdot HH_A = D_1H \cdot HH_D$ and $A_1HH_D\sim D_1HH_A$. Since
$\angle HH_DL_A = 90^\circ = \angle HH_AL_D$, figures $A_1HH_DL_A$ and
$D_1HH_AL_D$ are similar as well. Consequently,
\[
  \frac{HL_A}{L_AA_1}=\frac{HL_D}{L_DD_1}=s
\]
as a signed ratio.
%$L_A$ divides segment $HA_1$ in
%the same signed ratio $s$ as $L_D$ divides segment $HD_1$.

Let the line through $A_1$ parallel to $E_1F_1$ and the line through $D_1$
parallel to $B_1C_1$ meet at $Z_A$. Then $HX_A/X_AZ_A=s$
%$X_A$ divides segment $HZ_A$ in signed ratio $s$
and $Y_A$ is the midpoint of segment $X_AZ_A$. Therefore, $H$ lies on
line $X_AY_A$. This means that $T_A$ lies on the polar of $H$ with respect to
$\omega$, and by symmetry so do $T_B$ and $T_C$.


\subparagraph{Solution 2 (author)}

As in the first solution to part (a), let
$h$ be inversion with respect to $\omega$, let $\gamma$ of center $K$ be the common
Euler circle of triangles $A_1B_1C_1$ and $D_1E_1F_1$, and let $H$ be their
common orthocenter.

Again as in the solution to part (a), $h$ maps $\Omega_A$ onto the nine-point
circle $\gamma_A$ of triangle $A_1E_1F_1$ and $\Omega_D$ onto the nine-point
circle $\gamma_D$ of triangle $D_1B_1C_1$.

Let $K_A$ and $K_D$ be the centers of $\gamma_A$ and $\gamma_D$, respectively,
and let $\ell_A$ be the perpendicular bisector of segment $K_AK_D$. Since
$\gamma_A$ and $\gamma_D$ are congruent (both of them are of radius
$\frac{1}{2}r$), they are reflections of each other across $\ell_A$.

Inversion $h$ maps the two common external tangents of $\Omega_A$ and $\Omega_D$
onto the two circles $\alpha$ and $\beta$ through $I$ that are tangent to both
of $\gamma_A$ and $\gamma_D$ in the same way. (That is, either internally to both or
externally to both.) Consequently, $\alpha$ and $\beta$ are reflections of each
other in $\ell_A$ and so their second point of intersection $S_A$, which $h$
maps $T_A$ onto, is the reflection of $I$ in $\ell_A$.

Define $\ell_B$, $\ell_C$, $S_B$, and $S_C$ similarly. Then $S_B$ is the reflection of $I$ in $\ell_B$ and $S_C$ is the reflection of $I$ in $\ell_C$.

As in the solution to part (a),
$\overrightarrow{KK_A}=\frac{1}{2}\overrightarrow{D_1A_1}$ and
$\overrightarrow{KK_D}=\frac{1}{2}\overrightarrow{A_1D_1}$. Consequently, $K$ is
the midpoint of segment $K_AK_D$ and so $K$ lies on $\ell_A$. Similarly, $K$
lies on $\ell_B$ and $\ell_C$.

Therefore, all four points $I$, $S_A$, $S_B$, and $S_C$ lie on the circle of center $K$ that contains $I$. (This is also the circle on diameter $IH$.) Since points $S_A$, $S_B$, and $S_C$ are concyclic with $I$, their images $T_A$, $T_B$, and $T_C$ under $h$ are collinear, and the solution is complete.

\subparagraph{Solution 3 (Ankan Bhattacharya)}

From either of the first two solutions to part (a), we know that there is a
circle $\Omega_{AD}$ passing through $A$ and $D$ which is (internally) tangent
to $\Omega_A$ and $\Omega_D$. By Monge's theorem applied to $\Omega_A,
\Omega_D$, and $\Omega_{AD}$, it follows that $A, D$, and $T_A$ are collinear.

The inversion at $T_A$ swapping $\Omega_A$ with $\Omega_D$ also swaps $A$ with
$D$ because $T_A$ lies on $AD$ and $A$ is not homologous to $D$. Let $\Omega_A$
and $\Omega_D$ meet $\Omega$ again at $L_A$ and $L_D$. Since $ADL_AL_D$ is
cyclic, the same inversion at $T_A$ also swaps $L_A$ and $L_D$, so $T_A=AD\cap
L_AL_D$.

By \href{https://aops.com/community/c6h598547p3551881}{Taiwan TST
2014}, $L_A$ and $L_D$ are the tangency points of the $A$-mixtilinear and
$D$-mixtilinear incircles, respectively, with $\Omega$. Therefore $AL_A\cap
DL_D$ is the exsimilicenter $X$ of $\Omega$ and $\omega$. Then $T_A, T_B,$ and
$T_C$ all lie on the polar of $X$ with respect to $\Omega$.

\subparagraph{Solution 4 (Andrew Gu)}

We show that $T_A$ lies on the radical axis of the point circle at $I$ and
$\Omega$, which will solve the problem.  Let $I_A$ and $I_D$ be the centers of
$\omega_A$ and $\omega_D$ respectively. By the Monge's theorem applied to
$\omega$, $\omega_A$, and $\omega_D$, points $A$, $D$, and $T_A$ are collinear.
Additionally, these other triples are collinear: $(A, I_A, I), (D, I_D, I),
(I_A, I_D, T)$.  By Menelaus's theorem, we have

\[\frac{T_AD}{T_AA}=\frac{I_AI}{I_AA}\cdot\frac{I_DD}{I_DI}.\]
If $s$ is the length of the side opposite $A$ in $\Delta_A$, then we compute
\begin{align*}
  \frac{I_AI}{I_AA} &=\frac{s/\cos(A/2)}{r_A/\sin(A/2)} \\
                    &=\frac{2R_A\sin(A)\sin(A/2)}{\cos(A/2)} \\
                    &=\frac{4R_A\sin^2(A/2)}{r_A} \\
                    &=\frac{4R_Ar^2}{r_AAI^2}.
\end{align*}
From part (a), we know that $\frac{R_A}{r_A}=\frac{R_D}{r_D}$.
Therefore, doing a similar calculation for $\frac{I_DD}{I_DI}$, we get
\begin{align*}
  \frac{T_AD}{T_AA} &=\frac{I_AI}{I_AA}\cdot\frac{I_DD}{I_DI} \\
                    &=\frac{4R_Ar^2}{r_AAI^2}\cdot \frac{r_DDI^2}{4R_Dr^2} \\
                    &=\frac{DI^2}{AI^2}.
\end{align*}
Thus $T_A$ is the point where the tangent to $(AID)$ at $I$ meets $AD$ and
$T_AI^2=T_AA\cdot T_AD$. This shows what we claimed at the start.

\subparagraph{Solution 5 (Ankit Bisain)}

As in the previous solution, it suffices to show that $\frac{I_AI}{AI_A}\cdot
\frac{DI_D}{I_DI} = \frac{DI^2}{AI^2}$. Let $AI$ and $DI$ meet $\Omega$ again at
$M$ and $N$, respectively. Let $\ell$ be the line parallel to $BC$ and tangent
to $\omega$ but different from $BC$. Then
\[
  \frac{DI_D}{I_DI}=\frac{d(D, BC)}{d(BC, \ell)} = \frac{DB\cdot DC/2R}{2r} =
  \frac{MI^2-MD^2}{4Rr}.
\]
%(One way to prove $DB\cdot DC=MI^2-MD^2$ is to let $CD$ meet $(BIC)$ again at
%$B'$, show that $DB=DB'$, and apply power of a point.)
Since $IDM\sim IAN$, we have
\[\frac{DI_D}{I_DI}\cdot \frac{I_AI}{AI_A} =
\frac{MI^2-MD^2}{NI^2-NA^2}=\frac{DI^2}{AI^2},\]
as desired.

%\begin{remark}
%The proof of part (a) also shows that lines $I_AO_A$ and $I_DO_D$ are parallel, and similarly for lines $I_BO_B$ and $I_EO_E$ and lines $I_CO_C$ and $I_FO_F$.
%\end{remark}
%
%\bigskip
%
%\begin{remark}
%All proofs of part (b) also show that lines $IO$ and $T_AT_BT_C$ are perpendicular. (Since $h$ maps $k$ onto $e$, the center $K$ of $e$ lies on line $IO$. Therefore, the reflection $H$ of $I$ in $K$ lies on line $IO$ as well.)
%\end{remark}

\begin{remark*}[Author comments on generalization of part (b) with a
  circumscribed hexagram]
Let triangles $ABC$ and $DEF$ be circumscribed about the same circle $\omega$ so that they form a hexagram. However, we do not require anymore that they are inscribed in the same circle.

Define circles $\Omega_A$, $\omega_A, \dots, \omega_F$ as in the problem.
Let $T^\text{Circ}_A$ be the intersection point of the two common external
tangents to circles $\Omega_A$ and $\Omega_D$, and define points $T^\text{Circ}_B$
and $T^\text{Circ}_C$ similarly. Also let $T^\text{In}_A$ be the intersection
point of the two common external tangents to circles $\omega_A$ and $\omega_D$, and define
points $T^\text{In}_B$ and $T^\text{In}_C$ similarly.

Then points $T^\text{Circ}_A$, $T^\text{Circ}_B$, and $T^\text{Circ}_C$ are collinear
and points $T^\text{In}_A$, $T^\text{In}_B$, and $T^\text{In}_C$ are also
collinear.

The second solution to part (b) of the problem works also for the circumcircles
part of the generalisation. To see that segments $K_AK_D$, $K_BK_E$, and
$K_CK_F$ still have a common midpoint, let $M$ be the centroid of points $A$,
$B$, $C$, $D$, $E$, and $F$. Then the midpoint of segment $K_AK_D$ divides
segment $OM$ externally in ratio $3 : 1$, and so do the other two midpoints as
well.

For the incircles part of the generalisation, we start out as in the first
solution to part (b) of the problem, and eventually we reduce everything to the
following:

\emph{Let points $A_1$, $B_1$, $C_1$, $D_1$, $E_1$, and $F_1$ lie on circle $\omega$. Let lines $B_1C_1$ and $E_1F_1$ meet at point $X_A$, let the line through $A_1$ parallel to $B_1C_1$ and the line through $D_1$ parallel to $E_1F_1$ meet at point $Z_A$, and define points $X_B$, $Z_B$, $X_C$, and $Z_C$ similarly. Then lines $X_AZ_A$, $X_BZ_B$, and $X_CZ_C$ are concurrent.}

Take $\omega$ as the unit circle and assign complex numbers $u$, $v$, $w$, $x$, $y$, and $z$ to points $A_1$, $F_1$, $B_1$, $D_1$, $C_1$, and $E_1$, respectively, so that when we permute $u$, $v$, $w$, $x$, $y$, and $z$ cyclically the configuration remains unchanged. Then by standard complex bash formulas we obtain that each two out of our three lines meet at $\varphi/\psi$, where \[\varphi = \sum_\text{Cyc} u^2vw(wx - wy + xy)(y - z)\] and \[\psi = {} - u^2w^2y^2 - v^2x^2z^2 - 4uvwxyz + \sum_\text{Cyc} u^2(vwxy - vwxz + vwyz - vxyz + wxyz).\]

(But the calculations were too difficult for me to do by hand, so I used SymPy.)
\end{remark*}

\bigskip

\begin{remark*}[Author comments on generalization of part (b) with an
  inscribed hexagram]
Let triangles $ABC$ and $DEF$ be inscribed inside the same circle $\Omega$ so that they form a hexagram. However, we do not require anymore that they are circumscribed about the same circle.

Define points $T^\text{Circ}_A$, $T^\text{Circ}_B$, \dots, $T^\text{In}_C$ as
in the previous remark. It looks like once again points $T^\text{Circ}_A$,
$T^\text{Circ}_B$, and $T^\text{Circ}_C$ are collinear and points $T^\text{In}_A$,
$T^\text{In}_B$, and $T^\text{In}_C$ are also collinear. However, I do not
have proofs of these claims.
\end{remark*}

\begin{remark*}[Further generalization from Andrew Gu]
  Let $ABC$ and $DEF$ be triangles which share an
  inconic, or equivalently share a circumconic.  Define points $T^\text{Circ}_A$,
  $T^\text{Circ}_B, \dots, T^\text{In}_C$ as in the previous remarks. Then
  it is conjectured that points $T^\text{Circ}_A$, $T^\text{Circ}_B$, and
  $T^\text{Circ}_C$ are collinear and points $T^\text{In}_A$, $T^\text{In}_B$,
  and $T^\text{In}_C$ are also collinear. (Note that extraversion may be
  required depending on the configuration of points, e.g. excircles instead of
  incircles.) Additionally, it appears that the insimilicenters of the
  circumcircles lie on a line perpendicular to the line through
  $T^\text{Circ}_A$, $T^\text{Circ}_B$, and $T^\text{Circ}_C$.
\end{remark*}
