author: Merlijn Staps
desc: Centroid angle chase
hardness: 10
source: TSTST 2023/1
url: https://aops.com/community/p28015679
tags: [2023-06, good, anglechase, pop, complex]

---

Let $ABC$ be a triangle with centroid $G$.
Points $R$ and $S$ are chosen on rays $GB$ and $GC$, respectively, such that
\[ \angle ABS = \angle ACR = 180^\circ - \angle BGC. \]
Prove that $\angle RAS + \angle BAC = \angle BGC$.

---

In all the following solutions,
let $M$ and $N$ denote the midpoints of $\ol{AC}$ and $\ol{AB}$, respectively.

\begin{center}
\begin{asy}
size(12cm);

pair A = dir(97);
pair B = dir(190);
pair C = dir(350);
pair M = midpoint(A--C);
pair N = midpoint(A--B);
pair G = extension(B, M, C, N);
draw(A--G, blue);
pair Y = A*dir((C-G)/(B-G))**2;

pair X = A*dir((B-G)/(C-G))**2;

pair S = extension(B, Y, C, G);
pair R = extension(C, X, B, G);
filldraw(A--B--C--cycle, opacity(0.1)+lightblue, blue);
draw(B--M, blue);
draw(C--N, blue);
draw(R--A--S, lightred);
draw(C--R, deepgreen);
draw(B--S, deepgreen);

dot("$A$", A, dir(A));
dot("$B$", B, dir(B));
dot("$C$", C, dir(C));
dot("$M$", M, dir(M));
dot("$N$", N, dir(N));
dot("$G$", G, dir(280));
dot("$S$", S, dir(270));
dot("$R$", R, dir(150));

/* -----------------------------------------------------------------+
|                 TSQX: by CJ Quines and Evan Chen                  |
| https://github.com/vEnhance/dotfiles/blob/main/py-scripts/tsqx.py |
+-------------------------------------------------------------------+
!size(12cm);
A = dir 97
B = dir 190
C = dir 350
M = midpoint A--C
N = midpoint A--B
G 280 = extension B M C N
A--G blue
!pair Y = A*dir((C-G)/(B-G))**2;
!pair X = A*dir((B-G)/(C-G))**2;
S 270 = extension B Y C G
R 150 = extension C X B G
A--B--C--cycle / 0.1 lightblue / blue
B--M blue
C--N blue
R--A--S lightred
C--R deepgreen
B--S deepgreen
*/

\end{asy}
\end{center}

\paragraph{Solution 1 using power of a point.}
From the given condition that $\dang ACR = \dang CGM$, we get that
\[ MA^2 = MC^2 = MG \cdot MR \implies \dang RAC = \dang MGA. \]
Analogously,
\[ \dang BAS = \dang AGN. \]
Hence,
\[
  \dang RAS + \dang BAC
  = \dang RAC + \dang BAS
  = \dang MGA + \dang AGN = \dang MGN = \dang BGC.
\]

\paragraph{Solution 2 using similar triangles.}
As before, $\triangle MGC \sim \triangle MCR$ and $\triangle NGB \sim \triangle NBS$.
We obtain
\[
  \frac{|AC|}{|CR|} = \frac{2|MC|}{|CR|} = \frac{2|MG|}{|GC|}
  = \frac{|GB|}{2|NG|} = \frac{|BS|}{2|BN|} = \frac{|BS|}{|AB|}
\]
which together with $\angle ACR = \angle ABS$ yields
\[ \triangle ACR \sim \triangle SBA \implies \dang BAS = \dang CRA. \]
Hence
\[
  \dang RAS + \dang BAC = \dang RAC + \dang BAS
  = \dang RAC + \dang CRA = - \dang ACR = \dang BGC,
\]
which proves the statement.

\paragraph{Solution 3 using parallelograms.}
Let $M$ and $N$ be defined as above.
Let $P$ be the reflection of $G$ in $M$
and let $Q$ the reflection of $G$ in $N$.
Then $AGCP$ and $AGBQ$ are parallelograms.
\begin{center}
\begin{asy}
size(12cm);

pair A = dir(97);
pair B = dir(190);
pair C = dir(350);
pair M = midpoint(A--C);
pair N = midpoint(A--B);
pair G = extension(B, M, C, N);
draw(A--G, blue);
pair Y = A*dir((C-G)/(B-G))**2;

pair X = A*dir((B-G)/(C-G))**2;

pair S = extension(B, Y, C, G);
pair R = extension(C, X, B, G);
filldraw(A--B--C--cycle, opacity(0.1)+lightblue, blue);
draw(B--M, blue);
draw(C--N, blue);
draw(R--A--S, lightred);
draw(C--R, deepgreen);
draw(B--S, deepgreen);

pair P = A+C-G;
pair Q = A+B-G;
draw(A--P--C, heavycyan);
draw(P--M, heavycyan);
draw(A--Q--B, heavycyan);
draw(Q--N, heavycyan);

draw(circumcircle(A, C, R), dotted+deepcyan);
draw(circumcircle(A, B, S), dotted+deepcyan);

dot("$A$", A, dir(A));
dot("$B$", B, dir(B));
dot("$C$", C, dir(C));
dot("$M$", M, dir(70));
dot("$N$", N, dir(100));
dot("$G$", G, dir(280));
dot("$S$", S, dir(270));
dot("$R$", R, dir(150));
dot("$P$", P, dir(P));
dot("$Q$", Q, dir(Q));

/* -----------------------------------------------------------------+
|                 TSQX: by CJ Quines and Evan Chen                  |
| https://github.com/vEnhance/dotfiles/blob/main/py-scripts/tsqx.py |
+-------------------------------------------------------------------+
!size(12cm);
A = dir 97
B = dir 190
C = dir 350
M 70 = midpoint A--C
N 100 = midpoint A--B
G 280 = extension B M C N
A--G blue
!pair Y = A*dir((C-G)/(B-G))**2;
!pair X = A*dir((B-G)/(C-G))**2;
S 270 = extension B Y C G
R 150 = extension C X B G
A--B--C--cycle / 0.1 lightblue / blue
B--M blue
C--N blue
R--A--S lightred
C--R deepgreen
B--S deepgreen
P = A+C-G
Q = A+B-G
A--P--C heavycyan
P--M heavycyan
A--Q--B heavycyan
Q--N heavycyan
circumcircle A C R / dotted deepcyan
circumcircle A B S / dotted deepcyan
*/
\end{asy}
\end{center}

\begin{claim*}
  Quadrilaterals $APCR$ and $AQBS$ are concyclic.
\end{claim*}
\begin{proof}
  Because $\dang APR = \dang APG = \dang CGP = -\dang BGC = \dang ACR$.
\end{proof}
Thus from $\ol{PC} \parallel \ol{GA}$ we get
\[ \dang RAC = \dang RPC = \dang GPC = \dang PGA \]
and similarly
\[ \dang BAS = \dang BQS = \dang BQG = \dang AGQ. \]
We conclude that
\[
  \dang RAS + \dang BAC = \dang RAC + \dang BAS
  = \dang PGA + \dang AGQ = \dang PGQ = \dang BGC.
\]

\paragraph{Solution 4 also using parallelograms, by Ankan Bhattacharya.}
Construct parallelograms $ARCK$ and $ASBL$. Since
\[ \dang CAK = \dang ACR = \dang CGB = \dang CGK, \]
it follows that $AGCK$ is cyclic. Similarly, $AGBL$ is also cyclic.
\begin{center}
  \begin{asy}
    size(12cm);
    pair A = dir(95), B = dir(190), C = dir(350);
    pair G = (A+B+C)/3;
    pair K = 2*foot(circumcenter(A, G, C), B, G) - G;
    pair L = 2*foot(circumcenter(A, G, B), C, G) - G;
    pair R = A+C-K, S = A+B-L;

    draw(A--G);
    draw(B--K^^C--L);
    draw(A--R--C--K--cycle, heavycyan);
    draw(A--S--B--L--cycle, heavymagenta);
    draw(A--B--C--cycle);
    draw(circumcircle(A, G, C), heavycyan+dashed);
    draw(circumcircle(A, G, B), heavymagenta+dashed);

    dot("$A$", A, dir(A));
    dot("$B$", B, dir(B));
    dot("$C$", C, dir(C));
    dot("$G$", G, dir(dir(B-G)+dir(C-G)));
    dot("$K$", K, dir(dir(K-A)+dir(K-C)));
    dot("$L$", L, dir(dir(L-A)+dir(L-B)));
    dot("$R$", R, dir(dir(A-R)+dir(B-R)));
    dot("$S$", S, dir(dir(A-S)+dir(C-S)));
  \end{asy}
\end{center}
Finally, observe that
\begin{align*}
  \angle RAS + \angle BAC
  & = \dang BAS + \dang RAC\\
  & = \dang ABL + \dang KCA\\
  & = \dang AGL + \dang KGA\\
  & = \dang KGL\\
  & = \angle BGC
\end{align*}
as requested.

\paragraph{Solution 5 using complex numbers, by Milan Haiman.}
Note that $\angle RAS + \angle BAC=\angle BAS+\angle RAC$.
We compute $\angle BAS$ in complex numbers;
then $\angle RAC$ will then be known by symmetry.

Let $a$, $b$, $c$ be points on the unit circle representing $A$, $B$, $C$ respectively.
Let $g=\frac{1}{3}(a+b+c)$ represent the centroid $G$,
and let $s$ represent $S$.

\begin{claim*}
  We have
  \[ \frac{s-a}{b-a} = \frac{ab-2bc+ca}{2ab-bc-ca}. \]
\end{claim*}
\begin{proof}
Since $S$ is on line $CG$, which passes through the midpoint of segment $AB$, we
have that \[ s=\frac{a+b}{2}+t(c-g) \] for some $t\in\RR$.

By the given angle condition, we have that
\[ \frac{(s-b)/(b-a)}{(c-g)/(g-b)}\in\RR. \]
Note that \[ \frac{s-b}{b-a}=t\frac{c-g}{b-a}-\frac{1}{2}. \]
So, \[ t\frac{g-b}{b-a}-\frac{g-b}{2(c-g)}\in \RR. \]
Thus
\[
  t = \frac{\opname{Im} \left(\frac{g-b}{2(c-g)}\right)}
  {\opname{Im} \left(\frac{g-b}{b-a}\right)}
  = \frac{1}{2} \cdot \frac
    {\left(\frac{g-b}{c-g}\right)-\ol{\left(\frac{g-b}{c-g}\right)}}
    {\left(\frac{g-b}{b-a}\right)-\ol{\left(\frac{g-b}{b-a}\right)}}.
\]
Let $N$ and $D$ be the numerator and denominator of the second factor above.

We want to compute
\[ \frac{s-a}{b-a}
   = \frac{1}{2}+t\frac{c-g}{b-a}
   = \frac{(b-a)+2t(c-g)}{2(b-a)}
   = \frac{(b-a)D+(c-g)N}{2(b-a)D}. \]

We have
\begin{align*}
    (c-g)N &= g-b-(c-g)\ol{\left(\frac{g-b}{c-g}\right)} \\
    &= \frac{a+b+c}{3}-b-\left(c-\frac{a+b+c}{3}\right)\frac{\frac{1}{a}+\frac{1}{b}+\frac{1}{c}-\frac{3}{b}}{\frac{3}{c}-\frac{1}{a}-\frac{1}{b}-\frac{1}{c}}\\
    &= \frac{(a+c-2b)(2ab-bc-ca)-(2c-a-b)(ab+bc-2ca)}{3(2ab-bc-ca)}\\
    &= \frac{3(a^2b+b^2c+c^2a-ab^2-bc^2-ca^2)}{3(2ab-bc-ca)}\\
    &= \frac{(a-b)(b-c)(a-c)}{2ab-bc-ca}
\end{align*}

We also compute \begin{align*}
    (b-a)D&=g-b-(b-a)\ol{\left(\frac{g-b}{b-a}\right)} \\
    &=\frac{a+b+c}{3}-b-\left(b-a\right)\frac{\frac{1}{a}+\frac{1}{b}+\frac{1}{c}-\frac{3}{b}}{\frac{3}{b}-\frac{3}{a}}\\
    &=\frac{(a+c-2b)c+(ab+bc-2ca)}{3c}\\
    &=\frac{ab-bc-ca+c^2}{3c}\\
    &=\frac{(a-c)(b-c)}{3c}
\end{align*}

So, we obtain
\[
  \frac{s-a}{b-a}
  = \frac{\frac{1}{3c}+\frac{a-b}{2ab-bc-ca}}{\frac{2}{3c}}
  = \frac{2ab-bc-ca+3c(a-b)}{2(2ab-bc-ca)}=\frac{ab-2bc+ca}{2ab-bc-ca}.
\]
\end{proof}

By symmetry,
\[ \frac{r-a}{c-a}=\frac{ab-2bc+ca}{2ca-ab-bc}. \]
Hence their ratio
\[ \frac{s-a}{b-a} \div \frac{r-a}{c-a} = \frac{2ab-bc-ca}{2ca-ab-bc} \]
has argument $\angle RAC +\angle BAS$.

We also have that $\angle BGC$ is the argument of
\[ \frac{b-g}{c-g}=\frac{2b-a-c}{2c-a-b}. \]
Note that these two complex numbers are inverse-conjugates,
and thus have the same argument. So we're done.
