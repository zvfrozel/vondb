author: Nikolai Beluhov
desc: Cars running away
source: TSTST 2019/3
tags: [2019-07, troll, brave, favorite, grid, algorithm, design, magic, yod]
hardness: 45
url: https://aops.com/community/p12608769

---

On an infinite square grid we place finitely many \emph{cars},
which each occupy a single cell and face in one of the four cardinal directions.
Cars may never occupy the same cell.
It is given that the cell immediately in front of each car is empty,
and moreover no two cars face towards each other
(no right-facing car is to the left of a left-facing car within a row, etc.).
In a \emph{move}, one chooses a car and shifts it one cell forward to a vacant cell.
Prove that there exists an infinite sequence of valid moves
using each car infinitely many times.

---

Let $S$ be any rectangle containing all the cars.
Partition $S$ into horizontal strips of height $1$,
and color them red and green in an alternating fashion.
It is enough to prove all the cars may exit $S$.

\begin{center}
\begin{asy}
size(10cm);
defaultpen(fontsize(12pt));
usepackage("amssymb");
picture base;
draw(base, box((0,0), (7,5)), black+1);

fill(base, box((0,0), (7,1)), rgb(1,0.9,0.9));
fill(base, box((0,2), (7,3)), rgb(1,0.9,0.9));
fill(base, box((0,4), (7,5)), rgb(1,0.9,0.9));

fill(base, box((0,1), (7,2)), rgb(0.9,1,0.9));
fill(base, box((0,3), (7,4)), rgb(0.9,1,0.9));

for (int i=1; i<7; ++i) {
  draw(base, (i,0)--(i,5), grey);
}
for (int i=1; i<5; ++i) {
  draw(base, (0,i)--(7,i), grey);
}

picture phase0;
add(phase0, base);
draw(phase0, (1.5,1.5)--(1.5,2.5), dashed+orange, EndArrow(TeXHead));
draw(phase0, (4.5,3.5)--(4.5,2.5), dashed+orange, EndArrow(TeXHead));
draw(phase0, (5.5,3.5)--(5.5,2.5), dashed+orange, EndArrow(TeXHead));
draw(phase0, (6.5,3.5)--(6.5,4.5), dashed+orange, EndArrow(TeXHead));
label(phase0, "$\blacktriangleright$", (1.5, 4.5));
label(phase0, "$\bigtriangledown$", (3.5, 4.5));
label(phase0, "$\blacktriangleright$", (1.5, 3.5));
label(phase0, "$\bigtriangledown$", (4.5, 3.5));
label(phase0, "$\bigtriangledown$", (5.5, 3.5));
label(phase0, "$\bigtriangleup$", (6.5, 3.5));
label(phase0, "$\blacktriangleright$", (0.5, 2.5));
label(phase0, "$\blacktriangleright$", (3.5, 2.5));
label(phase0, "$\bigtriangleup$", (2.5, 2.5));
label(phase0, "$\bigtriangleup$", (1.5, 1.5));
label(phase0, "$\blacktriangleleft$", (3.5, 1.5));
label(phase0, "$\blacktriangleleft$", (5.5, 1.5));
label(phase0, "$\blacktriangleright$", (6.5, 1.5));
label(phase0, "$\blacktriangleleft$", (4.5, 0.5));
label(phase0, "Step 1", (3.5,0), dir(-90));

picture phase1;
add(phase1, base);
draw(phase1, (5.5,1.5)--(0,1.5), dashed+orange, EndArrow(TeXHead));
draw(phase1, (6.5,1.5)--(7,1.5), dashed+orange, EndArrow(TeXHead));
draw(phase1, (1.5,3.5)--(7,3.5), dashed+orange, EndArrow(TeXHead));
label(phase1, "$\blacktriangleright$", (1.5, 4.5));
label(phase1, "$\bigtriangledown$", (3.5, 4.5));
label(phase1, "$\blacktriangleright$", (1.5, 3.5));
label(phase1, "$\bigtriangledown$", (4.5, 2.5), blue); // moved down
label(phase1, "$\bigtriangledown$", (5.5, 2.5), blue); // moved down
label(phase1, "$\bigtriangleup$", (6.5, 4.5), blue); // moved up
label(phase1, "$\blacktriangleright$", (0.5, 2.5));
label(phase1, "$\blacktriangleright$", (3.5, 2.5));
label(phase1, "$\bigtriangleup$", (2.5, 2.5));
label(phase1, "$\bigtriangleup$", (1.5, 2.5), blue); // moved up
label(phase1, "$\blacktriangleleft$", (3.5, 1.5));
label(phase1, "$\blacktriangleleft$", (5.5, 1.5));
label(phase1, "$\blacktriangleright$", (6.5, 1.5));
label(phase1, "$\blacktriangleleft$", (4.5, 0.5));
label(phase1, "Step 2", (3.5,0), dir(-90));

picture phase2;
add(phase2, base);
draw(phase2, (1.5,2.5)--(1.5,3.5), dashed+orange, EndArrow(TeXHead));
draw(phase2, (2.5,2.5)--(2.5,3.5), dashed+orange, EndArrow(TeXHead));
draw(phase2, (3.5,4.5)--(3.5,3.5), dashed+orange, EndArrow(TeXHead));
draw(phase2, (4.5,2.5)--(4.5,1.5), dashed+orange, EndArrow(TeXHead));
draw(phase2, (5.5,2.5)--(5.5,1.5), dashed+orange, EndArrow(TeXHead));
draw(phase2, (6.5,4.5)--(6.5,5), dashed+orange, EndArrow(TeXHead));
label(phase2, "$\blacktriangleright$", (1.5, 4.5));
label(phase2, "$\bigtriangledown$", (3.5, 4.5));
label(phase2, "$\bigtriangleup$", (6.5, 4.5));
label(phase2, "$\bigtriangledown$", (4.5, 2.5));
label(phase2, "$\bigtriangledown$", (5.5, 2.5));
label(phase2, "$\blacktriangleright$", (0.5, 2.5));
label(phase2, "$\blacktriangleright$", (3.5, 2.5));
label(phase2, "$\bigtriangleup$", (2.5, 2.5));
label(phase2, "$\bigtriangleup$", (1.5, 2.5));
label(phase2, "$\blacktriangleleft$", (4.5, 0.5));
label(phase2, "Step 3", (3.5,0), dir(-90));

picture phase3;
add(phase3, base);
draw(phase3, (1.5,4.5)--(7,4.5), dashed+orange, EndArrow(TeXHead));
draw(phase3, (0.5,2.5)--(7,2.5), dashed+orange, EndArrow(TeXHead));
draw(phase3, (4.5,0.5)--(0,0.5), dashed+orange, EndArrow(TeXHead));
label(phase3, "$\blacktriangleright$", (1.5, 4.5));
label(phase3, "$\bigtriangledown$", (3.5, 3.5), blue);
label(phase3, "$\bigtriangledown$", (4.5, 1.5), blue);
label(phase3, "$\bigtriangledown$", (5.5, 1.5), blue);
label(phase3, "$\blacktriangleright$", (0.5, 2.5));
label(phase3, "$\blacktriangleright$", (3.5, 2.5));
label(phase3, "$\bigtriangleup$", (2.5, 3.5), blue);
label(phase3, "$\bigtriangleup$", (1.5, 3.5), blue);
label(phase3, "$\blacktriangleleft$", (4.5, 0.5));
label(phase3, "Step 4", (3.5,0), dir(-90));

add(phase0);
add(shift(8,0)*phase1);
add(shift(0,-7)*phase2);
add(shift(8,-7)*phase3);
\end{asy}
\end{center}

To do so, we outline a five-stage plan for the cars.
\begin{enumerate}
  \ii All vertical cars in a green cell may advance one cell into a red cell
  (or exit $S$ altogether),
  by the given condition.
  (This is the only place where the hypothesis about empty space is used!)

  \ii All horizontal cars on green cells may exit $S$,
  as no vertical cars occupy green cells.

  \ii All vertical cars in a red cell
  may advance one cell into a green cell
  (or exit $S$ altogether),
  as all green cells are empty.

  \ii All horizontal cars within red cells may exit $S$,
  as no vertical car occupy red cells.

  \ii The remaining cars exit $S$, as they are all vertical.
  The solution is complete.
\end{enumerate}

\begin{remark*}[Author's comments]
  The solution I've given for this problem is so short
  and simple that it might appear at first
  to be about IMO 1 difficulty. I don't believe that's true!
  There are very many approaches that look perfectly plausible at first,
  and then fall apart in this or that twisted special case.
\end{remark*}

\begin{remark*}[Higher-dimensional generalization by author]
  The natural higher-dimensional generalization
  is true, and can be proved in largely the same fashion.
  For example, in three dimensions,
  one may let $S$ be a rectangular prism
  and partition $S$ into horizontal slabs
  and color them red and green in an alternating fashion.
  Stages 1, 3, and 5 generalize immediately,
  and stages 2 and 4 reduce to an application of the two-dimensional problem.
  In the same way, the general problem is handled by induction on the dimension.
\end{remark*}

\begin{remark*}
  [Historical comments]
  For $k > 1$, we could consider a variant of the problem
  where cars are $1 \times k$ rectangles (moving parallel to the longer edge)
  instead of occupying single cells.
  In that case, if there are $2k-1$ empty spaces in front of each car,
  the above proof works
  (with the red and green strips having height $k$ instead).
  On the other hand, at least $k$ empty spaces are necessary.
  We don't know the best constant in this case.
\end{remark*}
