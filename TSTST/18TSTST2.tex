desc: Hamiltonian cycles on 2-regular digraph
source: TSTST 2018/2
tags: [2018-06, nice, linalg, parity, graph, bet]
author: Victor Wang
hardness: 15
url: https://aops.com/community/p10570985

---

In the nation of Onewaynia,
certain pairs of cities are connected by one-way roads.
Every road connects exactly two cities
(roads are allowed to cross each other, e.g., via bridges),
and each pair of cities has at most one road between them.
Moreover, every city has exactly two roads leaving it and
exactly two roads entering it.

We wish to close half the roads of Onewaynia in such a way that
every city has exactly one road leaving it and exactly one road entering it.
Show that the number of ways to do so is a power of $2$ greater than $1$
(i.e.\ of the form $2^n$ for some integer $n \ge 1$).

---

In the language of graph theory, we have a simple digraph $G$
which is 2-regular and we seek the
number of sub-digraphs which are $1$-regular.
We now present two solution paths.

\paragraph{First solution, combinatorial.}
We construct a simple undirected bipartite graph $\Gamma$ as follows:
\begin{itemize}
  \ii the vertex set consists of two copies of $V(G)$,
  say $V_{\text{out}}$ and $V_{\text{in}}$; and
  \ii for $v \in V_{\text{out}}$ and $w \in V_{\text{in}}$
  we have an undirected edge $vw \in E(\Gamma)$
  if and only if the directed edge $v \to w$ is in $G$.
\end{itemize}
Moreover, the desired sub-digraphs of $H$ correspond exactly
to perfect matchings of $\Gamma$.

However the graph $\Gamma$ is $2$-regular
and hence consists of several disjoint (simple) cycles of even length.
If there are $n$ such cycles, the number of perfect matchings
is $2^n$, as desired.

\begin{remark*}
  The construction of $\Gamma$ is not as magical as it may first seem.

  Suppose we pick a road $v_1 \to v_2$ to use.
  Then, the other road $v_3 \to v_2$ is certainly \emph{not} used;
  hence some other road $v_3 \to v_4$ must be used, etc.
  We thus get a cycle of forced decisions until
  we eventually return to the vertex $v_1$.

  These cycles in the original graph $G$
  (where the arrows alternate directions)
  correspond to the cycles we found in $\Gamma$.
  It's merely that phrasing the solution in terms of $\Gamma$
  makes it cleaner in a linguistic sense,
  but not really in a mathematical sense.
\end{remark*}

\paragraph{Second solution by linear algebra over $\FF_2$ (Brian Lawrence).}
This is actually not that different from the first solution.
For each edge $e$, we create an indicator variable $x_e$.
We then require for each vertex $v$ that:
\begin{itemize}
  \ii If $e_1$ and $e_2$ are the two edges leaving $v$,
  then we require $x_{e_1} + x_{e_2} \equiv 1 \pmod 2$.
  \ii If $e_3$ and $e_4$ are the two edges entering $v$,
  then we require $x_{e_3} + x_{e_4} \equiv 1 \pmod 2$.
\end{itemize}
We thus get a large system of equations.
Moreover, the solutions come in natural pairs $\vec x$ and $\vec x + \vec 1$
and therefore the number of solutions is either zero, or a power of two.
So we just have to prove there is at least one solution.

For linear algebra reasons, there can only be zero solutions
if some nontrivial linear combination of the equations gives the sum $0 \equiv 1$.
So suppose we added up some subset $S$ of the equations
for which every variable appeared on the left-hand side an even number of times.
Then every variable that did appear appeared exactly twice;
and accordingly we see that the edges corresponding to these variables
form one or more even cycles as in the previous solution.
Of course, this means $|S|$ is even, so we really have $0 \equiv 0 \pmod 2$ as needed.

\begin{remark*}
  The author's original proposal contained a second part
  asking to show that it was not always possible
  for the resulting $H$ to be connected,
  even if $G$ was strongly connected.
  This problem is related to IMO Shortlist 2002 C6,
  which gives an example of a strongly connected graph
  which does have a full directed Hamiltonian cycle.
\end{remark*}

---

%We wish to consider subgraphs $H$ for which each connected
%component is strongly connected.
%Notice, however that this requires every vertex to have
%indegree and outdegree at least $1$ (with respect to $H$);
%since $H$ has exactly half as many edges as $E$,
%this can only occur if every indegree and outdegree is exactly $1$.
%(Equivalently, we are looking for the number of ways
%to partition $G$ into directed cycles.)

%We say a sequence of edges is a \emph{wobbly cyclic trail}
%if they are distinct and have the form
%\[ v_1 \leftarrow w_1 \rightarrow v_2 \leftarrow w_2 \rightarrow v_3
%  \leftarrow \dots
%  \rightarrow v_m \leftarrow w_m \rightarrow v_1. \]
%(We allow repeated vertices but not repeated edges.)
%Note that every vertex is part of exactly two wobbly cycles,
%one containing its two out-neighbors
%and one containing its two in-neighbors.
%In particular, we can fix a partition of
%\[ E(G) = W_1 \sqcup W_2 \sqcup \dots \sqcup W_n \]
%into wobbly cycles.

%Then, a choice of half the edges of $E(G)$ with the desired
%property is exactly a choice of
%either the even or odd edges of each wobbly cycle.
%In particular, there are exactly $2^n$ choices.
