author: Merlijn Staps
desc: Cute angle chase
source: TSTST 2019/2
tags: [2019-07, anglechase, rich, good, bet]
hardness: 20
url: https://aops.com/community/p12608478

---

Let $ABC$ be an acute triangle with circumcircle $\Omega$
and orthocenter $H$.
Points $D$ and $E$ lie on segments $AB$ and $AC$
respectively, such that $AD = AE$.
The lines through $B$ and $C$ parallel to $\ol{DE}$
intersect $\Omega$ again at $P$ and $Q$, respectively.
Denote by $\omega$ the circumcircle of $\triangle ADE$.
\begin{enumerate}[(a)]
  \ii Show that lines $PE$ and $QD$ meet on $\omega$.
  \ii Prove that if $\omega$ passes through $H$,
  then lines $PD$ and $QE$ meet on $\omega$ as well.
\end{enumerate}

---

We will give one solution to (a),
then several solutions to (b).

\paragraph{Solution to (a).}
Note that $\dang AQP = \dang ABP = \dang ADE$
and $\dang APQ = \dang ACQ = \dang AED$,
so we have a spiral similarity $\triangle ADE \sim \triangle AQP$.
Therefore, lines $PE$ and $QD$ meet at the second
intersection of $\omega$ and $\Omega$ other than $A$.
Call this point $X$.
% (Note this does not even depend on $\omega$ passing through $H$.)

\paragraph{Solution to (b) using angle chasing.}
Let $L$ be the reflection of $H$ across $\ol{AB}$,
which lies on $\Omega$.
\begin{claim*}
  Points $L$, $D$, $P$ are collinear.
\end{claim*}
\begin{proof}
  This is just angle chasing:
  \begin{align*}
    \dang CLD &= \dang DHL
    = \dang DHA + \dang AHL = \dang DEA + \dang AHC \\
    &= \dang ADE + \dang CBA = \dang ABP + \dang CBA
    = \dang CBP = \dang CLP. \qedhere
  \end{align*}
\end{proof}

\begin{center}
\begin{asy}
size(7cm);

pair A = dir(125);
pair B = dir(210);
pair C = dir(-30);
pair O = (0,0);

pair H =orthocenter(A,B,C);
pair I = incenter(A,B,C);

pair J = 2*foot(H,A,I) - H;

pair D = intersectionpoints(A--B,circumcircle(A,H,J))[1];
pair E = intersectionpoints(A--C,circumcircle(A,H,J))[1];

pair S = intersectionpoints(circumcircle(A,B,C),circumcircle(A,H,J))[0];

pair P = extension(S,E,D,J);
pair Q = extension(S,D,E,J);
pair X = extension(P,E,Q,D);

draw(B--foot(B,A,C),dotted+gray);
draw(A--foot(A,B,C),dotted+gray);
draw(C--foot(C,A,B),dotted+gray);

draw(circumcircle(A,B,C),purple);
draw(circumcircle(A,H,J),red);

draw(D--E,red);
draw(B--P,purple);
draw(C--Q,purple);

pair HH = 1.9*circumcenter(A,D,E)-1.1*D;
pair BB = -B;

pair K = extension(D,P,E,Q);
pair L = extension(P,D,C,H);

draw(Q--E,dotted+red);
draw(P--L,dashed+red);
draw(H--L, dotted);
draw(P--X--Q, orange+dashed);


draw(A--B--C--A);
dot("$A$", A, dir(A));
//dot("$Z$",Z,dir(-90));
dot("$B$",B,dir(225));
dot("$C$",C,dir(C));
dot("$H$",H,dir(225));
dot("$D$",D,2*dir(56));
dot("$E$",E,dir(30));
dot("$P$",P,dir(P));
dot("$Q$",Q,dir(Q));
dot("$L$",L,dir(L));
dot("$K$",K,dir(K-A));
dot("$X$",X,dir(X), red);

label("$\omega$",HH,red);
label("$\Omega$",-1.1*B,purple);
\end{asy}
\end{center}

Now let $K \in \omega$ such that $DHKE$ is an isosceles trapezoid,
i.e.\ $\dang BAH = \dang KAE$.
\begin{claim*}
  Points $D$, $K$, $P$ are collinear.
\end{claim*}
\begin{proof}
  Using the previous claim,
  \[ \dang KDE = \dang KAE = \dang BAH = \dang LAB
    = \dang LPB = \dang DPB = \dang PDE. \qedhere \]
\end{proof}

By symmetry, $\ol{QE}$ will then pass through the same $K$, as needed.
\begin{remark*}
  These two claims imply each other,
  so guessing one of them allows one to realize the other.
  It is likely the latter is easiest to guess from the diagram,
  since it does not need any additional points.
\end{remark*}

\paragraph{Solution to (b) by orthogonal circles (found by contestants).}
We define $K$ as in the previous solution,
but do not claim that $K$ is the desired intersection.
Instead, we note that:

\begin{claim*}
  Point $K$ is the orthocenter of isosceles triangle $APQ$.
\end{claim*}
\begin{proof}
  Notice that $AH = AK$ and $BC = PQ$.
  Moreover from $\ol{AH} \perp \ol{BC}$
  we deduce $\ol{AK} \perp \ol{PQ}$ by reflection
  across the angle bisector.

  In light of the formula ``$AH^2 = 4R^2 - a^2$'', this implies the conclusion.
\end{proof}

Let $M$ be the midpoint of $\ol{PQ}$.
Since $\triangle APQ$ is isosceles,
\[ \ol{AKM} \perp \ol{PQ} \implies MK \cdot MA = MP^2 \]
by orthocenter properties.

So to summarize
\begin{itemize}
  \ii The circle with diameter $\ol{PQ}$ is orthogonal to $\omega$.
  In other words, point $P$ lies on the polar of $Q$ with respect to $\omega$.
  \ii The point $X = \ol{QD} \cap \ol{PE}$ is on $\omega$.
\end{itemize}
On the other hand, if we let $K' = \ol{QE} \cap \omega$,
then by Brokard theorem on $XDK'E$,
the polar of $Q = \ol{XD} \cap \ol{K'E}$ pass through $\ol{DK'} \cap \ol{XE}$;
this point must therefore be $P$ and $K' = K$ as desired.

\paragraph{Solution to (b) by complex numbers (Yang Liu and Michael Ma).}
Let $M$ be the arc midpoint of $\widehat{BC}$.
We use the standard arc midpoint configuration.
We have that
\[ A = a^2, \; B = b^2, \; C = c^2, \; M = -bc, \;
H = a^2+b^2+c^2, \; P = \frac{a^2c}{b}, \; Q = \frac{a^2b}{c}, \]
where $M$ is the arc midpoint of $\widehat{BC}$.
By direct angle chasing we can verify that $\ol{MB} \parallel \ol{DH}$.
Also, $D \in \ol{AB}$.
Therefore, we can compute $D$ as follows.
\[ d + a^2b^2 \bar{d} = a^2+b^2 \text{ and }
  \frac{d-h}{\bar{d} - \bar{h}} = -mb^2 = b^3c \implies
  d = \frac{a^2(a^2c+b^2c+c^3-b^3)}{c(bc+a^2)}. \]
By symmetry, we have that
\[ e = \frac{a^2(a^2b+bc^2+b^3-c^3)}{b(bc+a^2)}. \]
To finish, we want to show that the angle
between $\ol{DP}$ and $\ol{EQ}$ is angle $A$.
To show this, we compute
$\frac{d-p}{e-q} \Big/ \ol{\frac{d-p}{e-q}}$.
First, we compute
\begin{align*}
  d-p &= \frac{a^2(a^2c+b^2c+c^3-b^3)}{c(bc+a^2)} - \frac{a^2c}{b} \\
  &= a^2 \left(\frac{a^2c+b^2c+c^3-b^3}{c(bc+a^2)} - \frac{c}{b} \right)
  = \frac{a^2(a^2c-b^3)(b-c)}{bc(bc+a^2)}.
\end{align*}
By symmetry,
\[ \frac{d-p}{e-q} = -\frac{a^2c-b^3}{a^2b-c^3}
  \implies \frac{d-p}{e-q} \Big/ \ol{\frac{d-p}{e-q}}
  = \frac{a^2b^3c}{a^2bc^3} = \frac{b^2}{c^2} \]
as desired.

\paragraph{Solution to (b) using untethered moving points (Zack Chroman).}
We work in the real projective plane $\mathbb{RP}^2$,
and animate $C$ linearly on a fixed line through $A$.

Recall:
\begin{lemma*}
  [Zack's lemma]
  Suppose points $A$, $B$ have degree $d_1$, $d_2$,
  and there are $k$ values of $t$ for which $A=B$.
  Then line $AB$ has degree at most $d_1+d_2-k$.
  Similarly, if lines $\ell_1$, $\ell_2$ have degrees $d_1$, $d_2$,
  and there are $k$ values of $t$ for which $\ell_1=\ell_2$,
  then the intersection $\ell_1 \cap \ell_2$ has degree at most $d_1+d_2-k$.
\end{lemma*}

Now, note that $H$ moves linearly in $C$ on line $BH$.
Furthermore, angles $\angle AHE$, $\angle AHF$ are fixed,
we get that $D$ and $E$ have degree $2$.
One way to see this is using the lemma; $D$ lies on line $AB$,
which is fixed, and line $HD$ passes through a point at infinity
which is a constant rotation of the point at infinity on line $AH$,
and therefore has degree $1$.
Then $D$, $E$ have degree at most $1+1-0=2$.

Now, note that $P,Q$ move linearly in $C$.
Both of these are because the circumcenter $O$ moves linearly in $C$,
and $P$, $Q$ are reflections of $B$, $C$ in a
line through $O$ with fixed direction, which also moves linearly.

So by the lemma, the lines $PD$, $QE$ have degree at most $3$.
I claim they actually have degree $2$;
to show this it suffices to give an example of a choice of $C$
for which $P=D$ and one for which $Q=E$.
But an easy angle chase shows that in the unique case when $P=B$,
we get $D=B$ as well and thus $P=D$. Similarly when $Q=C$, $E=C$.
It follows from the lemma that lines $PD$, $QE$ have degree at most $2$.

Let $\ell_\infty$ denote the line at infinity.
I claim that the points $P_1=PD \cap \ell_\infty$,
$P_2=QE \cap \ell_\infty$ are projective in $C$.
Since $\ell_\infty$ is fixed, it suffices to show by the lemma that there
exists some value of $C$ for which $QE=\ell_\infty$ and $PD = \ell_\infty$.
But note that as $C \to \infty$, all four points $P,D,Q,E$ go to infinity.
It follows that $P_1$, $P_2$ are projective in $C$.

Then to finish,
recall that we want to show that $\angle (PD, QE)$ is constant.
It suffices then to show that there's a constant rotation
sending $P_1$ to $P_2$. Since $P_1$, $P_2$ are projective,
it suffices to verify this for $3$ values of $C$.

We can take $C$ such that $\angle ABC=90$, $\angle ACB = 90$,
or $AB=AC$, and all three cases are easy to check.
