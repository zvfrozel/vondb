desc: Ducks in a circle
hardness: 10
source: TSTST 2020/1
tags: [2020-11, invariant, find, bestpossible, aleph]
author: Ankan Bhattacharya
url: https://aops.com/community/p18933796

---

Let $a$, $b$, $c$ be fixed positive integers.
There are $a+b+c$ ducks sitting in a circle, one behind the other.
Each duck picks either \emph{rock}, \emph{paper}, or \emph{scissors},
with $a$ ducks picking rock, $b$ ducks picking paper,
and $c$ ducks picking scissors.

A \emph{move} consists of an operation of one of the following three forms:
\begin{itemize}
  \ii If a duck picking rock sits behind a duck
  picking scissors, they switch places.
  \ii If a duck picking paper sits behind a duck
  picking rock, they switch places.
  \ii If a duck picking scissors sits behind a duck
  picking paper, they switch places.
\end{itemize}
Determine, in terms of $a$, $b$, and $c$,
the maximum number of moves which could take place,
over all possible initial configurations.

---

The maximum possible number of moves is $\max(ab, ac, bc)$.

First, we prove this is best possible.
We define a \emph{feisty triplet} to be an unordered triple of ducks,
one of each of rock, paper, scissors,
such that the paper duck is between the rock and scissors duck
and facing the rock duck, as shown.
(There may be other ducks not pictured, but the orders are irrelevant.)
\begin{center}
\begin{asy}
size(11cm);
picture duck;
pen duckborder = black+1.2;

picture leg;
draw(leg, (0.5*dir(-70))--(1.6*dir(-70)), duckborder);
draw(leg, (1.3*dir(-70))--(1.3*dir(-70)+0.3*dir(-95)), duckborder);
draw(leg, (1.3*dir(-70))--(1.3*dir(-70)+0.3*dir(-50)), duckborder);
add(duck, leg);
add(duck, reflect(dir(-90), dir(90))*leg);
filldraw(duck, (1,0)--(-1,0)..(0,-1)..cycle, rgb("a67b5b"), duckborder); // body of duck
draw(duck, (0.1,-0.4)..(-0.2,-0.35)..(-0.6,-0.2), duckborder); // part of wing
draw(duck, (0.1,-0.8)..(-0.2,-0.7)..(-0.6,-0.2), duckborder); // part of wing
filldraw(duck, (1.2,0.3)--(1.2,0)--(1.7,0.15)--cycle, yellow, duckborder); // beak of duck
filldraw(duck, circle((0.7,0.3), 0.6), rgb("b9d3a4"), duckborder); // duck head
fill(duck, ellipse((1.05,0.45), 0.08, 0.12), black);
fill(duck, ellipse((1.07,0.49), 0.02, 0.03), white);
draw(duck, (1.6,0.5)--(1.8,0.7)..(2.0,0.75)..(2.4,1.2)..(1.8,1.5)..(1.2,1.2)..(1.6,0.7)--cycle );
label(duck, "\textsf{quack}", (1.8,1.1), fontsize(9pt));

draw(scale(4)*unitcircle, blue+2);
pen labelpen = red + fontsize(16pt);

add(shift(5*dir(90))*duck);
label("Rock", 3*dir(90), labelpen);
add(rotate(120)*shift(5*dir(90))*duck);
label("Paper", 3*dir(210), labelpen);
add(rotate(240)*shift(5*dir(90))*duck);
label("Scissors", 3*dir(330), labelpen);
\end{asy}
\end{center}

\begin{claim*}
  The number of feisty triplets decreases
  by $c$ if a paper duck swaps places with a rock duck, and so on.
\end{claim*}
\begin{proof}
  Clear.
\end{proof}
Obviously the number of feisty triples is at most $abc$ to start.
Thus at most $\max(ab,bc,ca)$ moves may occur,
since the number of feisty triplets should always be nonnegative,
at which point no moves are possible at all.

To see that this many moves is possible,
assume WLOG $a = \min(a, b, c)$
and suppose we have $a$ rocks, $b$ papers, and $c$ scissors
in that clockwise order.
\begin{center}
\begin{asy}
size(11cm);
picture duck;
pen duckborder = black+1.2;

picture leg;
draw(leg, (0.5*dir(-70))--(1.6*dir(-70)), duckborder);
draw(leg, (1.3*dir(-70))--(1.3*dir(-70)+0.3*dir(-95)), duckborder);
draw(leg, (1.3*dir(-70))--(1.3*dir(-70)+0.3*dir(-50)), duckborder);
add(duck, leg);
add(duck, reflect(dir(-90), dir(90))*leg);
filldraw(duck, (1,0)--(-1,0)..(0,-1)..cycle, rgb("a67b5b"), duckborder); // body of duck
draw(duck, (0.1,-0.4)..(-0.2,-0.35)..(-0.6,-0.2), duckborder); // part of wing
draw(duck, (0.1,-0.8)..(-0.2,-0.7)..(-0.6,-0.2), duckborder); // part of wing
filldraw(duck, (1.2,0.3)--(1.2,0)--(1.7,0.15)--cycle, yellow, duckborder); // beak of duck
filldraw(duck, circle((0.7,0.3), 0.6), rgb("b9d3a4"), duckborder); // duck head
fill(duck, ellipse((1.05,0.45), 0.08, 0.12), black);
fill(duck, ellipse((1.07,0.49), 0.02, 0.03), white);
draw(duck, (1.6,0.5)--(1.8,0.7)..(2.0,0.75)..(2.4,1.2)..(1.8,1.5)..(1.2,1.2)..(1.6,0.7)--cycle );
label(duck, "\textsf{quack}", (1.8,1.1), fontsize(9pt));

draw(scale(4)*unitcircle, blue+2);
pen labelpen = red + fontsize(16pt);

add(rotate(-10)*shift(5*dir(90))*duck);
add(rotate(10)*shift(5*dir(90))*duck);
label("Rocks", 3*dir(90), labelpen);

add(rotate(100)*shift(5*dir(90))*duck);
add(rotate(120)*shift(5*dir(90))*duck);
add(rotate(140)*shift(5*dir(90))*duck);
label("Papers", 3*dir(210), labelpen);

add(rotate(200)*shift(5*dir(90))*duck);
add(rotate(220)*shift(5*dir(90))*duck);
add(rotate(240)*shift(5*dir(90))*duck);
add(rotate(260)*shift(5*dir(90))*duck);
label("Scissors", 3*dir(330), labelpen);
\end{asy}
\end{center}
Then, allow the scissors to filter through the papers
while the rocks stay put.
Each of the $b$ papers swaps with $c$ scissors,
for a total of $bc = \max(ab, ac, bc)$ swaps.

\begin{remark*}
  [Common errors]
  One small possible mistake:
  it is not quite k\"{o}sher to say that ``WLOG $a \le b \le c$''
  because the condition is not symmetric, only cyclic.
  Therefore in this solution we only assume $a = \min(a,b,c)$.

  It is true here that every pair of ducks swaps at most once,
  and some solutions make use of this fact.
  However, this fact implicitly uses the fact that $a,b,c > 0$
  and is false without this hypothesis.
\end{remark*}
