author: Yannick Yao, Evan Chen
desc: Yannick Euclidea
source: TSTST 2018/3
tags: [2018-06, rich, anglechase, mine, gimel]
hardness: 30
url: https://aops.com/community/p10570988

---

Let $ABC$ be an acute triangle with incenter $I$, circumcenter $O$, and circumcircle $\Gamma$.
Let $M$ be the midpoint of $\ol{AB}$.
Ray $AI$ meets $\ol{BC}$ at $D$.
Denote by $\omega$ and $\gamma$ the circumcircles of $\triangle BIC$ and $\triangle BAD$, respectively.
Line $MO$ meets $\omega$ at $X$ and $Y$, while line $CO$ meets $\omega$ at $C$ and $Q$.
Assume that $Q$ lies inside $\triangle ABC$ and $\angle AQM = \angle ACB$.

Consider the tangents to $\omega$ at $X$ and $Y$ and the tangents to $\gamma$ at $A$ and $D$.
Given that $\angle BAC \neq 60\dg$, prove that these four lines are concurrent on $\Gamma$.

---


Henceforth assume $\angle A \neq 60\dg$; we prove the concurrence.
Let $L$ denote the center of $\omega$, which is the midpoint of minor arc $BC$.

\begin{claim*}
  Let $K$ be the point on $\omega$ such that
  $\ol{KL} \parallel \ol{AB}$ and $\ol{KC} \parallel \ol{AL}$.
  Then $\ol{KA}$ is tangent to $\gamma$, and we may put
  \[ x = KA = LB = LC = LX = LY = KX = KY. \]
\end{claim*}
\begin{proof}
  By construction, $KA = LB = LC$.
  Also, $\ol{MO}$ is the perpendicular bisector of $\ol{KL}$
  (since the chords $\ol{KL}$, $\ol{AB}$ of $\omega$ are parallel)
  and so $KXLY$ is a rhombus as well.

  Moreover, $\ol{KA}$ is tangent to $\gamma$ as well since
  \[ \dang KAD = \dang KAL = \dang KAC + \dang CAL
    = \dang KBC + \dang ABK = \dang ABC. \qedhere \]
\end{proof}

\begin{center}
\begin{asy}
size(9cm);
pair L = dir(270);
pair B = dir(196);
pair C = L*L/B;
real r = 2**0.5 * abs(C-L);

pair A = IP(unitcircle, CR(C, r));
filldraw(unitcircle, opacity(0.1)+lightcyan, lightblue);
draw(A--B--C--cycle, lightblue);

pair M = midpoint(A--B);
pair N = midpoint(A--C);
pair O = origin;
pair E = extension(M, O, B, C);
pair K = 2*E-L;
pair X = IP(M--O, CP(L, B));
pair Y = 2*E-X;

filldraw(CP(L, X), opacity(0.1)+lightgreen, heavygreen);

clip(box((-1.4,-1.4),(1.4,1.4)));
pair D = extension(A, L, B, C);
filldraw(X--L--Y--K--cycle, opacity(0.15)+lightred, red+0.8);

draw(A--K, red+dashed);
draw(B--L, red+dashed);
draw(K--D, red+dashed);
draw(C--L, red+dashed);

draw(A--L, heavycyan);
draw(K--C, heavycyan);
pair Q = foot(A, C, O);
draw(A--Q--C, orange);
draw(N--L, orange);
draw(M--Y, purple);
draw(K--L, purple);

dot("$L$", L, dir(L));
dot("$B$", B, dir(B));
dot("$C$", C, dir(C));
dot("$A$", A, dir(A));
dot("$M$", M, dir(M));
dot("$N$", N, dir(N));
dot("$O$", O, dir(45));
dot("$E$", E, dir(270));
dot("$K$", K, dir(K));
dot("$X$", X, dir(100));
dot("$Y$", Y, dir(Y));
dot("$D$", D, dir(315));
dot("$Q$", Q, dir(Q));

/* TSQ Source:

!size(9cm);
L = dir 270
B = dir 196
C = L*L/B
!real r = 2**0.5 * abs(C-L);

A = IP unitcircle CR C r
unitcircle 0.1 lightcyan / lightblue
A--B--C--cycle lightblue

M = midpoint A--B
N = midpoint A--C
O = origin R45
E = extension M O B C R270
K = 2*E-L
X = IP M--O CP L B R100
Y = 2*E-X

CP L X 0.1 lightgreen / heavygreen

!clip(box((-1.4,-1.4),(1.4,1.4)));
D = extension A L B C R315
X--L--Y--K--cycle 0.15 lightred / red+0.8

A--K red dashed
B--L red dashed
K--D red dashed
C--L red dashed

A--L heavycyan
K--C heavycyan
Q = foot A C O
A--Q--C orange
N--L orange
M--Y purple
K--L purple

*/
\end{asy}
\end{center}

Up to now we have not used the existence of $Q$; we henceforth do so.

Note that $Q \neq O$, since $\angle A \neq 60\dg \implies O \notin \omega$.
Moreover, we have $\angle AOM = \angle ACB$ too.
Since $O$ and $Q$ both lie inside $\triangle ABC$,
this implies that $A$, $M$, $O$, $Q$ are concyclic.
As $Q \neq O$ we conclude $\angle CQA = 90\dg$.

The main claim is now:
\begin{claim*}
  Assuming $Q$ exists, the rhombus $LXKY$ is a square.
  In particular, $\ol{KX}$ and $\ol{KY}$ are tangent to $\omega$.
\end{claim*}

\begin{proof}
  [First proof of Claim, communicated by Milan Haiman]
  Observe that $\triangle QLC \sim \triangle LOC$
  since both triangles are isosceles and share a base angle.
  Hence, $CL^2 = CO \cdot CQ$.

  Let $N$ be the midpoint of $\ol{AC}$, which lies on $(AMOQ)$.
  Then,
  \[ x^2 = CL^2 = CO \cdot CQ = CN \cdot CA = \half CA^2 = \half LK^2 \]
  where we have also used the fact $AQON$ is cyclic.
  Thus $LK = \sqrt2 x$ and so the rhombus $LXKY$ is actually a square.
\end{proof}

\begin{proof}
  [Second proof of Claim, Evan Chen]
  Observe that $Q$ lies on the circle with diameter $\ol{AC}$, centered at $N$, say.
  This means that $O$ lies on the radical axis of $\omega$ and $(N)$,
  hence $\ol{NL} \perp \ol{CO}$ implying
  \begin{align*}
    NO^2 + CL^2 &= NC^2 + LO^2
    = NC^2 + OC^2 = NC^2 + NO^2 + NC^2 \\
    \implies x^2 &= 2NC^2 \\
    \implies x &= \sqrt 2 NC = \frac{1}{\sqrt2} AC = \frac{1}{\sqrt2} LK.
  \end{align*}
  So $LXKY$ is a rhombus with $LK = \sqrt2 x$.
  Hence it is a square.
\end{proof}

\begin{proof}
  [Third proof of Claim]
  A solution by trig is also possible.
  As in the previous claims, it suffices to show that $AC = \sqrt 2 x$.

  First, we compute the length $CQ$ in two ways; by angle chasing one can show
  $\angle CBQ = 180\dg - (\angle BQC + \angle QCB) = \half \angle A$,
  and so
  \begin{align*}
    AC \sin B = CQ &= \frac{BC}{\sin(90\dg+\half\angle A)} \cdot \sin\half\angle A \\
    \iff \sin^2 B &= \frac{\sin A \cdot \sin \half \angle A}{\cos \half \angle A} \\
    \iff \sin^2 B &= 2\sin^2 \half \angle A \\
    \iff \sin B &= \sqrt 2 \sin \half \angle A \\
    \iff 2R \sin B &= \sqrt 2 \left(2R \sin \half \angle A\right) \\
    \iff AC &= \sqrt 2 x
  \end{align*}
  as desired (we have here used the fact $\triangle ABC$ is acute to take square roots).

  It is interesting to note that $\sin^2 B = 2 \sin^2 \half \angle A$
  can be rewritten as \[ \cos A = \cos^2 B \]
  since $\cos^2 B = 1 - \sin^2 B = 1 - 2 \sin^2 \half \angle A = \cos A$;
  this is the condition for the existence of the point $Q$.
\end{proof}

We finish by proving that
\[ KD = KA \]
and hence line $\ol{KD}$ is tangent to $\gamma$.
Let $E = \ol{BC} \cap \ol{KL}$.
Then
\[ LE \cdot LK = LC^2 = LX^2 = \half LK^2 \]
and so $E$ is the midpoint of $\ol{LK}$.
Thus $\ol{MXOY}$, $\ol{BC}$, $\ol{KL}$ are concurrent at $E$.
As $\ol{DL} \parallel \ol{KC}$, we find that $DLCK$ is a parallelogram,
so $KD = CL = KA$ as well.
Thus $\ol{KD}$ and $\ol{KA}$ are tangent to $\gamma$.

\begin{remark*}
  The condition $\angle A \neq 60\dg$ cannot be dropped, since if $Q = O$ the problem is not true.

  On the other hand, nearly all solutions begin by observing $Q \neq O$ and then obtaining $\angle AQO = 90\dg$.
  This gives a way to construct the diagram by hand with ruler and compass.
  One draws an arbitrary chord $\ol{BC}$ of a circle $\omega$ centered at $L$,
  and constructs $O$ as the circumcenter of $\triangle BLC$ (hence obtaining $\Gamma$).
  Then $Q$ is defined as the intersection of ray $CO$ with $\omega$,
  and $A$ is defined by taking the perpendicular line through $Q$ on the circle $\Gamma$.
  In this way we can draw a triangle $ABC$ satisfying the problem conditions.
\end{remark*}

\paragraph{Authorship comments.}
In the notation of the present points,
the question originally sent to me by Yannick Yao read:
\begin{quote}
  Circles $(L)$ and $(O)$ are drawn,
  meeting at $B$ and $C$, with $L$ on $(O)$.
  Ray $CO$ meets $(L)$ at $Q$,
  and $A$ is on $(O)$ such that $\angle CQA = 90\dg$.
  The angle bisector of $\angle AOB$ meets $(L)$ at $X$ and $Y$.
  Show that $\angle XLY = 90\dg$.
\end{quote}
Notice the points $M$ and $K$ are absent from the problem.
I am told this was found as part of the computer game ``Euclidea''.
Using this as the starting point,
I constructed the TSTST problem by recognizing the significance
of that special point $K$, which became the center of attention.

---

Yannick sent me
\url{https://github.com/MathsFans/Euclidea/blob/master/v1/solving/d12.6.1.png}
asking me if this could become a reasonable TSTST problem
(not knowing how to prove the coincidence).
It took me about an hour and a half before I figured out a way to do it;
then I took those ideas and extended them to arrive at a
more traditional-looking problem over the next 90 minutes.
