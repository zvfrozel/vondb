desc:  Geometry finale
author: Danielle Wang
source:  TSTST 2016/6
tags:  [anglechase, rich, favorite, projective, harmonic, pop, 2016-07, well, yod]
hardness: 45
url: https://aops.com/community/p6580553

---

Let $ABC$ be a triangle with incenter $I$, and whose incircle is tangent
to $\ol{BC}$, $\ol{CA}$, $\ol{AB}$ at $D$, $E$, $F$, respectively.
Let $K$ be the foot of the altitude from $D$ to $\ol{EF}$.
Suppose that the circumcircle of $\triangle AIB$
meets the incircle at two distinct points $C_1$ and $C_2$,
while the circumcircle of $\triangle AIC$ meets the incircle
at two distinct points $B_1$ and $B_2$.
Prove that the radical axis of the circumcircles of
$\triangle BB_1B_2$ and $\triangle CC_1C_2$ passes through the midpoint $M$ of $\ol{DK}$.

---

\paragraph{First solution (Allen Liu).}
Let $X$, $Y$, $Z$ be midpoints of $EF$, $FD$, $DE$, and let $G$ be the Gergonne point.
By radical axis on $(AEIF)$, $(DEF)$, $(AIC)$ we see that $B_1$, $X$, $B_2$ are collinear.
Likewise, $B_1$, $Z$, $B_2$ are collinear, so lines $B_1B_2$ and $XZ$ coincide.
Similarly, lines $C_1C_2$ and $XY$ coincide.
In particular lines $B_1B_2$ and $C_1C_2$ meet at $X$.

\begin{center}
\begin{asy}
size(12cm);

pair A = dir(70);
pair B = dir(240);
pair C = dir(300);
pair I = incenter(A, B, C);
pair D = foot(I, B, C);
pair E = foot(I, C, A);
pair F = foot(I, A, B);
pair G = extension(A, D, B, E);
filldraw(A--B--C--cycle, opacity(0.1)+lightcyan, blue);
draw(D--E--F--cycle, blue);
draw(incircle(A, B, C), blue);
draw(A--D, blue);
draw(B--E, blue);
draw(C--F, blue);

pair Q = extension(G, G+F-D, E, F);
pair S = extension(G, G+F-D, E, D);
pair R = extension(G, G+D-E, F, D);
pair U = extension(G, G+D-E, F, E);

pair X = midpoint(E--F);
pair Y = midpoint(F--D);
pair Z = midpoint(D--E);

pair Qp = 2*Q-G;
pair Rp = 2*R-G;
pair Sp = 2*S-G;
pair Up = 2*U-G;

pair B_1 = OP(circumcircle(B, Qp, Sp), incircle(A, B, C));
pair B_2 = IP(circumcircle(B, Qp, Sp), incircle(A, B, C));
pair C_1 = OP(circumcircle(C, Rp, Up), incircle(A, B, C));
pair C_2 = IP(circumcircle(C, Rp, Up), incircle(A, B, C));

draw(circumcircle(B, B_1, B_2), red);
draw(circumcircle(C, C_1, C_2), red);

pair V = extension(X, Y, A, B);
pair W = extension(X, Z, A, C);

pair T = extension(B, W, C, V);
draw(X--T, yellow);
draw(unitcircle, blue);
draw(B--W, heavygreen);
draw(C--V, heavygreen);
draw(B--V, blue);
draw(C--W, blue);

draw(C_2--V, heavygreen);
draw(B_2--W, heavygreen);

dot("$A$", A, dir(A));
dot("$B$", B, dir(B));
dot("$C$", C, dir(C));
dot("$D$", D, dir(D));
dot("$E$", E, dir(E));
dot("$F$", F, dir(170));
dot("$G$", G, dir(G));
dot("$X$", X, dir(110));
dot("$Y$", Y, dir(350));
dot("$Z$", Z, dir(Z));
dot("$B_1$", B_1, dir(B_1));
dot("$B_2$", B_2, dir(100));
dot("$C_1$", C_1, dir(C_1));
dot("$C_2$", C_2, dir(40));
dot("$V$", V, dir(V));
dot("$W$", W, dir(W));
dot("$T$", T, dir(T));

/* Source generated by TSQ */
\end{asy}
\end{center}

Note $G$ is the symmedian point of $DEF$, so
it is well-known that $XG$ passes through the midpoint of $DK$.
So we just have to prove $G$ lies on the radical axis.

First, note that $\triangle DEF$ is the cevian triangle of the Gergonne point $G$.
Set $V = \ol{XY} \cap \ol{AB}$, $W = \ol{XZ} \cap \ol{AC}$, and $T = \ol{BW} \cap \ol{CV}$.

We begin with the following completely projective claim.
\begin{claim*}
  The points $X$, $G$, $T$ are collinear.
\end{claim*}
\begin{proof}
  It suffices to view $\triangle XYZ$ as any cevian triangle of $\triangle DEF$
  (which is likewise any cevian triangle of $\triangle ABC$).
  Then
  \begin{itemize}
    \ii By Cevian Nest on $\triangle ABC$,
    it follows that $\ol{AX}$, $\ol{BY}$, $\ol{CZ}$ are concurrent.
    \ii Hence $\triangle BYV$ and $\triangle CZW$ are perspective.
    \ii Hence $\triangle BZW$ and $\triangle CYV$ are perspective too.
    \ii Hence we deduce by Desargues theorem that $T$, $X$,
    and $\ol{BZ} \cap \ol{CY}$ are collinear.
    \ii Finally, the Cevian Nest theorem applied on $\triangle GBC$
    (which has cevian triangles $\triangle DFE$, $\triangle XZY$)
    we deduce $G$, $X$, and $\ol{BZ} \cap \ol{CY}$, proving the claim.
  \end{itemize}
  One could also proceed by using barycentric coordinates on $\triangle DEF$.
\end{proof}
\begin{remark*}
  [Eric Shen]
  The first four bullets can be replaced by non-projective means:
  one can check that $\ol{BZ} \cap \ol{CY}$ is the radical center
  of $(BIC)$, $(BB_1B_2)$, $(CC_1C_2)$ and therefore it lies on line $\ol{XT}$.
\end{remark*}

Now, we contend point $V$ is the radical center $(CC_1C_2)$, $(ABC)$ and $(DEF)$.
To see this, let $V' = \ol{ED} \cap \ol{AB}$;
then $(FV';AB)$ is harmonic, and $V$ is the midpoint of $\ol{FV'}$,
and thus $VA \cdot VB = VF^2 = VC_1 \cdot VC_2$.

So in fact $\ol{CV}$ is the radical axis of $(ABC)$ and $(CC_1C_2)$.

Similarly, $\ol{BW}$ is the radical axis of $(ABC)$ and $(BB_1B_2)$.
Thus $T$ is the radical center of $(ABC)$, $(BB_1B_2)$, $(CC_1C_2)$.

This completes the proof, as now $\ol{XT}$ is the desired radical axis.

\paragraph{Second solution (Evan Chen).}
As before, we just have to prove $G$ lies on the radical axis.

\begin{center}
 \begin{asy}
  size(14cm);

  pair A = dir(50);
  pair B = dir(220);
  pair C = dir(320);
  pair I = incenter(A, B, C);
  pair D = foot(I, B, C);
  pair E = foot(I, C, A);
  pair F = foot(I, A, B);
  pair G = extension(A, D, B, E);
  filldraw(A--B--C--cycle, opacity(0.1)+lightcyan, blue);
  draw(D--E--F--cycle, blue);
  draw(incircle(A, B, C), blue);
  draw(A--D, blue);
  draw(B--E, blue);
  draw(C--F, blue);

  pair P = extension(G, G+E-F, D, F);
  pair T = extension(G, G+E-F, D, E);
  pair Q = extension(G, G+F-D, E, F);
  pair S = extension(G, G+F-D, E, D);
  pair R = extension(G, G+D-E, F, D);
  pair U = extension(G, G+D-E, F, E);

  pair X = midpoint(E--F);
  pair Y = midpoint(F--D);
  pair Z = midpoint(D--E);

  pair Pp = 2*P-G;
  pair Qp = 2*Q-G;
  pair Rp = 2*R-G;
  pair Sp = 2*S-G;
  pair Tp = 2*T-G;
  pair Up = 2*U-G;

  draw(Pp--Tp, heavygreen);
  draw(Rp--Up, heavygreen);
  draw(Qp--Sp, heavygreen);

  pair V = extension(B, X, Sp, Z);
  pair W = extension(B, R, Qp, X);

  pair B_1 = OP(circumcircle(B, Qp, Sp), incircle(A, B, C));
  pair B_2 = IP(circumcircle(B, Qp, Sp), incircle(A, B, C));
  pair C_1 = OP(circumcircle(C, Rp, Up), incircle(A, B, C));
  pair C_2 = IP(circumcircle(C, Rp, Up), incircle(A, B, C));

  pair K = foot(D, E, F);
  pair M = midpoint(D--K);
  draw(D--K, blue);

  draw(C_1--C_2, red);
  draw(B_1--B_2, red);
  filldraw(circumcircle(P, Q, R), opacity(0.1)+lightgreen, green+1.4);
  filldraw(circumcircle(Pp, Qp, Rp), opacity(0.1)+lightgreen, green);
  draw(circumcircle(Sp, Q, P), orange);
  draw(circumcircle(B, B_1, B_2), red);
  draw(circumcircle(C, C_1, C_2), red);

  draw(X--M, yellow);
  draw(B--V, orange);
  draw(B--W, orange);

  dot("$A$", A, dir(A));
  dot("$B$", B, dir(B));
  dot("$C$", C, dir(C));
  dot("$I$", I, dir(I));
  dot("$D$", D, dir(D));
  dot("$E$", E, dir(E));
  dot("$F$", F, dir(F));
  dot("$G$", G, dir(G));
  dot("$P$", P, dir(200));
  dot("$T$", T, dir(100));
  dot("$Q$", Q, dir(70));
  dot("$S$", S, dir(260));
  dot("$R$", R, dir(250));
  dot("$U$", U, dir(U));
  dot("$X$", X, dir(X));
  dot("$Y$", Y, dir(350));
  dot("$Z$", Z, dir(Z));
  dot("$P'$", Pp, dir(Pp));
  dot("$Q'$", Qp, dir(Qp));
  dot("$R'$", Rp, dir(240));
  dot("$S'$", Sp, dir(Sp));
  dot("$T'$", Tp, dir(Tp));
  dot("$U'$", Up, dir(60));
  dot("$V$", V, dir(V));
  dot("$W$", W, dir(W));
  dot("$B_1$", B_1, dir(B_1));
  dot("$B_2$", B_2, dir(B_2));
  dot("$C_1$", C_1, dir(C_1));
  dot("$C_2$", C_2, dir(C_2));
  dot("$K$", K, dir(200));
  dot("$M$", M, dir(M));
 \end{asy}
\end{center}

Construct parallelograms $GPFQ$, $GRDS$, $GTUE$
such that $P,R \in DF$, $S,T \in DE$, $Q,U \in EF$.
As $FG$ bisects $PQ$ and is isogonal to $FZ$,
we find $PQED$, hence $PQRU$, is cyclic.
Repeating the same logic and noticing $PR$, $ST$, $QU$ not concurrent,
all six points $PQRSTU$ are cyclic.
Moreover, since $PQ$ bisects $GF$, we see that
a dilation with factor $2$ at $G$ sends $PQ$ to $P', Q' \in AB$, say,
with $F$ the midpoint of $P'Q'$.
Define $R', S' \in BC$ similarly now and $T', U' \in CA$.

Note that $EQPDS'$ is in cyclic too,
as $\dang DS'Q = \dang DRS = \dang DEF$.
By homothety through $B$, points $B$, $P$, $X$ are collinear;
assume they meet $(EQPDS')$ again at $V$.
Thus $EVQPDS'$ is cyclic, and now
\[ \dang BVS' = \dang PVS' = \dang PQS = \dang PTS = \dang FED = \dang XEZ = \dang XVZ \]
hence $V$ lies on $(BQ'S')$.

Since $FB \parallel QP$, we get $EVFB$ is cyclic too,
so $XV \cdot XB = XE \cdot XF$ now;
thus $X$ lies on the radical axis of $(BS'Q')$ and $(DEF)$.
By the same argument with $W \in BZ$, we get $Z$ lies on the radical axis too.
Thus the radical axis of $(BS'Q')$ and $(DEF)$ must be line $XZ$,
which coincides with $B_1B_2$; so $(BB_1B_2) = (BS'Q')$.

Analogously, $(CC_1C_2) = (CR'U')$.
Since $G = Q'S' \cap R'U'$, we need only prove that $Q'R'S'U'$ is cyclic.
But $QRSU$ is cyclic, so we are done.

The circle $(PQRSTU)$ is called the \emph{Lemoine circle} of $ABC$.
