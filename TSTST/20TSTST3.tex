desc: Determine all quirky triangles with $(n-1,n,n+1)$
author: Evan Chen, Danielle Wang
source: TSTST 2020/3
tags: [2020-11, trig, divis, polynomial, nice, mine, find, hardanswer, zayin]
hardness: 40
url: https://aops.com/community/p18933954

---

We say a nondegenerate triangle whose angles have
measures $\theta_1$, $\theta_2$, $\theta_3$ is \emph{quirky}
if there exists integers $r_1$, $r_2$, $r_3$, not all zero,
such that \[ r_1 \theta_1 + r_2 \theta_2 + r_3 \theta_3 = 0. \]
Find all integers $n \ge 3$ for which a triangle with
side lengths $n-1$, $n$, $n+1$ is quirky.

---


The answer is $n = 3,\ 4,\ 5,\ 7$.

We first introduce a variant of the $k$th Chebyshev polynomials
in the following lemma
(which is standard, and easily shown by induction).
\begin{lemma*}
  For each $k \ge 0$ there exists $P_k(X) \in \ZZ[X]$,
  monic for $k \ge 1$ and with degree $k$, such that
  \[ P_k( X + X^{-1} ) \equiv X^k + X^{-k}. \]
  The first few are $P_0(X) \equiv 2$,
  $P_1(X) \equiv X$, $P_2(X) \equiv X^2-2$, $P_3(X) \equiv X^3-3X$.
  % In particular, $P_k(2\cos\theta) = 2\cos(k\theta)$.
\end{lemma*}
Suppose the angles of the triangle are $\alpha < \beta < \gamma$,
so the law of cosines implies that
\[ 2\cos\alpha =\frac{n+4}{n+1} \qquad \text{and} \qquad 2\cos\gamma = \frac{n-4}{n-1}. \]
%\begin{align*}
%  2\cos\alpha &= \tfrac{n^2+(n+1)^2-(n-1)^2}{n(n+1)}
%    = \frac{n+4}{n+1} \\
%  2\cos\beta &= \tfrac{(n-1)^2+(n+1)^2-n^2}{(n-1)(n+1)}
%    = \frac{n^2+2}{n^2-1} \\
%  2\cos\gamma &= \tfrac{n^2+(n-1)^2-(n+1)^2}{n(n-1)}
%    = \frac{n-4}{n-1}.
%\end{align*}
\begin{claim*}
  The triangle is quirky iff there exists $r,s \in \ZZ_{\ge0}$ not both zero such that
  \[ \cos(r\alpha) = \pm \cos(s\gamma)
    \qquad \text{or equivalently} \qquad
    P_r\left( \frac{n+4}{n+1} \right) = \pm P_s\left( \frac{n-4}{n-1} \right). \]
\end{claim*}
\begin{proof}
If there are integers $x$, $y$, $z$ for which $x\alpha + y\beta + z\gamma = 0$,
then we have that $(x-y)\alpha = (y-z)\gamma - \pi y$,
whence it follows that we may take $r = |x-y|$ and $s = |y-z|$
(noting $r=s=0$ implies the absurd $x=y=z$).
Conversely, given such $r$ and $s$ with $\cos(r\alpha) = \pm \cos(s\gamma)$,
then it follows that $r \alpha \pm s \gamma = k\pi = k(\alpha + \beta + \gamma)$
for some $k$, so the triangle is quirky.
\end{proof}

If $r=0$, then by rational root theorem on $P_s(X) \pm 2$ it follows
$\frac{n-4}{n-1}$ must be an integer which occurs only when $n = 4$ (recall $n \ge 3$).
Similarly we may discard the case $s = 0$.

Thus in what follows assume $n \neq 4$ and $r,s > 0$.
Then, from the fact that $P_r$ and $P_s$ are nonconstant
monic polynomials, we find
\begin{corollary*}
  If $n \neq 4$ works, then when $\frac{n+4}{n+1}$
  and $\frac{n-4}{n-1}$ are written as fractions in lowest terms,
  the denominators have the same set of prime factors.
\end{corollary*}
But $\gcd(n+1, n-1)$ divides $2$, and $\gcd(n+4, n+1)$, $\gcd(n-4,n-1)$ divide $3$.
So we only have three possibilities:
\begin{itemize}
  \ii $n+1 = 2^u$ and $n-1 = 2^v$ for some $u,v \ge 0$.
  This is only possible if $n = 3$.
  Here $2\cos\alpha = \frac74$ and $2\cos\gamma = -\frac12$,
  and indeed $P_2(-1/2) = -7/4$.
  \ii $n+1 = 3 \cdot 2^u$ and $n-1 = 2^v$ for some $u,v \ge 0$,
  which implies $n = 5$.
  Here $2\cos\alpha = \frac32$ and $2\cos\gamma = \frac14$,
  and indeed $P_2(3/2) = 1/4$.
  \ii $n+1 = 2^u$ and $n-1 = 3 \cdot 2^v$ for some $u,v \ge 0$,
  which implies $n = 7$.
  Here $2\cos\alpha = \frac{11}{8}$ and $2\cos\gamma = \frac12$,
  and indeed $P_3(1/2) = -11/8$.
\end{itemize}
Finally, $n=4$ works because the triangle is right, completing the solution.

\begin{remark*}
  [Major generalization due to Luke Robitaille]
  In fact one may find all quirky triangles
  whose sides are integers in arithmetic progression.

  Indeed, if the side lengths of the triangle are $x-y$, $x$, $x+y$
  with $\gcd(x,y) = 1$ then the problem becomes
  \[ P_r\left( \frac{x+4y}{x+y} \right) = \pm P_s \left( \frac{x-4y}{x-y} \right) \]
  and so in the same way as before,
  we ought to have $x+y$ and $x-y$ are both of the form $3 \cdot 2^{\ast}$ unless $rs = 0$.
  This time, when $rs=0$, we get the extra solutions $(1,0)$ and $(5,2)$.

  For $rs \neq 0$, by triangle inequality, we have $x-y \le x+y < 3(x-y)$,
  and $\min(\nu_2(x-y), \nu_2(x+y)) \le 1$,
  so it follows one of $x-y$ or $x+y$ must be in $\{1,2,3,6\}$.
  An exhaustive check then leads to
  \[ (x,y) \in \left\{ (3,1), (5,1), (7,1), (11,5) \right\}
    \cup \left\{ (1,0), (5,2), (4,1) \right\} \]
  as the solution set. And in fact they all work.

  In conclusion the equilateral triangle, $3-5-7$ triangle (which has a $120\dg$ angle)
  and $6-11-16$ triangle (which satisfies $B=3A+4C$)
  are exactly the new quirky triangles (up to similarity)
  whose sides are integers in arithmetic progression.
\end{remark*}

---

One day, dumby was thinking about triangle.
He notice 4-5-6 triangle has a linear relation from old problem
(since $\angle B = 2\angle C$),
and of course 3-4-5 does too since it has right angle.
Then he mistakenly think 5-6-7 triangle has $60\dg$ angle,
(in fact it is 5-7-8).
He became curious whether this always worked,
and use Python found relations for $n=3,4,5,7$ above,
although sadly not $n=6$.
Dumby show problem to babby and babby solve.
