desc: $\theta$ functional equation from polynomials to $\ZZ$
author: Evan Chen, Yang Liu
source: TSTST 2018/1
tags: [2018-06, FE, nice, mine, good, rushdown, rembound, intuitive, divis, induct, aleph]
hardness: 10
url: https://aops.com/community/p10570981

---

As usual, let $\ZZ[x]$ denote the set of single-variable
polynomials in $x$ with integer coefficients.
Find all functions $\theta \colon \ZZ[x] \to \ZZ$
such that for any polynomials $p,q \in \ZZ[x]$,
\begin{itemize}
  \ii $\theta(p+1) = \theta(p)+1$, and
  \ii if $\theta(p) \neq 0$
  then $\theta(p)$ divides $\theta(p \cdot q)$.
\end{itemize}

---

The answer is $\theta : p \mapsto p(c)$, for each choice of $c \in \ZZ$.
Obviously these work, so we prove these are the only ones.
In what follows, $x \in \ZZ[x]$ is the identity polynomial,
and $c = \theta(x)$.

\paragraph{First solution (Merlijn Staps).}
Consider an integer $n \neq c$.
Because $x-n \mid p(x)-p(n)$, we have
\[ \theta(x-n) \mid \theta(p(x)-p(n))
  \implies c - n \mid \theta(p(x)) - p(n).
\]
On the other hand, $c - n \mid p(c) - p(n)$.
Combining the previous two gives $c - n \mid \theta(p(x)) - p(c)$,
and by letting $n$ large we conclude
$\theta(p(x)) - p(c) = 0$, so $\theta(p(x)) = p(c)$.

\paragraph{Second solution.}
First, we settle the case $\deg p = 0$.
In that case, from the second property,
$\theta(m) = m + \theta(0)$ for every integer $m \in \ZZ$
(viewed as a constant polynomial).
Thus $m + \theta(0) \mid 2m + \theta(0)$,
hence $m + \theta(0) \mid -\theta(0)$,
so $\theta(0) = 0$ by taking $m$ large.
Thus $\theta(m) = m$ for $m \in \ZZ$.

Next, we address the case of $\deg p = 1$.
We know $\theta(x+b) = c+b$ for $b \in \ZZ$.
Now for each particular $a \in \ZZ$,
we have
\[ c+k \mid \theta(x+k) \mid \theta(ax+ak) = \theta(ax) + ak
  \implies c+k \mid \theta(ax) - ac. \]
for any $k \neq -c$.
Since this is true for large enough $k$,
we conclude $\theta(ax) = ac$.
Thus $\theta(ax+b) = ac+b$.

We now proceed by induction on $\deg p$.
Fix a polynomial $p$ and assume it's true
for all $p$ of smaller degree.
Choose a large integer $n$ (to be determined later)
for which $p(n) \neq p(c)$.
We then have
\[ \frac{p(c)-p(n)}{c-n}
  = \theta\left( \frac{p-p(n)}{x-n} \right)
  \mid \theta\left( p-p(n) \right)
  = \theta(p) - p(n). \]
Subtracting off $c-n$ times the left-hand side gives
\[ \frac{p(c)-p(n)}{c-n}
  \mid \theta(p) - p(c). \]
The left-hand side can be made arbitrarily large
by letting $n \to \infty$, since $\deg p \ge 2$.
Thus $\theta(p) = p(c)$, concluding the proof.

\paragraph{Authorship comments.}
I will tell you a story about the creation of this problem.
Yang Liu and I were looking over the drafts of December and
January TST in October 2017,
and both of us had the impression that the test was too difficult.
This sparked a non-serious suggestion that we should try to
come up with a problem \emph{now} that would be easy enough to use.
While we ended up just joking about changing the TST,
we did get this problem out of it.

Our idea was to come up with a functional equation that
was different from the usual fare:
at first we tried $\ZZ[x] \to \ZZ[x]$,
but then I suggested the idea of using $\ZZ[x] \to \ZZ$,
with the answer being the ``evaluation'' map.
Well, what properties does that satisfy?
One answer was $a-b \mid p(a)-p(b)$;
this didn't immediately lead to anything,
but eventually we hit on the form of the problem above off this idea.
At first we didn't require $\theta(p) \neq 0$ in the bullet,
but without the condition the problem was too easy,
since $0$ divides only itself;
and so the condition was added and we got the functional equation.

I proposed the problem to USAMO 2018, but it was rejected
(unsurprisingly; I think the problem may be
too abstract for novice contestants).
Instead it was used for TSTST, which I thought fit better.
