source: USA TST 2019/5
desc: Tasty and Stacy
author: Yannick Yao
tags: [2018-10, process, understand, algorithm, invariant, smallcases, nice, zayin]
hardness: 30
url: https://aops.com/community/p11625809

---

Let $n$ be a positive integer.
Tasty and Stacy are given a circular necklace
with $3n$ sapphire beads and $3n$ turquoise beads,
such that no three consecutive beads have the same color.
They play a cooperative game where they alternate turns
removing three consecutive beads, subject to the following conditions:
\begin{itemize}
   \ii Tasty must remove three consecutive beads
   which are turquoise, sapphire, and turquoise, in that order,
   on each of his turns.
   \ii Stacy must remove three consecutive beads
   which are sapphire, turquoise, and sapphire, in that order,
   on each of her turns.
\end{itemize}
They win if all the beads are removed in $2n$ turns.
Prove that if they can win with Tasty going first,
they can also win with Stacy going first.

---

In the necklace, we draw a \emph{divider}
between any two beads of the same color.
Unless there are no dividers,
this divides the necklace into several \emph{zigzags}
in which the beads in each zigzag alternate.
Each zigzag has two \emph{endpoints}
(adjacent to dividers).

Observe that the condition about not having
three consecutive matching beads is equivalent
to saying there are no zigzags of lengths $1$.

\begin{center}
\begin{asy}
  size(7cm);
  string[] s = {"T", "S", "S", "T", "T", "S", "T", "T", "S", "T", "T", "S", "S", "T", "S", "T", "S", "S", "T", "S", "S", "T", "S", "T"};
  pen[] colors = {white, red, deepgreen, blue, heavygreen, grey,
    purple, orange, deepcyan};
  int j = 0;
  for (int i=0; i<24; ++i) {
    if ((i == 0) || (s[i] == s[i-1])) {
      ++j;
      draw(0.9*dir(187.5-15*i)--1.1*dir(187.5-15*i), black+1.5);
    }
    label(s[i], dir(180-15*i), dir(180-15*i), colors[j]);
  }
  draw(unitcircle);
\end{asy}
\end{center}

The main claim is that
the game is winnable (for either player going first)
if and only if there are at most $2n$ dividers.
We prove this in two parts,
the first part not using the hypothesis about three consecutive letters.

\begin{claim*}
  The game cannot be won with Tasty going first
  if there are more than $2n$ dividers.
\end{claim*}
\begin{proof}
  We claim each move removes at most one divider,
  which proves the result.

  Consider removing a $TST$ in some zigzag
  (necessarily of length at least $3$).
  We illustrate the three possibilities in the following table,
  with Tasty's move shown in red.
  \begin{center}
  \begin{tabular}{ccl}
    Before & After & Change \\ \hline
    % ------------
    $\dots \mathtt{ST} \mid \mathtt{\color{red} TST}
      \mid \mathtt{TS}\dots$
      & $\dots \mathtt{ST} \mid \mathtt{TS} \dots$
      & One less divider; two zigzags merge \\
    $\dots \mathtt{ST} \mid \mathtt{\color{red} TST}
      \mathtt{ST}\dots$
      & $\dots \mathtt{STST} \dots$
      & One less divider; two zigzags merge \\
    $\dots \mathtt{S} \mathtt{\color{red} TST} \mathtt{S} \dots$
      & $\dots \mathtt{S} \mid \mathtt{S} \dots$
      & One more divider; a zigzag splits in two
  \end{tabular}
  \end{center}
  The analysis for Stacy's move is identical.
\end{proof}

\begin{claim*}
  If there are at most $2n$ dividers
  and there are no zigzags of length $1$
  then the game can be won (with either player going first).
\end{claim*}
\begin{proof}
  By symmetry it is enough to prove Tasty wins going first.

  At any point if there are no dividers at all,
  then the necklace alternates $TSTST\dots$
  and the game can be won.
  So we will prove that on each of Tasty's turns,
  if there exists at least one divider,
  then Tasty and Stacy can each make a move
  at an endpoint of some zigzag
  (i.e.\ the first two cases above).
  As we saw in the previous proof, such moves will
  (a) decrease the number of dividers by exactly one,
  (b) not introduce any singleton zigzags
  (because the old zigzags merge, rather than split).
  Since there are fewer than $2n$ dividers,
  our duo can eliminate all dividers and then win.

  Note that as the number of $S$ and $T$'s are equal,
  there must be an equal number of
  \begin{itemize}
    \ii zigzags of odd length ($\ge 3$) with $T$ at the endpoints
    (i.e.\ one more $T$ than $S$), and
    \ii zigzags of odd length ($\ge 3$) with $S$ at the endpoints
    (i.e.\ one more $S$ than $T$).
  \end{itemize}
  Now iff there is at least one of each,
  then Tasty removes a $TST$ from the end of such a zigzag
  while Stacy removes an $STS$ from the end of such a zigzag.

  Otherwise suppose all zigzags have even size.
  Then Tasty finds any zigzag of length $\ge 4$
  (which must exist since the \emph{average} zigzag length is $3$)
  and removes $TST$ from the end containing $T$.
  The resulting merged zigzag is odd and hence $S$ endpoints,
  hence Stacy can move as well.
\end{proof}

\begin{remark*}
  There are many equivalent ways to phrase the solution.
  For example, the number of dividers is equal to the
  number of pairs of two consecutive letters
  (rather than singleton letters).
  So the win condition can also be phrased in terms
  of the number of adjacent pairs of letters being at least $2n$,
  or equivalently the number of differing pairs being at least $4n$.

  If one thinks about the game as a process,
  this is a natural ``monovariant'' to consider anyways,
  so the solution is not so unmotivated.
\end{remark*}

\begin{remark*}
  The constraint of no three consecutive identical beads is actually
  needed: a counterexample without this constraint is
  $\mathtt{TTSTSTSTTSSS}$.
  (They win if Tasty goes first and lose if Stacy goes first.)
\end{remark*}

\begin{remark*}
  [Why induction is unlikely to work]
  Many contestants attempted induction.
  However, in doing so they often implicitly proved
  a different problem: ``prove that if they can win
  with Tasty going first \emph{without ever creating a triplet},
  they can also win in such a way with Stacy going first''.
  This essentially means nearly all induction attempts fail.

  Amusingly, even the modified problem
  (which is much more amenable to induction)
  sill seems difficult without \emph{some} sort of global argument.
  Consider a position in which Tasty wins going first,
  with the sequence of winning moves being Tasty's first move in red
  below and Stacy's second move in blue below:
  \[
    \dots \mathtt{TTSSTT}
    \underbrace{\mathtt{\color{blue}S}
    \overbrace{\mathtt{\color{red}TST}}^{\text{Tasty}}
    \mathtt{\color{blue}TS}}_{\text{Stacy}}
    \mathtt{STTSST} \dots.
  \]
  There is no ``nearby'' $STS$ that Stacy can remove
  instead on her first turn,
  without introducing a triple-$T$ and also
  preventing Tasty from taking a $TST$.
  So it does not seem possible to easily change
  a Tasty-first winning sequence to a Stacy-first one,
  even in the modified version.
\end{remark*}
