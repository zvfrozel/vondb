desc: $\binom{n}{k}$ partition into equal sizes
source: USA TST 2024/3
url: https://aops.com/community/p29409068
tags: [2023-12, genfunc, polynomial, cyclotomic, zayin]
author: Ankan Bhattacharya

---

Let $n > k \ge 1$ be integers and let $p$ be a prime dividing $\tbinom nk$.
Prove that the $k$-element subsets of $\{1, \dots, n\}$ can be
split into $p$ classes of equal size, such that
any two subsets with the same sum of elements belong to the same class.

---

Let $\sigma(S)$ denote the sum of the elements of $S$, so that
\[ P(x) \coloneqq \sum_{\substack{S\subseteq \{1, \dots, n\} \\ |S|=k}} x^{\sigma(S)} \]
is the generating function for the sums of $k$-element subsets of $\{1, \dots, n\}$.

By Legendre's formula,
\[ \nu_p\left(\binom{n}{k}\right) = \sum_{r=1}^{\infty}
  \left(\left\lfloor\frac{n}{p^r}\right\rfloor - \left\lfloor\frac{k}{p^r}\right\rfloor
  - \left\lfloor\frac{n-k}{p^r}\right\rfloor\right) \]
so there exists a positive integer $r$ with
\[ \left\lfloor\frac{n}{p^r}\right\rfloor - \left\lfloor\frac{k}{p^r}\right\rfloor
- \left\lfloor\frac{n-k}{p^r}\right\rfloor > 0. \]

The main claim is the following:
\begin{claim*}
  $P(x)$ is divisible by
  \[
    \Phi_{p^r}(x) = x^{(p-1)p^{r-1}} + \dots + x^{p^{r-1}} + 1.
  \]
\end{claim*}

Before proving this claim, we will show how it solves the problem. It implies
that there exists a polynomial $Q$ with integer coefficients satisfying
\begin{align*}
	P(x) & = \Phi_{p^r}(x) Q(x)\\
	& = (x^{(p-1)p^{r-1}} + \dots + x^{p^{r-1}} + 1) Q(x).
\end{align*}
Let $c_0,\ c_1,\ \dots$ denote the coefficients of $P$, and define
\[ s_i = \sum_{j \equiv i \pmod{p^r}} c_j. \]
Then it's easy to see that
\begin{alignat*}{4}
	& s_0 && = s_{p^{r-1}} && = \cdots && = s_{(p-1)p^{r-1}}\\
	& s_1 && = s_{p^{r-1}+1} && = \cdots && = s_{(p-1)p^{r-1}+1}\\
	&  &&  && \vdots &&\\
	& s_{p^{r-1}-1} && = s_{2p^{r-1}-1} && = \cdots && = s_{p^r-1}.
\end{alignat*}
This means we can construct the $p$ classes by placing a set with sum $z$ in
class $\left\lfloor\frac{z\mod{p^r}}{p^{r-1}}\right\rfloor$.

Now we present two ways to prove the claim.

\paragraph{First proof of claim.}
There's a natural bijection between $k$-element subsets of $\{1, \dots, n\}$
and binary strings consisting of $k$ zeroes and $\ell$ ones:
the set $\{a_1, \dots, a_k\}$ corresponds to the string
which has zeroes at positions $a_1,\ \dots,\ a_k$.
Moreover, the inversion count of this string is simply $(a_1 + \dots + a_k) - \tfrac12k(k+1)$,
so we only deal with these inversion counts (equivalently, we are factoring
$x^{\frac{k(k+1)}{2}}$ out of $P$).

Recall that the generating function for these inversion counts is given by the
$q$-binomial coefficient
\[
	P(x) = \frac{(x-1) \cdots (x^{k+\ell}-1)}
	{\big[(x-1)\cdots(x^k-1)\big] \times \big[(x-1)\cdots(x^\ell-1)\big]}.
\]
By choice of $r$, the numerator of $P(x)$ has more factors of $\Phi_{p^r}(x)$
than the denominator, so $\Phi_{p^r}(x)$ divides $P(x)$.

\begin{remark*}
  Here is a proof that $P(x)$ is divisible by $\Phi_{p^r}(x)$ for some $r$ using
  the $q$-binomial formula, without explicitly identifying $r$. We know that
  $P(x)$ is the product of several cyclotomic polynomials, and that $P(1)$ is a
  multiple of $p$. Thus there is a factor $\Phi_q(x)$ for which $\Phi_q(1)$ is a
  multiple of $p$, which is equivalent to $q$ being a power of $p$.
\end{remark*}

\paragraph{Second proof of claim.}
Note that $P(x)$ is the coefficient of $y^k$ in the polynomial
\[ Q(x, y) \coloneqq (1+xy)(1+x^2y)\dotsm(1+x^ny). \]

Let $a$ be the remainder when $n$ is divided by $p^r$, and let $b$ be the
remainder when $k$ is divided by $p^r$; then we have $a<b$ by the choice of $r$.
Let $q = \lfloor n/p^r\rfloor$ so $n=p^rq+a$.
Consider taking $x$ to be a primitive $p^r$th root of unity, say $\omega$. Then
\[
  Q(\omega, y) = \left[(1+\omega y)(1+\omega^2y)\cdots (1+\omega^{p^r}y)\right]^q
  (1+\omega y)(1+\omega^2y)\dots (1+\omega^a y).
\]
Now $\omega$, $\omega^2$, \dots, $\omega^{p^r}$ are all the $p^r$th roots of unity,
each exactly once; then we can see that
\begin{align*}
  & (1+\omega y)(1+\omega^2y)\cdots (1+\omega^{p^r}y) \\
  &= (1-\omega(-y))(1-\omega^2(-y))\cdots (1-\omega^{p^r}(-y)) \\
  &= 1-(-y)^{p^r},
\end{align*}
so
\[ Q(\omega,y) = (1-(-y)^{p^r})^q (1+\omega y)(1+\omega^2y)\dots (1+\omega^ay). \]
In particular, for any $m$, if the coefficient of $y^m$ in $Q(w,x)$ is nonzero,
then $m$ must be congruent to one of $0,1, \dots, a \pmod{p^r}$.
Therefore the coefficient of $y^k$ in $Q(\omega, y)$ is zero.
This means that $P(\omega)=0$ whenever $\omega$ is a primitive $p^r$th root
of unity, which proves the claim.
