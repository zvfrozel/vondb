desc: Cheating at trivia, $2^{n-1}$ version
author: Linus Hamilton
source: USA TST 2017/4
tags: [2017-01, nice, well, instructive, global, thinkbig, weak, meta, wishful, algorithm,
  smallcases, scouting, inefficient, understand, dalet]
hardness: 30
url: https://aops.com/community/p7732191

---

You are cheating at a trivia contest.
For each question, you can peek at each of the
$n > 1$ other contestant's guesses before writing your own.
For each question, after all guesses are submitted, the emcee announces the correct answer.
A correct guess is worth $0$ points.
An incorrect guess is worth $-2$ points for other contestants,
but only $-1$ point for you, because you hacked the scoring system.
After announcing the correct answer, the emcee proceeds to read out the next question.
Show that if you are leading by $2^{n-1}$ points at any time,
then you can surely win first place.

---

We will prove the result with $2^{n-1}$ replaced
even by $2^{n-2}+1$.

We first make the following reductions.
First, change the weights to be $+1$, $-1$, $0$ respectively
(rather than $0$, $-2$, $-1$); this clearly has no effect.
Also, WLOG that all contestants except you initially have score zero
(and that your score exceeds $2^{n-2}$).
WLOG ignore rounds in which all answers are the same.
Finally, ignore rounds in which you get the correct answer,
since that leaves you at least as well off as before --- in other words,
we'll assume your score is always fixed,
but you can pick any group of people with the same answers
and ensure they lose 1 point,
while some other group gains $1$ point.

The key observation is the following.
Consider two rounds $R_1$ and $R_2$ such that:
\begin{itemize}
  \ii In round $R_1$, some set $S$ of contestants gains a point.
  \ii In round $R_2$, the set $S$ of contestants all have the same answer.
\end{itemize}
Then, if we copy the answers of contestants in $S$ during $R_2$,
then the sum of the scorings in $R_1$ and $R_2$ cancel each other out.
In other words we can then ignore $R_1$ and $R_2$ forever.

We thus consider the following strategy.
We keep a list $\mathcal L$ of subsets of $\{1, \dots, n\}$, initially empty.
Now do the following strategy:
\begin{itemize}
  \ii On a round, suppose there exists a set $S$ of people
  with the same answer such that $S \in \mathcal L$.
  $($If multiple exist, choose one arbitrarily.)
  Then, copy the answer of $S$, causing them to lose a point.
  Delete $S$ from $\mathcal L$.
  (Importantly, we do not add any new sets to $\mathcal L$.)
  \ii Otherwise, copy any set $T$ of contestants,
  selecting $|T| \ge n/2$ if possible.
  Let $S$ be the set of contestants who answer correctly (if any),
  and add $S$ to the list $\mathcal L$.
  Note that $|S| \le n/2$, since $S$ is disjoint from $T$.
\end{itemize}
By construction, $\mathcal L$ has no duplicate sets.
So the score of any contestant $c$ is bounded above by the number of times
that $c$ appears among sets in $\mathcal L$.
The number of such sets is clearly at most $\half \cdot 2^{n-1}$.
So, if you lead by $2^{n-2}+1$ then you ensure victory.
This completes the proof!

\begin{remark*}
Several remarks are in order.
First, we comment on the bound $2^{n-2} + 1$ itself.
The most natural solution using only the
list idea gives an upper bound of $(2^n-2)+1$,
which is the number of nonempty proper subsets of $\{1, \dots, n\}$.
Then, there are two optimizations one can observe:
\begin{itemize}
  \ii In fact we can improve to the number of times
  any particular contestant $c$ appears in some set,
  rather than the total number of sets.
  \ii When adding new sets $S$ to $\mathcal L$, one can ensure $|S| \le n/2$.
\end{itemize}
Either observation alone improves the bound from $2^n-1$ to $2^{n-1}$,
but both together give the bound $2^{n-2} + 1$.
Additionally, when $n$ is odd the calculation of subsets
actually gives $2^{n-2} - \half \binom{n-1}{\frac{n-1}{2}} + 1$.
This gives the best possible value at both $n=2$ and $n=3$.
It seems likely some further improvements are possible,
and the true bound is suspected to be polynomial in $n$.

Secondly, the solution is highly motivated by considering a true/false contest
in which only two distinct answers are given per question.
However, a natural mistake (which graders assessed as a two-point deduction)
is to try and prove that in fact
one can ``WLOG'' we are in the two-question case.
The proof of this requires substantially more care than expected.
For instance, set $n = 3$.
If $\mathcal L = \left\{ \{1\}, \{2\}, \{3\}  \right\}$
then it becomes impossible to prevent a duplicate set from appearing
in $\mathcal L$ if all contestants give distinct answers.
One might attempt to fix this by instead adding to $\mathcal L$
the \emph{complement} of the set $T$ described above.
The example $\mathcal L = \left\{ \{1,2\}, \{2,3\}, \{3,1\}  \right\}$
(followed again by a round with all distinct answers)
shows that this proposed fix does not work either.
This issue affects all variations of the above approach.

Because the USA TST did not have any solution-writing process at this time,
this issue was not noticed until January 15 (less than a week before the exam).
When it was brought up by email by Evan,
every organizer who had testsolved the problem had apparently made the same error.
\end{remark*}
\begin{remark*}
  Here are some motivations for the solution:
  \begin{enumerate}
    \ii The exponential bound $2^n$ suggests looking at subsets.
    \ii The $n = 2$ case suggests the idea of ``repeated rounds''.
    (I think this $n=2$ case is actually really good.)
    \ii The ``two distinct answers'' case suggests looking at rounds
    as partitions (even though the WLOG does not work,
    at least not without further thought).
    \ii There's something weird about this problem:
    it's a finite bound over unbounded time.
    This is a hint to \emph{not} worry excessively about
    the actual scores, which turn out to be almost irrelevant.
  \end{enumerate}
\end{remark*}

---

Hint: \emph{not} monovariants.
(Also, to clarify, questions can have more than
two possible answers.)
\begin{walk}
  \ii Show that we can WLOG assume that you answer all questions incorrectly,
  by discarding any rounds where you get the answer correct.
  \ii By shifting the scores by $1$,
  re-frame the problem in such a way your score never changes.
  This should make the problem easier to think about.
  \ii Assume for now that in each round, only two distinct answers are given.
  Thus each round corresponds to a partition of $\{1, \dots, n\}$ into two halves.
  How many possible rounds are there?
  If two identical rounds occur, can you cancel them out?
  \ii Solve the problem in the special case described in (c).
  \ii Extend the solution to the general case.
  (This requires some care.  It is tempting to try and argue
  that the assumption in (c) can be made without loss of generality,
  but as far as I know this is not actually true.
  Instead, keep track of a history of sets of contestants answering correctly.
  See the remarks in the solution for further details, including a cautionary story
  of why you should always write solutions for contests you organize.)
  \ii Optional: can you improve the bound to $2^{n-2}+1$?
\end{walk}
The right mentality for the problem is to really think about the process,
which is what leads to the observations such as those in (a) - (c),
rather than immediately trying to contrive a monovariant
(which was a common failure mode for the problem).
