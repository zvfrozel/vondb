desc: Graph with 2017 edges
author: Evan Chen
source: USA TST 2018/3
tags: [2016-10, graph, mine, favorite, local, global, instructive, perturb, algorithm,
  find, bestpossible, yod]
hardness: 45
url: https://aops.com/community/p9513105

---

At a university dinner,
there are $2017$ mathematicians who each order two distinct entr\'ees,
with no two mathematicians ordering the same pair of entr\'{e}es.
The cost of each entr\'ee is equal
to the number of mathematicians who ordered it,
and the university pays for each mathematician's
less expensive entr\'ee (ties broken arbitrarily).
Over all possible sets of orders,
what is the maximum total amount the university could have paid?

---

In graph theoretic terms:
we wish to determine the maximum possible value of
\[ S(G) \coloneqq \sum_{e = vw} \min \left( \deg v, \deg w \right) \]
across all graphs $G$ with $2017$ edges.
We claim the answer is $63 \cdot \binom{64}{2} + 1 = 127009$.

\paragraph{First solution (combinatorial, Evan Chen).}
First define $L_k$ to consist of a clique on $k$ vertices,
plus a single vertex connected to exactly one vertex of the clique.
Hence $L_k$ has $k+1$ vertices, $\binom k2+1$ edges,
and $S(L_k) = (k-1) \binom k2 + 1$.
In particular, $L_{64}$ achieves the claimed maximum,
so it suffices to prove the upper bound.

\begin{lemma*}
  Let $G$ be a graph such that either
  \begin{itemize}
    \ii $G$ has $\binom k2$ edges for some $k \ge 3$ or
    \ii $G$ has $\binom k2 + 1$ edges for some $k \ge 4$.
  \end{itemize}
  Then there exists a graph $G^\ast$ with the same number of edges
  such that $S(G^\ast) \ge S(G)$, and moreover $G^\ast$ has a universal vertex
  (i.e.\ a vertex adjacent to every other vertex).
\end{lemma*}

\begin{proof}
  Fix $k$ and the number $m$ of edges.
  We prove the result by induction on the number $n$ of vertices in $G$.
  Since the lemma has two parts,
  we will need two different base cases:
  \begin{enumerate}
    \item Suppose $n = k$ and $m = \binom k2$.
    Then $G$ must be a clique so pick $G^\ast = G$.

    \item Suppose $n = k+1$ and $m = \binom k2 + 1$.
    If $G$ has no universal vertex, we claim we may take $G^\ast = L_k$.
    Indeed each vertex of $G$ has degree at most $k-1$, and the average degree is
    \[ \frac{2m}{n} = \frac{k^2-k+2}{k+1} < k-1 \]
    using here $k \ge 4$.
    Thus there exists a vertex $w$ of degree $1 \le d \le k-2$.
    The edges touching $w$ will have label at most $d$ and hence
    \begin{align*}
       S(G) &\le (k-1)(m-d) + d^2 = (k-1)m - d(k-1-d) \\
       &\le (k-1)m - (k-2) = (k-1) \binom k2 + 1 = S(G^\ast).
    \end{align*}
  \end{enumerate}
  Now we settle the inductive step.
  Let $w$ be a vertex with minimal degree $0 \le d < k-1$, with neighbors $w_1$, \dots, $w_d$.
  By our assumption, for each $w_i$ there exists a vertex $v_i$ for which $v_i w_i \notin E$.
  Now, we may delete all edges $ww_i$ and in their place put $v_i w_i$,
  and then delete the vertex $w$.
  This gives a graph $G'$, possibly with multiple edges
  (if $v_i = w_j$ and $w_j = v_i$), and with one fewer vertex.

  \begin{center}
  \begin{asy}
    defaultpen(fontsize(11pt));
    unitsize(1cm);

    dotfactor *= 1.5;
    pair A = (-1,1);
    pair B = (0,1);
    pair X = (-1,0);
    pair Y = (0,0);
    pair Z = (1,0);
    pair W = (0,-1);

    picture[] Gs = new picture[3];
    Gs[0] = new picture;
    Gs[1] = new picture;
    Gs[2] = new picture;


    draw(Gs[0], W--X, red);
    draw(Gs[0], W--Y, red);
    draw(Gs[0], W--Z, red);

    draw(Gs[1], X--B, red);
    draw(Gs[1], Y..(0.5, 0.2)..Z, red);
    draw(Gs[1], Y..(0.5,-0.2)..Z, red);

    draw(Gs[2], X--B, red);
    draw(Gs[2], Y..(0.5,-0.2)..Z, red);
    draw(Gs[2], A--Y, blue);

    for (int i=0; i<3; ++i) {
      dot(Gs[i], A);
      dot(Gs[i], B);
      dot(Gs[i], X);
      dot(Gs[i], Y);
      dot(Gs[i], Z);
      draw(Gs[i], A--B--Z--A--X--Y--B);
    }
    dot(Gs[0], "$w$", W, dir(-90));

    label(Gs[0], "$G$", B, 2*dir(90));
    label(Gs[1], "$G'$", B, 2*dir(90));
    label(Gs[2], "$G''$", B, 2*dir(90));

    add(shift(-3,0) * Gs[0]);
    add(shift(0,0) * Gs[1]);
    add(shift(3,0) * Gs[2]);
  \end{asy}
  \end{center}

  We then construct a graph $G''$ by taking any pair of double edges,
  deleting one of them, and adding any missing edge of $G''$ in its place.
  (This is always possible, since when $m = \binom k2$ we have $n-1 \ge k$
  and when $m = \binom k2 +1$ we have $n-1 \ge k+1$.)

  Thus we have arrived at a simple graph $G''$ with one fewer vertex.
  We also observe that we have $S(G'') \ge S(G)$;
  after all every vertex in $G''$ has degree at least as large as it did in $G$,
  and the $d$ edges we deleted have been replaced with
  new edges which will have labels at least $d$.
  Hence we may apply the inductive hypothesis to the graph $G''$
  to obtain $G^\ast$ with $S(G^\ast) \ge S(G'') \ge S(G)$.
\end{proof}

The problem then is completed once we prove the following:
\begin{claim*}
  For any graph $G$,
  \begin{itemize}
    \ii If $G$ has $\binom k2$ edges for $k \ge 3$,
    then $S(G) \le \binom k2 \cdot (k-1)$.
    \ii If $G$ has $\binom k2 + 1$ edges for $k \ge 4$,
    then $S(G) \le \binom k2 \cdot (k-1) + 1$.
  \end{itemize}
\end{claim*}

\begin{proof}
We prove both parts at once by induction on $k$,
with the base case $k = 3$ being plain
(there is nothing to prove in the second part for $k=3$).
Thus assume $k \ge 4$.
By the earlier lemma, we may assume $G$ has a universal vertex $v$.
For notational convenience, we say $G$ has $\binom k2 + \eps$ edges
for $\eps \in \{0,1\}$, and $G$ has $p+1$ vertices, where $p \ge k-1 + \eps$.

Let $H$ be the subgraph obtained when $v$ is deleted.
Then $m = \binom k2 + \eps - p$ is the number of edges in $H$;
from $p \ge k-1+\eps$ we have $m \le \binom{k-1}{2}$
and so we may apply the inductive hypothesis to $H$ to deduce
$S(H) \le \binom{k-1}{2} \cdot (k-2)$.

\begin{center}
\begin{asy}
defaultpen(fontsize(11pt));
size(4.5cm);

dotfactor *= 1.5;
pair V = (0,2);

pair A = (-1.5,0);
pair B = (-1.0,0);
pair C = (+1.5,0);

for (real x=-1.5; x<=1.5; x+=0.5) {
  draw(V--(x,0));
}
label("$\dots$", midpoint(B--C));
dot("$w_1$", A, dir(-90));
dot("$w_2$", B, dir(-90));
dot("$w_p$", C, dir(-90));
dot("$v$", V, dir(90), red);


real EX = 2;
real EY = 0.8;
draw(ellipse((0,0), EX, EY), dashed);
label("$H$", (EX, -EY));
\end{asy}
\end{center}
Now the labels of edges $vw_i$ have sum
\[
  \sum_{i=1}^p \min \left( \deg_G v, \deg_G w_i \right)
  = \sum_{i=1}^p \deg_G w_i
  = \sum_{i=1}^p (\deg_H w_i + 1) =  2m + p.
\]
For each of the edges contained in $H$,
the label on that edge has increased by exactly $1$,
so those edges contribute $S(H)+m$.
In total,
\begin{align*}
  S(G) &= 2m + p + \left( S(H)+m \right) = (m+p) + 2m + S(H) \\
  &\le \binom k2 + \eps + 2\binom{k-1}{2} + \binom{k-1}{2}(k-2)
  = \binom k2 (k-1) + \eps. \qedhere
\end{align*}
\end{proof}

\paragraph{Second solution (algebraic, submitted by contestant James Lin).}
We give a different proof of $S(G) \le 127009$.
The proof proceeds using the following two claims,
which will show that $S(G) \le 127010$ for all graphs $G$.
Then a careful analysis of the equality cases will show
that this bound is not achieved for any graph $G$.
Since the example $L_{64}$ earlier has $S(L_{64}) = 127009$,
this will solve the problem.

\begin{lemma*}
  [Combinatorial bound]
  Let $G$ be a graph with $2017$ edges and let
  $d_1 \ge d_2 \ge \dots \ge d_n$ be the degree sequence of the graph
  (thus $n \ge 65$).
  Then \[ S(G) \le d_2 + 2d_3 + 3d_4 + \dots + 63d_{64} + d_{65}. \]
\end{lemma*}
\begin{proof}
  Let $v_1$, \dots, $v_n$ be the corresponding vertices.
  For any edge $e = \{v_i, v_j\}$ with $i < j$,
  we consider associating each edge $e$ with $v_j$,
  and computing the sum $S(G)$ indexing over associated vertices.
  To be precise, if we let $a_i$ denote the number of edges
  associated to $v_i$, we now have $a_i \le i-1$, $\sum a_i = 2017$, and
  \[ S(G) = \sum_{i=1}^n a_i d_i. \]
  The inequality
  $\sum a_i d_i \le d_2 + 2d_3 + 3d_4 + \dots + 63d_{64} + d_{65}$
  then follows for smoothing reasons
  (by ``smoothing'' the $a_i$), since the $d_i$ are monotone.
  This proves the given inequality.
\end{proof}

Once we have this property, we handle the bounding completely algebraically.
\begin{lemma*}
  [Algebraic bound]
  Let $x_1 \ge x_2 \ge \dots \ge x_{65}$ be any nonnegative integers
  such that $\sum_{i=1}^{65} x_i \le 4034$.
  Then \[ x_2 + 2x_3 + \dots + 63x_{64} + x_{65} \le 127010. \]
  Moreover, equality occurs if and only if
  $x_1 = x_2 = x_3 = \dots = x_{64} = 63$ and $x_{65} = 2$.
\end{lemma*}
\begin{proof}
  Let $A$ denote the left-hand side of the inequality.
  We begin with a smoothing argument.
  \begin{itemize}
    \ii Suppose there are indices $1 \le i < j \le 64$
    such that $x_i > x_{i+1} \ge x_{j-1} > x_j$.
    Then replacing $(x_i, x_j)$ by $(x_i-1, x_j+1)$
    strictly increases $A$ preserving all conditions.
    Thus we may assume all numbers in $\{x_1, \dots, x_{64}\}$
    differ by at most $1$.
    \ii Suppose $x_{65} \ge 4$.
    Then we can replace $(x_1, x_2, x_3, x_4, x_{65})$
    by $(x_1+1, x_2+1, x_3+1, x_4+1, x_{65}-4)$
    and strictly increase $A$.
    Hence we may assume $x_{65} \le 3$.
  \end{itemize}
  We will also tacitly assume $\sum x_i = 4034$,
  since otherwise we can increase $x_1$.
  These two properties leave only four sequences to examine:
  \begin{itemize}
    \ii $x_1 = x_2 = x_3 = \dots = x_{63} = 63$, $x_{64} = 62$,
    and $x_{65} = 3$, which gives $A = 126948$.
    \ii $x_1 = x_2 = x_3 = \dots = x_{63} = x_{64} = 63$ and $x_{65} = 2$,
    which gives $A = 127010$.
    \ii $x_1 = 64$, $x_2 = x_3 = \dots = x_{63} = x_{64} = 63$ and $x_{65} = 1$,
    which gives $A = 127009$.
    \ii $x_1 = x_2 = 64$, $x_3 = \dots = x_{63} = x_{64} = 63$ and $x_{65} = 0$,
    which gives $A = 127009$.
  \end{itemize}
  This proves that $A \le 127010$.
  To see that equality occurs only in the second case above,
  note that all the smoothing operations other than incrementing $x_1$ were strict,
  and that $x_1$ could not have been incremented in this way
  as $x_1 = x_2 = 63$.
\end{proof}

This shows that $S(G) \le 127010$ for all graphs $G$,
so it remains to show equality never occurs.
Retain the notation $d_i$ and $a_i$ of the combinatorial bound now;
we would need to have $d_1 = \dots = d_{64} = 63$ and $d_{65} = 2$
(in particular, deleting isolated vertices from $G$, we may assume $n=65$).
In that case, we have $a_i \le i-1$ but also $a_{65} = 2$ by definition
(the last vertex gets all edges associated to it).
Finally,
\begin{align*}
  S(G) &= \sum_{i=1}^n a_i d_i = 63(a_1 + \dots + a_{64}) + a_{65} \\
  &= 63(2017-a_{65}) + a_{65} \le 63 \cdot 2015 + 2 = 126947
\end{align*}
completing the proof.

\begin{remark*}
  Another way to finish once $S(G) \le 127010$
  is note there is a unique graph
  (up to isomorphism and deletion of universal vertices)
  with degree sequence
  $(d_1, \dots, d_{65}) = (63, \dots, 63, 2)$.
  Indeed, the complement of the graph
  has degree sequence $(1, \dots, 1, 63)$,
  and so it must be a $63$-star plus a single edge.
  One can then compute $S(G)$ explicitly for this graph.
\end{remark*}

\paragraph{Some further remarks.}

\begin{remark*}
  Interestingly, the graph $C_4$ has $\binom 32+1 = 4$ edges
  and $S(C_4) = 8$, while $S(L_3) = 7$.
  This boundary case is visible in the combinatorial solution
  in the base case of the first claim.
  It also explains why we end up with the bound $S(G) \le 127010$
  in the second algebraic solution,
  and why it is necessary to analyze the equality cases so carefully;
  observe in $k=3$ the situation $d_1 = d_2 = d_3 = d_4 = 2$.
\end{remark*}

\begin{remark*}
  Some comments about further context for this problem:
  \begin{itemize}
  \ii The obvious generalization of $2017$ to any constant
  was resolved in September 2018 by Mehtaab Sawhney and Ashwin Sah.
  The relevant paper is
  \emph{On the discrepancy between two Zagreb indices},
  published in Discrete Mathematics, Volume 341, Issue 9, pages 2575-2589.
  The arXiv link is \url{https://arxiv.org/pdf/1801.02532.pdf}.

  \ii The quantity
  \[ S(G) = \sum_{e = vw} \min \left( \deg v, \deg w \right) \]
  in the problem has an interpretation:
  it can be used to provide a bound on the
  number of triangles in a graph $G$.
  To be precise, $\# E(G) \le \frac 13 S(G)$,
  since an edge $e = vw$ is part of at most $\min(\deg v, \deg w)$ triangles.

  \ii For \emph{planar} graphs it is known $S(G) \le 18n-36$
  and it is conjectured that for $n$ large enough, $S(G) \le 18n-72$.
  See \url{https://mathoverflow.net/a/273694/70654}.
  \end{itemize}
\end{remark*}

\paragraph{Authorship comments.}
I came up with the quantity $S(G)$
in a failed attempt to provide a bound on the number of triangles in a graph,
since this is natural to consider when you do a standard
double-counting via the edges of the triangle.
I think the problem was actually APMO 1989,
and I ended up not solving the problem (the solution is much simpler),
but the quantity $S(G)$ stuck in my head for a while after that.

Later on that month I was keeping Danielle company while
she was working on art project (flower necklace),
and with not much to do except doodle on tables
I began thinking about $S(G)$ again.
I did have the sense that $S(G)$ should be maximized
at a graph close to a complete graph.
But to my frustration I could not prove it for a long time.
Finally after many hours of trying various approaches
I was able to at least show that $S(G)$ was maximized
for complete graphs if the number of edges was a triangular number.

I had come up with this in March 2016,
which would have been perfect since $2016$ is a triangular number,
but it was too late to submit it to any contest
(the USAMO and IMO deadlines were long past).
So on December 31, 2016 I finally sat down and solved it for the case $2017$,
which took another few hours of thought, then submitted it to that year's IMO.
To my dismay it was rejected, but I passed it along to the USA TST after that,
thus making it just in time for the close of the calendar year.

---

%%fakesection Walk
This is a rather detail-oriented problem which requires a lot of care
in order to get all the arguments to be correct.
In my opinion, being able to work out a flawless solution
in the time limit of the exam is an impressive feat.

The original proposed solution is pretty combinatorial,
and the walkthrough for that is below;
it uses a \emph{local} smoothing idea in the first half
and then a \emph{global} idea in the second part.
(There is also an algebraic solution using
inequality smoothing instead.)

Viewing the problem in graph theory terms,
let \[ S(G) \coloneqq \sum_{e = vw} \min \left( \deg v, \deg w \right) \]
and thus we want to maximize $S(G)$ over simple graphs $G$ with $2017$ edges.

Let's get started.
\begin{walk}
  \ii Find an example of a graph achieving $S(G) = 127009$.
  We will prove this is the maximum value.

  \ii Make a conjecture what the answer is if $2017$
  is replaced by $\binom k2+1$ where $k \ge 4$.

  \ii Show that the natural extension of (b) does \emph{not} hold when $k = 3$.
  This edge case will haunt us briefly later on again.
\end{walk}
Here is the local half:
\begin{walk}[resume]
  \ii Suppose $G$ is a graph with $\binom k2$ edges for $k \ge 3$.
  Assume $G$ has no universal vertex,
  and let $w$ denote the vertex of minimal degree.

  Prove that one can delete $w$, and rewire the edges originally adjacent to $w$,
  to obtain a graph $G^\ast$ with fewer vertices and $S(G^\ast) \ge S(G)$.

  \ii Repeat part (d) where $G$ has $\binom k2 + 1$ edges
  and $k \ge 4$, and $\#V(G) > k+1$.
  Then check the existence of $G^\ast$ by hand for $\#V(G) = k+1$.

  \ii Show that (e) is false if $k=3$.
\end{walk}
So, we can reduce to the case where $G$ has a universal vertex.
We will now prove by induction on $k \ge 3$ the following claim:
\begin{quote}
  For any graph $G$,
  \begin{itemize}
    \ii If $G$ has at most $\binom k2$ edges for $k \ge 3$,
    then $S(G) \le \binom k2 \cdot (k-1)$.
    \ii If $G$ has at most $\binom k2 + 1$ edges for $k \ge 4$,
    then $S(G) \le \binom k2 \cdot (k-1) + 1$.
  \end{itemize}
\end{quote}
This is the global half.
\begin{walk}[resume]
  \ii Check the base case $k=3$.
  (We won't need a base case for the second bullet;
  it will reduce to the first one.)
  \ii WLOG, let $v$ be the universal vertex of $G$.
  Show that if $v$ is deleted,
  the resulting graph $H$ has at most $\binom{k-1}{2}$ edges
  (in both cases) and thus we can apply
  the induction hypothesis to the resulting graph $H$.
  \ii Prove that $S(G) = S(H) + 3\#E(H) + \#V(H)$.
  \ii Combine the previous two parts to complete the problem.
\end{walk}


---

For dress-up, I included both the degrees and the edge labels in the average:

There are $100$ marked points on a circle.
Some pairs of these points are joined by $300$ chords;
we then label each marked point with the number of chords
containing it as an endpoint, and each chord
with the smaller of the two labels at its endpoints.
What is the largest possible average of these $400$ labels?

Answer: $\frac{39}{2}$.

To achieve this, we take $25$ vertices and join them together.
In this case, all $25$ vertices and $300$ edges have a label
of $24$, so the sum of the numbers is $325 \cdot 24 = \frac{39}{2} \cdot 400$.

Let's now prove this is optimal.
If we take the graph $G = (V,E)$ with vertices the points and edges the chords,
then the sum in question is equal to
\[ \sum_v \deg v + \sum_{e = vw} \min \left( \deg v, \deg w \right). \]
Of course $\sum_v \deg v = 2 \# E = 600$.
So the core of the problem is the following lemma,
which will imply the answer.
