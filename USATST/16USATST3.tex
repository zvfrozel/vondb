desc:  gcd and Psi commute
author: Mark Sellke
source:  USA TST 2016/3
tags:  [criticalclaim, euclid, manysolutions, highermath, good, polynomial, primes, gimel]
hardness: 30
url: https://aops.com/community/p5679392

---

Let $p$ be a prime number. Let $\FF_p$ denote the integers modulo $p$,
and let $\FF_p[x]$ be the set of polynomials with coefficients in $\FF_p$.
Define $\Psi \colon \FF_p[x] \to \FF_p[x]$ by
\[ \Psi\left( \sum_{i=0}^n a_i x^i \right) = \sum_{i=0}^n a_i x^{p^i}. \]
Prove that for nonzero polynomials $F,G \in \FF_p[x]$,
\[ \Psi(\gcd(F,G)) = \gcd(\Psi(F), \Psi(G)). \]
%Here, a polynomial $Q$ divides $P$ if there exists $R \in \FF_p[x]$
%such that $P(x) - Q(x) R(x)$ is the polynomial with all coefficients $0$
%(with all addition and multiplication in the coefficients taken modulo $p$),
%and the gcd of two polynomials is the highest degree polynomial
%with leading coefficient $1$ which divides both of them.
%A non-zero polynomial is a polynomial with not all coefficients $0$.
%As an example of multiplication, $(x+1)(x+2)(x+3) = x^3+x^2+x+1$
%in $\FF_5[x]$.

---


Observe that $\Psi$ is also a linear map of $\FF_p$ vector spaces,
and that $\Psi(xP) = \Psi(P)^p$ for any $P \in \FF_p[x]$.
(In particular, $\Psi(1) = x$, not $1$, take caution!)

\paragraph{First solution (Ankan Bhattacharya).}
We start with:
\begin{claim*}
  If $P \mid Q$ then $\Psi(P) \mid \Psi(Q)$.
\end{claim*}
\begin{proof}
  Set $Q = PR$, where $R = \sum_{i=0}^k r_i x^i$.
  Then
  \[ \Psi(Q) = \Psi\left( P\sum_{i=0}^k r_i x^i \right)
    = \sum_{i=0}^k \Psi\left( P \cdot r_i x^i  \right)
    = \sum_{i=0}^k r_i \Psi(P)^{p^i} \]
  which is divisible by $\Psi(P)$.
\end{proof}

This already implies
\[ \Psi(\gcd(F,G)) \mid \gcd(\Psi(F), \Psi(G)). \]
For the converse, by Bezout there exists $A,B \in \FF_p[x]$
such that $AF + BG = \gcd(F,G)$, so taking $\Psi$ of both sides gives
\[ \Psi(AF) + \Psi(BG) = \Psi\left( \gcd(F,G) \right). \]
The left-hand side is divisible by $\gcd(\Psi(F), \Psi(G))$
since the first term is divisible by $\Psi(F)$
and the second term is divisible by $\Psi(G)$.
So $\gcd(\Psi(F), \Psi(G)) \mid \Psi(\gcd(F,G))$
and noting both sides are monic we are done.

\paragraph{Second solution.}
Here is an alternative (longer but more conceptual) way to finish without Bezout lemma.
Let $\beth \subseteq \FF_p[x]$ denote the set of polynomials in the image of $\Psi$,
thus $\Psi \colon \FF_p[x] \to \beth$ is a bijection on the level of sets.

\begin{claim*}
  If $A,B \in \beth$ then $\gcd(A,B) \in \beth$.
\end{claim*}
\begin{proof}
  It suffices to show that if $A$ and $B$ are monic,
  and $\deg A > \deg B$,
  then the remainder when $A$ is divided by $B$ is in $\beth$.
  Suppose $\deg A = p^k$ and $B = x^{p^{k-1}} - c_2x^{p^{k-2}} - \dots - c_k$.
  Then
  \begin{align*}
    x^{p^k} &\equiv \left( c_2x^{p^{k-2}} + c_3x^{p^{k-3}}
    + \dots + c_k  \right)^p \pmod B  \\
    &\equiv c_2x^{p^{k-1}} + c_3x^{p^{k-2}} \dots + c_k \pmod B
  \end{align*}
  since exponentiation by $p$ commutes with addition in $\FF_p$.
  This is enough to imply the conclusion.
  The proof if $\deg B$ is smaller less than $p^{k-1}$ is similar.
\end{proof}

Thus, if we view $\FF_p[x]$ and $\beth$ as partially ordered sets
under polynomial division, then $\gcd$ is the
``greatest lower bound'' or ``meet'' in both partially ordered sets.
We will now prove that $\Psi$ is an \emph{isomorphism} of the posets.
We have already seen that $P \mid Q \implies \Psi(P) \mid \Psi(Q)$
from the first solution. For the converse:

\begin{claim*}
  If $\Psi(P) \mid \Psi(Q)$ then $P \mid Q$.
\end{claim*}
\begin{proof}
  Suppose $\Psi(P) \mid \Psi(Q)$, but $Q=PA+B$ where $\deg B < \deg P$.
  Thus $\Psi(P) \mid \Psi(PA) + \Psi(B)$, hence $\Psi(P) \mid \Psi(B)$,
  but $\deg \Psi(P) > \deg \Psi(B)$ hence $\Psi(B) = 0 \implies B = 0$.
\end{proof}

This completes the proof.

\begin{remark*}
   In fact $\Psi \colon \FF_p[x] \to \beth$ is a ring isomorphism
   if we equip $\beth$ with function composition as the ring multiplication.
   Indeed in the proof of the first claim
   (that $P \mid Q \implies \Psi(P) \mid \Psi(Q)$) we saw that
   \[ \Psi(RP) = \sum_{i=0}^k r_i \Psi(P)^{p^i} = \Psi(R) \circ \Psi(P). \]
\end{remark*}
