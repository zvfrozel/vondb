author: Ankan Bhattacharya
desc: Great sequences
source: USA TST 2019/4
tags: [2018-10, divis, mods, good, grid, smallcases, bet]
hardness: 10
url: https://aops.com/community/p11625808

---

We say a function $f \colon \ZZ_{\ge 0} \times \ZZ_{\ge 0} \to \ZZ$
is \emph{great} if for any nonnegative integers $m$ and $n$,
\[ f(m+1, n+1) f(m,n) - f(m+1,n) f(m,n+1) = 1. \]
If $A = (a_0, a_1, \dots)$ and $B = (b_0, b_1, \dots)$
are two sequences of integers,
we write $A \sim B$ if there exists a great function $f$
satisfying $f(n,0) = a_n$ and $f(0,n) = b_n$
for every nonnegative integer $n$ (in particular, $a_0=b_0$).

Prove that if $A$, $B$, $C$, and $D$ are four sequences of integers
satisfying $A \sim B$, $B \sim C$, and $C \sim D$, then $D \sim A$.

---

We present two solutions.
In what follows, we say $(A, B)$ form a great pair
if $A \sim B$.

\paragraph{First solution (Nikolai Beluhov).}
Let $k = a_0 = b_0 = c_0 = d_0$.
We let $f$, $g$, $h$ be great functions for $(A,B)$, $(B,C)$, $(C,D)$
and write the following infinite array:
\[
  \begin{bmatrix}
    & \vdots & \vdots & {\color{blue} b_3} & \vdots & \vdots & \\[5pt]
    \dotsb & g(2,2) & g(2,1) & {\color{blue}b_2} & f(1,2) & f(2,2) & \dotsb \\[5pt]
    \dotsb & g(1,2) & g(1,1) & {\color{blue}b_1} & f(1,1) & f(2,1) & \dotsb \\[5pt]
    {\color{blue}c_3} & {\color{blue}c_2} & {\color{blue}c_1} &
      {\color{blue}k} &
      {\color{blue}a_1} & {\color{blue}a_2} & {\color{blue}a_3} \\[5pt]
    \dotsb & h(2,1) & h(1,1) & {\color{blue}d_1} & & & \\[5pt]
    \dotsb & h(2,2) & h(1,2) & {\color{blue}d_2} & & & \\[5pt]
    & \vdots & \vdots & {\color{blue}d_3} & & & \ddots \\[5pt]
  \end{bmatrix}
\]
The greatness condition is then equivalent to saying
that any $2 \times 2$ sub-grid has determinant $\pm1$
(the sign is $+1$ in two quadrants and $-1$ in the other two),
and we wish to fill in the lower-right quadrant.
To this end, it suffices to prove the following.
\begin{lemma*}
  Suppose we have a $3 \times 3$ sub-grid
  \[
    \begin{bmatrix}
      a & b & c \\
      x & y & z \\
      p & q &
    \end{bmatrix}
  \]
  satisfying the determinant conditions.
  Then we can fill in the ninth entry in the lower right
  with an integer while retaining greatness.
\end{lemma*}
\begin{proof}
  We consider only the case where the $3 \times 3$
  is completely contained inside the bottom-right quadrant,
  since the other cases are exactly the same
  (or even by flipping the signs of the top row
  or left column appropriately).

  If $y = 0$ we have $-1 = bz = bx = xq$, hence $qz = -1$,
  and we can fill in the entry arbitrarily.

  Otherwise, we have $bx \equiv xq \equiv bz \equiv -1 \pmod{y}$.
  This is enough to imply $qz \equiv -1 \pmod y$,
  and so we can fill in the integer $\frac{qz+1}{y}$.
\end{proof}

\begin{remark*}
  In this case (of all $+1$ determinants),
  I think it turns out the bottom entry is
  exactly equal to $qza - cyp - c - p$,
  which is obviously an integer.
\end{remark*}


\paragraph{Second solution (Ankan Bhattacharya).}
We will give an explicit classification of great sequences:
\begin{lemma*}
  The pair $(A,B)$ is great if and only if
  $a_0 = b_0$, $a_0 \mid a_1b_1 + 1$, and
  $a_n \mid a_{n-1} + a_{n+1}$ and $b_n \mid b_{n-1} + b_{n+1}$ for all $n$.
\end{lemma*}
\begin{proof}[Proof of necessity]
  It is clear that $a_0 = b_0$.
  Then $a_0f(1, 1) - a_1b_1 = 1$,
  i.e. $a_0 \mid a_1b_1 + 1$.

  Now, focus on six entries $f(x, y)$
  with $x \in \{n-1, n, n+1\}$ and $y \in \{0, 1\}$.
  Let $f(n-1, 1) = u$, $f(n, 1) = v$, and $f(n+1, 1) = w$, so
  \begin{align*}
    v a_{n-1} - u a_n & = 1,\\
    w a_n - v a_{n+1} & = 1.
  \end{align*}
  Then
  \[u + w = \frac{v(a_{n-1} + a_{n+1})}{a_n}\]
  and from above $\gcd(v, a_n) = 1$,
  so $a_n \mid a_{n-1} + a_{n+1}$;
  similarly for $b_n$.
  (If $a_n = 0$,
  we have $va_{n-1} = 1$ and $va_{n+1} = -1$,
  so this is OK.)
\end{proof}

\begin{proof}[Proof of sufficiency]
  Now consider two sequences $a_0, a_1, \dots$
  and $b_0, b_1, \dots$ satisfying our criteria.
  We build a great function $f$ by induction on $(x, y)$.
  More strongly, we will assume as part of the inductive hypothesis
  that any two adjacent entries of $f$ are relatively prime
  and that for any three consecutive entries horizontally or vertically,
  the middle one divides the sum of the other two.

  First we set $f(1, 1)$ so that $a_0 f(1, 1) = a_1 b_1 + 1$,
  which is possible.

  Consider an uninitialized $f(s, t)$;
  without loss of generality suppose $s \ge 2$.
  Then we know five values of $f$
  and wish to set a sixth one $z$,
  as in the matrix below:
  \[
    \begin{matrix}
    u & x\\
    v & y\\
    w & {\color{gray} z} \\
    \end{matrix}
  \]
  (We imagine $a$-indices to increase southwards and $b$-indices to increase eastwards.)
  If $v \neq 0$,
  then the choice $y \cdot \frac{u + w}{v} - x$ works
  as $uy - vx = 1$.
  If $v = 0$,
  it easily follows that $\{u, w\} = \{1, -1\}$
  and $y = w$ as $yw = 1$.
  Then we set the uninitialized entry to anything.

  Now we verify that this is compatible with the inductive hypothesis.
  From the determinant 1 condition,
  it easily follows that $\gcd(w, z) = \gcd(v, z) = 1$.
  The proof that $y \mid x + z$
  is almost identical to a step performed in the ``necessary'' part
  of the lemma and we do not repeat it here.
  By induction, a desired great function $f$ exists.
\end{proof}

We complete the solution.
Let $A$, $B$, $C$, and $D$ be integer sequences
for which $(A, B)$, $(B, C)$, and $(C, D)$ are great.
Then $a_0 = b_0 = c_0 = d_0$,
and each term in each sequence
(after the zeroth term)
divides the sum of its neighbors.
Since $a_0$ divides all three of $a_1b_1 + 1$, $b_1c_1 + 1$, and $c_1d_1 + 1$,
it follows $a_0$ divides $d_1a_1 + 1$,
and thus $(D, A)$ is great as desired.

\begin{remark*}
To simplify the problem,
we may restrict the codomain of great functions
and elements in great pairs of sequences
to $\ZZ_{> 0}$.
This allows the parts of the solution dealing with zero entries to be ignored.
\end{remark*}

\begin{remark*}
Of course, this solution also shows that any odd path
(in the graph induced by the great relation on sequences)
completes to an even cycle.
If we stipulate that great functions must have $f(0, 0) = \pm 1$,
then even paths complete to cycles as well.
Alternatively, we could change the great functional equation to
\[f(x + 1, y + 1) f(x, y) - f(x + 1, y) f(x, y + 1) = -1.\]

A quick counterexample to transitivity of $\sim$ as is
without the condition $f(0,0)=1$, for concreteness:
let $a_n = c_n = 3+n$ and $b_n = 3+2n$ for $n \ge 0$.
\end{remark*}
