desc: Power of $K$ to $XCD$ is constant
author: Andrew Gu, Frank Han
source: USA TST 2021/2
tags: [2021-02, pop, nice, anglechase, zayin]
hardness: 45
url: https://aops.com/community/p20672623

---

Points $A$, $V_1$, $V_2$, $B$, $U_2$, $U_1$
lie fixed on a circle $\Gamma$, in that order,
and such that $BU_2 > AU_1 > BV_2 > AV_1$.

Let $X$ be a variable point on the arc $V_1 V_2$ of $\Gamma$
not containing $A$ or $B$.
Line $XA$ meets line $U_1 V_1$ at $C$,
while line $XB$ meets line $U_2 V_2$ at $D$.

Prove there exists a fixed point $K$, independent of $X$,
such that the power of $K$ to the circumcircle
of $\triangle XCD$ is constant.

---

For brevity, we let $\ell_i$ denote line $U_iV_i$ for $i=1,2$.

We first give an explicit description of the fixed point $K$.
Let $E$ and $F$ be points on $\Gamma$ such that $\ol{AE} \parallel \ell_1$
and $\ol{BF} \parallel \ell_2$.
The problem conditions imply that $E$ lies between $U_1$ and $A$
while $F$ lies between $U_2$ and $B$.
Then we let \[ K = \ol{AF} \cap \ol{BE}. \]
This point exists because $AEFB$ are the vertices
of a convex quadrilateral.

\begin{remark*}
  [How to identify the fixed point]
  If we drop the condition that $X$ lies on the arc,
  then the choice above is motivated by choosing $X \in \{E,F\}$.
  Essentially, when one chooses $X \to E$,
  the point $C$ approaches an infinity point.
  So in this degenerate case, the only points whose
  power is finite to $(XCD)$ are bounded are those on line $BE$.
  The same logic shows that $K$ must lie on line $AF$.
  Therefore, if the problem is going to work,
  the fixed point must be exactly $\ol{AF} \cap \ol{BE}$.
\end{remark*}

We give two possible approaches for proving the power
of $K$ with respect to $(XCD)$ is fixed.

\paragraph{First approach by Vincent Huang.}
We need the following claim:
\begin{claim*}
  Suppose distinct lines $AC$ and $BD$ meet at $X$.
  Then for any point $K$
  \[ \opname{pow}(K, XAB) + \opname{pow}(K, XCD)
    = \opname{pow}(K, XAD) + \opname{pow}(K, XBC). \]
\end{claim*}
\begin{proof}
  The difference between the left-hand side and right-hand
  side is a linear function in $K$,
  which vanishes at all of $A$, $B$, $C$, $D$.
\end{proof}

Construct the points $P = \ell_1 \cap \ol{BE}$
and $Q = \ell_2 \cap \ol{AF}$, which do not depend on $X$.
\begin{claim*}
  Quadrilaterals $BPCX$ and $AQDX$ are cyclic.
\end{claim*}
\begin{proof}
  By Reim's theorem: $\dang CPB = \dang AEB = \dang AXB = \dang CXB$, etc.
\end{proof}

\begin{center}
\begin{asy}
pair A = dir(190);
pair B = -conj(A);
pair U_1 = dir(125);
pair V_1 = dir(220);
pair U_2 = dir(80);
pair V_2 = dir(310);
filldraw(unitcircle, opacity(0.1)+lightcyan, blue);
draw(A--B, blue);
draw(U_1--V_1, deepgreen);
draw(U_2--V_2, deepgreen);

pair E = V_1*U_1/A;
pair F = V_2*U_2/B;

pair X = dir(265);
pair C = extension(A, X, U_1, V_1);
pair D = extension(B, X, U_2, V_2);
draw(A--X--B, lightred);
draw(A--E, deepcyan);
draw(B--F, deepcyan);
pair K = extension(A, F, B, E);
draw(A--F, blue);
draw(B--E, blue);

pair P = extension(B, E, U_1, V_1);
pair Q = extension(A, F, U_2, V_2);

filldraw(circumcircle(X, C, B), opacity(0.05)+yellow, dotted+orange);
filldraw(circumcircle(X, A, D), opacity(0.05)+yellow, dotted+orange);

dot("$A$", A, dir(A));
dot("$B$", B, dir(B));
dot("$U_1$", U_1, dir(U_1));
dot("$V_1$", V_1, dir(V_1));
dot("$U_2$", U_2, dir(U_2));
dot("$V_2$", V_2, dir(V_2));
dot("$E$", E, dir(E));
dot("$F$", F, dir(F));
dot("$X$", X, dir(X));
dot("$C$", C, dir(C));
dot("$D$", D, dir(D));
dot("$K$", K, dir(270));
dot("$P$", P, dir(290));
dot("$Q$", Q, dir(240));

/* TSQ Source:

A = dir 190
B = -conj(A)
U_1 = dir 125
V_1 = dir 220
U_2 = dir 80
V_2 = dir 310
unitcircle 0.1 lightcyan / blue
A--B blue
U_1--V_1 deepgreen
U_2--V_2 deepgreen

E = V_1*U_1/A
F = V_2*U_2/B

X = dir 265
C = extension A X U_1 V_1
D = extension B X U_2 V_2
A--X--B lightred
A--E deepcyan
B--F deepcyan
K = extension A F B E R270
A--F blue
B--E blue

P = extension B E U_1 V_1 R290
Q = extension A F U_2 V_2 R240

circumcircle X C B 0.05 yellow / dotted orange
circumcircle X A D 0.05 yellow / dotted orange

*/
\end{asy}
\end{center}

Now, for the particular $K$ we choose, we have
\begin{align*}
  \opname{pow}(K, XCD) &=
  \opname{pow}(K, XAD) + \opname{pow}(K, XBC) - \opname{pow}(K, XAB) \\
  &= KA \cdot KQ + KB \cdot KP - \opname{pow}(K, \Gamma).
\end{align*}
This is fixed, so the proof is completed.

\paragraph{Second approach by authors.}
Let $Y$ be the second intersection of $(XCD)$ with $\Gamma$.
Let $S = \ol{EY} \cap \ell_1$ and $T = \ol{FY} \cap \ell_2$.
\begin{claim*}
  Points $S$ and $T$ lies on $(XCD)$ as well.
\end{claim*}
\begin{proof}
  By Reim's theorem: $\dang CSY = \dang AEY = \dang AXY = \dang CXY$, etc.
\end{proof}

Now let $X'$ be any other choice of $X$,
and define $C'$ and $D'$ in the obvious way.
We are going to show that $K$ lies on the radical axis
of $(XCD)$ and $(X'C'D')$.

\begin{center}
\begin{asy}
pair A = dir(190);
pair B = -conj(A);
pair U_1 = dir(125);
pair V_1 = dir(220);
pair U_2 = dir(80);
pair V_2 = dir(310);
filldraw(unitcircle, opacity(0.1)+lightcyan, blue);
draw(A--B, blue);
draw(U_1--V_1, deepgreen);
draw(U_2--V_2, deepgreen);

pair E = V_1*U_1/A;
pair F = V_2*U_2/B;

pair X = dir(265);
pair C = extension(A, X, U_1, V_1);
pair D = extension(B, X, U_2, V_2);
draw(A--X--B, lightred);
draw(A--E, deepcyan);
draw(B--F, deepcyan);
pair K = extension(A, F, B, E);
draw(A--F, blue);
draw(B--E, blue);

pair Y = -X+2*foot(X, origin, circumcenter(X, C, D));

draw(circumcircle(X, C, D), red);
pair S = extension(Y, E, U_1, V_1);
pair T = extension(Y, F, U_2, V_2);
draw(E--Y--F, brown);

pair Xp = dir(300);
pair Cp = extension(Xp, A, U_1, V_1);
pair Dp = extension(Xp, B, U_2, V_2);
draw(A--Xp--B, orange);

pair L = extension(S, Y, Cp, Xp);

dot("$A$", A, dir(A));
dot("$B$", B, dir(B));
dot("$U_1$", U_1, dir(U_1));
dot("$V_1$", V_1, dir(V_1));
dot("$U_2$", U_2, dir(U_2));
dot("$V_2$", V_2, dir(V_2));
dot("$E$", E, dir(E));
dot("$F$", F, dir(F));
dot("$X$", X, dir(X));
dot("$C$", C, dir(C));
dot("$D$", D, dir(D));
dot("$K$", K, dir(270));
dot("$Y$", Y, dir(Y));
dot("$S$", S, dir(S));
dot("$T$", T, dir(T));
dot("$X'$", Xp, dir(Xp));
dot("$C'$", Cp, dir(170));
dot("$D'$", Dp, dir(10));
dot("$L$", L, dir(L));

/* TSQ Source:

A = dir 190
B = -conj(A)
U_1 = dir 125
V_1 = dir 220
U_2 = dir 80
V_2 = dir 310
unitcircle 0.1 lightcyan / blue
A--B blue
U_1--V_1 deepgreen
U_2--V_2 deepgreen

E = V_1*U_1/A
F = V_2*U_2/B

X = dir 265
C = extension A X U_1 V_1
D = extension B X U_2 V_2
A--X--B lightred
A--E deepcyan
B--F deepcyan
K = extension A F B E R270
A--F blue
B--E blue

Y = -X+2*foot X origin circumcenter X C D

circumcircle X C D red
S = extension Y E U_1 V_1
T = extension Y F U_2 V_2
E--Y--F brown

X' = dir 300
C' = extension Xp A U_1 V_1 R170
D' = extension Xp B U_2 V_2 R10
A--Xp--B orange

L = extension S Y Cp Xp

*/
\end{asy}
\end{center}

The main idea is as follows:
\begin{claim*}
  The point $L = \ol{EY} \cap \ol{AX'}$ lies on the radical axis.
  By symmetry, so does the point $M = \ol{FY} \cap \ol{BX'}$ (not pictured).
\end{claim*}
\begin{proof}
  Again by Reim's theorem, $SC'YX'$ is cyclic.
  Hence we have
  \[ \opname{pow}(L, X'C'D') = LC' \cdot LX'
    = LS \cdot LY = \opname{pow}(L, XCD). \qedhere \]
\end{proof}

To conclude, note that by Pascal theorem on
\[ EYFAX'B \]
it follows $K$, $L$, $M$ are collinear,
as needed.

\begin{remark*}
  All the conditions about $U_1$, $V_1$, $U_2$, $V_2$
  at the beginning are there to eliminate configuration issues,
  making the problem less obnoxious to the contestant.

  In particular, without the various assumptions,
  there exist configurations in which the point $K$ is at infinity.
  In these cases, the center of $XCD$ moves along a fixed line.
\end{remark*}
