desc: Dies to inversion, concurrence of $AA_2$ et al
author: Evan Chen
source: USA TST 2017/2
tags: [2016-08, harmonic, spiralsim, mine, nice, inversion, rich, bet, well]
hardness: 25
url: https://aops.com/community/p7389108

---

Let $ABC$ be an acute scalene triangle with circumcenter $O$,
and let $T$ be on line $BC$ such that $\angle TAO = 90\dg$.
The circle with diameter $\ol{AT}$
intersects the circumcircle of $\triangle BOC$ at two points
$A_1$ and $A_2$, where $OA_1 < OA_2$.
Points $B_1$, $B_2$, $C_1$, $C_2$ are defined analogously.
\begin{enumerate}
  \ii[(a)]
  Prove that $\ol{AA_1}$, $\ol{BB_1}$, $\ol{CC_1}$ are concurrent.
  \ii[(b)]
  Prove that $\ol{AA_2}$, $\ol{BB_2}$, $\ol{CC_2}$ are concurrent
  on the Euler line of triangle $ABC$.
\end{enumerate}

---

Let triangle $ABC$ have circumcircle $\Gamma$.
Let $\triangle XYZ$ be the tangential triangle of $\triangle ABC$
(hence $\Gamma$ is the incircle of $\triangle XYZ$),
and denote by $\Omega$ its circumcircle.
Suppose the symmedian $\ol{AX}$ meets $\Gamma$ again at $D$,
and let $M$ be the midpoint of $\ol{AD}$.
Finally, let $K$ be the Miquel point of quadrilateral $ZBCY$,
meaning it is the intersection of $\Omega$
and the circumcircle of $\triangle BOC$ (other than $X$).

\begin{center}
\begin{asy}
size(12cm);
pair A = dir(115);
pair B = dir(200);
pair C = dir(340);
draw(A--B--C--cycle, blue);
filldraw(circumcircle(A, B, C), opacity(0.1)+lightblue, blue);
pair O = circumcenter(A, B, C);

pair X = 2*B*C/(B+C);
pair Y = 2*C*A/(C+A);
pair Z = 2*A*B/(A+B);

filldraw(circumcircle(X, Y, Z), opacity(0.1)+lightcyan, cyan);
draw(X--Y--Z--cycle, cyan);

filldraw(circumcircle(B, O, C), opacity(0.1)+heavycyan, heavycyan);
pair T = extension(Y, Z, B, C);
draw(Z--T, lightcyan);
draw(T--B, blue+dashed);

pair D = -A+2*foot(O, A, X);
pair U = foot(A, B, C);
filldraw(circumcircle(T, A, U), opacity(0.1)+lightgreen, heavygreen);

pair V = circumcenter(X, Y, Z);
pair M = midpoint(A--D);
pair L = IP(V--(Y+Z-V), circumcircle(X, Y, Z));
pair K = extension(L, A, O, U);
draw(A--X, heavycyan);
draw(K--L, heavygreen+dashed);

dot("$A$", A, dir(A));
dot("$B$", B, dir(B));
dot("$C$", C, dir(C));
dot("$O$", O, dir(O));
dot("$X$", X, dir(X));
dot("$Y$", Y, dir(Y));
dot("$Z$", Z, dir(Z));
dot("$T$", T, dir(T));
dot("$D$", D, dir(D));
dot("$M$", M, dir(M));
dot("$L$", L, dir(L));
dot("$K$", K, dir(K));

/* TSQ Source:

!size(12cm);
A = dir 115
B = dir 200
C = dir 340
A--B--C--cycle blue
circumcircle A B C 0.1 lightblue / blue
O = circumcenter A B C

X = 2*B*C/(B+C)
Y = 2*C*A/(C+A)
Z = 2*A*B/(A+B)

circumcircle X Y Z 0.1 lightcyan / cyan
X--Y--Z--cycle cyan

circumcircle B O C 0.1 heavycyan / heavycyan
T = extension Y Z B C
Z--T lightcyan
T--B blue dashed

D = -A+2*foot O A X
U := foot A B C
circumcircle T A U 0.1 lightgreen / heavygreen

V := circumcenter X Y Z
M = midpoint A--D
L = IP V--(Y+Z-V) circumcircle X Y Z
K = extension L A O U
A--X heavycyan
K--L heavygreen dashed

*/
\end{asy}
\end{center}

We first claim that $M$ and $K$ are $A_1$ and $A_2$.
In that case $OM < OA < OK$, so $M = A_1$, $K = A_2$.

To see that $M = A_1$, note that $\angle OMX = 90\dg$,
and moreover that $\ol{TA}$, $\ol{TD}$ are tangents to $\Gamma$,
whence we also have $M = \ol{TO} \cap \ol{AD}$.
Thus $M$ lies on both $(BOC)$ and $(AT)$.
This solves part (a) of the problem:
the concurrency point is the symmedian point of $\triangle ABC$.

Now, note that since $K$ is the Miquel point,
\[ \frac{ZK}{YK} = \frac{ZB}{YC} = \frac{ZA}{YA} \]
and hence $\ol{KA}$ is an angle bisector of $\angle ZKY$.
Thus from $(TA;YZ)=-1$ we obtain $\angle TKA = 90\dg$.

It remains to show $\ol{AK}$ passes through a fixed point on the Euler line.
We claim it is the exsimilicenter of $\Gamma$ and $\Omega$.
Let $L$ be the midpoint of the arc $YZ$ of $\triangle XYZ$ not containing $X$.
Then we know that $K$, $A$, $L$ are collinear.
Now the positive homothety sending $\Gamma$ to $\Omega$ maps $A$ to $L$;
this proves the claim.
Finally, it is well-known that the line through $O$
and the circumcenter of $\triangle XYZ$
coincides with the Euler line of $\triangle ABC$;
hence done.

A second approach to (b) presented by many contestants
is to take an inversion around the circumcircle of $ABC$.
In that situation, the part reduces to the following
known lemma: if $\ol{AH_a}$, $\ol{BH_b}$, $\ol{CH_c}$
are the altitudes of a triangle,
then the circumcircles of triangles $OAH_a$, $BOH_b$, $COH_c$
are coaxial, and the radical axis coincides with the Euler line.
Indeed one simply observes that the orthocenter
has equal power to all three circles.

\paragraph{Authorship comments.}
This problem was inspired by the fact that $K$, $A$, $L$
are collinear in the figure,
which was produced by one of my students (Ryan Kim)
in a solution to a homework problem.
I realized for example that this implied that line $AK$
passed through the $X_{56}$ point of $\triangle XYZ$
(which lies on the Euler line of $\triangle ABC$).

This problem was the result of playing around with
the resulting very nice picture:
all the power comes from the ``magic'' point $K$.
