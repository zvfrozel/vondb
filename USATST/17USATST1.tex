desc: Reading comprehension, sports league, greedy
author: Po-Shen Loh
source: USA TST 2017/1
tags: [2016-12, local, extreme, pitfall, find, bestpossible, troll, algorithm, aleph]
hardness: 15
url: https://aops.com/community/p7389115

---

In a sports league, each team uses a set of at most $t$ signature colors.
A set $S$ of teams is \emph{color-identifiable} if one can assign
each team in $S$ one of their signature colors,
such that no team in $S$ is assigned
\emph{any} signature color of a different team in $S$.
For all positive integers $n$ and $t$,
determine the maximum integer $g(n,t)$ such that:
In any sports league with exactly $n$ distinct colors
present over all teams, one can always
find a color-identifiable set of size at least $g(n,t)$.

---

Answer: $\left\lceil n/t \right\rceil$.

To see this is an upper bound, note that one can easily construct
a sports league with that many teams anyways.

A quick warning:
\begin{remark*}
  [Misreading the problem]
  It is common to misread the problem by ignoring the word ``any''.
  Here is an illustration.

  Suppose we have two teams, MIT and Harvard;
  the colors of MIT are red/grey/black, and
  the colors of Harvard are red/white.
  (Thus $n=4$ and $t=3$.)
  The assignment of MIT to grey and Harvard to red
  is \emph{not} acceptable because red is a
  signature color of MIT, even though not the one assigned.
\end{remark*}

We present two proofs of the lower bound.

\paragraph{Approach by deleting teams (Gopal Goel).}
Initially, place all teams in a set $S$.
Then we repeat the following algorithm:
\begin{quote}
If there is a team all of whose signature colors
are shared by some other team in $S$ already,
then we delete that team.
\end{quote}
(If there is more than one such team, we pick arbitrarily.)

At the end of the process,
all $n$ colors are still present at least once,
so at least $\left\lceil n/t \right\rceil$ teams remain.
Moreover, since the algorithm is no longer possible,
the remaining set $S$ is already color-identifiable.

\begin{remark*}
  [Gopal Goel]
  It might seem counter-intuitive
  that we are \emph{deleting} teams from the full set
  when the original problem is trying to get a large set $S$.

  This is less strange when one thinks
  of it instead as ``safely deleting useless teams''.
  Basically, if one deletes such a team,
  the problem statement implies that the task must still be possible,
  since $g(n,t)$ does not depend on the number of teams:
  $n$ is the number of colors present,
  and deleting a useless team does not change this.
  It turns out that this optimization is already enough to solve the problem.
\end{remark*}

\paragraph{Approach by adding colors.}
For a constructive algorithmic approach,
the idea is to greedy pick by color (rather than by team),
taking at each step the least used color.
Select the color $C_1$ with the \emph{fewest} teams using it,
and a team $T_1$ using it.
Then delete all colors $T_1$ uses, and all teams which use $C_1$.
Note that
\begin{itemize}
  \ii By problem condition,
  this deletes at most $t$ teams total.
  \ii Any remaining color $C$ still has at least one user.
  Indeed, if not, then $C$ had the same set of teams
  as $C_1$ did (by minimality of $C$),
  but then it should have deleted as a color of $T_1$.
\end{itemize}

Now repeat this algorithm with $C_2$ and $T_2$, and so on.
This operations uses at most $t$ colors each time,
so we select at least $\left\lceil n/t \right\rceil$ colors.

\begin{remark*}
  A greedy approach by team \emph{does not work}.
  For example, suppose we try to ``grab teams until no more can be added''.

  As before, assume our league has teams, MIT and Harvard;
  the colors of MIT are red/grey/black, and
  the colors of Harvard are red/white.
  (Thus $n=4$ and $t=3$.)
  If we start by selecting MIT and red, then
  it is impossible to select any more teams; but $g(n,t) = 2$.
\end{remark*}

---

A quick warning:
It is common to misread the problem by ignoring the word ``any''.
Here is an illustration.

Suppose we have two teams, MIT and Harvard;
the colors of MIT are red/grey/black, and
the colors of Harvard are red/white.
(Thus $n=4$ and $t=3$.)
The assignment of MIT to grey and Harvard to red
is \emph{not} acceptable because red is a
signature color of MIT, even though not the one assigned.

With that comment, let's focus on the problem.

\begin{walk}
  \ii Figure out what the answer should be
  and give a construction.

  \ii Consider the following greedy algorithm:
  \begin{itemize}
    \ii Take any team which one can still add;
    \ii Pick any color for that team;
    \ii Repeat until stuck.
  \end{itemize}
  Give an example to show this algorithm
  does \emph{not} work.
\end{walk}
However, maybe surprisingly, the quantity $g(n,t)$
apparently does not depend on the number of teams.
So, consider the following idea:
we call a team \emph{useless} if every signature color
of the team is shared by some other existing team.
\begin{walk}[resume]
  \ii Give an argument that one can delete useless teams
  without worry, hence justifying the name ``useless''.
  \ii Following the repeat-until-stuck (RUST) philosophy,
  continue doing this until no more deletions are possible.
  What property does the remaining set have?
  \ii Show that all $n$ colors are still present.
  \ii Conclude.
\end{walk}
Part (c) is related to the concept of ``circular optimization'',
described in the hyperlink
\url{https://usamo.wordpress.com/2020/04/15/circular-optimization/}.
