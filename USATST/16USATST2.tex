desc: Easy mixtilinear geometry problem
author: Evan Chen
source: USA TST 2016/2
tags: [2016-08, mine, dalet, mixtilinear, nice, rich, anglechase, well]
hardness: 20
url: https://aops.com/community/p5679361

---

Let $ABC$ be a scalene triangle with circumcircle $\Omega$,
and suppose the incircle of $ABC$ touches $BC$ at $D$.
The angle bisector of $\angle A$ meets $BC$ and $\Omega$ at $K$ and $M$.
The circumcircle of $\triangle DKM$ intersects the $A$-excircle
at $S_1$, $S_2$, and $\Omega$ at $T \neq M$.
Prove that line $AT$ passes through either $S_1$ or $S_2$.

---

We present an angle-chasing solution,
and then a more advanced alternative finish.

\paragraph{First solution (angle chasing).}
Assume for simplicity $AB < AC$.
Let $E$ be the contact point of the $A$-excircle on $BC$;
also let ray $TD$ meet $\Omega$ again at $L$.
From the fact that $\angle MTL = \angle MTD = 180^{\circ} - \angle MKD$,
we can deduce that $\angle MTL = \angle ACM$,
meaning that $L$ is the reflection of $A$ across the
perpendicular bisector $\ell$ of $BC$.
If we reflect $T$, $D$, $L$ over $\ell$, we deduce $A$, $E$ and the reflection of $T$ across
$\ell$ are collinear, which implies that $\angle BAT = \angle CAE$.

Now, consider the reflection point $E$ across line $AI$, say $S$.
Since ray $AI$ passes through the $A$-excenter, $S$ lies on the $A$-excircle.
Since $\angle BAT = \angle CAE$, $S$ also lies on ray $AT$.
But the circumcircles of triangles $DKM$ and $KME$ are congruent (from $DM = EM$),
so $S$ lies on the circumcircle of $\triangle DKM$ too.
Hence $S$ is the desired intersection point.

\begin{center}
\begin{asy}
size(12cm);
pair A = dir(40);
pair B = dir(210);
pair C = dir(330);
pair I = incenter(A, B, C);
pair D = foot(I, B, C);
pair K = extension(A, I, B, C);
pair M = dir(-90);
pair E = B+C-D;
pair I_A = 2*M-I;

pair S = -E+2*foot(E, K, M);

pair L = dir(90);

filldraw(A--B--C--cycle, opacity(0.2)+lightblue, blue);
filldraw(unitcircle, opacity(0.1)+lightblue, blue);

filldraw(circumcircle(D, K, M), opacity(0.2)+lightgreen, green);
filldraw(circumcircle(E, K, M), opacity(0.2)+lightred, red);

draw(A--I_A, blue);

draw(arc(I_A, abs(E-I_A), -20, 120), heavycyan);
draw(E--S, orange);

pair T = extension(I, L, A, S);
draw(E--A--S, orange+dashed);

dot("$A$", A, dir(A));
dot("$B$", B, dir(B));
dot("$C$", C, dir(C));
dot("$D$", D, dir(90));
dot("$K$", K, dir(100));
dot("$M$", M, dir(M));
dot("$E$", E, dir(E));
dot("$I_A$", I_A, dir(I_A));
dot("$S$", S, dir(10));
dot("$T$", T, dir(T));

/* Source generated by TSQ

!size(12cm);
A = dir 40
B = dir 210
C = dir 330
I := incenter A B C
D = foot I B C R90
K = extension A I B C R100
M = dir -90
E = B+C-D
I_A = 2*M-I

S = -E+2*foot E K M R10
L := dir 90

A--B--C--cycle 0.2 lightblue / blue
unitcircle 0.1 lightblue / blue

circumcircle D K M 0.2 lightgreen / green
circumcircle E K M 0.2 lightred / red

A--I_A blue

!draw(arc(I_A, abs(E-I_A), -20, 120), heavycyan);
E--S orange

T = extension I L A S
E--A--S orange dashed
*/
\end{asy}
\end{center}

\paragraph{Second solution (advanced).}
It's known that $T$ is the touch-point of the $A$-mixtilinear incircle.
Let $E$ be contact point of $A$-excircle on $BC$.
Now the circumcircles of $\triangle DKM$ and $\triangle KME$ are congruent,
since $DM = ME$ and the angles at $K$ are supplementary.
Let $S$ be the reflection of $E$ across line $KM$, which by
the above the above comment lies on the circumcircle of $\triangle DKM$.
Since $KM$ passes through the $A$-excenter, $S$ also lies on the $A$-excircle.
But $S$ also lies on line $AT$, since lines $AT$ and $AE$ are isogonal
(the mixtilinear cevian is isogonal to the Nagel line).
Thus $S$ is the desired intersection point.

\paragraph{Authorship comments.}
This problem comes from an observation of mine:
let $ABC$ be a triangle,
let the $\angle A$ bisector meet $\ol{BC}$ and $(ABC)$ at $E$ and $M$.
Let $W$ be the tangency point of the $A$-mixtilinear excircle
with the circumcircle of $ABC$.
Then $A$-Nagel line passed through a common intersection
of the circumcircle of $\triangle MEW$
and the $A$-mixtilinear incircle.

This problem is the inverted version of this observation.

---

This is secretly a mixtilinear problem,
although it turns out you don't need this to do the solution.
However, since we're practicing American geo anyways, why not?
\begin{walk}
  \ii Show that $T$ is the mixtilinear contact point.
\end{walk}
Now, this time we have the ``$S_1$ or $S_2$ construct''
and it's somewhat misleading.
Let's deal with it the following way --- we define
$S$ be the intersection of $\ol{AT}$ with $(DKM)$ other than $T$.
Then, we just want to show $S$ lies on the $A$-excircle.
\begin{walk}[resume]
  \ii Figure out where the reflection
  of $S$ across the $\angle A$ bisector should be.
  We'll denote this point by $E$ --- the definition of $E$
  should be a simple one (not involving $S$) if you got it right.
  \ii Show that $(DKM)$ and $(EKM)$ have equal radii.
  \ii Put everything together using the fact
  that $\ol{AE}$ and $\ol{AT}$ are isogonal.
\end{walk}
