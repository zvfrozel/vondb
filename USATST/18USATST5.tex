desc: IMO 2014 BESTED geometry
source: USA TST 2018/5
tags: [2017-09, desargues, nice, mine, spiralsim, miquel, anglechase, length, rich, gimel]
author: Evan Chen
hardness: 30
url: https://aops.com/community/p9735608

---

Let $ABCD$ be a convex cyclic quadrilateral which is not a kite,
but whose diagonals are perpendicular and meet at $H$.
Denote by $M$ and $N$ the midpoints of $\ol{BC}$ and $\ol{CD}$.
Rays $MH$ and $NH$ meet $\ol{AD}$ and $\ol{AB}$ at $S$ and $T$, respectively.
Prove there exists a point $E$, lying outside quadrilateral $ABCD$,
such that
\begin{itemize}
  \ii ray $EH$ bisects both angles $\angle BES$, $\angle TED$, and
  \ii $\angle BEN = \angle MED$.
\end{itemize}

---

The main claim is that $E$ is the
intersection of $(ABCD)$ with the circle with diameter $\ol{AH}$.
\begin{center}
\begin{asy}
size(10cm);
pair A = dir(100);
pair B = dir(190);
pair D = dir(-10);
pair F = -A;
pair H = foot(A, B, D);
pair E = foot(A, F, H);
pair C = -A+2*foot(origin, A, H);

filldraw(unitcircle, opacity(0.1)+lightcyan, lightblue);
draw(A--B--C--D--cycle, lightblue);
draw(A--B--F--D--cycle, lightblue);
draw(A--C, lightblue);
draw(B--D, lightblue);
pair M = midpoint(B--C);
pair N = midpoint(D--C);
pair S = foot(H, A, D);
pair T = foot(H, A, B);
draw(M--S, lightblue);
draw(N--T, lightblue);

filldraw(circumcircle(B, T, S), opacity(0.1)+yellow, orange);
// pair Q = midpoint(A--H);
draw(H--F, lightblue);
pair P = midpoint(H--F);
filldraw(B--T--S--D--cycle, opacity(0.1)+yellow, orange+1);
// draw(P--Q, blue);
draw(A--F, blue);
draw(circumcircle(E, B, M), lightgreen);
draw(circumcircle(E, D, N), lightgreen);
draw(H--E--A, orange);
draw(circumcircle(A, T, S), lightgreen);

dot("$A$", A, dir(A));
dot("$B$", B, dir(B));
dot("$D$", D, dir(D));
dot("$F$", F, dir(F));
dot("$H$", H, dir(H));
dot("$E$", E, dir(E));
dot("$C$", C, dir(C));
dot("$M$", M, dir(M));
dot("$N$", N, dir(N));
dot("$S$", S, dir(S));
dot("$T$", T, dir(T));
// dot("$Q$", Q, dir(200));
dot("$P$", P, dir(P));

/* TSQ Source:

!size(10cm);
A = dir 100
B = dir 190
D = dir -10
F = -A
H = foot A B D
E = foot A F H
C = -A+2*foot origin A H

unitcircle 0.1 lightcyan / lightblue
A--B--C--D--cycle lightblue
A--B--F--D--cycle lightblue
A--C lightblue
B--D lightblue
M = midpoint B--C
N = midpoint D--C
S = foot H A D
T = foot H A B
M--S lightblue
N--T lightblue

circumcircle B T S 0.1 yellow / orange
Q = midpoint A--H R200
H--F lightblue
P = midpoint H--F
B--T--S--D--cycle 0.1 yellow / orange+1
P--Q blue
A--F blue
circumcircle E B M lightgreen
circumcircle E D N lightgreen
H--E--A orange
circumcircle A T S lightgreen

*/
\end{asy}
\end{center}

The following observation can be quickly made
without reference to $E$.
\begin{lemma*}
  We have $\angle HSA = \angle HTA = 90\dg$.
  Consequently, quadrilateral $BTSD$ is cyclic.
\end{lemma*}
\begin{proof}
  This is direct angle chasing.
  In fact, $\ol{HM}$ passes through the circumcenter of $\triangle BHC$
  and $\triangle HAD \sim \triangle HCB$,
  so $\ol{HS}$ ought to be the altitude of $\triangle HAD$.
\end{proof}

From here it follows that $E$ is the Miquel point of
cyclic quadrilateral $BTSD$.
Define $F$ to be the point diametrically opposite $A$,
so that $E$, $H$, $F$ are collinear, $\ol{CF} \parallel \ol{BD}$.
By now we already have
\[ \dang BEH = \dang BEF = \dang BAF = \dang CAD = \dang HAS = \dang HES \]
so $\ol{EH}$ bisects $\angle BES$, and $\angle TED$.
Hence it only remains to show $\angle BEM = \angle NED$;
we present several proofs below.

\paragraph{First proof (original solution).}
Let $P$ be the circumcenter of $BTSD$.
The properties of the Miquel point imply $P$ lies on
the common bisector $\ol{EH}$ already,
and it also lies on the perpendicular bisector of $\ol{BD}$,
hence it must be the midpoint of $\ol{HF}$.

We now contend quadrilaterals $BMPS$ and $DNPT$ are cyclic.
Obviously $\ol{MP}$ is the external angle bisector of $\angle BMS$,
and $PB = PS$, so $P$ is the arc midpoint of $(BMS)$.
The proof for $DNPT$ is analogous.

It remains to show $\angle BEN = \angle MED$,
or equivalently $\angle BEM = \angle NED$.
By properties of Miquel point we have $E \in (BMPS) \cap (TPND)$, so
\[ \dang BEM = \dang BPM = \dang PBD = \dang BDP = \dang NPD = \dang NED \]
as desired.

\paragraph{Second proof (2011 G4).}
By 2011 G4, the circumcircle of $\triangle EMN$
is tangent to the circumcircle of $ABCD$.
Hence if we extend $\ol{EM}$ and $\ol{EN}$ to meet $(ABCD)$
again at $X$ and $Y$, we get $\ol{XY} \parallel \ol{MN} \parallel \ol{BD}$.
Thus $\dang BEM = \dang BEX = \dang YED = \dang NED$.

\paragraph{Third proof (involutions, submitted by Daniel Liu).}
Let $G = \ol{BN} \cap \ol{MD}$ denote the centroid of $\triangle BCD$,
and note that it lies on $\ol{EHF}$.

Now consider the dual of Desargues involution theorem on complete
quadrilateral $BMDNCG$ at point $E$.
We get
\[ (EB,ED), \quad (EM,EN), \quad (EC,EG) \]
form an involutive pairing.

However, the bisector of $\angle BED$, say $\ell$,
is also the angle bisector of $\angle CEF$ (since $\ol{CF} \parallel \ol{BD}$).
So the involution we found must coincide with reflection across $\ell$.
This means $\angle MEN$ is bisected by $\ell$ as well, as desired.

\paragraph{Authorship comments.}
This diagram actually comes from the inverted
picture in IMO 2014/3 (which I attended).
I had heard for many years that one could
solve this problem quickly by inversion at $H$ afterwards.
But when I actually tried to do it during an OTIS class
years later, I ended up with the picture in the TST problem,
and couldn't see why it was true!
In the process of trying to reconstruct this rumored solution,
I ended up finding most of the properties
that ended up in the January TST problem
(but were overkill for the original IMO problem).

Let us make the equivalence explicit
by deducing the IMO problem from our work.

Let rays $EM$ and $EN$ meet the circumcircles of $\triangle BHC$
and $\triangle BNC$ again at $X$ and $Y$, with $EM < EX$ and $EN < EY$.
As above we concluded $EM/EX = EN/EY$
and so $\ol{MN} \parallel \ol{XY} \implies \ol{XY} \perp \ol{AHC}$.

Now consider an inversion at $H$ which swaps
$B \leftrightarrow D$ and $A \leftrightarrow C$.
The point $E$ goes to $E^\ast$ diametrically opposite $A$.
Points $X$ and $Y$ go to points on $X^\ast \in \ol{AD}$ and $Y^\ast \in \ol{AB}$.
Since the reflection of $E$ across $\ol{PX}$ is supposed to lie on $(BAE)$,
it follows that the circumcenter of $\triangle HX^\ast E^\ast$
lies on $\ol{AD}$.
Consequently $X^\ast$ plays the role of point ``$T$'' in the IMO problem.
Then $Y^\ast$ plays the role of point ``$S$'' in the IMO problem.

Now the fact that $(HX^\ast Y^\ast)$ is tangent to $\ol{BD}$
is equivalent to $\ol{XY} \perp \ol{AHC}$ which we already knew.
